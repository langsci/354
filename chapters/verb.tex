\chapter{Verbal morphology}\label{chapter: verb}\largerpage

In {\iaIA}, regular verbs are divided into simple verbs and complex verbs. In their infinitive citation form, simple verbs consist of a root, theme vowel, and infinitive suffix. Of these simple verbs, there are two conjugation classes based on the theme vowel. Complex verbs include a valency-changing morpheme. These include passives, causatives, and inchoatives. In contrast, irregular verbs can be divided into four categories: nasal-infixed verbs, suppletive verbs, defective verbs, and miscellaneous verbs.

When comparing {\iaIA} with other Armenian lects, {\iaIA} is close to {\seaSEA}. Like {\seaSE}, {\iaIA} widely uses periphrasis for various inflectional paradigm cells. Periphrasis is used for the indicative present, indicative past imperfective, and various complex tenses (present perfect, past perfect, future). Periphrasis involves the use of a non-finite converb (which carries lexical meaning) alongside an inflected auxiliary that carries tense/agreement marking. Synthesis is used for less frequent inflectional cells, such as subjunctives, conditionals, futures, and imperatives. The most common synthetic form is the past perfective, also called the aorist. 

There is a larger literature on the verbal morphology of other Armenian lects. For {\swaSWA} morphotactics, see \citet{Donabedian-1997-NeutralisationdeLaDiatheseDesParticipesEnAcDeLArmenianModerneOccidental}, \citet{Boyacioglu-2010-HayPayVerbsArmenianOccidentalWestArmenian}, \citet{boyaciogluDolatian-2020-ArmenianVerbs}, \citet{DolatianGuekguezian-prep-TierBasedLocalityArmenianConjugationClass,DolatianGuekguezian-prep-Morphome}, and \citet{KarakasDolatainGuekguezian-prep-DisentanglingTesnseAgreementWesternArmenian}. For {\seaSEA}, most work on verbal morphology is on   verbal semantics \citep{Kozintseva-1995-EasternBook,DumTragut-2009-ArmenianReferenceGrammar,DanielKhurshudyan-2015-ValancyClassesinEasternArmenian,Plungian-2018-EastArmNotesVerbalParadigmStems}. For {\iaIA}, we focus on providing complete paradigms for the different conjugation classes. We provide a complete segmentation of all inflectional morphology.

For reference, {\iaIA} shows the following significant differences from {\seaSEA} in terms of verbal morphology. 

\begin{itemize}
%	\item 
%
%
%
%\begin{exe}
%	\ex Summary of differences between {\seaSE} and {\iaIA} verbs: \label{ex:Verb:summary}
%	
%	\begin{xlist}
	\item The 1SG marker /{-m}/ is used in both the present and past paradigms (\S\ref{section:verb:aux:past}). 
		\item The present 3SG auxiliary is /{ɒ}/ in {\iaIA}, /{e}/ in {\seaSE} (\S\ref{section:verb:aux:pres}). The form [ɑ] is also attested in \seaCEA. 
		\item {\iaIA} deletes the auxiliary /{e}/ or theme vowel /{e}/ before the past marker /{i}/ (\S\ref{section:verb:aux:past}, \S\ref{section:verb:synthesis:subj}). 
		\item There is optional leveling of the negated copula and negated auxiliary (\S\ref{section:verb:aux:neg}).
		\item The perfective converb suffix displays liquid-zero alternations, briefly illustrated in \S\ref{section:verb:periphrasis:perfect}, discussed more in \S\ref{section:morphophono:auxiliary}.
		\item The past perfective or aorist system has been significantly altered, by promoting the past morph /-ɒ/ from a restricted marked allomorph to an elsewhere allomorph (\S\ref{section:verb:synthesis:perf}).
		\item The imperative 2SG suffix differs across the lects (\S\ref{section:verb:synthesis:imp}).
		
		\item Some irregular verbs in {\seaSE} have become leveled or lost in {\iaIA} (\S\ref{section:verb:irregular}).
%	\end{xlist}
\end{itemize}

For contrast, we often show the verbal paradigms of both {\seaSEA} and {\iaIA}. This chapter provides complete paradigms for the simplex verbs, and partial paradigms for complex and irregular verbs. Complete paradigms are found in our online archive.\footnote{\url{https://github.com/jhdeov/iranian_armenian}}

\begin{sloppypar}
Across Armenian varieties, the conjugation classes utilize different stems when forming the different paradigm cells. These are often called the present stem and the past/aorist stem. The aorist stem can be formed via various morphological strategies, such as root allomorphy and affix deletion. The aorist stem can include either an overt aorist suffix \textit{-t͡sʰ-} or a covert aorist suffix -$\emptyset$-. Due to space limitations, we do not explicitly discuss the formation of present vs. aorist stems in {\iaIA}. Our paradigms indicate the use of the aorist stem and aorist suffix \textit{-t͡sʰ/$\emptyset$-} in both the past perfective and other paradigm cells as {\aor}. When used in the past perfective, the aorist morpheme contributes perfective meaning; but it is used meaninglessly as a morphomic element in other paradigm cells \citep[cf.][]{Aronoff-1994-MorphologyItselfStemsInflectionClass}. For discussion of the formation of aorist stems in Standard Armenian, see \citet{DolatianGuekguezian-prep-Morphome}. 
\end{sloppypar}




\section{Simple verbs and their classes}\label{section:verb:simple}
Like in {\seaSEA}, 
regular simple verbs in {\iaIA} are classified into two classes based on the choice of theme vowel: \textit{{-e-, -ɒ-}} (\tabref{tab:Verb:Simple:Inf}). We call these classes E-Class and A-Class. The citation form is the infinitive, called the [ɑnoɾoʃ deɾbɑj] \armenian{անորոշ դերբայ} `indefinite participle'  in {\seaSEA}. 

\begin{table}
	\caption{Simple infinitives from the two regular classes}\label{tab:Verb:Simple:Inf}
	\begin{tabular}{ll lll}
		\lsptoprule 
		\multicolumn{2}{c}{E-Class} & \multicolumn{2}{c}{A-Class} &\\\midrule
		{jeɻkʰ-e-l} & {ɒpɻ-e-l} 
		& {kɒɻtʰ-ɒ-l} & {χos-ɒ-l} 	&$\sqrt{~}$-{\thgloss}-{\infgloss} \\
		`to sing' %(AS)
		& `to live' %(KM)
		& `to read'& `to speak' % (KM)
		% (NK)
		& \\
		\armenian{երգել}	& \armenian{ապրել}	& \armenian{կարդալ}	& \armenian{խօսալ} 	& \\ 
		\lspbottomrule
		\end{tabular}
\end{table}

{\seaSEA} uses the same conjugation classes. In general, a given verb belongs to the same conjugation class in both lects. There are some exceptions though. For example, the verb `to speak' uses the root \textit{{χos-}}. In {\iaIA}, this verb belongs to the A-Class: \textit{{χos-ɒ-l}} `to speak'. In contrast in {\seaSEA}, this verb belongs to the E-Class: \textit{{χos-e-l}}.\footnote{It is possible that these few deviations have a diachronic reason. Modern {\seaSE} and {\iaIA} utilize only two theme vowels: \textit{{-e-}} and \textit{{-ɑ/ɒ-}}. But Classical Armenian had two additional theme vowels \textit{{-i-}} and \textit{{-u-}}. Reflexes of verbs with these theme vowels are assigned to one of the surviving classes, usually to the E-Class. For example, `to speak' was an I-Class verb in Classical Armenian: \textit{{χos-i-l}}. The fact that this verb became E-Class in {\seaSEA}, but A-Class in {\iaIA} suggests that more deviations would be found in the reflexes of Classical verbs with obsolete theme vowels.}

In terms of morphological structure, we treat theme vowels as meaningless empty morphs \citep{Aronoff-1994-MorphologyItselfStemsInflectionClass}. The choice of theme vowel is root-conditioned and meaningless. 
For a theoretical analysis of Armenian theme vowels, see \citet{GuekguezianDolatian-2021-WCCFLThemeVowelWesternArmenianVerb}. Their {\swaSWA} analysis can easily extend to {\iaIA}. 



Having set up the basic classes, the next sections describe  verbal inflection. Like {\seaSEA}, verbal inflection in {\iaIA} is highly periphrastic. Before we describe these periphrastic forms, we first describe the auxiliary system in {\iaIA}.

\section{Auxiliaries}\label{section:verb:aux}

The verb `to be' acts as both a copula in predicate sentences (\ref{example: v as cop}), and as an auxiliary in periphrastic forms (\ref{example: v as aux}). 

\begin{exe}
	\ex 
	\begin{xlist}
		\ex \gll {mɒɻtʰ-ə} {təχuɻ} {ɒ}	\\
		man-{\defgloss} sad {\auxgloss}.{\prs}.3{\sg}
		\\
		\trans 	`The man is sad.'\label{example: v as cop} \hfill (NK)
		\\
		\armenian{Մարդը տխուր ա։}
		\ex \gll {mɒɻtʰ-ə} {jeɻkʰ-um} {ɒ} 
		\\
		man-{\defgloss} sing-{\impfcvb} {\auxgloss}.{\prs}.3{\sg}
		\\
		\trans 	`The man is singing.'\label{example: v as aux} \hfill (NK)
		\\
		\armenian{Մարդը երգում ա։}
		
	\end{xlist}
\end{exe}

In this section, we gloss the present 3SG auxiliary [ɒ] as `{\auxgloss}.{\prs}.3{\sg}' for explanation. But throughout the rest of the grammar, we usually just gloss it as {\auxgloss}.

In periphrastic constructions, the verb is in a converb form, e.g., the imperfective converb in (\ref{example: v as aux}). Before discussing these converbs, we first lay out the paradigm of the auxiliary. 
The name of the auxiliary is [oʒɑndɑk bɑj] \armenian{օժանդակ բայ} `helper verb' in {\seaSEA}. 

\subsection{Present auxiliary}\label{section:verb:aux:pres}

We show the present tense paradigm of the auxiliary in Table \ref{tab:present aux paradigm}. Because the auxiliary can also function as a copula, we gloss both as {\auxgloss}. In the present tense, the auxiliary consists of the auxiliary's marker \textit{{-e-}}, and then a fused tense-agreement marker ({\tense}/{\agr} or just {\agr}). In the 3SG, there is no T/Agr marker. Instead, the inflected auxiliary is just the auxiliary marker /e/  in {\seaSEA}. In contrast, in {\iaIA}, the 3SG present uses an allomorph /ɒ/ of the auxiliary.




\begin{table}
	\centering
	\caption{Paradigm of the present auxiliary and copula in {\seaSE} and {\iaIA}}
	\label{tab:present aux paradigm}
	\begin{tabular}{ll l  l l }
		\lsptoprule
		&\multicolumn{2}{l}{{\seaSE}} &\multicolumn{2}{l}{{\iaIA}}\\\midrule
		1SG & {{e-m}}& \armenian{եմ}& {{e-m}} & \armenian{եմ} \\
		&`I am' & & `I am' & \\
		2SG & {{e-s}} & \armenian{ես}& {{e-s}} & \armenian{ես} \\
		3SG & {{e}} & \armenian{է} & {{ɒ}} & \armenian{ա} \\
		1PL & {{e-ŋkʰ}} & \armenian{ենք} & {{e-ŋkʰ}} & \armenian{ենք}\\
		2PL &{{e-kʰ}} & \armenian{եք} & {{e-kʰ}} & \armenian{էք}\\
		3PL & {{e-n}} & \armenian{են}& {{e-n}} & \armenian{են}\\
		& \multicolumn{4}{l}{{\auxgloss}-{\agr}}\\
		\lspbottomrule
	\end{tabular} 
\end{table}

The {\iaIA} 3SG form /{ɒ}/ is likely diachronically derived from an earlier /{e}/ form. In fact, the 3SG auxiliary /{ɒ}/ is found in the colloquial speech of {\seaSE} speakers in Armenia as /ɑ/. For {\iaIA}, the low-vowel form /{ɒ}/ form is simply grammaticalized as the only realization of the present 3SG auxiliary.



We utilize the following rules for {\iaIA} (Rule \ref{rule aux present t/agr}). Tense and agreement are expressed via a single marker in the present. 

\begin{newruleblock}[rule aux present t/agr]{Rules for marking present agreement} %\label{rule aux present t/agr}
	\begin{center}
		\begin{tabular}{lll ll}
			1{\sg} &$\leftrightarrow$ &\textit{{-m}}\\
			2{\sg}, present &$\leftrightarrow$ &\textit{{-s}}\\
			3{\sg}, present &$\leftrightarrow$ &-$\emptyset$\\
			1{\pl} &$\leftrightarrow$ &\textit{{-nkʰ}}\\
			2{\pl} &$\leftrightarrow$ &\textit{{-kʰ}}\\
			3{\pl} &$\leftrightarrow$ &\textit{{-n}}
		\end{tabular}
	\end{center}
\end{newruleblock} 

Note that the 1PL suffix is underlyingly /-nkʰ/ and the nasal assimilates in place to become [-ŋkʰ] (\S\ref{section:phono:segmental:other cons}). This plural morpheme is a reflex of Classical *\textit{-m-kʰ}. Compare modern [eŋkʰ] against Classical \armenian{եմք} <emk'> \citep[26]{Thomson-1989-IntroClassicalArmenian}. 

The markers of the 1SG and the plurals do not specify tense. As we see later, these markers are used throughout {\iaIA} for these person-number combinations. 

As for the auxiliary itself (Rule \ref{rule aux present}), it has allomorphs /{e}/ and /{ɒ}/. For the present 3SG, the auxiliary is expressed by /{ɒ}/ without an extra tense marker. We later revise the marker rules for the auxiliary.

\begin{newruleblock}[rule aux present]{Rules for the form of the auxiliary verb `to be' in the present (to be revised)} % \label{rule aux present}
	\begin{center}
		\begin{tabular}{lll ll}
			`be' or {\auxgloss} &$\leftrightarrow$ &{{ɒ-}} & / \_ {\prs}.3{\sg}\\
			& &{{e-}} & / elsewhere\\
		\end{tabular}
	\end{center}
\end{newruleblock} 



\subsection{Past auxiliary}\label{section:verb:aux:past}

For the present auxiliary, {\seaSE} and {\iaIA} have few differences. But in the past form of the auxiliary, we find two major differences between the two lects. In Table \ref{tab:past aux}, we provide zero markers for easier illustration. Note the glide is epenthetic.


\begin{table}
	\caption{Paradigm of the past auxiliary in {\seaSE} and {\iaIA}\label{tab:past aux}}
	\begin{tabular}{ l ll ll  ll  }
		\lsptoprule 
		& \multicolumn{4}{c}{Without zero markers}& \multicolumn{2}{c}{With zero markers}\\\cmidrule(lr){2-5}\cmidrule(lr){6-7}
		&\multicolumn{2}{l}{{\seaAbbre}}&\multicolumn{2}{l}{{\iaAbbre}} &{\seaAbbre} & {\iaAbbre}\\\midrule
		1SG & {{ej-i}} & \armenian{էի} & {{i-m}}&\armenian{իմ} & {{ej-i}-$\emptyset$} & {$\emptyset$-{i-m}} \\ 
		    & `I was' & & `I was' & & & \\ 
		2SG & {{ej-i-ɾ}} & \armenian{էիր}& {{i-ɻ}} & \armenian{իր} & {{ej-i-ɾ}} & {$\emptyset$-{i-ɻ}} \\
		3SG & {{e-ɾ}} &\armenian{էր}& {{e-ɻ}} & \armenian{էր} & {{e-$\emptyset$-ɾ}} & {{e-$\emptyset$-ɻ}} \\
		1PL & {{ej-i-ŋkʰ}} & \armenian{էինք} & {{i-ŋkʰ}} & \armenian{ինք} & {{ej-i-ŋkʰ}} & {$\emptyset$-{i-ŋkʰ}}\\
		2PL &{{ej-i-kʰ}} & \armenian{էիք}& {{i-kʰ}} & \armenian{իք} &{{ej-i-kʰ}} & {$\emptyset$-{i-kʰ}}\\
		3PL & {{ej-i-n}} & \armenian{էին} &{{i-n}} & \armenian{ին} & {{ej-i-n}} & {$\emptyset$-{i-n}} \\
		    & & 	& & 	& \multicolumn{2}{l}{{\auxgloss}-{\pst}-{\agr}}\\
		\lspbottomrule 
	\end{tabular}
\end{table}



Consider first the non-3SG forms. In {\seaSEA}, the past form of the auxiliary is made up of three overt morphs: the auxiliary \textit{{e}}, a past suffix \textit{{-i}}, and then agreement. Tense and agreement are thus separate suffixes in the past. Vowel hiatus between the auxiliary and past suffix triggers glide epenthesis: 1PL /e-i-nkʰ/ $\rightarrow$ [{{ej-i-ŋkʰ}}]. In contrast, in {\iaIA}, the auxiliary morpheme is covert in these contexts. Outside of the 3SG, there are only two overt morphs and these are the past suffix and the agreement suffix. For example, 1PL is [{ej-i-ŋkʰ}] in {\seaSE} but [{{i-ŋkʰ}}] in {\iaIA}. 

We analyze this difference as due to a morpheme-specific rule of vowel deletion in hiatus (Rule \ref{rule e deletion}). This rule will delete the vowel \textit{{e}} before the past morpheme \textit{{-i}}. We call this rule \textit{e-deletion}. The target of this rule is just a segment, while the trigger is a specific morph. 

\vfill
\begin{newruleblock}[rule e deletion]{\textbf{\textit{e}-Deletion:   Rule for deleting  /{e}/  before past  /{i}/}}%%\label{rule e deletion}
	\begin{center}
		\begin{tabular}{lll l}
			/{e}/ & $\rightarrow$ & $\emptyset$ & / \_ {i}
			\\
			& & & where \textit{i} is the past suffix
		\end{tabular}
	\end{center}
\end{newruleblock}
\vfill\pagebreak

In morphological theory, the use of morpheme-specific phonological processes is controversial \citep{Pater-2007-LocusExceptionalityMorphemeSpecificConstraitnIndexation,siddiqi-2009-syntaxWithinWordEconomyAllomorphyArgumentSeelctionDistributedMorphology,haugenSiddiqi-2016-towardsRestrictedTheoryMultimorphemicMonolistemicityPortmanteauxPostLinearizationSpanning,Haugen-2016-ReadjustmentRejected,EmbickSchwayer-MorphoPhonoMisapplications}. There are two pieces of evidence for treating the absence of the auxiliary \textit{{-e-}} as morpheme-specific phonology instead of allomorphy. First, in the 3SG, the past suffix is covert, and the auxiliary is overt: \textit{{e-$\emptyset$-ɻ}} instead of *\textit{{$\emptyset$-i-ɻ}} or *\textit{{$\emptyset$-$\emptyset$-ɻ}}. It thus seems that the absence of the auxiliary is conditioned by making the past suffix an overt vowel. Second, we will see in the subjunctive past (\S\ref{section:verb:synthesis:subj})  that the \textit{{-e-}} theme vowel likewise deletes before the past \textit{{-i-}} suffix. In sum, the above rule was possibly developed in {\iaIA} as a morpheme-specific rule for repairing vowel hiatus. 


%{\added}
Outside of {\iaIA}, there are other Armenian dialects where the past auxiliary has this reduced form. For example, in 1911, the dialect of Armenian spoken in Yerevan had past auxiliaries like 3PL [$\emptyset$-i-n]  (\citealt[43]{Adjarian-1911-DialectologyBook}; translated by \citealt{Dolatian-prep-Adjarian}). Such auxiliary forms were lost in Yerevan, due to the language shift from (Old) Yerevan Armenian to {\seaSEA}. But they remain as grammaticalized in {\iaIA}. 

The second difference between the lects concerns the 1SG. In {\seaSE}, the Agr morph is covert: \textit{{e-i-$\emptyset$}} `I was'. In {\iaIA}, the Agr morph is an overt /{m}/: \textit{{$\emptyset$-i-m}}. This /{m}/ morph is the same suffix used in the present 1SG [e-m]. Thus this morph /m/ has a more general distribution in {\iaIA} than in {\seaSE}. We list the rules for the 1SG below for the two lects for the two tenses (Rule \ref{rule 1sg}).

The use of \textit{-m} as a general 1SG marker is rather common in Armenian lects in Iran \citep[p. 103, feature 100.6]{Jahukyan-1972-ArmenianDiaolectology}. See \citet[55--56]{Vaux-Salmast} for useful maps on the spread of this phenomenon across Iran. For the spread of the \textit{{-m}} morph, it is possible that a contributing factor is that Persian uses a morph \textit{{-æm}} as a generalized 1SG marker for both the present and past \citep[229ff]{Mahootian-2002-PersianGrammar}. 

\begin{newruleblock}[rule 1sg]{Rules for the 1SG in the two lects}%\label{rule 1sg}
	\begin{center}
		\begin{tabular}{lllll}
			\textit{{\seaSE}} & 1{\sg} & $\leftrightarrow$ & -$\emptyset$ & / in the past\\
							  & & & \textit{{-m}} & / elsewhere\\
			\textit{{\iaIA}}  &	1{\sg} & $\leftrightarrow$ & \textit{{-m}} & \\
		\end{tabular}
	\end{center}
\end{newruleblock} 


We list below the additional rules that are needed for the {\iaIA} 3SG (Rule \ref{rule past impf i}). The past morph is covert in the 3SG: [e-$\emptyset$-ɻ], while an overt /-i/ elsewhere.  We do not need to list any rules for plural Agr, because they are the same as for the present (\S\ref{section:verb:aux:pres}). 

\begin{newruleblock}[rule past impf i]{Rules for past tense and agreement in 3SG}%%\label{rule past impf i}
	\begin{center}
		\begin{tabular}{llll}
			\pst{} & $\leftrightarrow$ & \textit{{-$\emptyset$}} & / in 3{\sg}\\
			& & \textit{{-i}} &/ elsewhere\\
			singular non-1st person & $\leftrightarrow$ & \textit{{-ɻ}}& / in the past\\
		\end{tabular}
	\end{center}
\end{newruleblock} 

The past 2SG and 3SG are syncretic for the agreement suffix \citep{KarakasDolatainGuekguezian-prep-DisentanglingTesnseAgreementWesternArmenian}. They both use the morph \textit{{ɻ}}. The two paradigm cells are distinguished by tense being overt in the 2SG, but covert in the 3SG: \textit{{$\emptyset$-i-ɻ}} `you were' vs. \textit{{e-$\emptyset$-ɻ}} `he was'. 

%For illustration, we provide below the morphological structure of the past 2SG and 3SG in {\iaIA}. The 2SG form is provided in two forms: its underlying form before the application of \textit{{e}}-deletion, and its surface form after. 

%
%\begin{exe}
%	\ex \textit{Forms of the past 2SG and 3SG of the auxiliary}
%	
%	\begin{tabular}{|l|l|l|}
	%		\hline 
	%		2SG before \textit{{e}-}Deletion& 2SG after \textit{{e}}-Deletion & 3SG \\
	%		/{e-i-ɻ}/ & [{$\emptyset$-i-ɻ}] & /{e-$\emptyset$-ɻ}/
	%		\\
	%		\hline \begin{tikzpicture} 
		%			\Tree 
		%			[.{\agr} [.{\tense} [.\textit{v} {e} ] [.{\tense} {-i} ] ] [ [.{\agr} {-ɻ} ] ] ]
		%		\end{tikzpicture}
	%		&
	%		\begin{tikzpicture} 
		%			\Tree 
		%			[.{\agr} [.{\tense} [.\textit{v} $\emptyset$ ] [.{\tense} {-i} ] ] [ [.{\agr} {-ɻ} ] ] ]
		%		\end{tikzpicture}
	%		&
	%		\begin{tikzpicture} 
		%			\Tree 
		%			[.{\agr} [.{\tense} [.\textit{v} {e} ] [.{\tense} $\emptyset$ ] ] [ [.{\agr} {-ɻ} ] ] ]
		%		\end{tikzpicture}
	%		
	%		\\ \hline\end{tabular}
%\end{exe}


\subsection{Negation}\label{section:verb:aux:neg}\largerpage
The previous subsections described the inflection of the auxiliary in the positive. Negation is straightforwardly marked by adding the negation prefix \textit{{t͡ʃʰ-}}. However, we see some divergences in the present 3SG.

%%\begin{exe}
%	\ex \textit{Rule for the negative}
%	
%	\begin{tabular}{llll}
	%		{\neggloss} & $\leftrightarrow$ & {{t͡ʃʰ-}} 
	%	\end{tabular}
%\end{exe}

Table \ref{tab:neg aux present} shows the paradigm for the negated present auxiliary for {\seaSE} and {\iaIA}. For all but the present 3SG, negation is marked by adding the negation prefix \textit{{t͡ʃʰ-}} to the auxiliary.


\begin{table}
	\caption{Paradigm of negated present auxiliary in {\seaSE} and {\iaIA}}
	\label{tab:neg aux present}
	\begin{tabular}{lllll llll}
		\lsptoprule 
		&\multicolumn{8}{l}{Present: ({\neggloss})-{\auxgloss}-{\agr}} \\ 
		&\multicolumn{4}{c}{{\seaSE}} & \multicolumn{4}{c}{{\iaIA}}\\\cmidrule(lr){2-5}\cmidrule(lr){6-9}
		& Positive& & Negaive&& Positive & &Negaive& \\\midrule
		1SG & {{e-m}} &\armenian{եմ} & {{t͡ʃʰ-e-m}} & \armenian{չեմ} & {{e-m}} & \armenian{եմ} & {{t͡ʃʰ-e-m}} & \armenian{չեմ} \\
		2SG & {{e-s}} & \armenian{ես}& {{t͡ʃʰ-e-s}} & \armenian{չես} & {{e-s}} & \armenian{ես} & {{t͡ʃʰ-e-s}} & \armenian{չես} \\
		3SG & {{e-$\emptyset$}} & \armenian{է}& {{t͡ʃʰ-i-$\emptyset$}} & \armenian{չի} & {{ɒ-$\emptyset$}} & \armenian{ա} & {{t͡ʃʰ-i-$\emptyset$}} & \armenian{չի}\\
		1PL & {{e-ŋkʰ}} &\armenian{ենք} & {{t͡ʃʰ-e-ŋkʰ}} &\armenian{չենք} & {{e-ŋkʰ}} & \armenian{ենք} & {{t͡ʃʰ-e-ŋkʰ}} & \armenian{չենք}\\
		2PL & {{e-kʰ}} & \armenian{եք} & {{t͡ʃʰ-e-kʰ}} & \armenian{չեք}& {{e-kʰ}} & \armenian{էք} & {{t͡ʃʰ-e-kʰ}} &\armenian{չէք}\\
		3PL & {{e-n}} & \armenian{են} & {{t͡ʃʰ-e-n}} & \armenian{չեն}& {{e-n}} & \armenian{են} & {{t͡ʃʰ-e-n}} &\armenian{չեն}\\
		\lspbottomrule
	\end{tabular}
\end{table}

Table \ref{tab:neg aux past} shows the paradigm of the negated past auxiliary. Negation is marked by adding the negation prefix.

\begin{table}
	\caption{Paradigm of negated past auxiliary in {\seaSE} and {\iaIA}\label{tab:neg aux past}}
	\resizebox{\textwidth}{!}{%
		\begin{tabular}{lllll llll}
			\lsptoprule 
			&\multicolumn{8}{l}{Past: ({\neggloss})-{\auxgloss}-{\pst}-{\agr}}\\
			&\multicolumn{4}{c}{{\seaSE}} & \multicolumn{4}{c}{{\iaIA}} \\\cmidrule(lr){2-5}\cmidrule(lr){6-9}
			& Positive & &Negaive&& Positive && Negaive& \\\midrule
			1SG & {{ej-i-$\emptyset$}} &\armenian{էի} & {{t͡ʃʰ-ej-i-$\emptyset$}} &\armenian{չէի} & {{$\emptyset$-i-m}} &\armenian{իմ} & {{t͡ʃʰ-$\emptyset$-i-m}} & \armenian{չիմ}\\
			2SG & {{ej-i-ɾ}} & \armenian{էիր}& {{t͡ʃʰ-ej-i-ɾ}} & \armenian{չէիր} & {{$\emptyset$-i-ɻ}} & \armenian{իր}& {{t͡ʃʰ-$\emptyset$-i-ɻ}} & \armenian{չիր}\\
			3SG & {{e-$\emptyset$-ɾ}} & \armenian{էր}& {{t͡ʃʰ-e-$\emptyset$-ɾ}} & \armenian{չէր} & {{e-$\emptyset$-ɻ}} & \armenian{էր} & {{t͡ʃʰ-e-$\emptyset$-ɻ}} & \armenian{չէր}\\
			1PL & {{ej-i-ŋkʰ}} & \armenian{էինք}& {{t͡ʃʰ-ej-i-ŋkʰ}} &\armenian{չէինք} & {{$\emptyset$-i-ŋkʰ}} & \armenian{ինք} & {{t͡ʃʰ-$\emptyset$-i-ŋkʰ}} & \armenian{չինք}\\
			2PL & {{ej-i-kʰ}} &\armenian{էիք} & {{t͡ʃʰ-ej-i-kʰ}} & \armenian{չէիք} & {{$\emptyset$-i-kʰ}} &\armenian{իք} & {{t͡ʃʰ-$\emptyset$-i-kʰ}} & \armenian{չիք}\\
			3PL & {{ej-i-n}} & \armenian{էին} & {{t͡ʃʰ-ej-i-n}} & \armenian{չէին} & {{$\emptyset$-i-n}} & \armenian{ին} & {{t͡ʃʰ-$\emptyset$-i-n}} & \armenian{չին}\\
			\lspbottomrule
		\end{tabular}}
\end{table}


Differences emerge in the present 3SG. When used as a verbal auxiliary in Table \ref{tab:neg aux variation}, the positive form is /{ɒ}/ in {\iaIA}, and /{e}/ in {\seaSE}. The negative form is /{{t͡ʃʰ-i}}/ for both lects. The negative auxiliary is placed before the verb.{\interfootnotelinepenalty=10000\footnote{The complete gloss for the copula and auxiliary in the tables is {\auxgloss}.{\prs}.3{\sg}. The complete gloss for the verb `singing' is sing-{\impfcvb}.}}

\begin{table}
	\caption{Forms of negative auxiliary across {\seaSE} and {\iaIA}\label{tab:neg aux variation}}
	\begin{tabular}{lll llll ll}
		\lsptoprule 
		 & \multicolumn{4}{c}{Positive}&\multicolumn{4}{c}{Negaive} \\\cmidrule(lr){2-5}\cmidrule(lr){6-9}
		{\seaAbbre} & {{jeɾkʰ-um}}& {{e}} & \multicolumn{2}{l }{\armenian{Երգում է։}} & {{t͡ʃʰ-i}}& {{jeɾkʰ-um}} & \multicolumn{2}{l }{\armenian{Չի երգում։}}\\
		{\iaAbbre} & {{jeɻkʰ-um}} &{{ɒ}}& \multicolumn{2}{l }{\armenian{Երգում ա։}} & {{t͡ʃʰ-i}} &{{jeɻkʰ-um}} & \multicolumn{2}{l }{\armenian{Չի երգում։}}\\
		Gloss: & singing & is & & & {\neggloss}-is& singing & & \\
		&\multicolumn{4}{l}{`He is singing.'}&\multicolumn{4}{l}{`He isn't singing.'}\\ 
		\lspbottomrule
	\end{tabular}
\end{table}


But when used as a copula, we find more significant dialectal differences in Table \ref{tab:neg cop variation}. In both the positive and negative, the copula is placed after the predicate. The positive form is /{ɒ}/ in {\iaIA} and /{e}/ in {\seaSE}, as expected. When negated, {\seaSE} uses /{t͡ʃʰ-e}/. In {\iaIA}, speakers can use either /{t͡ʃʰ-e}/ or /{t͡ʃʰ-i}/. We call the use of /{t͡ʃʰ-e}/   the un-leveled form, while the use of /{t͡ʃʰ-}i/ is the leveled form. Such variation is also documented for {\seaCEA} \citep[216]{DumTragut-2009-ArmenianReferenceGrammar}.\largerpage


\begin{table}
	\caption{Forms of negative copula across {\seaSE} and {\iaIA}\label{tab:neg cop variation}}
	\begin{tabular}{lll llll ll}
		\lsptoprule
		&\multicolumn{4}{c}{Positive} & \multicolumn{4}{c}{Negaive} \\\cmidrule(lr){2-5}\cmidrule(lr){6-9}
		{\seaAbbre}& {{uɾɑχ}} & {{e}} & \multicolumn{2}{l }{\armenian{Ուրախ է։}}& {{uɾɑχ}}& {{t͡ʃʰ-e}}
		
		& \multicolumn{2}{l }{\armenian{Ուրախ չէ։ }}\\
		
		{\iaAbbre} (un-leveled) & {{uɻɒχ}} &{{ɒ}}& \multicolumn{2}{l }{\armenian{Ուրախ ա։}} & {{uɻɒχ}} &{{t͡ʃʰ-e}} & \multicolumn{2}{l }{\armenian{Ուրախ չէ։ }}\\
		
		{\iaAbbre} (leveled) & {{uɻɒχ}} &{{ɒ}}& \multicolumn{2}{l }{\armenian{Ուրախ ա։}} & {{uɻɒχ}} &{{t͡ʃʰ-i}} & \multicolumn{2}{l }{\armenian{Ուրախ չի։ }}\\
		Gloss:               & happy    & is  &   & & happy & {\neggloss}-is & &\\
		&\multicolumn{4}{l}{`He is happy.'}&\multicolumn{4}{l}{`He isn't happy.'} \\ 
		\lspbottomrule
	\end{tabular}
\end{table}


% \begin{exe}
	% \ex 
	% \begin{multicols}{2}
		% \begin{xlist}
			% \ex {\seaSE}
			% \begin{xlist}
				% \ex \gll {uɾɑχ} e 
				% \\
				% happy {\auxgloss}.{\prs}.3{\sg} 
				% \\
				% \trans `He is happy.'
				
				% \ex \gll {uɾɑχ} t͡ʃʰ-e
				% \\
				% happy {\neggloss}-{\auxgloss}.{\prs}.3{\sg}
				% \\
				% \trans `He isn't happy.'
				% \ex \gll {jeɾkʰ-um} e 
				% \\
				% sing-{\impfcvb} {\auxgloss}.{\prs}.3{\sg} 
				% \\
				% \trans `He is singing.'
				% \ex \gll {t͡ʃʰ-i} {jeɾkʰ-um} 
				% \\
				% {\neggloss}-{\auxgloss}.{\prs}.3{\sg} sing-{\impfcvb}
				% \\
				% \trans `He isn't singing.'
				% \end{xlist}
			% \ex {\iaIA}
			% \begin{xlist}
				% \ex \gll {uɻɒχ} ɒ
				% \\
				% happy {\auxgloss}.{\prs}.3{\sg} 
				% \\
				% \trans `He is happy.'
				
				% \ex \gll uɻɒχ t͡ʃʰ-e
				% \\
				% happy {\neggloss}-{\auxgloss}.{\prs}.3{\sg}
				% \\
				% \trans `He isn't happy.'
				% \ex \gll jeɻkʰ-um ɒ
				% \\
				% sing-{\impfcvb} {\auxgloss}.{\prs}.3{\sg} 
				% \\
				% \trans `He is singing.'
				% \ex \gll t͡ʃʰ-i jeɻkʰ-um 
				% \\
				% {\neggloss}-{\auxgloss}.{\prs}.3{\sg} sing-{\impfcvb}
				% \\
				% \trans `He isn't singing.'
				% \end{xlist}
			% \end{xlist}
		
		% \end{multicols}
	% \end{exe}

The above patterns require the following rules (Rule \ref{rule verb is}). For {\seaSE}, the verb `to be' surfaces as /{i}/ only when it is an auxiliary, negative, and present 3SG. In all other contexts (including as a copula), it surfaces as the elsewhere morph /{e}/.

\begin{newruleblock}[rule verb is]{Rules for the auxiliary verb `to be' in {\seaSE}}%%\label{rule verb is}
	
	\begin{center}
		\begin{tabular}{llll}
			`be' or {\auxgloss}  & $\leftrightarrow$ & {{i-}} &/ {\neggloss} \_ {\prs}.3{\sg}, used as verbal auxiliary\\
								 &                   &        & \hphantom{/}  (not a copula)\\
								 &                 & {e-} &/ elsewhere
		\end{tabular}
	\end{center}
\end{newruleblock}


For {\iaIA}, matters are slightly more complicated. Some speakers can use /{i}/ in the negative of both the auxiliary and the copula. All speakers use the form /{ɒ}/ in the positive of both the auxiliary and copula. This simpler leveled system uses the rules below (Rule \ref{rule verb is level}). The rule for /{i}/ simply does not reference the auxiliary vs. copula status of the verb. The verb surfaces as [{ɒ}] in the positive present 3SG, and as [{e}] elsewhere.

\begin{newruleblock}[rule verb is level]{Rules for the auxiliary verb `to be' in {\iaIA} with full leveling}%%\label{rule verb is level}
	
	\begin{center}
		\begin{tabular}{llll}
					`be' or {\auxgloss} & $\leftrightarrow$ & {i-} &/ {\neggloss} \_ {\prs}.3{\sg} \\
			&&{ɒ-} &/ \_ {\prs}.3{\sg} \\
			& & {e-} &/ elsewhere
		\end{tabular}
	\end{center}
\end{newruleblock}


As for the speakers who have not leveled the negative copula towards the negative auxiliary, they need the more complicated system below (Rule \ref{rule verb is level no}). These speakers use /{i}/ for the negative auxiliary, /{ɒ}/ for the positive verb, and /{e}/ elsewhere. 

\begin{newruleblock}[rule verb is level no]
	{Rules for the auxiliary verb `to be' without leveling}%%\label{rule verb is level no}
	
	\begin{center}
		\begin{tabular}{llll}
			`be' or {\auxgloss} & $\leftrightarrow$ & {i-} &/ {\neggloss} \_ {\prs}.3{\sg}, used as auxiliary verb \\
			& & &  \hphantom{/} (not as copula)\\
			& & {ɒ-} &/ \_ {\prs}.3{\sg} \\
			& & {e-} &/ elsewhere
		\end{tabular}
	\end{center}
	
\end{newruleblock}






\section{Periphrastic structures}\label{section:verb:periphrasis}
{\iaIA} uses periphrasis     to realize most tense-aspect-mood combinations. These periphrastic forms all utilize a special form of the verb called the converb. Tense and agreement are marked on the auxiliary. The auxiliary follows the converb in the positive, and it precedes the converb in the negative.\footnote{ The auxiliary can further move around the sentence because of focus and other syntactic factors (\S\ref{section:morphophono:auxiliary:syntax}).} Note that the future is marked with both synthetic and periphrastic constructions, discussed in \S\ref{section:verb:fut}. 

Throughout this grammar, we reserve the term “converb” for non-finite verb forms that are restricted to verbal periphrasis. We use the term “participle” for non-finite verb forms that can be used outside of periphrasis. This seems to be the intuition behind the use of these terms in the Eastern Armenian National Corpus.\footnote{\url{eanc.net/}}


\subsection{Indicative present and past imperfective}\label{section:verb:periphrasis:indicative}\largerpage
The first periphrastic construction that we describe is the indicative imperfective forms, called [sɑhmɑnɑkɑn jeʁɑnɑk] \armenian{սահմանական եղանակ} in {\seaSEA}. This construction is used in the indicative present and the indicative past imperfective (also called the past imperfect). This construction is formed identically in {\seaSE} and {\iaIA}. 

\begin{sloppypar}
The verb is in a converb form called the imperfective converb (\ref{sent:Verb:Peri:IndcPres:basic}). Some grammars also use the term present participle \citep[219]{DumTragut-2009-ArmenianReferenceGrammar}. In {\seaSEA}, this converb is called [ɑŋkɑtɑɾ deɾbɑj] \armenian{անկատար դերբայ}. Given the infinitive for a verb like \textit{{jeɻkʰ-e-l}} `to sing', the imperfective converb is formed by replacing the theme vowel and infinitive with the suffix \textit{{-um}}: \textit{{jeɻkʰ-um}}. Tense and subject agreement are marked on the auxiliary. The present auxiliary is used to form the indicative present; the past auxiliary is used to form the indicative past imperfective. 
\end{sloppypar}

\begin{exe}
	\ex \label{sent:Verb:Peri:IndcPres:basic}
	\begin{xlist}
		\ex \gll {jeɻkʰ-um} {e-ŋkʰ}
		\\
		sing-{\impfcvb} {\auxgloss}-1{\pl}
		\\
		\trans			`We are singing.' \hfill (NK)
		\\
		\armenian{Երգում ենք։}
		\ex \gll {jeɻkʰ-um} {$\emptyset$-i-ŋkʰ}
		\\
		sing-{\impfcvb} {\auxgloss}-{\pst}-1{\pl}
		\\
		\trans			`We were singing.' \hfill (NK)
		\\
		\armenian{Երգում ինք։}
	\end{xlist}
\end{exe}

Negation is marked by placing the negated form of the auxiliary before the converb (\ref{sent:Verb:Peri:IndcPres:basic:neg}).

\begin{exe}
	\ex \label{sent:Verb:Peri:IndcPres:basic:neg}
	\begin{xlist}
		\ex \gll {t͡ʃʰ-e-ŋkʰ} {jeɻkʰ-um}
		\\
		{\neggloss}-{\auxgloss}-1{\pl} sing-{\impfcvb} 
		\\
		\trans			`We are not singing.' \hfill (NK)
		\\
		\armenian{Չենք երգում։}
		\ex \gll {t͡ʃʰ-$\emptyset$-i-ŋkʰ} {jeɻkʰ-um}
		\\
		{\neggloss}-{\auxgloss}-{\pst}-1{\pl} sing-{\impfcvb} 
		\\
		\trans			`We were not singing.' \hfill (NK)
		\\
		\armenian{Չինք երգում։}
	\end{xlist}
\end{exe}

The two conjugation classes (E-Class and A-Class) do not differ in constructing the imperfective converb, e.g., the converb of \textit{{kɒɻtʰ-ɒ-l}} `to read' is \textit{{kɒɻtʰ-um}}. All tense-number-person combinations are straightforwardly marked by using the appropriate inflected auxiliary. The complete paradigm is given in Table \ref{tab:indc pres past}. For clarity of presentation, we do not segment the internal structure of the auxiliary.\largerpage

The imperfective converb suffix is simply \textit{-um}. If we assume that the theme vowels /{{e, ɒ}}/ are underlyingly present, then we need a rule that deletes the theme vowels before the converb suffix, as a type of morpheme-specific vowel hiatus repair (Rule \ref{rule delete theme in converb}). For example, 	/{jeɻkʰ-e-um}/ $ \rightarrow$ [{jeɻkʰ-$\emptyset$-um}]. 

\begin{newruleblock}[rule delete theme in converb]
	{Deleting theme vowels before the converb suffix}%%\label{rule delete theme in converb}
	
	\begin{center}
		
		\begin{tabular}{lllll}
			V & $\rightarrow$ & $\emptyset$ & / \_ V\textsubscript{2} & (where V\textsubscript{2} is part of a converb suffix)
			\\
		\end{tabular}
	\end{center}
\end{newruleblock}

\begin{table}
	\caption{Paradigm for indicative present and indicative past imperfective for E-Class [{jeɻkʰ-e-l}] `to sing'} \label{tab:indc pres past}
	\resizebox{\textwidth}{!}{%
		\begin{tabular}{l ll ll ll ll}
			\lsptoprule 
			&\multicolumn{4}{c}{Positive} &\multicolumn{4}{c}{Negaive} \\\cmidrule(lr){2-5}\cmidrule(lr){6-9}
			& \multicolumn{2}{l}{Indc. present} & \multicolumn{2}{l}{Indc. past imperf.} & \multicolumn{2}{l}{Indc. present} & \multicolumn{2}{l}{Indc. past imperf.}\\\midrule
			1SG
			&
			{jeɻkʰ-um} &			{em}
			&
			{jeɻkʰ-um} &			{im}
			&
			{t͡ʃʰ-em } & {jeɻkʰ-um}
			&
			{t͡ʃʰ-im } &{jeɻkʰ-um}
			\\
			& \multicolumn{2}{l}{`I am singing'}
			& \multicolumn{2}{l}{`I was singing'}
			& \multicolumn{2}{l}{`I am not singing'}
			& \multicolumn{2}{l}{`I was not singing'}
			\\
			& \armenian{երգում}& \armenian{եմ}
			& \armenian{երգում} & \armenian{իմ}
			& \armenian{չեմ} & \armenian{երգում}
			& \armenian{չիմ} & \armenian{երգում}
			\\
			\addlinespace 
			2SG 
			&
			{jeɻkʰ-um} &{es}
			&
			{jeɻkʰ-um} &{iɻ}
			&
			{t͡ʃʰ-es } &{jeɻkʰ-um}
			&
			{t͡ʃʰ-iɻ } &{jeɻkʰ-um}
			\\
			& \armenian{երգում} & \armenian{ես}
			& \armenian{երգում} & \armenian{իր}
			& \armenian{չես} & \armenian{երգում}
			& \armenian{չիր} & \armenian{երգում}
			\\
			\addlinespace 
			3SG
			&
			{jeɻkʰ-um} &{ɒ}
			&
			{jeɻkʰ-um} &{eɻ}
			&
			{t͡ʃʰ-i } &{jeɻkʰ-um}
			&
			{t͡ʃʰ-eɻ } &{jeɻkʰ-um}
			\\
			& \armenian{երգում} & \armenian{ա}
			& \armenian{երգում} & \armenian{էր}
			&\armenian{չի} & \armenian{երգում}
			& \armenian{չէր} & \armenian{երգում}
			\\
			\addlinespace 
			1PL
			&
			{jeɻkʰ-um} &{eŋkʰ}
			&
			{jeɻkʰ-um} &{iŋkʰ}
			&
			{t͡ʃʰ-eŋkʰ } &{jeɻkʰ-um}
			&
			{t͡ʃʰ-iŋkʰ } &{jeɻkʰ-um}
			\\
			& \armenian{երգում}& \armenian{}
			& \armenian{երգում} & \armenian{ինք}
			& \armenian{չենք} & \armenian{երգում}
			& \armenian{չինք} & \armenian{երգում}
			\\
			\addlinespace 
			2PL
			&
			{jeɻkʰ-um} &{ekʰ}
			&
			{jeɻkʰ-um} &{ikʰ}
			&
			{t͡ʃʰ-ekʰ } &{jeɻkʰ-um}
			&
			{t͡ʃʰ-ikʰ } &{jeɻkʰ-um}
			\\
			& \armenian{երգում} & \armenian{էք}
			& \armenian{երգում} & \armenian{իք}
			& \armenian{չէք} & \armenian{երգում}
			& \armenian{չիք} & \armenian{երգում}
			\\
			\addlinespace 
			3PL
			&
			{jeɻkʰ-um} &{en}
			&
			{jeɻkʰ-um} &{in}
			&
			{t͡ʃʰ-en } &{jeɻkʰ-um}
			&
			{t͡ʃʰ-in} &{jeɻkʰ-um}
			\\ 
			& \armenian{երգում} & \armenian{են}
			& \armenian{երգում} & \armenian{ին}
			& \armenian{չեն} & \armenian{երգում}
			& \armenian{չին} & \armenian{երգում}
			\\
			\addlinespace 
			&\multicolumn{4}{l}{$\sqrt{~}$-{\impfcvb} {\auxgloss}}
			&\multicolumn{4}{l}{{\neggloss}-{\auxgloss}
				$\sqrt{~}$-{\impfcvb} } 
			\\ \lspbottomrule
		\end{tabular}
	}
\end{table} 



\subsection{Present perfect and pluperfect}\label{section:verb:periphrasis:perfect}
The next periphrastic construction that we discuss is the periphrastic perfective. Like the other periphrastic forms, this construction utilizes a special converb and the inflected auxiliary. The converb is called the perfective converb. Some grammars also use the term perfect participle \citep[213]{DumTragut-2009-ArmenianReferenceGrammar} or past participle \citep{Adjarian-1911-DialectologyBook,Dolatian-prep-Adjarian}. It is called [vɑʁɑkɑtɑɾ deɾbɑj] \armenian{վաղակատար դերբայ} in {\seaSEA}. 

The perfective converb has subtle differences across the two lects (Table \ref{tab:Verb:Perip:Perf:Liquid}). In {\seaSEA}, the perfective converb is formed by adding the suffix \textit{{-el}}. The theme vowel is deleted thanks to the vowel-hiatus rule in Rule \ref{rule delete theme in converb}. In {\iaIA}, this suffix is \textit{{-el}} or  \textit{-eɻ}. AS and NK report that \textit{-eɻ} form is more common among younger generations than older ones.\footnote{The variation has some connections with geography. Osik Movses, an {\iaIA} speaker from Tehran, informs us that some Tehran neighborhoods use the [-eɻ] form because these speakers' ancestors originate from villages that used such a form. } 


\begin{table}
\caption{Liquid quality of the perfective converb in {\seaSE} and {\iaIA} for E-Class `to sing'}\label{tab:Verb:Perip:Perf:Liquid}
\begin{tabular}{lll}
	\lsptoprule
	&Infinitive   & Perfective converb  \\\midrule
	{\seaSE} & {jeɾkʰ-e-l} & {jeɾkʰ-el} \\
	{\iaIA} & {jeɻkʰ-e-l} & {{jeɻkʰ-el}}\\
	& & {{jeɻkʰ-eɻ}} \\
	&$\sqrt{~}$-{\thgloss}-{\infgloss}&$\sqrt{~}$-{\perfcvb}
 \\
 & \armenian{երգել}& \armenian{երգել, երգեր}
 \\
	\lspbottomrule
\end{tabular}
\end{table}






For the same speaker, the choice of liquid can vary between [-el] or [-eɻ] without semantic motivation (\ref{sent:Verb:Perip:Perf:LiquidAlt}). It is possible that [-el] feels more formal for our speakers. 


\begin{exe}
	\ex \gll {es} {jeɻkʰ-ə} {voɻ} {mɒm-it͡sʰ} {sovoɻ-eɻ/el} {e-m}
	\\
	this song-{\defgloss} that mom-{\abl} learn-{\perfcvb} {\auxgloss}-1{\sg}
	\\
	\trans `This song that I learned from my mom.' \hfill (NK)\label{sent:Verb:Perip:Perf:LiquidAlt}
	\\
	\armenian{Էս երգը որ մամից սովորեր/սովորել եմ։}
	
\end{exe}


In some social phrases, AS reports that the liquid is conventionally a lateral (\ref{sent:Verb:Perip:Perf:LiquidAltAS}). 

\begin{exe}
	\ex \gll {kɒɻot-el} =e-m {kʰez} 
	\\ 
	miss-{\perfcvb} ={\auxgloss}-1{\sg} you.{\sg}.{\dat}
	\\
	\trans `I've missed you.' \hfill (AS)\label{sent:Verb:Perip:Perf:LiquidAltAS}
	\\
	\armenian{Կարօտել եմ քեզ։}
\end{exe}


Diachronically, the rhotic form [-eɻ] may have developed from the lateral form [-el]. This development has been attested in other Armenian lects \citep{grigoryan-2018-FallOfLiquidLPastParticipleInversionSpokenLanguage}. 

 AS reports that some archaic registers use the form [-i], such as (\ref{ex:sent:perf archaic}). We found this sentence in our transcribed sample text, uttered by an actor who was putting on an archaic accent.

\begin{exe}
    \ex \gll vɒksɒn e-n t͡ʃɒɻ-i \\  vaccine {\auxgloss}-3{\pl} find-{\perfcvb}\\
    \trans `They've found a vaccine.' \label{ex:sent:perf archaic}
    \\ <vaccine> \armenian{են ճարի։}
\end{exe}
 

{\iaIA} has grammaticalized a process of liquid deletion for the perfective converb suffix  [-eɻ] or [-el] (Table \ref{tab:Verb:Perip:Perf:Cons}). When this suffix is used in the positive before the inflected auxiliary, the liquid surfaces. But when the auxiliary has shifted leftward as in negation, the suffix's liquid is deleted, and sometimes pronounced as [{h}].


\begin{table}
	\caption{Perfective converb in {\seaSE} and {\iaIA} for the E-Class verb `to sing'} \label{tab:Verb:Perip:Perf:Cons}
\begin{tabular}{ll ll l}
	\lsptoprule 
	& \multicolumn{2}{l}{Positive present perfect 1SG} & \multicolumn{2}{l}{Negative present perfect 1SG} \\\midrule
	{\seaAbbre} & {jeɾkʰ-el} & {em}& {t͡ʃʰ-em} & {jeɾkʰ-el}\\
	&\multicolumn{2}{l}{\armenian{Երգել եմ։}} &\multicolumn{2}{l}{\armenian{Չեմ երգել։}}\\
	{\iaAbbre} & {jeɻkʰ-el}& {em}& {t͡ʃʰ-em} & {jeɻkʰ-e}\\
	&\multicolumn{2}{l}{\armenian{Երգել եմ։}} &\multicolumn{2}{l}{\armenian{Չեմ երգէ։}}\\
	& {jeɻkʰ-eɻ}& {em}& {t͡ʃʰ-em} & {jeɻkʰ-eh}\\
	&\multicolumn{2}{l}{\armenian{Երգեր եմ։}} &\multicolumn{2}{l}{\armenian{Չեմ երգէ։}}\\
	&$\sqrt{~}$-{\perfcvb}& {\auxgloss}& {\neggloss}-{\auxgloss} & $\sqrt{~}$-{\perfcvb}\\
	&\multicolumn{2}{l}{`I have sung.'} &\multicolumn{2}{l}{`I have not sung.'}\\
	\lspbottomrule
\end{tabular}
\end{table}
	
	
	The behavior of the perfective suffix in {\iaIA} suggests that the final liquid is a floating segment or latent segment: \textit{{-e(l)}} or \textit{{-e(ɻ)}} \citep[cf. ghost consonants:][]{tranel-1996-frenchLiaisonElisionRevisitedUnfiiedAccountOT,Cote-2011-FrenchLiaison,zimmermann-2019-gradientSymbolicRepresentationTypologyGhostSegment}. The above paradigm suggests that the liquid is licensed when it is followed by the auxiliary. The conditions for surfacing or deleting this liquid are discussed in \S\ref{section:morphophono:auxiliary}. For now, we just provide the relevant rules (Rule \ref{ruke:Verb:Perip:Perf:AltRule}). 
	
	\begin{newruleblock}[ruke:Verb:Perip:Perf:AltRule]
		{Rule for the perfective converb}%%\label{ruke:Verb:Perip:Perf:AltRule}
		
		\begin{center}
			\begin{tabular}{llll}
				{\perfcvb} & $\leftrightarrow$ & {{-e(l)}} / (older speakers)
				\\
				&& {{-e(ɻ)}} / (some younger speakers)
			\end{tabular}
		\end{center}
	\end{newruleblock} 
	
	
	The above data concerns constructing the perfective converb for the E-Class. In the A-Class, the same suffix is used. However, a meaningless affix \textit{{-t͡sʰ-}} is added between the theme vowel and the converb suffix (Table \ref{tab:Verb:Periph:Perf:class}). 
	
	
	
\begin{table}
	\caption{Perfective converb in {\seaSE} and {\iaIA} for the E-Class vs. A-Class verb}\label{tab:Verb:Periph:Perf:class}
	
	\begin{tabular}{lllll}
		\lsptoprule 
		&\multicolumn{2}{c}{E-Class}&\multicolumn{2}{c}{A-Class}\\\cmidrule(lr){2-3}\cmidrule(lr){4-5}
		&Infinitive & Pfv. converb &Infinitive & Pfv. converb\\
		& \armenian{երգել}& \armenian{երգել, երգեր}& \armenian{կարդալ}& \armenian{կարդացել, կարդացեր}\\\midrule
		{\seaAbbre} & {jeɾkʰ-e-l} & {jeɾkʰ-el} & {{kɑɾtʰ-ɑ-l}} & {{kɑɾtʰ-ɑ-t͡sʰ-el}}\\
		{\iaAbbre} & {jeɻkʰ-e-l} & {{jeɻkʰ-el}} & {{kɒɻtʰ-ɒ-l}}& {{kɒɻtʰ-ɒ-t͡sʰ-el}}\\
		& & {{jeɻkʰ-eɻ}}& & {{kɒɻtʰ-ɒ-t͡sʰ-eɻ}}\\
		&$\sqrt{~}$-{\thgloss}-{\infgloss}&$\sqrt{~}$-{\perfcvb}&$\sqrt{~}$-{\thgloss}-{\infgloss}& $\sqrt{~}$-{\thgloss}-{\aorother}-{\perfcvb}\\
		\lspbottomrule
		\end{tabular}
\end{table}
	
	
	In the traditional literature on Armenian, the meaningless \textit{{-t͡sʰ-}} is called the aorist suffix. We gloss the additional meaningless suffix \textit{{t͡sʰ}} as {\aorother}. The suffix is used to mark synthetic past perfective verbs for the A-Class, but it is also used meaninglessly in other constructions. In the case of the perfective converb, this \textit{{-t͡sʰ}-} is being used morphomically. The use of this suffix in the A-Class perfective converb is treated as using an aorist stem. Such a stem is morphomic \citep{Aronoff-1994-MorphologyItselfStemsInflectionClass}. For a discussion and analysis of aorist stems in Armenian, see \citet{DolatianGuekguezian-prep-Morphome}. In this grammar, we do not provide rules for generating this meaningless aorist suffix. 
	 For descriptive purposes, the full paradigm is given in Table \ref{tab:perf conv paradigm} for the E-Class. 
	
	\begin{table}
		\caption{Paradigm for the present perfect and the pluperfect for E-Class [{jeɻkʰ-e-l}] `to sing'}
		\label{tab:perf conv paradigm}
		\resizebox{\textwidth}{!}{%
			\begin{tabular}{l llll llll}
				\lsptoprule
				& \multicolumn{4}{l}{Positive} & \multicolumn{4}{l}{Negaive} \\\cmidrule(lr){2-5}\cmidrule(lr){6-9}
				& \multicolumn{2}{l}{Present perfect}& \multicolumn{2}{l}{Pluperfect} & \multicolumn{2}{l}{Present perfect}& \multicolumn{2}{l}{Pluperfect}\\\midrule
				1SG
				&
				{jeɻkʰ-el} &{em}
				&
				{jeɻkʰ-el } &{im}
				&
				{t͡ʃʰ-em } &{jeɻkʰ-e}
				&
				{t͡ʃʰ-im} & {jeɻkʰ-e}
				\\
				&
				{jeɻkʰ-eɻ} &{em}
				&
				{jeɻkʰ-eɻ } &{im}
				&
				&	&	&
				\\
				&
				\multicolumn{2}{l}{`I have sung'}&
				\multicolumn{2}{l}{`I had sung'}&
				\multicolumn{2}{l}{`I haven't sung'}&
				\multicolumn{2}{l}{`I hadn't sung'}
				\\
				& \armenian{երգել} & \armenian{եմ}
				& \armenian{երգել} & \armenian{իմ}
				& \armenian{չեմ}& \armenian{երգէ}
				& \armenian{չիմ} & \armenian{երգէ}
				\\
				& \armenian{երգեր} & \armenian{եմ}
				& \armenian{երգեր} & \armenian{իմ}
				& & & &
				\\\addlinespace
				2SG
				&
				{jeɻkʰ-el } &{es}
				&
				{jeɻkʰ-el } &{iɻ}
				&
				{t͡ʃʰ-es} & {jeɻkʰ-e }
				&
				{t͡ʃʰ-iɻ} & {jeɻkʰ-e }
				\\
				&
				{jeɻkʰ-eɻ } &{es}
				&
				{jeɻkʰ-eɻ } & {iɻ}
				&
				& & 
				&
				\\
				& \armenian{երգել} & \armenian{ես}
				& \armenian{երգել} & \armenian{իր}
				& \armenian{չես} & \armenian{երգէ}
				& \armenian{չիր} & \armenian{երգէ}
				\\
				& \armenian{երգեր} & \armenian{ես}
				& \armenian{երգեր} & \armenian{իր}
				& & & &
				\\\addlinespace
				3SG
				&
				{jeɻkʰ-el } & {ɒ}
				&
				{jeɻkʰ-el } & {eɻ}
				&
				{t͡ʃʰ-i} & {jeɻkʰ-e }
				&
				{t͡ʃʰ-eɻ} & {jeɻkʰ-e }
				\\
				&
				{jeɻkʰ-eɻ } & {ɒ}
				&
				{jeɻkʰ-eɻ } & {eɻ}
				&
				& 	& &
				\\
				& \armenian{երգել} &\armenian{ա}
				& \armenian{երգել} &\armenian{էր}
				& \armenian{չի} & \armenian{երգէ}
				& \armenian{չէր} & \armenian{երգէ}
				\\
				& \armenian{երգեր} & \armenian{ա}
				& \armenian{երգեր} & \armenian{էր}
				& & & &
				\\
				\addlinespace
				1PL
				&
				{jeɻkʰ-el } & {eŋkʰ}
				&
				{jeɻkʰ-el} & {iŋkʰ}
				&
				{t͡ʃʰ-eŋkʰ} & {jeɻkʰ-e}
				&
				{t͡ʃʰ-iŋkʰ} & {jeɻkʰ-e}
				\\ 
				&
				{jeɻkʰ-eɻ } & {eŋkʰ}
				&
				{jeɻkʰ-eɻ} & {iŋkʰ}
				& & & &
				\\ 
				& \armenian{երգել} &\armenian{ենք}
				& \armenian{երգել} &\armenian{ինք}
				& \armenian{չենք} & \armenian{երգէ}
				& \armenian{չինք} & \armenian{երգէ}
				\\
				& \armenian{երգեր} & \armenian{ենք}
				& \armenian{երգեր} & \armenian{ինք}
				& & & &
				\\
				\addlinespace
				2PL
				&
				{jeɻkʰ-el} & {ekʰ}
				&
				{jeɻkʰ-el} & {ikʰ}
				&
				{t͡ʃʰ-ekʰ} &{jeɻkʰ-e}
				&
				{t͡ʃʰ-ikʰ} & {jeɻkʰ-e}
				\\
				&
				{jeɻkʰ-eɻ} & {ekʰ}
				&
				{jeɻkʰ-eɻ} & {ikʰ}
				& & & &
				\\
				& \armenian{երգել} &\armenian{էք}
				& \armenian{երգել} &\armenian{իք}
				& \armenian{չէք} & \armenian{երգէ}
				& \armenian{չիք} & \armenian{երգէ}
				\\
				& \armenian{երգեր} & \armenian{էք}
				& \armenian{երգեր} & \armenian{իք}
				& & & &
				\\
				\addlinespace
				3PL
				&
				{jeɻkʰ-el} & {en}
				&
				{jeɻkʰ-el} & {in}
				&
				{t͡ʃʰ-en} & {jeɻkʰ-e}
				&
				{t͡ʃʰ-in} & {jeɻkʰ-e}
				\\
				&
				{jeɻkʰ-eɻ} & {en}
				&
				{jeɻkʰ-eɻ} & {in}
				&
				& & &
				
				
				\\
				& \armenian{երգել} &\armenian{են}
				& \armenian{երգել} &\armenian{ին}
				& \armenian{չեն}& \armenian{երգէ}
				& \armenian{չին} & \armenian{երգէ}
				\\
				& \armenian{երգեր} & \armenian{են}
				& \armenian{երգեր} & \armenian{ին}
				& & & &
				
				\\
				\addlinespace
				&\multicolumn{4}{l}{$\sqrt{~}$-{\perfcvb} {\auxgloss}}
				&\multicolumn{4}{l}{{\neggloss}-{\auxgloss}
					$\sqrt{~}$-{\perfcvb} } 
				\\ \lspbottomrule
			\end{tabular}}
	\end{table}
	
	
	When the perfective converb is used with the present auxiliary, the construction denotes the present perfect. If we use the past auxiliary, then the construction denotes the pluperfect. The paradigm for the A-Class `to read' is analogously constructed with the converb [{{kɒɻtʰ-ɒ-t͡sʰ-el}}]. We do not segment the auxiliary. As before, the auxiliary shifts its position in the negated form.
	
	
	
	
	
	
	\subsection{Simultaneous converb}
	
	
	{\seaSEA} has an additional periphrastic construction that uses the simultaneous converb (Table \ref{tab:Verb:simultaneos}), also called the processual participle \citep[205]{DumTragut-2009-ArmenianReferenceGrammar}. The converb is called [hɑmɑkɑtɑɾ deɾbɑj] \armenian{համակատար դերբայ} in {\seaSEA}. This converb is built by adding the suffix \textit{{-is}} to infinitives. This construction is quite infrequent in {\seaSEA}. For {\iaIA}, NK reports that she never uses this participle, while KM reports that she does use it. AS reports that his consultants never use it. We do not report further on this converb because of the the limited data available to us. 
	
\begin{table}
\caption{Forming the simultaneous converb\label{tab:Verb:simultaneos}}
\begin{tabular}{llll}
	\lsptoprule
	\multicolumn{2}{c}{E-Class `to sing'}& \multicolumn{2}{c}{A-Class `to read'}\\\cmidrule(lr){1-2}\cmidrule(lr){3-4}
	Infinitive & Simultaneous converb & Infinitive & Simultaneous converb\\\midrule
	{jeɻkʰ-e-l}&{jeɻkʰ-e-l-is} & {{kɒɻtʰ-ɒ-l}}&{{kɒɻtʰ-ɒ-l-is}} \\
	$\sqrt{~}$-{\thgloss}-{\infgloss}&$\sqrt{~}$-{\thgloss}-{\infgloss}-{\simcvb}&$\sqrt{~}$-{\thgloss}-{\infgloss}&$\sqrt{~}$-{\thgloss}-{\infgloss}-{\simcvb}\\
	\armenian{երգել} & \armenian{երգելիս} &\armenian{կարդալ} &\armenian{կարդալիս}\\ 
	\lspbottomrule
\end{tabular}
\end{table}
	
	
	
	
	
\section{Synthetic forms}\label{section:verb:synthesis}
A large chunk of {\iaIA} verbal inflection is handled via periphrasis. There are however some pockets of synthetic constructions. These include the aorist (past perfective), subjunctives, and imperatives. Prohibitives are derived from imperatives via the addition of a particle. Note that the future is marked with both synthetic and periphrastic constructions, discussed in \S\ref{section:verb:fut}. 

\subsection{Past perfective or aorist form}\label{section:verb:synthesis:perf}
Impressionistically, the past perfective or aorist is the most common synthetic construction. It is used to denote the simple past. But as the examples in Table \ref{example past perf} illustrate, the two classes use markedly different affixes to generate the past perfective. The past perfective of the A-Class is formed in essentially the same way for the two lects, while the E-Class uses a markedly different construction.


\begin{table}
	\caption{Past perfective 1PL for E-Class and A-Class}\label{example past perf}
	\resizebox{\textwidth}{!}{%
		\begin{tabular}{llll l}
			\lsptoprule
			&\multicolumn{2}{c}{E-Class} & \multicolumn{2}{c}{A-Class} \\\cmidrule(lr){2-3}\cmidrule(lr){4-5}
			&{\iaAbbre}&{\seaAbbre} &{\iaAbbre}&{\seaAbbre}\\\midrule
			Infinitive& {jeɻkʰ-e-l} & {jeɾkʰ-e-l}& {{kɒɻtʰ-ɒ-l}}& {kɑɾtʰ-ɑ-l}\\
			& $\sqrt{~}$-{\thgloss}-{\infgloss}& $\sqrt{~}$-{\thgloss}-{\infgloss}& $\sqrt{~}$-{\thgloss}-{\infgloss}& $\sqrt{~}$-{\thgloss}-{\infgloss}\\
			&`to sing'&`to sing' & `to read' & `to read' \\
			& \armenian{երգել}& \armenian{երգել}& \armenian{կարդալ}& \armenian{կարդալ}\\
			\addlinespace
			Past Pfv. & {{jeɻkʰ-ɒ-ŋkʰ}}
			& {{jeɾkʰ-e-t͡sʰ-i-ŋkʰ}}& {{kɒɻtʰ-ɒ-t͡sʰ-i-ŋkʰ}}& {{kɑɾtʰ-ɑ-t͡sʰ-i-ŋkʰ}}\\
			&$\sqrt{~}$-{\pst}-1{\pl}&$\sqrt{~}$-{\thgloss}-{\aorperf}-{\pst}-1{\pl}& $\sqrt{~}$-{\thgloss}-{\aorperf}-{\pst}-1{\pl} & $\sqrt{~}$-{\thgloss}-{\aorperf}-{\pst}-1{\pl}\\
			&`we sang'&`we sang' & `we read (past)'& `we read (past)'\\ 
			& \armenian{երգանք}& \armenian{երգեցինք}& \armenian{կարդացինք}& \armenian{կարդացինք}\\
			\lspbottomrule
		\end{tabular}}
\end{table}

The name of the past perfective is [ɑnt͡sʰjɑl kɑtɑɾjɑl] \armenian{անցյալ կատարյալ} in {\seaSEA}. 

We first describe the A-Class in {\iaIA}, whose past perfective is formed essentially the same in {\seaSE}. The past perfective is formed by taking the stem of the A-Class (root and theme vowel), and adding the aorist suffix \textit{{-t͡sʰ-}}. The \textit{{-t͡sʰ-}} is a marker of perfectivity \citep{Donabedian-2016-aoristModernArmenian}. We then add the past marker /{i}/ and agreement markers. For brevity, we say that A-Class verbs use the /-t͡sʰ-i/ template for marking the past perfective. We gloss \textit{-t͡sʰ-} as -{\aorperf}- both in  the past perfective (where it is meaningful) and in non-past paradigms, as in the perfective converb of the A-Class (\S\ref{section:verb:periphrasis:perfect}). 


The complete paradigm is shown in Table \ref{tab:aorist a class} for the A-Class in {\seaSE} and {\iaIA}. Negation is formed by adding the prefix \textit{{t͡ʃʰ-}}, which surfaces with a schwa before consonant-initial verbs. The only morphological difference between the two lects is that the 1SG marker /{-m}/ is used in {\iaIA} (\S\ref{section:verb:aux:past}), while {\seaSE} uses a zero suffix.\largerpage[2]

\vfill
\begin{table}[H]
	\caption{Paradigm of past perfective of A-Class [{kɒɻtʰ-ɒ-l}] `to read' in {\seaSE} and {\iaIA}\label{tab:aorist a class}}
	\resizebox{\textwidth}{!}{%
		\begin{tabular}{l llll}
			\lsptoprule
			&\multicolumn{2}{c}{Positive} &\multicolumn{2}{c}{Negative}\\ \cmidrule(lr){2-3}\cmidrule(lr){4-5}
			& {\seaAbbre} & {\iaAbbre}& {\seaAbbre} & {\iaAbbre}\\\midrule
			1SG & {kɑɾtʰ-ɑ-t͡sʰ-i} & {{kɒɻtʰ-ɒ-t͡sʰ-i-m}} & {{t͡ʃʰə-kɑɾtʰ-ɑ-t͡sʰ-i}} & {{t͡ʃʰə-kɒɻtʰ-ɒ-t͡sʰ-i-m}} 
			\\
			& 	`I read (past)'	& 	`I read (past)'& `I did not read'& `I did not read'\\
			& \armenian{կարդացի}& \armenian{կարդացիմ}& \armenian{չկարդացի}& \armenian{չկարդացիմ}
			\\
			\addlinespace
			2SG & {{kɑɾtʰ-ɑ-t͡sʰ-i-ɾ}} & {{kɒɻtʰ-ɒ-t͡sʰ-i-ɻ}} & {{t͡ʃʰə-kɑɾtʰ-ɑ-t͡sʰ-i-ɾ}} & {{t͡ʃʰə-kɒɻtʰ-ɒ-t͡sʰ-i-ɻ}} \\
			& \armenian{կարդացիր}		& \armenian{կարդացիր}& \armenian{չկարդացիր}
			&\armenian{չկարդացիր}\\\addlinespace
			3SG & {{kɑɾtʰ-ɑ-t͡sʰ}} & {{kɒɻtʰ-ɒ-t͡sʰ}} & {{t͡ʃʰə-kɑɾtʰ-ɑ-t͡sʰ}}& {{t͡ʃʰə-kɒɻtʰ-ɒ-t͡sʰ}} \\
			& \armenian{կարդաց}	& \armenian{կարդաց}& \armenian{չկարդաց}
			& \armenian{չկարդաց}\\\addlinespace		
			1PL& {{kɑɾtʰ-ɑ-t͡sʰ-i-ŋkʰ}} & {{kɒɻtʰ-ɒ-t͡sʰ-i-ŋkʰ}} & {{t͡ʃʰə-kɑɾtʰ-ɑ-t͡sʰ-i-ŋkʰ}} & {{t͡ʃʰə-kɒɻtʰ-ɒ-t͡sʰ-i-ŋkʰ}} \\
			& \armenian{կարդացինք}& \armenian{կարդացինք}& \armenian{չկարդացինք}& \armenian{չկարդացինք}\\
			
			\addlinespace		
			2PL& {{kɑɾtʰ-ɑ-t͡sʰ-i-kʰ}} & {{kɒɻtʰ-ɒ-t͡sʰ-i-kʰ}} & {{t͡ʃʰə-kɑɾtʰ-ɑ-t͡sʰ-i-kʰ}}& {{t͡ʃʰə-kɒɻtʰ-ɒ-t͡sʰ-i-kʰ}}\\
			& \armenian{կարդացիք}& \armenian{կարդացիք}& \armenian{չկարդացիք}& \armenian{չկարդացիք}\\
			\addlinespace		
			3PL& {{kɑɾtʰ-ɑ-t͡sʰ-i-n}} & {{kɒɻtʰ-ɒ-t͡sʰ-i-n}} & {{t͡ʃʰə-kɑɾtʰ-ɑ-t͡sʰ-i-n}} & {{t͡ʃʰə-kɒɻtʰ-ɒ-t͡sʰ-i-n}} 
			\\
			& \armenian{կարդացին}& \armenian{կարդացին}& \armenian{չկարդացին}& \armenian{չկարդացին}
			\\
			\addlinespace 		&	\multicolumn{2}{l}{$\sqrt{~}$-{\thgloss}-{\aorperf}-{\pst}-{\agr}}& \multicolumn{2}{l}{{\neggloss}-$\sqrt{~}$-{\thgloss}-{\aorperf}-{\pst}-{\agr} }
			\\
			\lspbottomrule
		\end{tabular}}
\end{table}\vfill\pagebreak


For illustration, \tabref{tab:past perf a class zero aux} provides a fuller segmentation that shows zero markers for the positive. For contrast, we also repeat the paradigm of the past auxiliary.


\begin{table}
	\caption{Full segmentation of past perfective for A-Class [{kɒɻtʰ-ɒ-l}] `to read' and past auxiliary}
	\label{tab:past perf a class zero aux}
	\begin{tabular}{l llll}
		\lsptoprule 
		&\multicolumn{2}{l}{Past Pfv. with zero markers} &\multicolumn{2}{l}{Past auxiliary}\\\midrule
		1SG & {{kɒɻtʰ-ɒ-t͡sʰ-i-m}} & \armenian{կարդացիմ} &{{$\emptyset$-i-m}} & \armenian{իմ}\\
		2SG & {{kɒɻtʰ-ɒ-t͡sʰ-i-ɻ}} & \armenian{կարդացիր}&{{$\emptyset$-i-ɻ}} & \armenian{իր}\\
		3SG & {{kɒɻtʰ-ɒ-t͡sʰ-$\emptyset$-$\emptyset$}} & \armenian{կարդաց}&{{e-$\emptyset$-ɻ}} & \armenian{էր}\\
		1PL & {{kɒɻtʰ-ɒ-t͡sʰ-i-ŋkʰ}} &\armenian{կարդացինք}&{{$\emptyset$-i-ŋkʰ}} & \armenian{ինք}\\
		2PL& {{kɒɻtʰ-ɒ-t͡sʰ-i-kʰ}} & \armenian{կարդացիք} &{{$\emptyset$-i-kʰ}} & \armenian{իք}\\
		3PL& {{kɒɻtʰ-ɒ-t͡sʰ-i-n}} & \armenian{կարդացին} &{{$\emptyset$-i-n}} & \armenian{ին}\\
		& \multicolumn{2}{l}{$\sqrt{~}$-{\thgloss}-{\aorperf}-{\pst}-{\agr}} & \multicolumn{2}{l}{{\auxgloss}-{\pst}-{\agr}}\\
		\lspbottomrule
	\end{tabular}
\end{table}


For the past perfective in the 3SG, both the past suffix and the agreement suffix are covert. Elsewhere for the A-Class, the past suffix is /{i}/ in the past perfective, just as in past auxiliaries. Outside of the 3SG, the agreement morphs likewise match the morphs used in the past auxiliary: \textit{{i-ŋkʰ}} `we were'. We list below some other example A-Class words in the past perfective that we have collected (\tabref{tab:Verb:Synthn:Aor:Example}).

\begin{table}
\caption{Past perfective  form of some A-Class verbs}\label{tab:Verb:Synthn:Aor:Example}
\resizebox{\textwidth}{!}{%	
\begin{tabular}{lllll}
	\lsptoprule 
	\multicolumn{2}{l}{Infinitive} & \multicolumn{3}{l}{Past perfective}\\\cmidrule(lr){1-2}\cmidrule(lr){3-5}
	{\iaAbbre}& & {\iaAbbre} & {\seaAbbre} &\\\midrule
	% {ɡən-ɒ-l} & {ɡən-ɒ-t͡sʰ-i-ɻ} & {ɡən-ɑ-t͡sʰ-i-ɾ} & `to go' % (KM) \\
	{ʒəpt-ɒ-l}&`to smile' & {ʒəpt-ɒ-t͡sʰ-i-ɻ}&{ʒəpt-ɑ-t͡sʰ-i-ɾ} & `You.{\sg} smiled' %& (NK)
	\\
	\armenian{ժպտալ} & & \armenian{ժպտացիր}& \armenian{ժպտացիր} & \\
	{hɒvɒt-ɒ-l} & `to believe' & {hɒvɒt-ɒ-t͡sʰ-i-m}& {hɑvɑt-ɑ-t͡sʰ-i-$\emptyset$} & `I believed' %& (NK)
	\\
	\armenian{հաւատալ}& & \armenian{հաւատացիմ}& \armenian{հավատացի} & \\
	$\sqrt{~}$-{\thgloss}-{\infgloss} & &\multicolumn{2}{l}{$\sqrt{~}$-{\thgloss}-{\aorperf}-{\pst}-{\agr}}& \\
	\lspbottomrule
\end{tabular}}
\end{table}


For the E-Class, the past perfective has a more complicated construction. In {\seaSE}, the past perfective is formed in the same way as for the A-Class, except for a difference in   theme vowel: [{{jeɾkʰ-e-t͡sʰ-i-ŋkʰ}}] `we sang'. Thus the {\seaSE} E-Class uses the template /-t͡sʰ-i/. In contrast, the {\iaIA} form drops the theme vowel and the aorist, and uses a different past allomorph /ɒ/: [{{jeɻkʰ-ɒ-ŋkʰ}}] `we sang'. For brevity, we say that the {\iaIA} E-Class uses the template /-$\emptyset$-ɒ/ where -$\emptyset$ is a covert perfective or aorist marker. 

The paradigm is given below for both lects (Table \ref{tab:aorist e class}). The negative is formed by just adding the negation prefix \textit{{t͡ʃʰə-}}. In order to save space we do not show zero morphs. 
 In the 3SG of the E-Class, {\iaIA} uses an overt /{ɒ}/ morph for past, and /{v}/ for agreement. {\seaSE} uses covert nodes for both. The 1SG uses an overt agreement morph /{m}/ in {\iaIA}, but covert in {\seaSE}.

\begin{table}
	\caption{Paradigm of past perfective of E-Class `to sing' in both lects} 
	\label{tab:aorist e class}
	\begin{tabular}{lllll}
		\lsptoprule
		& \multicolumn{2}{l}{{\seaSE}} & \multicolumn{2}{l}{{\iaIA}} \\ \cmidrule(lr){2-3}\cmidrule(lr){4-5}
		&Positive&Negative&Positive&Negative\\\midrule 
		1SG & {{jeɾkʰ-e-t͡sʰ-i}}& {{t͡ʃʰə-jeɾkʰ-e-t͡sʰ-i}}& {{jeɻkʰ-ɒ-m}}& {{t͡ʃʰə-jeɻkʰ-ɒ-m}}\\
		&`I sang'&`I did not sing'&`I sang'&`I did not sing'\\
		& \armenian{երգեցի} & \armenian{չերգեցի} & \armenian{երգամ} & \armenian{չերգամ}\\
		\addlinespace	2SG
		& {{jeɾkʰ-e-t͡sʰ-i-ɾ}} & {{t͡ʃʰə-jeɾkʰ-e-t͡sʰ-i-ɾ}}& {{jeɻkʰ-ɒ-ɻ}}& {{t͡ʃʰə-jeɻkʰ-ɒ-ɻ}}
		\\
		& \armenian{երգեցիր}
		& \armenian{չերգեցիր}
		& \armenian{երգար}
		& \armenian{չերգար}
		\\
		\addlinespace		3SG
		& {{jeɾkʰ-e-t͡sʰ}} & {{t͡ʃʰə-jeɾkʰ-e-t͡sʰ}}& {{jeɻkʰ-ɒ-v}}& {{t͡ʃʰə-jeɻkʰ-ɒ-v}}
		\\
		& \armenian{երգեց}
		& \armenian{չերգեց}
		& \armenian{երգաւ}
		& \armenian{չերգաւ}
		\\
		
		\addlinespace		1PL
		& {{jeɾkʰ-e-t͡sʰ-i-ŋkʰ}} & {{t͡ʃʰə-jeɾkʰ-e-t͡sʰ-i-ŋkʰ}}& {{jeɻkʰ-ɒ-ŋkʰ}}& {{t͡ʃʰə-jeɻkʰ-ɒ-ŋkʰ}}
		\\
		& \armenian{երգեցիինք}
		& \armenian{չերգեցինք}
		& \armenian{երգանք}
		&\armenian{չերգանք}
		\\
		
		\addlinespace		2PL
		& {{jeɾkʰ-e-t͡sʰ-i-kʰ}}&{{t͡ʃʰə-jeɾkʰ-e-t͡sʰ-i-kʰ}}& {{jeɻkʰ-ɒ-kʰ}}& {{t͡ʃʰə-jeɻkʰ-ɒ-kʰ}}
		\\
		& \armenian{երգեցիք}
		& \armenian{չերգեցիք}
		& \armenian{երգաք}
		& \armenian{չերգաք}
		\\
		\addlinespace		3PL
		& {{jeɾkʰ-e-t͡sʰ-i-n}} & {{t͡ʃʰə-jeɾkʰ-e-t͡sʰ-i-n}}& {{jeɻkʰ-ɒ-n}}& {{t͡ʃʰə-jeɻkʰ-ɒ-n}}
		\\
		& \armenian{երգեցին}
		& \armenian{չերգեցին}
		& \armenian{երգան}
		& \armenian{չերգան}
		\\
		\addlinespace
		&\multicolumn{2}{l}{( {\neggloss})-$\sqrt{~}$-{\thgloss}-{\aorperf}-{\pst}-{\agr}}&\multicolumn{2}{l}{( {\neggloss})-$\sqrt{~}$-{\pst}-{\agr}}\\ 
		\lspbottomrule
	\end{tabular}
\end{table}


To showcase the widespread difference between {\seaSE} and {\iaIA} for the E-Class perfective, Table \ref{tab:aorist e class:example} lists some frequent E-Class verbs, and an example past perfective form.


\begin{table}
	\caption{Past perfective form of some E-Class verbs}\label{tab:aorist e class:example}
	\resizebox{\textwidth}{!}{%
		\begin{tabular}{l lll ll l}
			\lsptoprule
			\multicolumn{2}{c}{Infinitive} & \multicolumn{5}{c}{Past perfective form}\\\cmidrule(lr){1-2}\cmidrule(lr){3-7}
			\multicolumn{2}{l}{{\iaIA}} &   \multicolumn{2}{l}{{\iaIA}}& \multicolumn{2}{l}{{\seaSEA}} &  \\\midrule
			{χəm-e-l} & \armenian{խմել} & {χəm-ɒ-m} &\armenian{խմամ} & {χəm-e-t͡sʰ-i} &\armenian{խմեցի} & `I drank'%& (KM)
			\\
			{t͡sɒk-e-l} & \armenian{ծակել}& {t͡sɒk-ɒ-ɻ} &\armenian{ծակար} & {t͡sɑk-e-t͡sʰ-i-ɾ} & \armenian{ծակեցիր}& `you.{\sg} made a hole' %& (KM)
			\\
			{t͡sɒχ-e-l} &\armenian{ծախել} & {t͡sɒχ-ɒ-v} & \armenian{ծախաւ}& {t͡sɒχ-e-t͡sʰ} & \armenian{ծախեց}& `he sold' %& (KM)
			\\
			{voɻoʃ-e-l} & \armenian{որոշել}& {voɻoʃ-ɒ-v} & \armenian{որոշաւ} & {voɾoʃ-e-t͡sʰ} & \armenian{որոշեց} & `he decided' %& (AS)
			\\
			{kɒnt͡ʃʰ-e-l} & \armenian{կանչել}& {kɒnt͡ʃʰ-ɒ-v} &\armenian{կանչաւ} & {kɑnt͡ʃʰ-e-t͡sʰ} & \armenian{կանչեց} & `he called' %& (AS)
			\\
			{mekn-e-l} & \armenian{մեկնել}& {mekn-ɒ-v} & \armenian{մեկնաւ} & {mekn-e-t͡sʰ} &\armenian{մեկնեց} & `he went away'% & (AS)
			\\
			{bərn-e-l}& \armenian{բռնել} & {bərn-ɒ-v} & \armenian{բռնաւ}& {bərn-e-t͡sʰ} &\armenian{բռնեց} & `he caught' %& (NK)
			\\
			{kɒŋɡ(ə)n-e-l}& \armenian{կանգնել} & {kɒŋɡ(ə)n-ɒ-v} & \armenian{կանգնաւ}& {kɑŋɡn-e-t͡sʰ} & \armenian{կանգնեց} & `he stood' %& (NK)
			\\
			{kʰɒjl-e-l}& \armenian{քայլել}& {kʰɒjl-ɒ-v} & \armenian{քայլաւ}& {kʰɑjl-e-t͡sʰ} &\armenian{քայլեց}& `he walked' %& (NK)
			\\
			{uʁɒɻk-e-l}& \armenian{ուղարկել}& {uʁɒɻk-ɒ-ŋkʰ} & \armenian{ուղարկանք}& {uʁɑɾk-e-t͡sʰ-i-ŋkʰ} & \armenian{ուղարկեցինք}& `we sent' %& (NK)
			\\
			{ɒpɻ-e-l} & \armenian{ապրել} & {ɒpɻ-ɒ-kʰ} & \armenian{ապրաք} & {ɑpɾ-e-t͡sʰ-i-kʰ} & \armenian{ապրեցիք} & `you.{\pl} lived' %& (KM)
			\\
			{ɡəɻ-e-l} & \armenian{գրել} & {ɡəɻ-ɒ-n} & \armenian{գրան} & {ɡəɾ-e-t͡sʰ-i-n} & \armenian{գրեցին}& `they wrote' %& (KM)
			\\
			\multicolumn{2}{l}{$\sqrt{~}$-{\thgloss}-{\infgloss}} & \multicolumn{2}{l}{$\sqrt{~}$-{\pst}-2{\sg}}& \multicolumn{2}{l}{$\sqrt{~}$-{\thgloss}-{\aorperf}-{\pst}-{\agr} } & \\
			\lspbottomrule 
		\end{tabular}}
\end{table}


In terms of morphological structure, we assume that the {\iaIA} past perfective of the E-Class contains a covert aorist perfective suffix to license perfective meaning. The theme vowel is then deleted before the /{ɒ}/ vowel as a morpheme-specific rule of vowel-hiatus repair (Rule \ref{rule: delete theme before pst a}). 

We show below the underlying and surface structure of the past perfective 1PL for both the A-Class and E-Class in {\iaIA} (\tabref{reprsentaiton pst perf}). The aorist suffix marks perfective aspect {\asp}. 

\begin{newruleblock}[rule: delete theme before pst a]
	{Delete theme vowels before the past suffix /{ɒ}/}%%\label{rule: delete theme before pst a}
	
	\begin{center}
		\begin{tabular}{llll}
			/{e}/ &$\rightarrow$&$\emptyset$ & / \_ {ɒ} 
			\\
			\multicolumn{4}{l}{(where /e/ is a theme vowel, and /ɒ/ is a past marker)}
		\end{tabular}
	\end{center}
\end{newruleblock}


\begin{table}
\caption{Underlying and surface structure of past perfective 1PL in {\iaIA}\label{reprsentaiton pst perf}}
\resizebox{\textwidth}{!}{%
	\begin{tabular}{lll}
		\lsptoprule
		A-Class& E-Class & \\
		Underlying and surface&Underlying&Surface\\\midrule
		`we read' &`we sang'&\\
		/{kɒɻtʰ-ɒ-t͡sʰ-i-ŋkʰ}/ & /{jeɻkʰ-e-$\emptyset$-ɒ-ŋkʰ}/& $\rightarrow$ [{jeɻkʰ-$\emptyset$-$\emptyset$-ɒ-ŋkʰ}]
		\\
		\begin{tikzpicture} 
			\Tree 
			[.{\agr} [.{\tense} [.{\asp} [.{\thgloss} [.$\sqrt{~}$ {kɒɻtʰ} ] [.{\thgloss} {{-ɒ}} ] ] [ [.{\aorperf} {-t͡sʰ} ] ] ] [ [ [.{\tense} {-i} ] ] ] ] [ [ [ [.{\agr} {-ŋkʰ} ] ] ] ] ] 
		\end{tikzpicture}
		&
		\begin{tikzpicture} 
			\Tree 
			[.{\agr} [.{\tense} [.{\asp} [.{\thgloss} [.$\sqrt{~}$ {jeɻkʰ} ] [.{\thgloss} {{-e}} ] ] [ [.{\aorperf} {-$\emptyset$} ] ] ] [ [ [.{\tense} {-ɒ} ] ] ] ] [ [ [ [.{\agr} {-ŋkʰ} ] ] ] ] ] 
		\end{tikzpicture}
		&
		\begin{tikzpicture} 
			\Tree 
			[.{\agr} [.{\tense} [.{\asp} [.{\thgloss} [.$\sqrt{~}$ {jeɻkʰ} ] [.{\thgloss} {{-$\emptyset$}} ] ] [ [.{\aorperf} {-$\emptyset$} ] ] ] [ [ [.{\tense} {-ɒ} ] ] ] ] [ [ [ [.{\agr} {-ŋkʰ} ] ] ] ] ] 
		\end{tikzpicture}\\ 
		\lspbottomrule
		\end{tabular}}
\end{table}


Before we provide complete rules for these morphemes in {\iaIA}, readers might wonder about the origin of this /{ɒ}/ morph. In {\seaSE}, the cognate of this morph is the past morph /{ɑ}/. This /{ɑ}/ is restricted to certain irregular classes and in some regular complex verbs such as inchoatives. In fact, the /{ɑ}/ morph is treated as the restricted or marked past allomorph in {\seaSE} and in {\swaWA} \citep{DolatianGuekguezian-prep-TierBasedLocalityArmenianConjugationClass,KarakasDolatainGuekguezian-prep-DisentanglingTesnseAgreementWesternArmenian}, while /{i}/ is the elsewhere morph. In contrast, in {\iaIA}, the /{ɒ}/ morph has developed a larger distribution, while /{i}/ shrank in its distribution. Similarly for the aorist/perfective suffix, the morph /{t͡sʰ}/ is the elsewhere morph in {\seaSE}, while a covert -$\emptyset$ is restricted to some irregular verbs. 

Table \ref{tab:Verb:Synthn:Aor:Suppletive} illustrates the distribution of these four morphs. For {\seaSE}, the perfective-past sequence of morphs is /{-t͡sʰ-i}/ for E-Class and A-Class verbs, while this sequence is /{-$\emptyset$-ɑ}/ for suppletive verbs like \textit{{ut-e-l}} `to eat'. In contrast, for {\iaIA}, the /{$\emptyset$-ɒ}/ sequence is now generalized to the perfective of E-Class, while /{-t͡sʰ-i}/ shrank in its distribution. We show the deleted theme vowels and covert aspect.

\begin{table}
	\caption{Past perfective 1PL for E-Class, A-Class, and suppletive verbs}\label{tab:Verb:Synthn:Aor:Suppletive}
	
%	\begin{subtable}[h]{\textwidth}
%		\caption{Infinitive}% and 1PL subjunctive past form}
%		\centering
%		\resizebox{\textwidth}{!}{%
%			\begin{tabular}{|l|l|lll l|}
%				\hline 
%				& & A-Class & E-Class & Suppletive& \\
%				& 			
%				&`to read' & `to sing' & `to eat' 
%				& \\
%				\hline
%				{\seaAbbre} & Inf. &{kɑɾtʰ-ɑ-l} &{jeɾkʰ-e-l} &{{ut-e-l}}&$\sqrt{~}$-{\thgloss}-{\infgloss}
%				\\
%%				& Sbjv. Past   3PL &{kɑɾtʰ-ɑj-i-n} &{jeɾkʰ-ej-i-n} &{{ut-ej-i-n}}&$\sqrt{~}$-{\thgloss}-{\pst}-3{\pl}
%%				\\
%				\hline {\iaAbbre}& Inf. &{{kɒɻtʰ-ɒ-l}} &{jeɻkʰ-e-l} &{{ut-e-l}}&$\sqrt{~}$-{\thgloss}-{\infgloss}
%				\\ 
%%				&Sbjv.  Past   3PL &{{kɒɻtʰ-ɒj-i-n}} &{jeɻkʰ-$\emptyset$-i-n} &{{ut-$\emptyset$-i-n}}&$\sqrt{~}$-{\thgloss}-{\pst}-3{\pl}
%%				\\ 
%				& 		& \armenian{կարդալ}& \armenian{երգել}& \armenian{ուտել}& 
%				\\
%				\hline
%			\end{tabular}
%		}
%	\end{subtable}
%	
%	\begin{subtable}[h]{\textwidth}
%		\caption{Past perfective 1PL form}
		\begin{tabular}{llll}
			\lsptoprule 
			& A-Class & E-Class & Suppletive \\	
			&`we read' & `we sang' & `we ate' \\\midrule
			{\seaAbbre} &{{kɑɾtʰ-ɑ\textbf{-t͡sʰ-i}-ŋkʰ}} &{{jeɾkʰ-e\textbf{-t͡sʰ-i}-ŋkʰ}} &{{keɾ-$\emptyset$\textbf{-$\emptyset$-ɑ}-ŋkʰ}}\\
			&$\sqrt{~}$-{\thgloss}-{\aorperf}-{\pst}-1{\pl}&$\sqrt{~}$-{\thgloss}-{\aorperf}-{\pst}-1{\pl}&$\sqrt{~}$-{\thgloss}-{\aorperf}-{\pst}-1{\pl}\\
			& \armenian{կարդացինք} & \armenian{երգեցինք} & \armenian{կերանք}\\
			\addlinespace 
			
			{\iaAbbre} &{{kɒɻtʰ-ɒ\textbf{-t͡sʰ-i}-ŋkʰ}} &{{jeɻkʰ-$\emptyset$\textbf{-$\emptyset$-ɒ}-ŋkʰ}} &{{keɻ-$\emptyset$\textbf{-$\emptyset$-ɒ}-ŋkʰ}}\\
			&$\sqrt{~}$-{\thgloss}-{\aorperf}-{\pst}-1{\pl}&$\sqrt{~}$-{\thgloss}-{\aorperf}-{\pst}-1{\pl}&$\sqrt{~}$-{\thgloss}-{\aorperf}-{\pst}-1{\pl}\\ 
			& \armenian{կարդացինք} & \armenian{երգանք} & \armenian{կերանք}\\
			\lspbottomrule
		\end{tabular}
%	\end{subtable}
\end{table}



It is a separate diachronic question to determine what caused these changes. One possible source is that the /{ɑ}/ morph is used in high-frequency irregular and suppletive verbs in {\seaSEA}. {\iaIA} speakers thus generalized the distribution of /ɑ, {ɒ}/ from high-frequency verbs to regular verbs, as illustrated above. Such a diachronic change is attested across different Armenian lects of Iran (\cites[201]{Adjarian-1961-Liakatar4Book2Conj}{Martirosyan-2019-Armeniandialects}) and {\seaCEA} in Yerevan (\citealt[230]{DumTragut-2009-ArmenianReferenceGrammar} citing \cites[98]{Gharagyulyan-1981-ColloquialArmenian}{Avetyan-2020-TendenciesAnalogicalArmenianAorist}). Tehrani {\iaIA} is special in how wide-scale this change is.\footnote{Some dialectological sources are more vague because they conflate the use of a zero perfective -$\emptyset$ with a past /-ɑ, -ɒ/ \citep[p. 102, feature 95]{Jahukyan-1972-ArmenianDiaolectology}. }


We leave a full-scale diachronic investigation to future work. For now, we focus on a synchronic analysis of {\iaIA}.\footnote{For the perfective of the A-Class, one could argue that the reason why the aorist \textit{{-t͡sʰ-}} and past suffix /{i}/ are used is to maintain a contrast between a past perfective 1PL form like [{{kɒɻtʰ-ɒ-t͡sʰ-i-ŋkʰ}}] `we read.{\pst}' (where /{ɒ}/ is the past morph) vs. a subjunctive present form \textit{{kɒɻtʰ-ɒ-ŋkʰ}} and subjunctive past [{{kɒɻtʰ-ɒj-i-ŋkʰ}}] `if we read.{\pst}' (where /{ɒ}/ is the theme vowel). See \S\ref{section:verb:synthesis:subj} for a fuller discussion of subjunctives.} The generalization is that in {\seaSE}, the default template for the past perfective is /{-t͡sʰ-i}/, while it is /{-$\emptyset$-ɒ}/ in {\iaIA}. In auxiliaries and in the subjunctive past   (\S\ref{section:verb:synthesis:subj}), the past is uniformly just /{-i}/ for all classes (Table \ref{tab:subj for aor}); it is zero for the 3SG. 

\begin{table}
	\caption{Infinitive  and  subjunctive past forms}\label{tab:subj for aor}
	\resizebox{\textwidth}{!}{%
		\begin{tabular}{lllll l}
			\lsptoprule 
			& & A-Class & E-Class & Suppletive& \\
			& &`to read' & `to sing' & `to eat' & \\\midrule
			Inf. & 	{\seaAbbre}   &{kɑɾtʰ-ɑ-l} &{jeɾkʰ-e-l} &{{ut-e-l}}&$\sqrt{~}$-{\thgloss}-{\infgloss}\\
			& 		{\iaAbbre}&  {{kɒɻtʰ-ɒ-l}} &{jeɻkʰ-e-l} &{{ut-e-l}}&$\sqrt{~}$-{\thgloss}-{\infgloss}\\ 
			& 		& \armenian{կարդալ}& \armenian{երգել}& \armenian{ուտել}& \\
			\addlinespace 		
			Sbjv.  Past   3PL & 	{\seaAbbre}	  &{kɑɾtʰ-ɑj-i-n} &{jeɾkʰ-ej-i-n} &{{ut-ej-i-n}}&$\sqrt{~}$-{\thgloss}-{\pst}-3{\pl}\\
			& {\iaAbbre}&{{kɒɻtʰ-ɒj-i-n}} &{jeɻkʰ-$\emptyset$-i-n} &{{ut-$\emptyset$-i-n}}&$\sqrt{~}$-{\thgloss}-{\pst}-3{\pl}\\ 
			& & \armenian{կարդային}& \armenian{երգեին, երգին}& \armenian{ուտեին, ուտին} & 	\\			\addlinespace 
			Sbjv.  Past   3SG &			{\seaAbbre}& {{kɑɾtʰ-ɑ-$\emptyset$-ɾ}} &{jeɾkʰ-e-$\emptyset$-ɾ} &{{ut-e-$\emptyset$-ɾ}}&$\sqrt{~}$-{\thgloss}-{\pst}-3{\sg}\\ 
			& 			{\iaAbbre} &{{kɒɻtʰ-ɒ-$\emptyset$-ɻ}} &{jeɻkʰ-e-$\emptyset$-ɻ} &{{ut-e-$\emptyset$-ɻ}}&$\sqrt{~}$-{\thgloss}-{\pst}-3{\sg}\\ 
			& & \armenian{կարդար}& \armenian{երգեր, երգէր}& \armenian{ուտեր, ուտէր} & \\
			\lspbottomrule
		\end{tabular}}
\end{table}


These generalizations are formalized below, based on the A-Class, suppletive `to eat', and E-Class.  %{\added}
For illustration, we use rules that realize  templates of morphemes like {\aor}-{\pst} because the exponents   for the two morpheme slots are highly correlated. 

%\textcolor{red}{all of the following was added or rewritten {\added}} 


For the past perfective, this paradigm cell uses the morpheme template {\aor}-{\pst} (Rule \ref{rule:pstperf:aorpst}). In {\seaSEA}, this template is realized as /-$\emptyset$-ɑ/ for a handful of irregular verbs like `to eat', while it is /-t͡sʰ-i/ elsewhere for the E-Class and A-Class. In contrast, in {\iaIA}, the template /-t͡sʰ-i/ is for the A-Class, and   /-$\emptyset$-ɒ/ is elsewhere.

\begin{newruleblock}[rule:pstperf:aorpst]
	{Rules for exponing the template /AOR-PST/ in the past perfective for the E-Class, A-Class, and suppletive `to eat'} %%\label{rule:pstperf:aorpst} 
	
	\begin{itemize}
	\item \seaSE: \begin{tabular}[t]{@{}llll@{}} 
				  {\aor}-{\pst} &$\rightarrow$ & -$\emptyset$-ɑ & / $\sqrt{\text{eat}}$ {\thgloss} \_ \\
				                 &             & -t͡sʰ-i &/ elsewhere\\
	               \end{tabular}
   \item \iaIA: \begin{tabular}[t]{@{}llll@{}}
			    {\aor}-{\pst} &$\rightarrow$ & -t͡sʰ-i & / $\sqrt{\text{A-Class}}$ {\thgloss} \_     \\
			                  &              & -$\emptyset$-ɒ &/ elsewhere \\
                \end{tabular}               
     \end{itemize}
\end{newruleblock}


Table \ref{tab:deriv:pstperf}  illustrates the application of the above rules. 

\begin{table}
	\caption{Deriving or exponing the template {\aor}-{\pst} in the past perfective}
	\label{tab:deriv:pstperf}
	\resizebox{\textwidth}{!}{%
		\begin{tabular}{llll}
			\lsptoprule
			&   E-Class & A-Class & Suppletive \\
			&  `they sang' & `they read' & `they ate'\\\midrule 
			Input & $\sqrt{\text{sing}}$-{\thgloss}-\textbf{{\aor}-{\pst}}-3{\pl} & $\sqrt{\text{read}}$-{\thgloss}-\textbf{{\aor}-{\pst}}-3{\pl} & $\sqrt{\text{eat}}$-{\thgloss}-\textbf{{\aor}-{\pst}}-3{\pl}\\
			{\seaAbbre} & jeɾkʰ-e-\textbf{t͡sʰ-i}-n  & kɑɾtʰ-ɑ-\textbf{t͡sʰ-i}-n & keɾ-$\emptyset$-\textbf{$\pmb{\emptyset}$-ɑ}-n \\
 {\iaAbbre} & jeɻkʰ-$\emptyset$-\textbf{$\pmb{\emptyset}$-ɒ}-n  & kɒɻtʰ-ɒ-\textbf{t͡sʰ-i}-n & keɻ-$\emptyset$-\textbf{$\pmb{\emptyset}$-ɒ}-n \\ 
 \lspbottomrule\end{tabular}}
\end{table}

In the past auxiliary and subjunctive past, there is no perfective or aorist morpheme {\aor}. Instead, the template is just {\pst}. This morpheme is realized in the same way in both dialects as just /i/ for all but the 3SG. We illustrate a rule below (Rule \ref{rule:impf:pstimpf}). It is /-$\emptyset$/ for the 3SG, and /-i/ elsewhere. 

Table \ref{tab:deriv:nonperfpast}   illustrates the application of the above rules. 

\begin{newruleblock}[rule:impf:pstimpf]
	{Rules for exponing the template /PST/ in the past auxiliary and subjunctive past} %%  \label{rule:impf:pstimpf}
	
	\begin{center}
		
		
		\begin{tabular}{llllllll}
			\lsptoprule
			\multicolumn{4}{l}{{\seaSE}} & \multicolumn{4}{l}{{\iaIA}} \\\midrule 
			{\pst} &$\rightarrow$ & -$\emptyset$  & / \_ 3SG &    {\pst} &$\rightarrow$ & -$\emptyset$  & / \_  3SG      \\
			&& -i & / elsewhere  
			&
			&& -i & / elsewhere  
			\\ \lspbottomrule 
			
		\end{tabular}
	\end{center}
	
\end{newruleblock} 


\begin{table}
	\caption{Deriving or exponing the template {\aor}-{\pst} in the past auxiliary or subjunctive past}
	\label{tab:deriv:nonperfpast}  
	{%\resizebox{\textwidth}{!}{%
			\begin{tabular}{lll}
				\lsptoprule
				&    A-Class & A-Class  \\
				&  `if he were reading' & `if they were reading'\\%%\cmidrule(lr){2-2}\cmidrule(lr){3-3}
				Input & $\sqrt{\text{read}}$-{\thgloss}-\textbf{{\pst}}-3{\sg} & $\sqrt{\text{read}}$-{\thgloss}-\textbf{{\pst}}-3{\pl}  
				\\\midrule
				{\seaAbbre} & kɑɾtʰ-ɑ-\textbf{$\pmb{\emptyset}$}-ɾ  & kɑɾtʰ-ɑj-\textbf{i}-n
				\\
				{\iaAbbre} & kɒɻtʰ-ɒ-\textbf{$\pmb{\emptyset}$}-ɻ  & kɒɻtʰ-ɒj-\textbf{i}-n
				\\
				\lspbottomrule    \end{tabular}
			
		}
	\end{table}
	
	
	If we try to decompose the template {\aor}-{\pst} into two separate realizations, so that we can unite the rules for the perfective and non-perfective (sbjv. past),  then it is difficult to write a coherent set of rules to expone the past morpheme (Rule \ref{rule:pst:all}). For {\seaSEA}, the past morpheme is /-ɑ/ for irregular perfectives, /-$\emptyset$/ for   3SG, and /-i/ elsewhere (regular perfectives and non-3SG non-perfectives). For {\iaIA}, the past morpheme is /-$\emptyset$/ for A-Class 3SG perfectives,  /-$\emptyset$/ for  3SG non-perfectives,  /-i/ for A-Class non-3SG perfectives, /-ɒ/ for other perfectives (for other classes), and then /-i/ again for non-3SG non-perfectives. We use the notation {\neg\aor} to denote non-perfective contexts \citep[cf.][49]{siddiqi-2009-syntaxWithinWordEconomyAllomorphyArgumentSeelctionDistributedMorphology}. 
	
	
	
	
	Table \ref{tab:deriv:allpst} illustrates the application of the above rules for {\iaIA}. 
	
	
	
	
	The rules are quite convoluted. But the core generalization is that in the past perfective, the default template is /{-t͡sʰ-i}/, while /{-$\emptyset$-ɑ}/ is the restricted or marked template. {\iaIA} instead does the reverse, with /{-$\emptyset$-ɒ}/ as default while /{-t͡sʰ-i}/ is restricted or marked. When there is no aorist morpheme, the past morpheme reverts back to /-i/ as the elsewhere form. We next discuss subjunctives, where we again find the past marker /{-i}/.
	
	\begin{newruleblock}[rule:pst:all]
		{Rules for exponing the morpheme {PST} in the past auxiliary, subjunctive past, and past perfective for E/A-Class and `to eat'}%%\label{rule:pst:all}
		\resizebox{\linewidth}{!}{%
			\begin{tabular}{llllllll}
				\lsptoprule
				\multicolumn{4}{l}{{\seaSE}} & \multicolumn{4}{l}{{\iaIA}} \\\cmidrule(lr){1-4}\cmidrule(lr){5-8}
				{\pst} &$\rightarrow$ & -ɑ  & / $\sqrt{\text{eat}}$ {\aor} \_ &  {\pst} &$\rightarrow$ & -$\emptyset$  & / $\sqrt{\text{A-Class}}$ {\thgloss} {\aor} \_ 3SG \\
				&& -$\emptyset$ & /   \_ 3SG  
				&
				&& -$\emptyset$ & / {\neg\aor} \_ 3SG  
				\\
				&& -i & / elsewhere
				& 
				&& -i & / $\sqrt{\text{A-Class}}$ {\thgloss} {\aor} \_    
				\\
				&& & 
				& 
				&& -ɒ & /  {\aor} \_  
				\\   
				&& & 
				& 
				&& -i & /  {\neg\aor} \_  
				\\   \lspbottomrule 
				
			\end{tabular}}
		
	\end{newruleblock} 
	
	\begin{table}
		\caption{Deriving or exponing the past morpheme in {\iaIA}}
		\label{tab:deriv:allpst}  
		 	\resizebox{\textwidth}{!}{%
					\begin{tabular}{ll lll }
				\lsptoprule
				A-Class & `to read' & perfective 3SG    &  $\sqrt{\text{read}}$-{\thgloss}-{\aor}-\textbf{{\pst}}-3{\sg} & kɒɻtʰ-ɒ-t͡sʰ-\textbf{$\pmb{\emptyset}$}-$\emptyset$    \\
				E-Class & `to sing' & sbjv. past 3SG    &  $\sqrt{\text{sing}}$-{\thgloss}-\textbf{{\pst}}-3{\sg} & jeɻkʰ-e-\textbf{$\pmb{\emptyset}$}-ɻ\\
				A-Class & `to read' & perfective 3PL    &  $\sqrt{\text{read}}$-{\thgloss}-{\aor}-\textbf{{\pst}}-3{\pl} & kɒɻtʰ-ɒ-t͡sʰ-\textbf{i}-n\\
				E-Class & `to sing' & perfective 3PL    &  $\sqrt{\text{sing}}$-{\thgloss}-{\aor}-\textbf{{\pst}}-3{\sg} & jeɻkʰ-$\emptyset$-$\emptyset$-\textbf{ɒ}-n\\
				E-Class & `to sing' & sbjv. past 3PL    &  $\sqrt{\text{sing}}$-{\thgloss}-\textbf{{\pst}}-3{\pl} & jeɻkʰ-$\emptyset$-\textbf{i}-n\\
				\lspbottomrule
			\end{tabular}}  
	\end{table}
	
	
	
	
	\subsection{Subjunctive}\label{section:verb:synthesis:subj}
	
	The subjunctive is a synthetic construction. It includes present and past subjunctives. In brief, these synthetic subjunctive forms differ from the periphrastic indicative forms by placing T/Agr suffixes on the verb itself instead of on the auxiliary. We illustrate below for the A-Class verb [{{kɒɻtʰ-ɒ-l}}] `to read' in {\iaIA} (Table \ref{tab:Verb:Synthn:Subj:Contrast}).
	
	\begin{table}
		\caption{Synthetic subjunctives vs. periphrastic indicatives for the 1PL in {\iaIA}}\label{tab:Verb:Synthn:Subj:Contrast}
		
		\begin{tabular}{lllll}
			\lsptoprule
			& \multicolumn{2}{l}{Present 1PL} & \multicolumn{2}{l}{Past PL } \\\midrule 
			Indicative & {{kɒɻtʰ-um}}& {{e-ŋkʰ}} & {{kɒɻtʰ-um}} &{{$\emptyset$-i-ŋkʰ}}\\
			& $\sqrt{~}$-{\impfcvb} & {\auxgloss}-1{\pl}& $\sqrt{~}$-{\impfcvb} & {\auxgloss}-{\pst}-1{\pl}\\
			& \multicolumn{2}{l}{`we read' } & \multicolumn{2}{l}{`we were reading'}\\
			& \multicolumn{2}{l}{\armenian{կարդում ենք}}& \multicolumn{2}{l}{\armenian{կարդում ինք}}\\ \addlinespace 
			Subjunctive &\multicolumn{2}{l}{{{kɒɻtʰ-ɒ-ŋkʰ}}}& \multicolumn{2}{l}{{{kɒɻtʰ-ɒj-i-ŋkʰ}}}\\
			&\multicolumn{2}{l}{$\sqrt{~}$-{\thgloss}-1{\pl}}& \multicolumn{2}{l}{$\sqrt{~}$-{\thgloss}-{\pst}-1{\pl}}\\
			& \multicolumn{2}{l}{`(if) we read' } & \multicolumn{2}{l}{`(if) we were reading' }\\
			& \multicolumn{2}{l}{\armenian{կարդանք}}& \multicolumn{2}{l}{\armenian{կարդայինք}}\\ 
			\lspbottomrule
		\end{tabular}
	\end{table}
	
	Diachronically, the modern subjunctive construction is a reflex of the Classical indicative \citep{Vaux-1995-ArmenianVerbDiachrony}. Subjunctive forms can also combine with other particles to create more nuanced meanings. For example, subjunctives can combine with the debitive proclitic [piti] to create the debitive mood (\S\ref{section:funct:other}). 
	
	We discuss the two types of subjunctives below. 
	
	
	
	\subsubsection{Subjunctive with present-tense agreement}\label{section:verb:synthesis:subj:pres}
	
	
	
	
	
	
	
	We first discuss the present subjunctive. When the finite verb uses present-tense agreement morphemes, the construction has been called “subjunctive present” (\citealt[190]{Minassian-1980-EastArmenianGrammar}, \citealt[160]{Hagopian-2007-ArmenianTextbookEveryone}), “subjunctive future” (\citealt[174]{BardakjianVaux-1999-easternArmeniantextbook}, \citealt[150]{Sakayan-2007-TextbookEasternArmenian}, \citealt[239]{DumTragut-2009-ArmenianReferenceGrammar}), “present optative” (\citealt[149]{fairbanksStevick-1975-spokenEastArmenian}). We label this construction as just the “subjunctive present”, in order to emphasize the connection between the indicative and subjunctive forms. 
	
	Paradigms for the E-Class are in Table \ref{tab:subj pres e} and for the A-Class in Table \ref{tab:subj pres a}. 
	Negation is marked by adding the prefix /{t͡ʃʰ-}/, which triggers schwa epenthesis before a consonant. We juxtapose these subjunctive forms with their indicative periphrastic forms. For illustration, we also provide the {\seaSE} subjunctive present which does not morphologically differ from {\iaIA}. As before, we treat the present tense suffix as fused with the agreement suffix.
	
	\begin{table}
		\caption{Paradigm of subjunctive present in simple E-Class verbs in {\seaSE} and {\iaIA} }
		\label{tab:subj pres e}
		\resizebox{\textwidth}{!}{%
			\begin{tabular}{lllll}
				\lsptoprule
				& \multicolumn{3}{l}{{\iaIA}} & \multicolumn{1}{l}{{\seaSE}} \\\cmidrule(lr){2-4}\cmidrule(lr){5-5}
				& \multicolumn{2}{l}{Sbjv. present} & \multicolumn{1}{l}{Indc. present} & \multicolumn{1}{l}{Sbjv. present}\\
				& Positive & Negaive & Positive & Positive \\\midrule
				1SG
				& {{jeɻkʰ-e-m}} 
				& {{t͡ʃʰə-jeɻkʰ-e-m}} 
				& {{jeɻkʰ-um e-m}}
				& {{jeɾkʰ-e-m}} 
				\\
				& `(if) I sing'&`(if) I did not sing' & `I sing'& `(if) I sing'
				\\
				& \armenian{երգեմ}
				& \armenian{չերգեմ}
				& \armenian{երգում եմ}
				& \armenian{երգեմ}
				\\
				\addlinespace
				2SG
				& {{jeɻkʰ-e-s}}
				& {{t͡ʃʰə-jeɻkʰ-e-s}}
				& {{jeɻkʰ-um e-s}}
				& {{jeɾkʰ-e-s}}
				\\
				& \armenian{երգես}
				& \armenian{չերգես}
				& \armenian{երգում ես}
				& \armenian{երգես}
				\\
				\addlinespace		3SG
				& {{jeɻkʰ-i}} 
				& {{t͡ʃʰə-jeɻkʰ-i}}
				& {{jeɻkʰ-um ɒ}} 
				& {{jeɾkʰ-i}} 
				\\
				& \armenian{երգի}
				& \armenian{չերգի}
				&\armenian{երգում ա}
				& \armenian{երգի}
				\\
				\addlinespace		1PL
				& {{jeɻkʰ-e-ŋkʰ}}
				& {{t͡ʃʰə-jeɻkʰ-e-ŋkʰ}}
				& {{jeɻkʰ-um e-ŋkʰ}} 
				& {{jeɾkʰ-e-ŋkʰ}} 
				\\
				& \armenian{երգենք}
				& \armenian{չերգենք}
				& \armenian{երգում ենք}
				& \armenian{երգենք}
				\\
				\addlinespace		2PL
				& {{jeɻkʰ-e-kʰ}} 
				& {{t͡ʃʰə-jeɻkʰ-e-kʰ}}
				& {{jeɻkʰ-um e-kʰ}} 
				& {{jeɾkʰ-e-kʰ}} 
				\\
				& \armenian{երգէք}
				& \armenian{չերգէք}
				& \armenian{երգում էք}
				& \armenian{երգեք}
				\\
				\addlinespace		3PL
				& {{jeɻkʰ-e-n}} 
				& {{t͡ʃʰə-jeɻkʰ-e-n}}
				& {{jeɻkʰ-um e-n}}
				& {{jeɾkʰ-e-n}} 
				\\
				& \armenian{երգեն}
				& \armenian{չերգեն}
				& \armenian{երգում են}
				& \armenian{երգեն}
				\\
				\addlinespace 
				&\multicolumn{2}{l}{( {\neggloss})-$\sqrt{~}$-{\thgloss}-{\agr}}&\multicolumn{1}{l}{$\sqrt{~}$-{\impfcvb} {\auxgloss}-{\agr}}
				&\multicolumn{1}{l}{$\sqrt{~}$-{\thgloss}-{\agr}}
				\\\lspbottomrule
			\end{tabular}}
	\end{table}
	
	
	For all but the 3SG, the distribution of the Agr suffixes follows straightforwardly. The same Agr suffixes as used in the present auxiliary are placed onto the subjunctive verb. In the A-Class, we see that the 3SG morph is covert in the present subjunctive: [{{kɒɻtʰ-ɒ-$\emptyset$}}] $\sqrt{\text{read}}$-{\thgloss}-3{\sg} `he reads'. Similarly, the present 3SG auxiliary is just /{ɒ-$\emptyset$}/. But for the E-Class, the /{e}/ theme vowel is replaced by /i/: {{[jeɻkʰ-i-$\emptyset$}}] `he sings' instead of *\textit{{jeɻkʰ-e-$\emptyset$}}. 

\begin{sloppypar}
In terms of explanation, this apparent allomorphy has multiple options (Rule~\ref{subj 3sg i options}). 
\end{sloppypar}


\begin{newruleblock}[subj 3sg i options]
	{Hypothetical rules to explain the subjunctive present 3SG} %%\label{ruke:Verb:Perip:Perf:AltRule}

  \begin{enumerate}
     	\item /{i}/ is the marker of the theme vowel /{e}/ but it has changed to [i] in the 3SG.
			
			/{jeɻkʰ-e-$\emptyset$}/ $\rightarrow$ [{jeɻkʰ-i-$\emptyset$}]
			\item /{i}/ is the allomorph of the E-Class theme vowel in the present 3SG.
			
			/{jeɻkʰ-i-$\emptyset$}/ 
					\item /{i}/ is the marker of the E-Class 3SG Agr suffix, and the theme /{e}/ is deleted before /{i}/.
			
			/{jeɻkʰ-e-i}/ $\rightarrow$ [{jeɻkʰ-$\emptyset$-i}]
			\item /{i}/ is the fused marker of the theme vowel /{e}/ and 3SG.
			
			/{jeɻkʰ-i}/ with glossing $\sqrt{~}$-{\thgloss}.3{\sg}
			\item /{i}/ is the result of autosegmental docking of the theme vowel /e/ and the E-Class 3SG floating feature [\textsc{+high}]
			
			/{jeɻkʰ-e}-[+ \textsc{high}]/ $\rightarrow$ [{jeɻkʰ-i}]
			
			Glossing as [jeɻkʰ-i-$\emptyset$]
  \end{enumerate}
		
  
	\end{newruleblock} 

  
  \begin{table}
\caption{Paradigm of subjunctive present in simple A-Class verbs in {\seaSE} and {\iaIA} }
\label{tab:subj pres a}
\resizebox{\textwidth}{!}{%
	\begin{tabular}{lll ll}
		\lsptoprule 
		& \multicolumn{3}{c}{{\iaIA}} & \multicolumn{1}{c}{{\seaSE}}\\\cmidrule(lr){2-4}\cmidrule(lr){5-5}
		& \multicolumn{2}{l}{Sbjv. present}
		& \multicolumn{1}{l}{Indc. present}
		& \multicolumn{1}{l}{Sbjv. present}
		\\
		& Positive & Negaive & Positive & Positive\\\midrule
		1SG
		& {{kɒɻtʰ-ɒ-m}}
		& {{t͡ʃʰə-kɒɻtʰ-ɒ-m}}
		& {{kɒɻtʰ-um e-m}}
		& {{kɑɾtʰ-ɑ-m}}
		\\
		& `(if) I read'& `(if) I did not read' & `I read'	& `(If) I read'
		\\
		& \armenian{կարդամ}
		& \armenian{չկարդամ}
		& \armenian{կարդում եմ}
		& \armenian{կարդամ}
		\\
		\addlinespace
		2SG
		& {{kɒɻtʰ-ɒ-s}}
		& {{t͡ʃʰə-kɒɻtʰ-ɒ-s}}
		& {{kɒɻtʰ-um e-s}}
		& {{kɑɾtʰ-ɑ-s}}
		\\
		& \armenian{կարդաս}
		& \armenian{չկարդաս}
		& \armenian{կարդում ես}
		& \armenian{կարդաս}
		\\
		\addlinespace		3SG
		& {{kɒɻtʰ-ɒ}}
		& {{t͡ʃʰə-kɒɻtʰ-ɒ}}
		& {{kɒɻtʰ-um ɒ}}
		& {{kɑɾtʰ-ɑ}}
		\\
		& \armenian{կարդայ}
		& \armenian{չկարդայ}
		& \armenian{կարդում ա}
		& \armenian{կարդա}
		\\
		\addlinespace		1PL
		& {{kɒɻtʰ-ɒ-ŋkʰ}}
		& {{t͡ʃʰə-kɒɻtʰ-ɒ-ŋkʰ}}
		& {{kɒɻtʰ-um e-ŋkʰ}}
		& {{kɑɾtʰ-ɑ-ŋkʰ}}
		\\
		& \armenian{կարդանք}
		&\armenian{չկարդանք}
		& \armenian{կարդում ենք}
		& \armenian{կարդանք}
		\\
		\addlinespace		2PL
		& {{kɒɻtʰ-ɒ-kʰ}}
		& {{t͡ʃʰə-kɒɻtʰ-ɒ-kʰ}}
		& {{kɒɻtʰ-um e-kʰ}}
		& {{kɑɾtʰ-ɑ-kʰ}}
		\\
		& \armenian{կարդաք}
		& \armenian{չկարդաք}
		& \armenian{կարդում էք}
		& \armenian{կարդաք}
		\\
		\addlinespace		3PL
		& {{kɒɻtʰ-ɒ-n}}
		& {{t͡ʃʰə-kɒɻtʰ-ɒ-n}}
		& {{kɒɻtʰ-um e-n}}
		& {{kɑɾtʰ-ɑ-n}}
		\\
		& \armenian{կարդան}
		& \armenian{չկարդան}
		& \armenian{կարդում են}
		& \armenian{կարդան}
		\\
		\addlinespace 
		&\multicolumn{2}{l}{( {\neggloss})-$\sqrt{~}$-{\thgloss}-{\agr}}&\multicolumn{1}{l}{$\sqrt{~}$-{\impfcvb} {\auxgloss}-{\agr}}
		&\multicolumn{1}{l}{$\sqrt{~}$-{\thgloss}-{\agr}}
		\\\lspbottomrule
	\end{tabular}} 
\end{table}
	
	
	
	Any of the above options must restrict the relevant change to the E-Class, while the A-Class and auxiliary would use a zero morph for the 3SG. We are partial to a floating feature analysis \citep[cf.][]{Akinlabi-2011-FeaturalAffix} and we use that for illustration. We likewise suspect that such allomorphy is not triggered by classes themselves, but by the identity of the actual theme vowel. That is, the present 3SG is [\textsc{+high}] after the /{e}/ theme vowel, but a zero -$\emptyset$ elsewhere (Rule \ref{subj 3sg i rule}).
	
	\begin{newruleblock}[subj 3sg i rule]
		{Rule for the present 3SG agreement suffix}%%\label{subj 3sg i rule}
		
		\begin{center}
			\begin{tabular}{llll}
				{\prs}.3{\sg} & $\leftrightarrow$ & [\textsc{+high}] & / {e}\textsubscript{Th} \_ \\
				& & -$\emptyset$ & / elsewhere
				
			\end{tabular}
		\end{center} 
	\end{newruleblock}
	
	
	One reason why we are partial to this floating feature analysis over alternatives involving allomorphs is that in {\swaSWA}, the present 3SG suffix is uniformly a zero for both the E-Class and the A-Class, e.g., the subjunctive forms [{{jeɾkʰ-e-$\emptyset$}}] `(if) he sings' and [{{ɡɑɾtʰ-ɑ-$\emptyset$}}] `(if) he reads' [$\sqrt{~}$-{\thgloss}-{\prs}/1{\sg}]. Thus, it is likely that {\seaSE} and {\iaIA} are innovative in causing this /{e}/$\rightarrow$[{i}] change in the present 3SG.
	
	\subsubsection{Subjunctive with past-tense agreement}\label{section:verb:synthesis:subj:pst}\largerpage
	
	
	
	Moving on to the past tense, the subjunctive forms again involve placing the T/Agr suffixes directly onto the verb instead of the auxiliary. 
	
	When the verb has the tense-agreement morphemes of the past, then the construction has been called the “subjunctive past” in grammars of {\seaSEA} that are written in English (\citealt[174]{BardakjianVaux-1999-easternArmeniantextbook}, \citealt[160]{Hagopian-2007-ArmenianTextbookEveryone}, \citealt[150]{Sakayan-2007-TextbookEasternArmenian}, \citealt[249]{DumTragut-2009-ArmenianReferenceGrammar}). One French grammar uses the “subjunctive imperfect” (\citealt[191]{Minassian-1980-EastArmenianGrammar}). However, for grammars of {\swaSWA}, the cognate construction is called either the “subjunctive past” (\citealt[113]{Sakayan-2000-TextbookWesternArmenian}, \citealt[143]{Hagopian-2007-ArmenianTextbookEveryone}), “subjunctive imperfect” (\citealt[35]{Riggs-1856-GrammarModernArmenianConstantinople}, \citealt[50]{Gulian-1902-ElementaryModernGrammarArmenian}, \citealt[107]{Feydit-1948-manuelWesternArmenian}, \citealt[89]{Kogian-1949-ArmenianWestGrammar}, \citealt[154]{BardakjianThomson-1977-WestArmenianTextbook}, \citealt[47]{Andonian-1999-BeginnerArmenian}, \citealt[181]{BardakjianVaux-2001-WasternArmeniantextbook}), “past optative” (\cites[78]{Fairbanks-1948-PhonologyMorphoWestern}{Fairbanks-1958-SpokenWestArmenian}), “hypothetical imperfect” (\citealt{Boyacioglu-2010-HayPayVerbsArmenianOccidentalWestArmenian}). The large set of names for the past-based subjunctive is due to the fact that the Armenian name for it is variably the subjunctive imperfect or subjunctive past.{\interfootnotelinepenalty=10000\footnote{The {\seaAbbre} name for the word “subjunctive” is variably [əʁd͡zɑkɑn] \armenian{ըղձական} or [stoɾɑdɑsɑkɑn] \armenian{ստորադասական}. }}  We call this construction the “subjunctive past”. 
	
	
	
	We provide paradigms below for E-Class \textit{{jeɻkʰ-e-l}} (Table \ref{tab:subj past e}) and A-Class \textit{{kɒɻtʰ-ɒ-l}} (Table \ref{tab:subj past a}). The abstract morphological structure of subjunctive past verbs is the same in {\seaSE} and {\iaIA}. We show deleted and zero morphs. Negation is again formed by adding [{t͡ʃʰ(ə)-}].
	
	
	\begin{table}
		\caption{Paradigm of subjunctive past in simple E-Class verbs in {\seaSE} and {\iaIA} }
		\label{tab:subj past e}
		\resizebox{\textwidth}{!}{%
			\begin{tabular}{lllll }
				\lsptoprule 
				& \multicolumn{3}{c}{{\iaIA}} & \multicolumn{1}{c}{{\seaSE}} \\\cmidrule(lr){2-4}\cmidrule(lr){5-5}
				& \multicolumn{2}{c}{Sbjv. past} & \multicolumn{1}{l}{Indc. past impf.} & \multicolumn{1}{l}{Sbjv. past}\\ \cmidrule(lr){2-3}\cmidrule(lr){4-4}\cmidrule(lr){5-5}
				& Positive & Negaive & Positive & Positive \\\midrule
				1SG
				& {{jeɻkʰ-$\emptyset$-i-m}} 
				& {{t͡ʃʰə-jeɻkʰ-$\emptyset$-i-m}} 
				& {{jeɻkʰ-um $\emptyset$-i-m}}
				& {{jeɾkʰ-ej-i-$\emptyset$}}
				\\
				& `(if) I were& `(if) I were not & `I was singing'& `(if) I were \\
				& singing'& singing' & & singing'\\
				& \armenian{երգիմ}
				& \armenian{չերգիմ}
				& \armenian{երգում իմ}
				& \armenian{երգեի}
				\\
				\addlinespace	2SG
				& {{jeɻkʰ-$\emptyset$-i-ɻ}} 
				& {{t͡ʃʰə-jeɻkʰ-$\emptyset$-i-ɻ}} 
				& {{jeɻkʰ-um $\emptyset$-i-ɻ}}
				& {{jeɾkʰ-ej-i-ɾ}} 
				\\
				& \armenian{երգիր}
				& \armenian{չերգիր}
				& \armenian{երգում իր}
				& \armenian{երգեիր}
				\\
				\addlinespace	3SG
				& {{jeɻkʰ-e-$\emptyset$-ɻ}}
				& {{t͡ʃʰə-jeɻkʰ-e-$\emptyset$-ɻ}}
				& {{jeɻkʰ-um e-$\emptyset$-ɻ}} 
				& {{jeɾkʰ-e-$\emptyset$-ɾ}} 
				\\
				& \armenian{երգէր}
				& \armenian{չերգէր}
				& \armenian{երգում էր}
				& \armenian{երգեր}
				\\
				\addlinespace	1PL
				& {{jeɻkʰ-$\emptyset$-i-ŋkʰ}} 
				& {{t͡ʃʰə-jeɻkʰ-$\emptyset$-i-ŋkʰ}} 
				& {{jeɻkʰ-um $\emptyset$-i-ŋkʰ}}
				& {{jeɾkʰ-ej-i-ŋkʰ}} 
				\\
				& \armenian{երգինք}
				& \armenian{չերգինք}
				& \armenian{երգում ինք}
				& \armenian{երգենք}
				\\
				\addlinespace		2PL
				& {{jeɻkʰ-$\emptyset$-i-kʰ}} 
				& {{t͡ʃʰə-jeɻkʰ-$\emptyset$-i-kʰ}} 
				& {{jeɻkʰ-um $\emptyset$-i-kʰ}}
				& {{jeɾkʰ-ej-i-kʰ}}\\
				& \armenian{երգիք}
				& \armenian{չերգիք}
				& \armenian{երգում իք}
				& \armenian{երգեիք}
				\\
				\addlinespace		3PL
				& {{jeɻkʰ-$\emptyset$-i-n}}
				& {{t͡ʃʰə-jeɻkʰ-$\emptyset$-i-n}}
				& {{jeɻkʰ-um $\emptyset$-i-n}} 
				& {{jeɾkʰ-ej-i-n}} 
				\\
				& \armenian{երգին}
				& \armenian{չերգին}
				& \armenian{երգում ին}
				& \armenian{երգեին}
				\\
				\addlinespace
				&$\sqrt{~}$-{\thgloss}-{\pst}-{\agr}&{ {\neggloss}-$\sqrt{~}$-{\thgloss}-{\pst}-{\agr}}&\multicolumn{1}{l}{$\sqrt{~}$-{\impfcvb} {\auxgloss}-{\pst}-{\agr}}
				& $\sqrt{~}$-{\thgloss}-{\pst}-{\agr}
				\\\lspbottomrule
			\end{tabular}
	} \end{table}
	
\begin{table}
	\caption{Paradigm of subjunctive past in simple A-Class verbs in {\seaSE} and {\iaIA} }
	\label{tab:subj past a}
	\resizebox{\textwidth}{!}{%
		\begin{tabular}{lllll}
			\lsptoprule 
			& \multicolumn{3}{c}{{\iaIA}} & \multicolumn{1}{l}{{\seaSE}} \\\cmidrule(lr){2-4}\cmidrule(lr){5-5}
			& \multicolumn{2}{c}{Sbjv. past}& \multicolumn{1}{l}{Indc. past impf.} & \multicolumn{1}{l}{Sbjv. past}\\\cmidrule(lr){2-3}\cmidrule(lr){4-4}\cmidrule(lr){5-5}
			& Positive & Negaive & Positive & Positive \\\midrule
			1SG
			& {{kɒɻtʰ-ɒj-i-m}}
			& {{t͡ʃʰə-kɒɻtʰ-ɒj-i-m}}
			& {{kɒɻtʰ-um $\emptyset$-i-m}}
			& {{kɑɾtʰ-ɑj-i-$\emptyset$}}
			\\
			& `(if) I were& `(if) I were not & `I was reading'& `(if) I were \\
			& reading'& reading' & & reading' 
			\\
			& \armenian{կարդայիմ}
			& \armenian{չկարդայիմ}
			& \armenian{կարդում իմ}
			& \armenian{կարդայի}
			\\
			\addlinespace		2SG
			& {{kɒɻtʰ-ɒj-i-ɻ}}
			& {{t͡ʃʰə-kɒɻtʰ-ɒj-i-ɻ}}
			& {{kɒɻtʰ-um $\emptyset$-i-ɻ}}
			& {{kɑɾtʰ-ɑj-i-ɾ}}
			\\
			& \armenian{կարդայիր}
			& \armenian{չկարդայիր}
			& \armenian{կարդում իր}
			& \armenian{կարդայիր}
			\\
			\addlinespace		3SG
			& {{kɒɻtʰ-ɒ-$\emptyset$-ɻ}}
			& {{t͡ʃʰə-kɒɻtʰ-ɒ-$\emptyset$-ɻ}}
			& {{kɒɻtʰ-um e-$\emptyset$-ɻ}}
			& {{kɑɾtʰ-ɑ-$\emptyset$-ɾ}}
			\\
			& \armenian{կարդար}
			& \armenian{չկարդար}
			& \armenian{կարդում էր}
			& \armenian{կարդար}
			\\
			\addlinespace		1PL
			& {{kɒɻtʰ-ɒj-i-ŋkʰ}}
			& {{t͡ʃʰə-kɒɻtʰ-ɒj-i-ŋkʰ}}
			& {{kɒɻtʰ-um $\emptyset$-i-ŋkʰ}}
			& {{kɑɾtʰ-ɑj-i-ŋkʰ}}
			\\
			& \armenian{կարդայինք}
			& \armenian{չկարդայինք}
			& \armenian{կարդում ինք}
			& \armenian{կարդայինք}
			\\
			\addlinespace		2PL
			& {{kɒɻtʰ-ɒj-i-kʰ}}
			& {{t͡ʃʰə-kɒɻtʰ-ɒj-i-kʰ}}
			& {{kɒɻtʰ-um $\emptyset$-i-kʰ}}
			& {{kɑɾtʰ-ɑj-i-kʰ}}
			\\
			& \armenian{կարդայիք}
			& \armenian{չկարդայիք}
			& \armenian{կարդում իք}
			& \armenian{կարդայիք}
			
			\\
			\addlinespace		3PL
			& {{kɒɻtʰ-ɒj-i-n}}
			& {{t͡ʃʰə-kɒɻtʰ-ɒj-i-n}}
			& {{kɒɻtʰ-um $\emptyset$-i-n}}
			& {{kɑɾtʰ-ɑj-i-n}}
			\\
			& \armenian{կարդային}
			& \armenian{չկարդային}
			& \armenian{կարդում ին}
			& \armenian{կարդային}
			
			\\
			\addlinespace		
			& $\sqrt{~}$-{\thgloss}-{\pst}-{\agr}& {\neggloss}-$\sqrt{~}$-{\thgloss}-{\pst}-{\agr}&\multicolumn{1}{l}{$\sqrt{~}$-{\impfcvb} {\auxgloss}-{\pst}-{\agr}}
			&$\sqrt{~}$-{\thgloss}-{\pst}-{\agr}\\\lspbottomrule
		\end{tabular}} 
	\end{table} 

	The markers of tense and agreement in the subjunctive past all follow from the same rules used for auxiliaries. 
	
	Morphophonologically, vowel hiatus between the theme vowel and past /{i}/ causes deletion of the /{e}/ theme vowel in {\iaIA}, while [{j}] is epenthesized after the /{ɒ}/ theme vowel. In {\seaSE}, the /{e}/ theme vowel is not deleted; instead [j{}] is epenthesized to resolve vowel hiatus. We illustrate this below for the 1PL (see the derivation in \tabref{Derivation:Verb:Synthn:Subj:Hiatus}).
	

	Glide epenthesis is a general rule of hiatus repair in Armenian, while deletion requires morpheme-specific deletion rules (Rule \ref{rule:Verb:Synthn:Subj:Hiatus}).\largerpage
	
	\begin{newruleblock}[rule:Verb:Synthn:Subj:Hiatus]
		{Delete the /{e}/ theme vowel before past /{i}/}%%\label{rule:Verb:Synthn:Subj:Hiatus}
		
		\begin{center}
			
			\begin{tabular}{lllll}
				/{e}/ &$\rightarrow$&$\emptyset$ & / \_ {i} \\
				\multicolumn{5}{l}{(where /{e}/ is a theme vowel, /{i}/ is past)}
			\end{tabular}
		\end{center}
	\end{newruleblock} 
	
	\begin{table}
\caption{Vowel hiatus repair in subjunctive past }\label{Derivation:Verb:Synthn:Subj:Hiatus}
\resizebox{\textwidth}{!}{%
	\begin{tabular}{lllll}
		\lsptoprule
		&\multicolumn{2}{l}{A-Class 1PL}&\multicolumn{2}{l}{E-Class 1PL}\\
		&\multicolumn{2}{l}{`(if) we were reading'}&\multicolumn{2}{l}{`(if) we were singing'}\\\cmidrule(lr){2-3}\cmidrule(lr){4-5}
		& {\seaAbbre} & {\iaAbbre} & {\seaAbbre} & {\iaAbbre}\\ \midrule
		Input& /{kɑɾtʰ-ɑ-i-ŋkʰ}/ & /{kɒɻtʰ-ɒ-i-ŋkʰ}/& /{jeɾkʰ-e-i-ŋkʰ}/& /{jeɻkʰ-e-i-ŋkʰ}/\\
		Epenthesis & {kɑɾtʰ-ɑj-i-ŋkʰ} & {kɒɻtʰ-ɒj-i-ŋkʰ} & {jeɾkʰ-ej-i-ŋkʰ} & \\
		Deletion& & & & {jeɻkʰ-$\emptyset$-i-ŋkʰ}\\
		& \armenian{կարդայինք}& \armenian{կարդայինք}& \armenian{երգեինք}& \armenian{երգինք}\\ 
		\lspbottomrule
	\end{tabular}}
\end{table}
	
	There is evidence that the Armenian dialects of Iran vary in the application of theme vowel deletion before the past marker /i/. In {\seaSEA}, neither the theme vowel /ɑ/ nor the theme vowel /e/ is deleted before past /-i/. In Tehrani {\iaIA}, only /e/ is deleted. But in New Julfa Armenian (Isfahan), both theme vowels are deleted \citep[\S 275]{Adjarian-1940-NewJulfaDialect,Vaux-prep-NewJulfa}. 
	
%{\added}	
As with the past auxiliary (\S\ref{section:verb:aux:past}), the deletion of the theme vowel /e/ before past /i/ is not rare among Armenian dialects. Old Yerevan Armenian likewise had such a rule in the subjunctive past (\citealt[42]{Adjarian-1911-DialectologyBook}; translated: \citealt{Dolatian-prep-Adjarian}). 
	
	
	
	\subsubsection{Eliciting the subjunctive}\label{section:verb:synthesis:subj:elicit}
	
	Before closing this section, we document how we elicited such subjunctives. These subjunctive forms can be elicited in diverse contexts with various meanings \citep[239ff]{DumTragut-2009-ArmenianReferenceGrammar}. In our fieldwork, we used the following sentence where the verb `to want' selects for a subjunctive clause (\ref{sent:Verb:Synthn:Subj:ex}). Note that this sentence is not a control or ECM (exceptional case-marking) construction. The embedded clause can have a different subject than the main clause. The embedded subject can be made overt as a pronoun. The complementizer \textit{{voɻ}} can be optionally added.\largerpage[2]
	
	\begin{exe}
		\ex \label{sent:Verb:Synthn:Subj:ex}
		\begin{xlist}
			
			\ex \gll {uz-um} {e-m} {(voɻ)} {(iɻɒŋkʰ)} {jeɻkʰ-e-n}
			\\
			want-{\impfcvb} {\auxgloss}-1{\sg} (that) (they.{\nom}) sing-{\thgloss}-3{\pl}
			\\
			\trans 		`I want them to sing.' \hfill (NK)
			\\
			\armenian{Ուզում եմ որ իրանք երգեն։}
			\ex \gll {uz-um} {$\emptyset$-i-m} {(voɻ)} {(iɻɒŋkʰ)} {jeɻkʰ-$\emptyset$-i-n}
			\\
			want-{\impfcvb} {\auxgloss}-{\pst}-1{\sg} (that) (they.{\nom}) sing-{\thgloss}-{\pst}-3{\pl}
			\\
			\trans 			`I wanted them to sing.'\hfill (NK)
			\\
			\armenian{Ուզում իմ որ իրանք երգին։}
		\end{xlist}
		
	\end{exe}
	

	
	\subsection{Imperatives and prohibitives}\label{section:verb:synthesis:imp}
	
	Imperatives and prohibitives are formed almost identically between {\seaSE} and {\iaIA}. They are restricted to the second person. The markers of imperative and prohibitive morphology depend on verb class. We show the imperative paradigms in Table \ref{tab:imp proh}. We use zero morphs to represent deleted theme vowels and covert 2SG suffixes.
	
	\begin{table}
		\caption{Paradigm of imperatives in {\seaSE} and {\iaIA}}
		\label{tab:imp proh}
		\resizebox{\textwidth}{!}{%
			\begin{tabular}{l l l l l}
				\lsptoprule 
				& \multicolumn{2}{c}{E-Class `to sing'} & \multicolumn{2}{c}{A-Class `to read'}	\\\cmidrule(lr){2-3}\cmidrule(lr){4-5}
				& {\seaAbbre} & {\iaAbbre}& {\seaAbbre} & 	{\iaAbbre}\\\midrule
				Infinitive 	& {jeɾkʰ-e-l} 	&{jeɻkʰ-e-l} & {{kɑrtʰ-ɑ-l}} & {{kɒrtʰ-ɒ-l}}\\
				&\multicolumn{2}{l}{$\sqrt{~}$-{\thgloss}-{\infgloss}}&\multicolumn{2}{l}{$\sqrt{~}$-{\thgloss}-{\infgloss}}\\
				& \armenian{երգել} 	& \armenian{երգել}	&\armenian{կարդալ}&\armenian{կարդալ}\\\addlinespace
				Imperative 2SG& 	{{jeɾkʰ-$\emptyset$-iɾ}}&{{jeɻkʰ-$\emptyset$-i}}& {{kɑɾtʰ-ɑ-$\emptyset$}}& {{kɒɻtʰ-ɒ-$\emptyset$}}\\
				~~Colloquial & {{jeɾkʰ-$\emptyset$-i}}&&&\\
					&\multicolumn{2}{l}{$\sqrt{~}$-{\thgloss}-{\imp}.2{\sg}}&\multicolumn{2}{l}{$\sqrt{~}$-{\thgloss}-{\imp}.2{\sg}}
				\\
				& \armenian{երգիր}& \armenian{երգի}	&\armenian{կարդա}&\armenian{կարդա}\\ \addlinespace
				Imperative 2PL 	&{{jeɾkʰ-$\emptyset$-ekʰ}}&	{{jeɻkʰ-$\emptyset$-ekʰ}}&	{{kɑɾtʰ-ɑ-t͡sʰ-ekʰ}}&{{kɒɻtʰ-ɒ-t͡sʰ-ekʰ}}\\
				&\multicolumn{2}{l}{$\sqrt{~}$-{\thgloss}-{\imp}.2{\pl}}&\multicolumn{2}{l}{$\sqrt{~}$-{\thgloss}-{\aorother}-{\imp}.2{\pl}}\\
				& \armenian{երգեք} & \armenian{երգէք} &\armenian{կարդացեք} &\armenian{կարդացէք}\\
				\lspbottomrule
			\end{tabular}}
	\end{table} 
	
	The imperative is called [həɾɑmɑjɑkɑn jeʁɑnɑk] \armenian{հրամայական եղանակ} in {\seaSEA}. 
	In the imperative 2SG, the A-Class is inflected by adding nothing to the theme vowel in both lects. The imperative 2SG suffix is thus covert for the A-Class: \textit{kɒɻtʰ-ɒ} `read!'. 
	
	But for the E-Class, there is significant cross-dialectal variation. In {\seaSEA}, the theme vowel is deleted, and followed by the overt imperative 2SG suffix \textit{{-iɾ}}: \textit{jeɾkʰ-iɾ} `sing!'. In {\seaCEA}, the suffix can be optionally reduced to \textit{-i}: \textit{jeɾkʰ-i} (\cites[273]{DumTragut-2009-ArmenianReferenceGrammar}[164]{kamalyan-2015-LiteraryColloquialEasternArmenianChangesStandardziation}{grigoryan-2019-FallOfLiquidRImperative}). {\iaIA} uses only \textit{-i}: \textit{jeɻkʰ-i} `sing!'. In contrast, in {\swaSWA}, both the E-Class and A-Class use a covert suffix without a vowel change: \textit{jeɾkʰ-e} `sing!' \armenian{երգէ}, \textit{ɡɑɾtʰ-ɑ} `read!' \armenian{կարդա}.\largerpage
	
	
	For the imperative 2PL, the two lects align. The E-Class is inflected by adding the imperative 2PL suffix \textit{{-ekʰ}} to the root, deleting the theme vowel: \textit{jeɻkʰ-ekʰ} `sing.{\pl}'. In the A-Class, the aorist suffix \textit{{-t͡sʰ-}} is added between the theme vowel and the \textit{{-ekʰ}}: : \textit{kɒɻtʰ-ɒ-t͡sʰ-ekʰ} `read.{\pl}'. The use of the aorist here is morphomic and meaningless, and is traditionally analyzed as part of an “aorist stem”.{\interfootnotelinepenalty=10000\footnote{One could argue that the reason why the A-Class imperative 2PL uses the morphomic aorist in [{{kɒɻtʰ-ɒ-t͡sʰ-ekʰ}}] `read.{\pl}' is to prevent ambiguity with the present subjunctive 2PL [{{kɒɻtʰ-ɒ-kʰ}}] `if you.{\pl} read'. Analyzing the use of morphomic aorist as due to contrast-preservation is attractive. However, it would not extend to other paradigm cells for the A-Class like the subject participle, which also uses the morphomic aorist [{{kɒɻtʰ-ɒ-t͡sʰ-oʁ}} ] `reader’ without any contrasting form [{{*kɒɻtʰ-oʁ}}].}}
	For the E-Class, more prescriptive uses of {\seaSEA} utilize the aorist stem for the E-Class imperative 2PL as well \citep[272]{DumTragut-2009-ArmenianReferenceGrammar}. But it has become increasingly common to abandon the aorist stem for the E-Class imperative 2PL in {\seaSEA}.
	
	The prohibitive is formed by simply adding the proclitic \textit{{mi}} before the imperative form: \textit{mi kɒɻtʰ-ɒ} `don't read!' (Table \ref{tab: proh}). 
	
	
	
\begin{table}
\caption{Paradigm of prohibitives in {\seaSE} and {\iaIA}}
\label{tab: proh}
\resizebox{\textwidth}{!}{%
	\begin{tabular}{l l l l l}
		\lsptoprule
		& \multicolumn{2}{l}{E-Class} & \multicolumn{2}{l}{A-Class}\\\cmidrule(lr){2-3}\cmidrule(lr){4-5}
		& {\seaAbbre} & {\iaAbbre}& {\seaAbbre} & {\iaAbbre} \\\midrule
		Infinitive & {jeɾkʰ-e-l} & {jeɻkʰ-e-l} & {{kɑrtʰ-ɑ-l}}& {{kɒrtʰ-ɒ-l}}\\
		&\multicolumn{2}{l}{$\sqrt{~}$-{\thgloss}-{\infgloss}}&\multicolumn{2}{l}{$\sqrt{~}$-{\thgloss}-{\infgloss}}
		\\
		& `to sing’
		& `to sing’
		& `to read’
		& `to read’
		\\
		& \armenian{երգել}
		& \armenian{երգել}
		&\armenian{կարդալ}
		&\armenian{կարդալ}
		\\
		\addlinespace 
		Prohibitive 2SG
		&
		{{mi jeɾkʰ-$\emptyset$-iɾ}}
		&{{mi jeɻkʰ-$\emptyset$-i}}
		&
		{{mi kɑɾtʰ-ɑ-$\emptyset$}}
		&{{mi kɒɻtʰ-ɒ-$\emptyset$}}
		\\
		~~Colloquial &
		{{mi jeɾkʰ-$\emptyset$-i}}
		&&&
		\\
		&\multicolumn{2}{l}{{\proh} $\sqrt{~}$-{\thgloss}-{\imp}.2{\sg}}&\multicolumn{2}{l}{{\proh} $\sqrt{~}$-{\thgloss}-{\imp}.2{\sg}}
		\\
		& \armenian{մի երգիր}
		& \armenian{մի երգի}
		&\armenian{մի կարդա}
		&\armenian{մի կարդա}
		\\\addlinespace
		Prohibitive 2PL
		&
		{{mi jeɾkʰ-$\emptyset$-ekʰ}
		}&
		{{mi jeɻkʰ-$\emptyset$-ekʰ}}
		&
		{{mi kɑɾtʰ-ɑ-t͡sʰ-ekʰ}
		}&
		{{mi kɒɻtʰ-ɒ-t͡sʰ-ekʰ}
		}
		\\
		&\multicolumn{2}{l}{{\proh} $\sqrt{~}$-{\thgloss}-{\imp}.2{\pl}}&\multicolumn{2}{l}{{\proh} $\sqrt{~}$-{\thgloss}-{\aorother}-{\imp}.2{\pl}}
		
		\\
		&\armenian{մի երգեք}
		& \armenian{մի երգէք}
		&\armenian{մի կարդացեք}
		&\armenian{մի կարդացէք}
		\\ \lspbottomrule 
	\end{tabular}}
\end{table} 
	
	
	
	
	For illustration, the verbs below show the imperative and prohibitive form of various verbs that we had elicited over the years (Table \ref{tab: proh:ex}). We omit zero morphs for space.\largerpage
	
	
	\begin{table}
		\caption{Elicited imperatives and prohibitives}\label{tab: proh:ex}
		
		\resizebox{\textwidth}{!}{%	
			\begin{tabular}{lllll l}
				\lsptoprule
				Infinitive & && Finite form & Quality & \\ \midrule
				{nəst-e-l} & `to sit' & \armenian{նստել} & {nəst-i}& Imp 2SG& \armenian{նստի} % (AS)
				\\
				{kʰən-e-l}& `to sleep'& \armenian{քնել} & {kʰən-i} &Imp 2SG&\armenian{քնի} % (AS)
				\\
				{ɡəɻ-e-l} &`to write' & \armenian{գրել} & {ɡəɻ-i} & Imp 2SG & \armenian{գրի} % (AS, NK)
				\\
				& & &{mi ɡəɻ-i} & Proh 2SG & \armenian{մի գրի}
				\\
				& & & {mi ɡəɻ-ekʰ} &Proh 2PL &\armenian{մի գրէք} %(KM)
				\\
				
				
				{bərn-e-l} & `to hold/catch' &\armenian{բռնել}&{mi bərn-i} & Proh 2SG& \armenian{մի բռնի}
				\\
				& & 
				& {mi bərn-ekʰ} & Proh 2PL & \armenian{մի բռնէք} % (KM)
				\\
				{t͡səχ-e-l} & `to smoke'&\armenian{ծխել}& {mi t͡səχ-i} & Proh 2SG & \armenian{մի ծխի}
				\\
				& & & {mi t͡səχ-ekʰ} &Proh 2PL &\armenian{մի ծխէք} % (KM)
				\\
				{χɒʁ-ɒ-l} & `to play' &\armenian{խաղալ}&{mi χɒʁ-ɒ} & Proh 2SG & \armenian{մի խաղա}
				\\
				& & & {mi χɒʁ-ɒ-t͡sʰ-ekʰ} &Proh 2PL & \armenian{մի խաղացէք} %(KM)
				\\
				{mən-ɒ-l} & `to remain'&\armenian{մնալ}& {mi mən-ɒ}& Proh 2SG & \armenian{մի մնա}
				\\
				& & & {mi mən-ɒ-t͡sʰ-ekʰ} &Proh 2PL &\armenian{մի մնացէք}% (KM)
				\\
				% {ɡən-ɒ-l } &&& {mi ɡən-ɒ} & {mi ɡən-ɒ-t͡sʰ-ekʰ} & `to go'& % (KM)
				% \\
				{ʒəpt-ɒ-l} & `to smile'&\armenian{ժպտալ}& {mi ʒəpt-ɒ} & Proh 2SG & \armenian{մի ժպտա}
				\\
				& & & {mi ʒəpt-ɒ-t͡sʰ-ekʰ} &Proh 2PL &\armenian{մի ժպտացէք} % (NK)
				\\ \lspbottomrule
			\end{tabular}}
	\end{table}
	
	One thing to note though is that our {\iaIA} speakers frequently prefer to use the negative subjunctive present 2PL in lieu of the prohibitive 2PL (Table \ref{tab:Verb:Synthn:Imp:proh subj}). We suspect this is an influence from Persian. AS reports that Persian often utilizes the subjunctive 2PL in lieu of the negative imperative 2PL. Note how for the E-Class, the surface sequence \textit{{-ekʰ}} has different morphological parses in the subjunctive vs. the prohibitive. 
	
	
\begin{table}
\caption{Negative subjunctive vs. prohibitive 2PL in {\iaIA}}\label{tab:Verb:Synthn:Imp:proh subj}
\begin{tabular}{lll}
	\lsptoprule
	& Prohibitive 2PL&Negative sbjv. 2PL \\\midrule
	E-Class `to sing'& {{mi jeɻkʰ-$\emptyset$-ekʰ}} & {{t͡ʃə-jeɻkʰ-e-kʰ}}\\
	& {\proh} $\sqrt{~}$-{\thgloss}-{\imp}.2{\pl} & {\neggloss}-$\sqrt{~}$-{\thgloss}-2{\pl}\\
	& \armenian{մի երգէք} & \armenian{չերգէք}\\\addlinespace 
	A-Class `to read'& {{mi kɒɻtʰ-ɒ-t͡sʰ-ekʰ}} & {{t͡ʃə-kɒɻtʰ-ɒ-kʰ}}\\
	& {\proh} $\sqrt{~}$-{\thgloss}-{\aorother}-{\imp}.2{\pl} & {\neggloss}-$\sqrt{~}$-{\thgloss}-2{\pl}\\
	& \armenian{մի կարդացէք} & \armenian{չկարդաք}\\
	\lspbottomrule
\end{tabular}
\end{table}
	
	
	
	

\subsection{Participles}\label{section:verb:synthesis:part}

Alongside converbs, {\iaIA} utilizes a set of participles derived from verbs. These participles cannot be used in periphrastic constructions. They are restricted to use as adjectives or nouns. Participle formation in {\iaIA} is identical to that in {\seaSE}.

There are two types of participles: the subject participle and the resultative participle (Table \ref{tab:participles}). The subject participle uses the suffix [{{-oʁ}}]. The resultative participle uses the suffix [{{-ɒt͡sʰ}}] in {\iaIA}, [{{-ɑt͡s}}] in {\seaSE}.\footnote{As explained in \S\ref{section:phono:segmental:laryngeal cons}, some {\iaIA} speakers aspirate the resultative suffix as [{{-ɒt͡sʰ}}], while some do not. Throughout this section, we aspirate this suffix because our main consultant NK used aspiration. } For the E-Class, these suffixes are added directly after the root, deleting the theme vowel. We use zero morphs to show the deleted theme vowel. For A-Class verbs, these suffixes trigger a morphomic aorist suffix \textit{{-t͡sʰ-}} between the theme and suffix, i.e., an aorist stem. 

\begin{table}
	\caption{Paradigm of subject and resultative participles}
	\label{tab:participles}
	\resizebox{\textwidth}{!}{%	
		\begin{tabular}{lllll}
			\lsptoprule
			&\multicolumn{2}{l}{E-Class} &\multicolumn{2}{l}{A-Class}\\\cmidrule(lr){2-3}\cmidrule(lr){4-5}
			&{\seaAbbre} & {\iaAbbre} &{\seaAbbre} 	& {\iaAbbre}\\\midrule
			Infinitive
			&
			{jeɾkʰ-e-l}
			&
			{jeɻkʰ-e-l}
			&
			{kɑɾtʰ-ɑ-l}
			&
			{kɒɻtʰ-ɒ-l}
			\\
			&`to sing'&&`to read' & 
			\\
			
			&\multicolumn{2}{l}{$\sqrt{~}$-{\thgloss}-{\infgloss}}
			&\multicolumn{2}{l}{$\sqrt{~}$-{\thgloss}-{\infgloss}}
			\\
			& \armenian{երգել} & & \armenian{կարդալ} & 
			\\\addlinespace 
			Subject participle
			&
			{jeɾkʰ-$\emptyset$-oʁ}
			&
			{jeɻkʰ-$\emptyset$-oʁ}
			&
			{kɑɾtʰ-ɑ-t͡sʰ-oʁ}
			&
			{kɒɻtʰ-ɒ-t͡sʰ-oʁ}
			\\
			&\multicolumn{2}{l}{$\sqrt{~}$-{\thgloss}-{\sptcp}}
			&\multicolumn{2}{l}{$\sqrt{~}$-{\thgloss}-{\aorother}-{\sptcp}}
			\\
			& \armenian{երգող} & & \armenian{կարդացող} & 
			
			\\\addlinespace
			Resultative participle
			&
			{jeɾkʰ-$\emptyset$-ɑt͡s}
			&
			{jeɻkʰ-$\emptyset$-ɒt͡sʰ}
			&
			{kɑɾtʰ-ɑ-t͡sʰ-ɑt͡s}
			&
			{kɒɻtʰ-ɒ-t͡sʰ-ɒt͡sʰ}
			\\
			&\multicolumn{2}{l}{$\sqrt{~}$-{\thgloss}-{\rptcp}}
			&\multicolumn{2}{l}{$\sqrt{~}$-{\thgloss}-{\aorother}-{\rptcp}}
			
			\\
			& \armenian{երգած} & & \armenian{կարդացած} & 
			
			\\
			\lspbottomrule
		\end{tabular}}
\end{table}
	
	
	In {\seaSEA}, the resultative participle is called [hɑɾɑkɑtɑɾ deɾbɑj] \armenian{հարակատար դերբայ}, and the subject participle is called [jentʰɑkɑjɑkɑn deɾbɑj] \armenian{ենթակայական դերբայ}. 
	
	The following are examples with these participles in {\iaIA} (\ref {sent:participles:ex}).
	
	
	\begin{exe}
		\ex \label{sent:participles:ex}
		\begin{xlist}
			\ex Subject participle
			
			\gll {jeɻkʰ-oʁ-ə} jev {kɒɻtʰ-ɒ-t͡sʰ-oʁ-ə}
			\\
			sing-{\sptcp}-{\defgloss} and 	read-{\thgloss}-{\aorother}-{\sptcp}-{\defgloss}
			\\
			\trans `the singer and the reader'
			\\
			\armenian{երգողը եւ կարդացողը}
			
			\pagebreak\ex Resultative participle
			
			\gll {jeɻkʰ-ɒt͡sʰ} {jeɻkʰ } jev {kɒɻtʰ-ɒ-t͡sʰ-ɒt͡sʰ} {ɡiɻkʰ} % NK)
			\\
			sing-{\rptcp} song and read-{\thgloss}-{\aorother}-{\rptcp} book
			\\
			\trans 		`a sung song and a read book'
			\\
			\armenian{երգած երգ եւ կարդացած գիրք}
		\end{xlist}
		
	\end{exe}

	\section{Future: Synthetic and periphrastic constructions}\label{section:verb:fut}
This section discusses the two morphological strategies that are used to mark the future. One strategy is periphrastic with a converb, while the other is synthetic with a prefix. The same strategies are used in both {\seaSEA} and {\iaIA}. 

The existing literature on Armenian is quite inconsistent in how these two categories are classified and analyzed. To minimize these inconsistencies, we discuss them both together here. 

\subsection{Variation in future marking}\label{section:verb:fut:vary}
To mark the simple future in {\seaSEA}, most traditional grammars (both descriptive and pedagogical) report a periphrastic construction (\ref{ex:Verb:Fut:periph}). \citet[233]{DumTragut-2009-ArmenianReferenceGrammar} labels this as the “simple future”. The verb is in a non-finite form called the future converb (with suffix \textit{-u}) while tense-agreement is on an auxiliary. {\iaIA} has the same periphrastic construction. 

\begin{exe}
	\ex Periphrastic future \\
	\glll ɡəɾ-e-l-u e-m ({\seaAbbre}) \\
	ɡəɻ-e-l-uw e-m ({\iaAbbre}) \\
	write-{\thgloss}-{\infgloss}-{\futcvb} {\auxgloss}-1{\sg} \\
	\trans `I will write.' \label{ex:Verb:Fut:periph}\\
	\armenian{Գրելու եմ։}
\end{exe}

An alternative synthetic construction is to add the prefix \textit{k(ə)-} before a finite subjunctive verb (\ref{ex:Verb:Fut:synth}). \citet[253]{DumTragut-2009-ArmenianReferenceGrammar} calls this the “conditional future”.


\begin{exe}
	\ex Synthetic future \\
	\glll kə-ɡəɾ-e-m ({\seaAbbre}) \\
	kə-ɡəɻ-e-m ({\iaAbbre}) \\
	{\fut}-write-{\thgloss}-1{\sg} \\
	\trans `I will write.' \label{ex:Verb:Fut:synth}\\
	\armenian{Կը գրեմ։}
\end{exe}


Note that our translations for the periphrastic future (\ref{ex:Verb:Fut:periph}) and synthetic future (\ref{ex:Verb:Fut:synth}) are identical. The problem is that it is quite unclear what are the fixed semantic and functional differences between the periphrastic and synthetic future.\footnote{Sometimes NK would say that the periphrastic construction means `I will X' while the synthetic one means `I am going to X'. But then we get the opposite order from AS's consultants. } To quote \citet[253]{DumTragut-2009-ArmenianReferenceGrammar}:

\begin{quote}
	In [{\seaAbbre}], however, [the synthetic future] is more often used to express simple actions in the future and as such has no major semantic differences to the [periphrastic future] and is even more often used [than] the [periphrastic future]. 
\end{quote}

There are some subtle semantic distinctions between the periphrastic and synthetic forms. For example, the synthetic form implies a stronger sense of intentionality or volition. For our consultants, it can denote a wish, a future condition, or an optative. It can be used to denote an action in the immediate future, where the agent has a strong desire to perform the action. The synthetic future has a sense of being more temporally immediate than the periphrastic future. But in general, the two types of futures can be used interchangeably. 


The above semantic observations concerning the future contrast strongly with  the traditional names that grammars use. The periphrastic future is always labeled as “the future” (\citealt[182]{Minassian-1980-EastArmenianGrammar}, \citealt[209]{fairbanksStevick-1975-spokenEastArmenian}, \citealt[71]{BardakjianVaux-1999-easternArmeniantextbook}, \citealt[94]{Hagopian-2007-ArmenianTextbookEveryone}, \citealt[124]{Sakayan-2007-TextbookEasternArmenian}, \citealt[233]{DumTragut-2009-ArmenianReferenceGrammar}). This shows that these grammarians think that the main function of this construction is to mark the future. In contrast, the synthetic form has multiple names, each of which make the synthetic form seem subordinate to the periphrastic form. It has been called the “conditional present” (\citealt[192]{Minassian-1980-EastArmenianGrammar}, \citealt[160]{Hagopian-2007-ArmenianTextbookEveryone}), “hypothetical future” (\citealt[224]{Sakayan-2007-TextbookEasternArmenian}), “future” (\citealt[85]{Johnson-1954-EastArmGrammar}, \citealt[93]{fairbanksStevick-1975-spokenEastArmenian}), and “conditional future” (\citealt[196]{BardakjianVaux-1999-easternArmeniantextbook}, \citealt[253]{DumTragut-2009-ArmenianReferenceGrammar}). In contrast, in Armenian dialectology, Adjarian (\citealt{Adjarian-1911-DialectologyBook}, translated in \citealt{Dolatian-prep-Adjarian}) labels the synthetic future as just the future.\footnote{Among modern grammars written in Armenian, there is also some inconsistency. The periphrastic future has been called the “future” [ɑpɑrni] \armenian{ապառնի} (\citealt[292]{Ezekyan-2007-Armenian}) or the “future present” [ɑpɑrni neɾkɑ] \armenian{ապառնի ներկա} (\citealt[295]{Sevak-2009-Coursebook}). In contrast,  the synthetic is called the “conditional future present” or “(conditional) future” (\citealt[292]{Ezekyan-2007-Armenian}, \citealt[295]{Sevak-2009-Coursebook}). The word for “conditional” can be [jentʰɑdɾɑkɑn] \armenian{ենթադրական} or [pɑjmɑnɑkɑn] \armenian{պայմանական}.}


There is thus a  mismatch between the names and functions of the two future constructions. Traditional grammars and names treat the periphrastic future as the default, while the synthetic future is argued to be restricted to special types of conditional clauses. However, more recent semantic work on Armenian argues that the synthetic future is the default way to mark the future tense \citep{avetyan-2022-Future}. The periphrastic future is instead an expected (predetermined) future. Personal communication with Avetyan   \citep{avetyan-2022-Future} then suggests the following two translations for these two types of futures (\ref{ex:Verb:Fut:avetyanNames}). 

\begin{exe}
	\ex Alternative translations \label{ex:Verb:Fut:avetyanNames}\begin{xlist}
		
		\ex Periphrastic future (expected) \\
		\glll ɡəɾ-e-l-u e-m ({\seaAbbre}) \\
		ɡəɻ-e-l-uw e-m ({\iaAbbre}) \\
		write-{\thgloss}-{\infgloss}-{\futcvb} {\auxgloss}-1{\sg} \\
		\trans `I am (going) to write.'\\
		\armenian{Գրելու եմ։}
		\ex Synthetic future (simple) \\
		\glll kə-ɡəɾ-e-m ({\seaAbbre}) \\
		kə-ɡəɻ-e-m ({\iaAbbre}) \\
		{\fut}-write-{\thgloss}-1{\sg} \\
		\trans `I will write.' \\
		\armenian{Կը գրեմ։}
	\end{xlist}
\end{exe}

As can be seen, it is difficult to know how to label these two morphological constructions. The traditional names obfuscate the fact that the synthetic structure is more common than the periphrastic structure, and that the synthetic can be used in non-conditional contexts. But, if we use  new names based on semantic functions, then we run the risk that future more in-depth work may contradict our grammar. With new semantically-based names, a future reader might also have trouble seeing the connection between our grammar and past grammars. 

As a compromise, we use morphological names for the two types of futures: the periphrastic future and the synthetic future \citep{fairbanksStevick-1975-spokenEastArmenian}. We gloss the morpheme /-u/ for periphrastic future as -{\futcvb} `future converb', and the prefix /k-/ for the synthetic future as {\fut}- `future'. 

The rest of this section discusses  in more detail   the morphology of these two constructions. More specifically, we discuss their   past forms  and their negation. 

\subsection{Periphrastic future with a converb}
\label{section:verb:fut:peri}
The periphrastic future is made by combining the future converb with an inflected auxiliary. The future converb is formed by taking the infinitive  and then adding the suffix \textit{{-u}} (Table \ref{tab:Verb:Periph:FutureRule}). Both the E-Class and A-Class keep their theme vowel. This construction is formed identically in {\seaSE} and {\iaIA}. The future converb is also called the future participle   [ɑpɑrni deɾbɑj] \armenian{ապառնի դերբայ}  in {\seaSEA} \citep[206]{DumTragut-2009-ArmenianReferenceGrammar}.


\begin{table}
	\caption{Forming the future converb for simple regular verbs}\label{tab:Verb:Periph:FutureRule}
	\begin{tabular}{llll}
		\lsptoprule 
		\multicolumn{2}{c}{E-Class `to sing'}& \multicolumn{2}{c}{A-Class `to read'}\\\cmidrule(lr){1-2}\cmidrule(lr){3-4}
		Infinitive & Future converb & Infinitive & Future converb\\\midrule
		{jeɻkʰ-e-l}&{jeɻkʰ-e-l-u} & {{kɒɻtʰ-ɒ-l}}&{{kɒɻtʰ-ɒ-l-u}} \\
		$\sqrt{~}$-{\thgloss}-{\infgloss}&$\sqrt{~}$-{\thgloss}-{\infgloss}-{\futcvb}&$\sqrt{~}$-{\thgloss}-{\infgloss}&$\sqrt{~}$-{\thgloss}-{\infgloss}-{\futcvb}
		\\
		\armenian{երգել} & \armenian{երգելու} &\armenian{կարդալ} &\armenian{կարդալու}\\ 
		\lspbottomrule
	\end{tabular}
\end{table}

The future converb suffix \textit{{-u}} likely originates from the   genitive/dative suffix \textit{-u} that is used by some declension classes (traditionally called the second declension). Its use is grammaticalized here as part of the future converb. % We provide rules below. 

%	\begin{exe}
	%	\ex 
	%	\begin{xlist}
		%			\ex \textit{Rule for the future converb}
		
		%			\begin{tabular}{llll}
			% {\futcvb}&$\leftrightarrow$ & {{-u}} 
			%			\end{tabular}
		%		\ex \textit{Structure of the future converb of E-Class} jeɻkʰ-e-l
		%		
		%%		\begin{tikzpicture} 
			%			\Tree 
			% 	[.Cvb} [.{\tense} [.\textit{v} [ [.$\sqrt{~}$ jeɻkʰ_E ] ] [.\textit{v} [.\textit{v} -$\emptyset$ ] [.{\thgloss} {-e} ] ] ] [ [ [.{\tense} {-l} ] ] ] ] [ [ [ [.\textsc{cvb} {-u} ] ] ] ] ]
		%		\end{tikzpicture}
	%%	\end{xlist}
%	\end{exe}

The converb can take the present or past auxiliaries to respectively create the simple future or the past future (“future in the past”, \citealt[235]{DumTragut-2009-ArmenianReferenceGrammar}). We show in Table \ref{tab:fut converb periphrasis} the complete paradigm for the E-Class \textit{jeɻkʰ-e-l}. The paradigm for the A-Class is analogously constructed with the converb [{{kɒɻtʰ-ɒ-l-u}}]. We do not segment the auxiliary. 


\begin{table}
\caption{Paradigm for the periphrastic future and the periphrastic past future for E-Class [{jeɻkʰ-e-l}] `to sing'}
\label{tab:fut converb periphrasis}
\resizebox{\textwidth}{!}{%
	\begin{tabular}{l llll llll}
		\lsptoprule 
		& \multicolumn{4}{c}{Positive} & \multicolumn{4}{c}{Negaive} \\ \cmidrule(lr){2-5}\cmidrule(lr){6-9}
		& \multicolumn{2}{l}{Future}& \multicolumn{2}{l}{Past future} & \multicolumn{2}{l}{Future}& \multicolumn{2}{l}{Past future}\\\midrule
		1SG
		&
		{jeɻkʰ-e-l-uw}&{em}
		&
		{jeɻkʰe-l-uw } &{im}
		&
		{t͡ʃʰ-em } &{jeɻkʰ-e-l-u}
		&
		{t͡ʃʰ-im} & {jeɻkʰe-l-u}
		\\
		&
		\multicolumn{2}{l}{`I will sing'}&
		\multicolumn{2}{l}{`I was going to sing'}&
		\multicolumn{2}{l}{`I will not sing'}&
		\multicolumn{2}{l}{`I wasn't going to sing'}
		\\
		& \armenian{երգելու}& \armenian{եմ}
		& \armenian{երգելու}& \armenian{իմ}
		&\armenian{չեմ}& \armenian{երգելու}
		& \armenian{չիմ}& \armenian{երգելու}
		\\
		\addlinespace	2SG
		&
		{jeɻkʰ-e-l-uw} &{es}
		&
		{jeɻkʰ-e-l-uw} & {iɻ}
		&
		{t͡ʃʰ-es} & {jeɻkʰ-e-l-u}
		&
		{{t͡ʃʰ-iɻ}} & {jeɻkʰ-e-l-u}
		\\
		& \armenian{երգելու}& \armenian{ես}
		& \armenian{երգելու}& \armenian{իր}
		&\armenian{չես}& \armenian{երգելու} 
		& \armenian{չիր}& \armenian{երգելու} 
		\\
		\addlinespace		3SG
		&
		{jeɻkʰ-e-l-uw} & {ɒ}
		&
		{jeɻkʰ-e-l-uw} & {eɻ}
		&
		{t͡ʃʰ-i} & {jeɻkʰ-e-l-u}
		&
		{t͡ʃʰ-eɻ} & {jeɻkʰ-e-l-u}
		\\
		& \armenian{երգելու}& \armenian{ա}
		& \armenian{երգելու}& \armenian{էր}
		&\armenian{չի}& \armenian{երգելու }
		& \armenian{չէր}& \armenian{երգելու} 
		\\
		\addlinespace		1PL
		&
		{jeɻkʰ-e-l-uw} & {eŋkʰ}
		&
		{jeɻkʰ-e-l-uw} & {iŋkʰ}
		&
		{t͡ʃʰ-eŋkʰ} & {jeɻkʰ-e-l-u}
		&
		{t͡ʃʰ-iŋkʰ} & {jeɻkʰ-e-l-u}
		\\
		& \armenian{երգելու}& \armenian{ենք}
		& \armenian{երգելու}& \armenian{ինք}
		&\armenian{չենք}& \armenian{երգելու }
		& \armenian{չինք}& \armenian{երգելու}
		\\
		\addlinespace	2PL
		&
		{jeɻkʰ-e-l-uw} & {ekʰ}
		&
		{jeɻkʰ-e-l-uw} & {ikʰ}
		&
		{t͡ʃʰ-ekʰ} & {jeɻkʰ-e-l-u}
		&
		{t͡ʃʰ-ikʰ} & {jeɻkʰ-e-l-u}
		\\
		& \armenian{երգելու}& \armenian{էք}
		& \armenian{երգելու}& \armenian{իք}
		&\armenian{չէք}& \armenian{երգելու} 
		& \armenian{չիք}& \armenian{երգելու} 
		\\
		\addlinespace	3PL
		&
		{jeɻkʰ-e-l-uw}& {en}
		&
		{jeɻkʰ-e-l-uw} & {in}
		&
		{t͡ʃʰ-en} & {jeɻkʰ-e-l-u}
		&
		{t͡ʃʰ-in} & {jeɻkʰ-e-l-u}
		\\
		& \armenian{երգելու}& \armenian{են}
		& \armenian{երգելու}& \armenian{ին}
		&\armenian{չեն}& \armenian{երգելու }
		& \armenian{չին}& \armenian{երգելու}
		\\ \addlinespace 
		&\multicolumn{4}{l}{$\sqrt{~}$-{\thgloss}-{\infgloss}-{\futcvb} {\auxgloss}}
		&\multicolumn{4}{l}{{\neggloss}-{\auxgloss}
			$\sqrt{~}$-{\thgloss}-{\infgloss}-{\futcvb} } 
		\\ \lspbottomrule
	\end{tabular}}
\end{table}

  Vowel hiatus between the converb and the auxiliary triggers the insertion of [{w}], discussed in \S\ref{section:morphophono:morphophono:vowel hiatus}.


When the converb is combined with the past auxiliary, the usual name for this construction is the “past future” or “future in the past” (\citealt[182]{Minassian-1980-EastArmenianGrammar}, \citealt[71]{BardakjianVaux-1999-easternArmeniantextbook}, \citealt[94]{Hagopian-2007-ArmenianTextbookEveryone}, \citealt[235]{DumTragut-2009-ArmenianReferenceGrammar}). Other names include the “future imperfect” (\citealt[126]{Sakayan-2007-TextbookEasternArmenian}) and “past future” (\citealt[210]{fairbanksStevick-1975-spokenEastArmenian}). 


As before, the auxiliary shifts its position in the negated form.    

\subsection{Synthetic future with a prefix}
\label{section:verb:fut:synn}

The synthetic future is derived from subjunctives via prefixation in the positive. But its negative form uses periphrasis with a converb called the connegative (\ref{sent:Verb:Synth:Cond:ex}).


\begin{exe}
\ex \label{sent:Verb:Synth:Cond:ex}
\begin{xlist}
	\ex \gll {kə-} {kɒɻtʰ} {-ɒ} {-ŋkʰ}
	\\
	{\fut}- read-{\thgloss} -1{\pl}
	\\
	\trans 			`We will read.' \hfill (NK)
	\\
	\armenian{Կը կարդանք։}
	\ex \gll {t͡ʃʰ-} {e} {-ŋkʰ} ~ {kɒɻtʰ}{-ɒ}-$\emptyset$
	\\
	{\neggloss}- is -1{\pl} ~ read-{\thgloss}-{\cncvb}
	\\
	\trans 			`We will not read.' \hfill (NK)
	\\
	\armenian{Չենք կարդայ։}
	
\end{xlist}
\end{exe}






In the positive, these synthetic forms are created by adding the prefix \textit{{k-}} to the subjunctive form (\S\ref{section:verb:synthesis:subj}). A schwa is added to repair any consonant clusters created by this prefix. Complications arise when the root starts with [je] (\S\ref{section:morphophono:morphophono:root initial glide}).


When the prefix is added to a subjunctive present verb, it produces a future meaning, but with various nuances (\S\ref{section:verb:fut:vary}). When this prefix is added to a subjunctive past verb, the meaning is more conditional-oriented. Grammars give many divergent names for this construction: conditional past (\citealt[160]{Hagopian-2007-ArmenianTextbookEveryone}, \citealt[260]{DumTragut-2009-ArmenianReferenceGrammar}), conditional imperfect (\citealt[192]{Minassian-1980-EastArmenianGrammar}, \citealt[196]{BardakjianVaux-1999-easternArmeniantextbook}), hypothetical past \citep[225]{Sakayan-2007-TextbookEasternArmenian}, past future \citep[132]{fairbanksStevick-1975-spokenEastArmenian}. Because NK   translates this construction as `I would X', we decided to call it the conditional past. 

Table \ref{tab:cond positive} shows the paradigm of the synthetic future and of the conditional past. We do not provide the {\seaSEA} forms because {\seaSEA} likewise builds this tense from the subjunctive.

\begin{table}[p]
\caption{Paradigm of positive synthetic future and the conditional past in {\iaIA}\label{tab:cond positive}}
\resizebox{\textwidth}{!}{%
	\begin{tabular}{lllll}
		\lsptoprule 
		& \multicolumn{2}{c}{Future} & \multicolumn{2}{c}{Conditional past}\\\cmidrule(lr){2-3}\cmidrule(lr){4-5}
		&		E-Class&A-Class&E-Class&A-Class\\\midrule
		1SG
		& {{kə-jeɻkʰ-e-m}} & {{kə-kɒɻtʰ-ɒ-m}}
		
		& {{kə-jeɻkʰ-$\emptyset$-i-m}} & {{kə-kɒɻtʰ-ɒj-i-m}}
		\\
		& `I will sing.'& `I will read.' & `I would sing' & `I would read.'
		\\
		& \armenian{կը երգեմ}
		& \armenian{կը կարդամ}
		& \armenian{կը երգիմ}
		& \armenian{կը կարդայիմ}
		\\
		\addlinespace	2SG
		& {{kə-jeɻkʰ-e-s}} & {{kə-kɒɻtʰ-ɒ-s}}
		& {{kə-jeɻkʰ-$\emptyset$-i-ɻ}} & {{kə-kɒɻtʰ-ɒj-i-ɻ}}
		\\
		& \armenian{կը երգես}
		& \armenian{կը կարդաս}
		& \armenian{կը երգիր}
		& \armenian{կը կարդայիր}
		\\\addlinespace		3SG
		& {{kə-jeɻkʰ-i-$\emptyset$}} & {{kə-kɒɻtʰ-ɒ}}
		& {{kə-jeɻkʰ-e-$\emptyset$-ɻ}} & {{kə-kɒɻtʰ-ɒ-$\emptyset$-ɻ}}
		\\
		& \armenian{կը երգի}
		& \armenian{կը կարդայ}
		& \armenian{կը երգէր}
		& \armenian{կը կարդար}
		\\\addlinespace		1PL
		& {{kə-jeɻkʰ-e-ŋkʰ}} & {{kə-kɒɻtʰ-ɒ-ŋkʰ}}
		& {{kə-jeɻkʰ-$\emptyset$-i-ŋkʰ}} & {{kə-kɒɻtʰ-ɒj-i-ŋkʰ}}
		\\
		& \armenian{կը երգենք}
		& \armenian{կը կարդանք}
		& \armenian{կը երգինք}
		& \armenian{կը կարդայինք}
		\\\addlinespace		2PL
		& {{kə-jeɻkʰ-e-kʰ}} & {{kə-kɒɻtʰ-ɒ-kʰ}}
		& {{kə-jeɻkʰ-$\emptyset$-i-kʰ}} & {{kə-kɒɻtʰ-ɒj-i-kʰ}}
		\\
		& \armenian{կը երգեք}
		& \armenian{կը կարդաք}
		& \armenian{կը երգիք}
		& \armenian{կը կարդայիք}
		\\
		\addlinespace		3PL
		& {{kə-jeɻkʰ-e-n}} & {{kə-kɒɻtʰ-ɒ-n}}
		& {{kə-jeɻkʰ-$\emptyset$-i-n}} & {{kə-kɒɻtʰ-ɒj-i-n}}
		\\
		& \armenian{կը երգեն}
		& \armenian{կը կարդան}
		& \armenian{կը երգին}
		& \armenian{կը կարդային}
		\\	\addlinespace
		&\multicolumn{2}{l}{{\fut}-$\sqrt{~}$-{\agr}}
		&\multicolumn{2}{l}{{\fut}-$\sqrt{~}$-{\pst}-{\agr}}
		\\\lspbottomrule
	\end{tabular}}
\end{table}

\begin{table}[p]
\caption{Connegative converbs for the E-Class and A-Class}\label{tab:Verb:Sythn:Cond:connegative}
\begin{tabular}{llll}
	
	\lsptoprule
	 &E-Class & A-Class & \\
	& `to sing' & `to read' & 
	\\ 
	\midrule 
	Infinitive& {jeɻkʰ-e-l}& {{kɒɻtʰ-ɒ-l}} & $\sqrt{~}$-{\thgloss}-{\infgloss} 
	\\
	& \armenian{երգել}& \armenian{կարդալ}& \\
	\addlinespace 
	Connegative& {{jeɻkʰ-i}}& {{kɒɻtʰ-ɒ}} & 
	\\
	~~ Possible analysis: &
	{{jeɻkʰ-i-$\emptyset$}}& {{kɒɻtʰ-ɒ-$\emptyset$}} & $\sqrt{~}$-{\thgloss}-{\cncvb} 
	\\
	& \armenian{երգի}& \armenian{կարդայ} & 
	\\
	\lspbottomrule
\end{tabular}
\end{table}



The above focused on the synthetic future and conditional when the verb is positive. When the verb is negative, then     an entirely different periphrastic construction is used. Tense and agreement are placed on a negative auxiliary (\S\ref{section:verb:aux}). The verb is in the connegative form (Table \ref{tab:Verb:Sythn:Cond:connegative}), also called the negative participle \citep[214]{DumTragut-2009-ArmenianReferenceGrammar}. The converb is called [ʒəχtɑkɑn deɾbɑj] \armenian{ժխտական դերբայ} in {\seaSEA}. The converb is constructed differently for the two classes. The converb suffix is a zero morph in the A-Class. In the E-Class, the theme vowel is replaced by /{i}/.

In terms of segmentation, we treat the connegative converb as a zero suffix in the A-Class. In the E-Class, we assume the connegative is a floating [\textsc{+high}] feature that docks onto the /{e}/ theme vowel, thus changing /{e}/ to [{i}] (Rule \ref{rule:Verb:Sythn:Cond:connegative}). This is the same analytical strategy that we used for the subjunctive present 3SG (Rule \ref{subj 3sg i rule}). The alternatives in  Rule \ref{subj 3sg i options}  would also work. 

\begin{newruleblock}[rule:Verb:Sythn:Cond:connegative]
{Rule for the connegative converb}%%\label{rule:Verb:Sythn:Cond:connegative}

\begin{center}
	
	\begin{tabular}{lllll}
		{\cncvb} & $\leftrightarrow$ & [\textsc{+high}] & / {e} \_
		& (where /e/ is theme) \\
		& & -$\emptyset$ & / elsewhere
		
	\end{tabular}
\end{center}
\end{newruleblock} 

We show the negative paradigm in Table \ref{tab:neg conditional}. Note that because we are defining the future constructions in terms of their morphology, then the negative paradigm is actually “the negative periphrastic of the synthetic future”. 

\begin{table}
\caption{Paradigm of the negative periphrastic form of the synthetic future and of the conditional past in {\iaIA}}
\label{tab:neg conditional}
\resizebox{\textwidth}{!}{%
	\begin{tabular}{lllllllll}
		\lsptoprule 
		& \multicolumn{4}{c}{Future} & \multicolumn{4}{c}{Conditional past}\\\cmidrule(lr){2-5}\cmidrule(lr){6-9}
		&\multicolumn{2}{l}{E-Class}&\multicolumn{2}{l}{A-Class}&\multicolumn{2}{l}{E-Class}&\multicolumn{2}{l}{A-Class}\\
		\midrule 		1SG
		& {{t͡ʃʰ-em}} & {{jeɻkʰ-i-$\emptyset$}} 
		& {{t͡ʃʰ-em}} &{{kɒɻtʰ-ɒ-$\emptyset$}} 
		& {{t͡ʃʰ-im}} &{{jeɻkʰ-i-$\emptyset$}} 
		& {{t͡ʃʰ-im}} &{{kɒɻtʰ-ɒ-$\emptyset$}} 
		\\
		& \multicolumn{2}{l}{`I will not sing'}
		& \multicolumn{2}{l}{`I will not read'}
		& \multicolumn{2}{l}{`I would not sing'}
		& \multicolumn{2}{l}{`I would not read'}
		%			\\
		%			& & 
		%			& & 
		%			& \multicolumn{2}{l}{have sing' }
		%			& \multicolumn{2}{l}{have read' }
		
		\\
		& \armenian{չեմ} & \armenian{երգի}
		& \armenian{չեմ} &\armenian{կարդայ}
		& \armenian{չիմ} & \armenian{երգի}
		& \armenian{չիմ} & \armenian{կարդայ}
		\\\addlinespace 		2SG
		& {{t͡ʃʰ-es}} & {{jeɻkʰ-i-$\emptyset$}} 
		& {{t͡ʃʰ-es}} &{{kɒɻtʰ-ɒ-$\emptyset$}} 
		& {{t͡ʃʰ-iɻ}} &{{jeɻkʰ-i-$\emptyset$}} 
		& {{t͡ʃʰ-i-ɻ}} &{{kɒɻtʰ-ɒ-$\emptyset$}} 
		\\
		& \armenian{չես} & \armenian{երգի}
		& \armenian{չես} &\armenian{կարդայ}
		& \armenian{չիր} & \armenian{երգի}
		& \armenian{չիր} & \armenian{կարդայ}
		
		\\
		\addlinespace 		3SG
		& {{t͡ʃʰ-i}} & {{jeɻkʰ-i-$\emptyset$}} 
		& {{t͡ʃʰ-i}} &{{kɒɻtʰ-ɒ-$\emptyset$}} 
		& {{t͡ʃʰ-eɻ}} &{{jeɻkʰ-i-$\emptyset$}} 
		& {{t͡ʃʰ-eɻ}} &{{kɒɻtʰ-ɒ-$\emptyset$}} 
		\\
		& \armenian{չի} & \armenian{երգի}
		& \armenian{չի} &\armenian{կարդայ}
		& \armenian{չէր} & \armenian{երգի}
		& \armenian{չէր} & \armenian{կարդայ}
		\\
		
		\addlinespace 		1PL
		& {{t͡ʃʰ-eŋkʰ}} & {{jeɻkʰ-i-$\emptyset$}} 
		& {{t͡ʃʰ-eŋkʰ}} &{{kɒɻtʰ-ɒ-$\emptyset$}} 
		& {{t͡ʃʰ-iŋkʰ}} &{{jeɻkʰ-i-$\emptyset$}} 
		& {{t͡ʃʰ-iŋkʰ}} &{{kɒɻtʰ-ɒ-$\emptyset$}} 
		\\
		& \armenian{չենք} & \armenian{երգի}
		& \armenian{չենք} &\armenian{կարդայ}
		& \armenian{չինք} & \armenian{երգի}
		& \armenian{չինք} & \armenian{կարդայ}
		
		\\
		\addlinespace 	
		2PL
		& {{t͡ʃʰ-ekʰ}} & {{jeɻkʰ-i-$\emptyset$}} 
		& {{t͡ʃʰ-ekʰ}} &{{kɒɻtʰ-ɒ-$\emptyset$}} 
		& {{t͡ʃʰ-ikʰ}} &{{jeɻkʰ-i-$\emptyset$}} 
		& {{t͡ʃʰ-ikʰ}} &{{kɒɻtʰ-ɒ-$\emptyset$}} 
		\\
		& \armenian{չէք} & \armenian{երգի}
		& \armenian{չէք} &\armenian{կարդայ}
		& \armenian{չիք} &\armenian{երգի}
		& \armenian{չիք} & \armenian{կարդայ}
		\\
		
		\addlinespace 		3PL
		& {{t͡ʃʰ-en}} & {{jeɻkʰ-i-$\emptyset$}} 
		& {{t͡ʃʰ-en}} &{{kɒɻtʰ-ɒ-$\emptyset$}} 
		& {{t͡ʃʰ-in}} &{{jeɻkʰ-i-$\emptyset$}} 
		& {{t͡ʃʰ-in}} &{{kɒɻtʰ-ɒ-$\emptyset$}} 
		\\
		& \armenian{չեն} & \armenian{երգի}
		& \armenian{չեն} &\armenian{կարդայ}
		& \armenian{չին} & \armenian{երգի}
		& \armenian{չին} & \armenian{կարդայ}
		\\
		\addlinespace 
		&\multicolumn{4}{l}{{\neggloss}-{\auxgloss}.{\prs}.{\agr} $\sqrt{~}$-{\thgloss}-{\cncvb}}
		&\multicolumn{4}{l}{{\neggloss}-{\auxgloss}.{\pst}.{\agr} $\sqrt{~}$-{\thgloss}-{\cncvb}}
		
		\\\lspbottomrule
	\end{tabular}}
\end{table} 

We do not show {\seaSEA} because it displays the exact same patterns, factoring out the phonological differences in the low vowel and rhotic, i.e., the connegative of `to read' in {\iaIA} [{{kɒɻtʰ-ɒ}}] corresponds to [{{kɑɾtʰ-ɑ}}] in {\seaSE}. We do not provide full segmentation for the auxiliary; for that see \S\ref{section:verb:aux:neg}.



\section{Complex regular verb class}\label{section:verb:complex}
The previous section provided the synthetic and periphrastic inflection of simple regular verbs. This section describes the inflection of complex verbs. Complex verbs are divided into passives, causatives, and inchoatives. These differ from simple verbs by including additional verbal material, such as the passive suffix. Their inflections differ from simple verbs in some but not all paradigm cells.

\subsection{Passives}\label{section:verb:complex:pass}
Passive verbs are formed by adding the suffix \textit{{-v-}} (Table \ref{tab:passive formation}). The suffix is added directly after the root of an E-Class verb. For an A-Class verb, the passive triggers the morphomic aorist {\textit{-t͡sʰ-}} (an aorist stem). Passive formation is the same in the two lects. We show the deleted theme vowel as a zero morph.


\begin{table}[p]
	\caption{Passive verbs in {\seaSE} and {\iaIA} }
	\label{tab:passive formation}
	\begin{tabular}{l llll}
		\lsptoprule
		&\multicolumn{2}{c}{E-Class}&\multicolumn{2}{c}{A-Class}\\\cmidrule(lr){2-3}\cmidrule(lr){4-5}
		&{\seaAbbre} &{\iaAbbre}&{\seaAbbre} &{\iaAbbre}\\ \midrule
		Infinitive 
		&
		{jeɾkʰ-e-l}
		&
		{jeɻkʰ-e-l}
		&
		{kɑɾtʰ-ɑ-l}
		&
		{{kɒɻtʰ-ɒ-l}}
		\\
		&\multicolumn{2}{l}{$\sqrt{~}$-{\thgloss}-{\infgloss}}&\multicolumn{2}{l}{$\sqrt{~}$-{\thgloss}-{\infgloss}}
		\\
		&`to sing'&&`to read' & \\
		& \armenian{երգել} &\armenian{երգել} & \armenian{կարդալ} & \armenian{կարդալ}
		\\
		
		\addlinespace 
		Passive
		&
		{{jeɾkʰ-$\emptyset$-v-e-l}}
		&
		{{jeɻkʰ-$\emptyset$-v-e-l}}
		&
		{{kɑɾtʰ-ɒ-t͡sʰ-v-e-l}}&
		{{kɒɻtʰ-ɒ-t͡sʰ-v-e-l}}
		\\
		&\multicolumn{2}{l}{$\sqrt{~}$-{\thgloss}-{\pass}-{\thgloss}-{\infgloss}}&\multicolumn{2}{l}{$\sqrt{~}$-{\thgloss}-{\aorother}-{\pass}-{\thgloss}-{\infgloss}}
		\\
		&`to be sung'&&`to be read' & \\
		& \armenian{երգվել} & \armenian{երգուել} & \armenian{կարդացվել} & \armenian{կարդացուել}
		\\
		\lspbottomrule 
	\end{tabular}
\end{table}

The name of the passive is [kəɾɑvoɾɑkɑn] \armenian{կրավորական} in {\seaSEA}. 

Semantically, the passive suffix demotes the object argument of the active verb. The passive can likewise trigger a host of other argument-reducing operations such as reflexivization, anticausativization, and so on (\citealt{haspelmath-1993-moreTypologyInchoativeCausativeVerbAltenrations}, \citealt[175]{DumTragut-2009-ArmenianReferenceGrammar}). However, there are some high-frequency intransitive verbs that have the passive suffix, like \textit{skəs-v-e-l} `to begin', but do not really have passive semantics, just intransitive semantics. For consistency, we gloss all instances of the passive suffix \textit{-v-} as just {\pass} even though its semantics can vary for some verbs. 

Morphologically, the passive takes its own theme vowel \textit{{-e-}}. We list some passives in Table \ref{tab:passive formation:ex}. 


\begin{table}[p]
	\tabcolsep=.66\tabcolsep
	\caption{Example passive verbs in {\iaIA}}\label{tab:passive formation:ex}
	\resizebox{\textwidth}{!}{%
		\begin{tabular}{ll l ll l }
			\lsptoprule 
			Active&&&Passive&& \\\midrule
			{bərn-e-l} &`to catch'& \armenian{բռնել}& {bərnə-v-e-l} & `to be caught' &\armenian{բռնուել}\\
			{kotɻ-e-l}&`to break'&\armenian{կոտրել}& {kotəɻ-v-e-l} & `to be broken' &\armenian{կոտրուել}\\
			{skəs-e-l}&`to start (trans.)'&\armenian{սկսել}& {skəs-v-e-l} & `to begin' &\armenian{սկսուել} %(KM)
			\\
			{ɒzɒt-e-l} &`to free'&\armenian{ազատել}&{ɒzɒt-v-e-l} & `to be freed' &\armenian{ազատուել}\\
			
			{ɒvɒɻt-e-l} & `to finish' &\armenian{աւարտել}& {ɒvɒɻt-v-e-l} &`to graduate (school)' & \armenian{աւարտուել} % & (AS)
			\\ \addlinespace
			{kɒɻtʰ-ɒ-l} & `to read' &\armenian{կարդալ}& {kɒɻtʰ-ɒ-t͡sʰ-v-e-l} &`to be read' & \armenian{կարդացուել} % & (AS)
			\\ \lspbottomrule
		\end{tabular}}
\end{table}



\begin{table}[p]
	\tabcolsep=.66\tabcolsep
	\caption{Past perfective of passive verbs in {\iaIA}}\label{tab:passive:pst}
	\resizebox{\textwidth}{!}{%
		\begin{tabular}{ll l ll l}
			\lsptoprule
			Active&&&Passive&& \\\midrule
			{bərnə-v-e-l} & `to be caught' &\armenian{բռնուել} & {bərnə-v-ɒ-m} & `I was caught' &\armenian{բռնուամ}\\
			{kotəɻ-v-e-l} & `to be broken' &\armenian{կոտրուել}& {kotəɻ-v-ɒ-v} & `it broke' & \armenian{կոտրուաւ}\\
			{ɒvɒɻt-v-e-l} &`to graduate' & \armenian{աւարտուել}& {ɒvɒɻtv-ɒ-v} & `he graduated' & \armenian{աւարտուաւ}% & (AS)& % & (AS)
   \\
			{ɒzɒt-v-e-l} & `to be freed' &\armenian{ազատուել} & {ɒzɒt-v-ɒ-n} & `they were freed' &\armenian{ազատուան}
			\\ \lspbottomrule
		\end{tabular}}
\end{table}

Passive verbs are inflected as simple E-Class verbs. For example, in the past perfective, they take the past morph /{-ɒ}/ (Table \ref{tab:passive:pst}).


The passive triggers schwa epenthesis after a CC cluster that cannot form a licit word-medial complex coda. For example, we see a schwa in [{{bərnə-v-e-l}}] `to be caught' but not in [jeɻkʰ-v-e-l] `to be sung'.\footnote{It is not completely clear to us why [rn] cannot form a complex coda in the passive verb [bəɾnə-v-e-l] `to be caught'. An open question is whether complex codas like [rn] are truly banned across the entire lexicon, or just passives. See discussion of complex codas in Armenian in \citet{Dolatian-prep-Schwa} } For an analysis of this phenomenon   in {\seaSE} and {\swaSWA}, see \citet[29,82]{Vaux-1998-ArmenianPhono} and \citet{Dolatian-prep-ArmenianPassive}.

\subsection{Inchoatives}\label{section:verb:complex:inch}

Inchoatives are productively formed by adding the sequence [{{-ɒ-n-ɒ-l}}] to a noun or adjective (Table \ref{tab:Verb:Complex:Inch:basic}). The nasal is the inchoative affix. It is followed by the /ɒ/ theme vowel. Depending on the lexeme, the pre-nasal vowel is either /{ɒ}/ or /{e}/. But the low vowel is more common. We assume this pre-nasal vowel is a meaningless linking vowel (LV) \citep{DolatianGuekguezian-prep-TierBasedLocalityArmenianConjugationClass}. 


\begin{table}
	\caption{Inchoative constructions}\label{tab:Verb:Complex:Inch:basic}
	
	\begin{tabular}{llll}
		\lsptoprule
		\multicolumn{2}{c}{LV is /{ɒ}/} & \multicolumn{2}{c}{LV is /{e}/}\\\cmidrule(lr){1-2}\cmidrule(lr){3-4}
		Base & Inchoative & Base & Inchoative \\\midrule
		{t͡ʃʰoɻ} & {t͡ʃʰoɻ-ɒ-n-ɒ-l}  & {vɒχ}  &{vɒχ-e-n-ɒ-l} \\
		$\sqrt{~}$&$\sqrt{~}$-{\lvgloss}-{\inch}-{\thgloss}-{\infgloss}&$\sqrt{~}$&$\sqrt{~}$-{\lvgloss}-{\inch}-{\thgloss}-{\infgloss}\\
		`dry'&`to become dry'&`fear'&`to fear'\\
		\armenian{չոր} & \armenian{չորանալ} & \armenian{վախ}& \armenian{վախենալ}\\ 
		\lspbottomrule
	\end{tabular}
\end{table}


The meaning of an inchoative can be loosely paraphrased as `to become X’. Note the contrast below between using the adjective as a predicate vs. as an inchoativized verb (\ref{sent:Verb:Complex:Inch:ex}).

\begin{exe}
	\ex \label{sent:Verb:Complex:Inch:ex}
	\begin{multicols}{2}
		\begin{xlist}
			\ex \gll {uɻɒχ} {el-n-e-l} %(NK)
			\\
			happy be-{\vx}-{\thgloss}-{\infgloss}
			\\
			\trans `to be happy'
			\\
			\armenian{ուրախ էլնել}
			\ex \gll {uɻɒχ-ɒ-n-ɒ-l}% (NK)
			\\
			happy-{\lvgloss}-{\inch}-{\thgloss}-{\infgloss}
			\\
			\trans			`to become happy'
			\\
			\armenian{ուրախանալ}
		\end{xlist}
	\end{multicols}
	
	
\end{exe}




We list below various morphologically inchoative verbs that we have elicited (Table \ref{tab:Verb:Complex:Inch:exMore}).\footnote{Some of these verbs like \textit{{ɡoʁ-ɒ-n-ɒ-l}} `to steal' have inchoative morphology, but are transitive in their semantics and argument structure. %{\added}
	And for some verbs like `to understand' /hɒsk-ɒ-n-ɒ-l/ or `to know' /im-ɒ-n-ɒ-l/, the root is a bound, and not an independent adjective or noun.}

\begin{table}
	\caption{Example inchoative verbs}\label{tab:Verb:Complex:Inch:exMore}
	%left column forom NK
	\begin{tabular}{l l l}
		\lsptoprule
		{mɒh-ɒ-n-ɒ-l} & `to die' & \armenian{մահանալ}\\
		{hɒsk-ɒ-n-ɒ-l} & `to understand' & \armenian{հասկանալ} %& (AS?)
		\\
		{ɡoʁ-ɒ-n-ɒ-l} & `to steal' & \armenian{գողանալ}\\
		{im-ɒ-n-ɒ-l} & `to know' & \armenian{իմանալ}%& (AS?)
		\\
		{ləv-ɒ-n-ɒ-l} & `to wash' & \armenian{լուանալ}\\
		{ɒɻtʰn-ɒ-n-ɒ-l} & `to awake' & \armenian{արթնանալ}% (NK)
		\\
		{t͡sʰɒŋk-ɒ-n-ɒ-l}& `to wish' & \armenian{ցանկանալ}\\
		{hɒŋɡəst-ɒ-n-ɒ-l} & `to relax' & \armenian{հանգստանալ}%& (NK)
		\\
		{un-e-n-ɒ-l} & `to have/own' & \armenian{ունենալ}\\
		\lspbottomrule
	\end{tabular}
\end{table}




Inchoatives are inflected similarly to A-Class verbs but with some deviations, such as the imperative 2SG (Table \ref{tab:inchoative partial}). Inchoatives use the morphomic aorist suffix (aorist stem) in more contexts than typical A-Class verbs. When the aorist is used, the inchoative affix and its theme vowel are deleted. We show a partial paradigm below, just for the {\iaIA} forms. We show only the deviations between the inchoative and A-Class. All other paradigm cells are formed the same. We do not use zero morphs to show deleted theme vowels and deleted inchoatives.\footnote{Inchoatives are inflected similarly in {\seaSE}. The main difference is that in {\seaSE}, inchoatives are exceptional because they are inflected with the past tense morph /{ɑ}/. {\iaIA} on the other hand uses the past tense morph /ɒ/ which is the default form for the past perfective. For an analysis and documentation of similar facts in {\swaSWA}, see \citet{DolatianGuekguezian-prep-TierBasedLocalityArmenianConjugationClass}.} We place an asterisk for those paradigm cells where the inchoative nasal is deleted, and where the aorist stem is  used instead.\largerpage


\begin{table}
	\caption{Partial paradigm of inchoatives vs. A-Class verbs\label{tab:inchoative partial}}
		\begin{tabular}{lll}
			\lsptoprule 
			&A-Class&Inchoative \\
			& `to read' & `to become happy' \\
			\midrule	
			Infinitive
			&
			{{kɒɻtʰ-ɒ-l}}
			&
			{{uɻɒχ-ɒ-n-ɒ-l}}
			\\
			&$\sqrt{~}$-{\thgloss}-{\infgloss}
			&$\sqrt{~}$-{\lvgloss}-{\inch}-{\thgloss}-{\infgloss}
			\\
			& \armenian{կարդալ} 
			& \armenian{ուրախանալ}
			\\ \addlinespace
			Past. Pfv. 1SG *
			&
			{{kɒɻtʰ-ɒ-t͡sʰ-i-m}}
			&
			{{uɻɒχ-ɒ-t͡sʰ-ɒ-m}}
			\\
			&$\sqrt{~}$-{\thgloss}-{\aorperf}-{\pst}-1{\sg}
			&$\sqrt{~}$-{\lvgloss}-{\aorperf}-{\pst}-1{\sg}
			\\
			
			& \armenian{կարդացիմ}
			& \armenian{ուրախացամ}
			\\ \addlinespace
			
			Imp. 2SG *
			&
			{{kɒɻtʰ-ɒ-$\emptyset$}}
			&
			{{uɻɒχ-ɒ-t͡sʰ-i}}
			\\
			&$\sqrt{~}$-{\thgloss}-{\imp}.2{\sg}
			&$\sqrt{~}$-{\lvgloss}-{\aorother}-{\imp}.2{\sg}
			\\
			& \armenian{կարդալ}
			& \armenian{ուրախացի}
			\\ \addlinespace
			Imp. 2PL *
			&
			{{kɒɻtʰ-ɒ-t͡sʰ-ekʰ}}
			&
			{{uɻɒχ-ɒ-t͡sʰ-ekʰ}}
			\\
			&$\sqrt{~}$-{\thgloss}-{\aorother}-{\imp}.2{\pl}
			&
			$\sqrt{~}$-{\lvgloss}-{\aorother}-{\imp}.2{\pl}
			
			\\
			& \armenian{կարդացէք}
			& \armenian{ուրախացէք}
			\\ \addlinespace
			Subj. Ptcp. *
			&
			{{kɒɻtʰ-ɒ-t͡sʰ-oʁ}}
			&
			{{uɻɒχ-ɒ-t͡sʰ-oʁ}}
			\\
			&
			$\sqrt{~}$-{\thgloss}-{\aorother}-{\sptcp}
			&$\sqrt{~}$-{\lvgloss}-{\aorother}-{\sptcp}
			\\
			& \armenian{կարդացող}
			& \armenian{ուրախացող}
			\\ \addlinespace
			Res. Ptcp. *
			&
			{{kɒɻtʰ-ɒ-t͡sʰ-ɒt͡sʰ}}
			&
			{{uɻɒχ-ɒ-t͡sʰ-ɒt͡sʰ}}
			\\
			& \armenian{կարդացած}
			& \armenian{ուրախացած}
			\\ 
			&
			$\sqrt{~}$-{\thgloss}-{\aorother}-{\rptcp}
			&$\sqrt{~}$-{\lvgloss}-{\aorother}-{\rptcp}
			\\\addlinespace
			Pfv. Cvb. *
			&
			{{kɒɻtʰ-ɒ-t͡sʰ-el}}
			&
			{{uɻɒχ-ɒ-t͡sʰ-el}}
			\\
			&
   			{{kɒɻtʰ-ɒ-t͡sʰ-eɻ}}
&
			{{uɻɒχ-ɒ-t͡sʰ-eɻ}}
			\\
			&
			$\sqrt{~}$-{\thgloss}-{\aorother}-{\perfcvb}
			&$\sqrt{~}$-{\lvgloss}-{\aorother}-{\perfcvb}
			\\ & \armenian{կարդացել, կարդացեր} 
			& \armenian{ուրախացել, ուրախացեր}
			\\ \lspbottomrule			
		\end{tabular} 
\end{table}

Prohibitives are formed by adding the proclitic \textit{{mi-}} before the imperative forms. For the other paradigm cells, inchoatives are inflected like A-Class verbs. These cells are the other converbs, the subjunctive,   the synthetic future, and the conditional past. Complete paradigms are provided in the online archive.\largerpage[-2]

\subsection{Causatives}\label{section:verb:complex:caus}

A causative infinitive consists of a stem plus the sequence \textit{{-t͡sʰn-e-l}} (Table \ref{tab:Verb:Complex:Caus:deriv}). The causative suffix is \textit{{-t͡sʰn-}} and it takes the \textit{{-e-}} theme vowel. The stem of the causative can be the root of a simple verb and its theme vowel. Causatives can also be derived from non-verbs and from inchoative verbs. When a causative is derived from an inchoative, the inchoative suffix and its theme vowel are deleted.\footnote{The causative suffix can sometimes surface with a schwa [{-t͡sən-}] in  {\iaIA}. This is likewise reported for {\seaSE}   (\citealt[47]{Abeghyan-1933-Meter}, \citealt[163]{Gharagulyan-1974-BookArmenianOrthoepy}, \citeyear[42]{Gharagjulyan-1979-SchwaRules}, \citealt[59]{Margaryan-1997-ArmenianPhonology}).} The name of the causative is [pɑtt͡ʃɑrɑkɑn] \armenian{պատճառական} in {\seaSEA}. 



\begin{table}
	\caption{Forming causatives}\label{tab:Verb:Complex:Caus:deriv}
	\footnotesize
	\begin{subtable}[h]{.4\textwidth}
		\centering%
		\caption{Causatives from simple verbs}
		\begin{tabular}{ll}
			\lsptoprule 
			Simple verb & Causative\\\midrule 
			sovoɻ-e-l & {sovoɻ-e-t͡sʰn-e-l }\\
			$\sqrt{~}$-{\thgloss}-{\infgloss}
			&
			$\sqrt{~}$-{\thgloss}-{\caus}-{\thgloss}-{\infgloss}
			\\
			`to learn' 
			
			& `teach'
			\\
			\armenian{սովորել}
			& \armenian{սովորեցնել}
			\\ \lspbottomrule
		\end{tabular}
	\end{subtable}\begin{subtable}[h]{.6\textwidth}
		\centering%
		\caption{Causatives from non-verbs or inchoatives}
		\begin{tabular}{lll}
			\lsptoprule Non-verb& Inchoative verb & Causative\\
			\midrule
			uɻɒχ&	{uɻɒχ-ɒ-n-ɒ-l }
			& {uɻɒχ-ɒ-t͡sʰn-e-l }
			\\
			$\sqrt{~}$ 	&	$\sqrt{~}$-{\lvgloss}-{\inch}-{\thgloss}-{\infgloss}
			&
			$\sqrt{~}$-{\lvgloss}-{\caus}-{\thgloss}-{\infgloss}
			\\
			`happy'&	`to become happy'
			&`to make happy'
			\\
			\armenian{ուրախ} & 			\armenian{ուրախանալ}
			& \armenian{ուրախացնել}
			\\ \lspbottomrule
		\end{tabular}
	\end{subtable}
	
\end{table}


Our consultants feel that deriving causatives from simple verbs is not very productive in {\iaIA}.\footnote{ %{\added}
	Don Stilo (p.c.) suggests that language contact with Persian may be the reason why our {\iaAbbre} consultants disprefer such causatives. He reports that:
 
 
\begin{quote}
    There are very few causative verbs in Persian that are formed on transitive verbs and those transitive verbs that are causativized are not commonly used verbs. The causative verbs in Persian are for the most part cases of valency changing strategies, i.e., intransitive > transitive (`be afraid of' > `scare'). (Stilo, p.c)
\end{quote}} In contrast, causativization is more productive in {\seaSE} and {\swaWA} \citep{DanielKhurshudyan-2015-ValancyClassesinEasternArmenian,DolatianGuekguezian-prep-TierBasedLocalityArmenianConjugationClass}. Deriving causatives from inchoatives is productive in {\iaIA} \citep{megerdoomian-2005-transitivityAlternationVerbCausativeEasternArmenian}. 


In many cases when a causative is derived from a simple verb, the post-root theme vowel differs between the simple verb and causative in {\iaIA} (Table \ref{tab:Verb:Complex:Caus:theme}).\footnote{ \citet{megerdoomian-2005-transitivityAlternationVerbCausativeEasternArmenian} lists many more cases of causative verbs that are derived from simple verbs but utilize a theme-vowel change.}

\begin{table}
	\caption{Differing pre-causative theme vowels}\label{tab:Verb:Complex:Caus:theme}
	\begin{tabular}{llll}
		\lsptoprule
		\multicolumn{2}{l}{Theme vowel changes} & \multicolumn{2}{l}{Theme vowel stays constant} \\\cmidrule(lr){1-2}\cmidrule(lr){3-4}
		{kʰən-e-l} & {kʰən-ɒ-t͡sʰn-e-l} & {kɒɻtʰ-ɒ-l } & {kɒɻtʰ-ɒ-t͡sʰn-e-l} \\ 
		$\sqrt{~}$-{\thgloss}-{\infgloss}&$\sqrt{~}$-{\thgloss}-{\caus}-{\thgloss}-{\infgloss}&$\sqrt{~}$-{\thgloss}-{\infgloss}&$\sqrt{~}$-{\thgloss}-{\caus}-{\thgloss}-{\infgloss}\\
		`to sleep' &`to make sleep' & %(NK)
		`to read' & `to make read'
		%(KM)
		\\
		\armenian{քնել} & \armenian{քնացնել} & \armenian{կարդալ} & \armenian{կարդացնել}\\ 
		\lspbottomrule
	\end{tabular}
\end{table}

Some common causatives are listed in Table \ref{tab:Verb:Complex:Caus:exMore}. It is common to find causative verbs without any pre-causative vowel.



\begin{table}
	\caption{Other common causative verbs in {\iaIA}}\label{tab:Verb:Complex:Caus:exMore}
	\begin{tabular}{lll}
		\lsptoprule 
		hɒŋɡəst-ɒ-t͡sʰn-e-l & `to calm down' & \armenian{հանգստացնել}\\%& (NK)
		{ve(ɻ)-t͡sʰn-e-l} & `to take' &\armenian{վերցնել}\\
		{lə-t͡sʰn-e-l} & `to fill/pour' & \armenian{լցնել}\\
		{dɒɻ-t͡sʰn-e-l} & `to turn into' &\armenian{դարձնել}\\
		\midrule 
		\multicolumn{2}{l}{$\sqrt{~}$(-{\lvgloss})-{\caus}-{\thgloss}-{\infgloss}}& \\ 
		\lspbottomrule 
	\end{tabular}
\end{table}


In terms of inflection, causatives are inflected primarily as E-Class verbs but with some deviation (Table \ref{tab:causative partial}). In the past perfective, the causative suffix uses a special allomorph \textit{{-t͡sʰɻ}-}. This allomorph is likewise used in disparate paradigm slots. These are slots which tend to show morphomic aorist stems in other verb classes. We show a partial paradigm below. We only show the causative paradigm cells which differ from simple E-Class verbs. We place an asterisk for those paradigm cells where the \textit{{-t͡sʰɻ}-} allomorph is used, meaning where we see the aorist stem. The theme vowel is deleted in most of these cells. 



\begin{table}
	\caption{Partial paradigm of causatives vs. E-Class verbs}
	\label{tab:causative partial}
	\begin{tabular}{ll l}
		\lsptoprule 
		&E-Class&Causative \\\midrule
		Infinitive
		&
		{sovoɻ-e-l}
		&
		{sovoɻ-e-t͡sʰn-e-l}
		\\
		&$\sqrt{~}$-{\thgloss}-{\infgloss}
		&$\sqrt{~}$-{\thgloss}-{\caus}-{\thgloss}-{\infgloss}
		\\
		& \armenian{սովորել} 
		& \armenian{սովորեցնել}
		\\
		\addlinespace 	
		Past. Pfv. 1SG *
		&
		{sovoɻ-ɒ-m}
		&
		{sovoɻ-e-t͡sʰɻ-ɒ-m}
		\\
		&$\sqrt{~}$-{\pst}-1{\sg}
		&$\sqrt{~}$-{\thgloss}-{\caus}-{\pst}-1{\sg}
		\\
		& \armenian{սովորամ}
		& \armenian{սովորեցրամ}
		\\
		\addlinespace 	Imp. 2SG *
		&
		{sovoɻ-i}
		&
		{sovoɻ-e-t͡sʰɻ-u}
		\\
		&$\sqrt{~}$-{\imp}.2{\sg}
		&$\sqrt{~}$-{\thgloss}-{\caus}-{\imp}.2{\sg}
		\\
		& \armenian{սովորի}
		& \armenian{սովորեցրու}
		\\
		\addlinespace 	Imp. 2PL *
		&
		{sovoɻ-ekʰ}
		&
		{sovoɻ-e-t͡sʰɻ-ekʰ}
		\\
		&$\sqrt{~}$-{\imp}.2{\pl}
		&$\sqrt{~}$-{\thgloss}-{\caus}-{\imp}.2{\pl}
		\\
		& \armenian{սովորէք}
		& \armenian{սովորեցրէք}
		
		\\
		\addlinespace 	Subj. Ptcp.
		&
		{sovoɻ-oʁ}
		&
		{sovoɻ-e-t͡sʰn-oʁ}
		\\
		&$\sqrt{~}$-{\sptcp}
		&$\sqrt{~}$-{\thgloss}-{\caus}-{\sptcp}
		\\
		& \armenian{սովորող}
		& \armenian{սովորեցնող}
		\\
		\addlinespace 	Res. Ptcp. *
		&
		{sovoɻ-ɒt͡sʰ}
		&
		{sovoɻ-e-t͡sʰɻ-ɒt͡sʰ}
		\\
		&$\sqrt{~}$-{\rptcp}
		&$\sqrt{~}$-{\thgloss}-{\caus}-{\rptcp}
		\\
		& \armenian{սովորած}
		& \armenian{սովորեցրած}
		
		\\
		\addlinespace 		Pfv. Cvb. *
		&
		{sovoɻ-el}
		&
		{sovoɻ-e-t͡sʰɻ-el}
		\\
		&
		{sovoɻ-eɻ}
		&
		{sovoɻ-e-t͡sʰɻ-eɻ}
		\\
		&$\sqrt{~}$-{\perfcvb}
		&$\sqrt{~}$-{\thgloss}-{\caus}-{\perfcvb}
		
		\\
		& \armenian{սովորել, սովորեր}
		& \armenian{սովորեցրել, սովորեցրեր}
		
		\\\lspbottomrule
	\end{tabular}
\end{table}



Prohibitives are formed by adding the proclitic \textit{{mi-}} before the imperative forms. For the other paradigm cells, causatives are inflected like E-Class verbs. These cells are the other converbs, the subjunctive,   the synthetic future, and the conditional past. The {\iaIA} forms do not significantly differ from {\seaSE} except for the past perfective. The {\seaSE} past perfective of causatives uses the past tense morph /i/ instead of {/ɒ/}: {\iaIA} \textit{{sovoɻ-e-t͡sʰɻ-ɒ-n}} vs. {\seaSE} \textit{{sovoɾ-e-t͡sʰɾ-i-n}} `they taught'. Complete paradigms are provided in the online archive. 

\section{Irregular verbs}\label{section:verb:irregular}
The regular verb classes were discussed in the previous section. These classes constitute the majority of verbs in the {\iaIA} lexicon. This section goes over some irregular classes. These are all rather low-frequency in terms of types, but seem high-frequency in their tokens. These irregulars can be divided into different subclasses: infixed verbs, suppletive verbs, defective verbs, and other verbs.

This section focuses on providing paradigms just for {\iaIA}. To contrast these irregular paradigms with {\seaSE}, see \citet[277ff]{DumTragut-2009-ArmenianReferenceGrammar}. Complete paradigms are provided in the online archive. 

\subsection{Infixed verbs}\label{section:verb:irregular:infixed}
In the infinitive form, simple regular verbs consist of a root, theme vowel, and the infinitive suffix \textit{-l}. {\iaIA} likewise has a set of irregular verbs where a meaningless morph /{-n-}/ surfaces between the root and theme vowel (Table \ref{tab:Verb:Irr:Infix:ex}). We gloss this meaningless verbal stem-extender as {\vx}.\footnote{For the verb \textit{{ənɡə-n-e-l}} `to fall,' the second schwa is epenthetic. It is absent before a vowel: \textit{{ənɡ-ɒ-ŋkʰ}} `we fell' [$\sqrt{~}$-{\pst}-1{\pl}].}


\begin{table}
	\caption{Infixed irregular verbs in {\seaSE} and {\iaIA} }\label{tab:Verb:Irr:Infix:ex}
	\begin{tabular}{lllll}
		\lsptoprule
		{\iaAbbre} && {\seaAbbre} & & \\\midrule
		{mət-n-e-l}& `to enter' & {mət-n-e-l} & `to enter'& \armenian{մտնել} \\
		{tes-n-e-l} &`to see'&{tes-n-e-l} &`to see' &\armenian{տեսնել}\\
		{ɒr-n-e-l} & `to buy' &{ɑr-n-e-l}& `to take'&\armenian{առնել}\\
		{el-n-e-l}&`to be'&{jel-n-e-l}& `to get up'&\armenian{ելնել}\\
		{tʰoʁ-n-e-l}&`to let/leave'&{tʰoʁ-n-e-l} &`to let/leave'&\armenian{թողնել}\\
		{əŋɡə-n-e-l}&`to fall'&{əŋk-n-e-l}& `to fall'& \armenian{ընկնել}\\
		{it͡ʃʰ-n-e-l}& `to descend'&{it͡ʃʰ-n-e-l}&`to descend'&\armenian{իջնել} \\
		$\sqrt{~}$-{\vx}-{\thgloss}-{\infgloss} && $\sqrt{~}$-{\vx}-{\thgloss}-{\infgloss}& &\\
		\lspbottomrule
	\end{tabular}
\end{table}


Across Armenian lects, this nasal morph /{-n-}/ is diachronically a reflex of the Proto-Indo-European nasal infix \citep{greppin-1973-originArmenianNasalSuffixVerb,hamp-1975-nasalPresentOfArmenian,kocharov-2019-oldArmenianLasalVerbs}. {\seaSEA} has these same verbs. However, for some of these verbs, the meaningless morph is an affricate /t͡ʃʰ/ in {\seaSEA}. It seems that {\iaIA} has lost the affricate morph, and now all the infixed verbs just use the nasal morph (Table \ref{tab:Verb:Irr:Infix:nnotch}).\footnote{The replacement of the affricate infix with the nasal infix is likewise attested in {\seaCEA} \citep[172]{DumTragut-2009-ArmenianReferenceGrammar},  Khoy/Urmia \citep[98]{Asatryan-1962-KhoyUrmiaDialect}, Salmast \citep[\S 3.2.7]{Vaux-Salmast},  and all of the Southeastern group of dialects,  and  Van \citep[165]{Adjarian-1952-VanDialect}. We could also find it perhaps in  Alashkert, Mush, Agulis,    New Julfa and other dialects that often pattern with Salmast.}

\begin{table}
	\caption{Infixed irregular verbs with affricates in {\seaSE}, but nasals in {\iaIA} }\label{tab:Verb:Irr:Infix:nnotch}
	\begin{tabular}{lllll}
		\lsptoprule 
		\multicolumn{2}{c}{\iaAbbre} & \multicolumn{3}{c}{\seaAbbre}\\\cmidrule(lr){1-2}\cmidrule(lr){3-5}
		{pʰɒχ-n-e-l}& \armenian{փախնել}& {pʰɑχ-t͡ʃʰ-e-l} & `to escape'&\armenian{փախչել} \\
		{tʰər-n-e-l}& \armenian{թռնել}&{tʰər-t͡ʃʰ-e-l} & `to fly' &\armenian{թռչել}\\
		$\sqrt{~}$-{\vx}-{\thgloss}-{\infgloss} && $\sqrt{~}$-{\vx}-{\thgloss}-{\infgloss}& & \\
		\lspbottomrule
	\end{tabular}
\end{table}


What is irregular about this class is that the nasal is dropped in some but not all paradigm cells (Table \ref{tab:Verb:Irr:Infix:NDel}). Whenever the verb lacks this nasal, the verb is said to use its aorist stem. For example, the nasal surfaces in the subjunctive present and the subjunctive past. But the nasal is deleted in the past perfective. The surface morphs are just the root and T-Agr suffixes.


\begin{table}
	\caption{Nasal deletion in infixed verbs vs. E-Class verbs in {\iaIA}}\label{tab:Verb:Irr:Infix:NDel}
	\resizebox{\textwidth}{!}{%
		\begin{tabular}{lllll}
			\lsptoprule
			&\multicolumn{2}{l}{Irregular infixed verb}& \multicolumn{2}{l}{Regular E-Class} \\\midrule
			Infinitive& {mer-n-e-l}& $\sqrt{~}$-{\vx}-{\thgloss}-{\infgloss} &{jeɻkʰ-e-l} & $\sqrt{~}$-{\thgloss}-{\infgloss}\\
			&`to die'&&`to sing'&\\
			& \armenian{մեռնել}
			& & \armenian{երգել}
			& 
			\\
			\addlinespace
			Sbjv. Present 1PL & {mer-n-e-ŋkʰ} & $\sqrt{~}$-{\vx}-{\thgloss}-1{\pl}&{jeɻkʰ-e-ŋkʰ}& $\sqrt{~}$-{\thgloss}-1{\pl}
			\\
			& \armenian{մեռնենք}
			& & \armenian{երգենք}
			& 
			\\
			\addlinespace
			Past Pfv. 1PL & {mer-ɒ-ŋkʰ} & $\sqrt{~}$-{\pst}-1{\pl}&{jeɻkʰ-ɒ-ŋkʰ}& $\sqrt{~}$-{\pst}-1{\pl}\\
			& \armenian{մեռանք}
			& & \armenian{երգանք}
			& 
			\\\lspbottomrule 
		\end{tabular}%
	}
\end{table}

The partial paradigm below shows the finite and non-finite forms of this irregular class (Table \ref{tab:nasal infix drop}). An asterisk is placed next to each cell that shows the deletion of this nasal morph. This class is inflected the same as the regular E-Class; the only difference is the deletion of the nasal morph in certain slots.\footnote{In {\seaSEA}, the infixed verbs are irregular in the past perfective not only because they drop the nasal, but also because they use the past T marker /{ɑ}/: [{{mer-ɑ-v}}] `he died' [$\sqrt{~}$-{\pst}-3{\sg}]. But in {\iaIA}, the use of the past T marker /ɒ/ is a regular feature. } 

\begin{table}
	\caption{Distribution of nasal deletion in {\iaIA} with [{mer-n-e-l}] `to die'}
	\label{tab:nasal infix drop}
	\resizebox{\textwidth}{!}{%
		\begin{tabular}{llll}
			\lsptoprule
			Cell&Form&Gloss& \\\midrule
			Infinitive & {mer-n-e-l} &$\sqrt{~}$-{\vx}-{\thgloss}-{\infgloss}
			& \armenian{մեռնել}
			\\
			Imperfective converb & {mer-n-um} &$\sqrt{~}$-{\vx}-{\impfcvb}
			& \armenian{մեռնում}	\\
			Future converb & {mer-n-e-l-u} &$\sqrt{~}$-{\vx}-{\thgloss}-{\infgloss}-{\futcvb} & \armenian{մեռնելու}
			\\
			Perfective converb * & {mer-el, mer-eɻ} &$\sqrt{~}$-{\perfcvb} 
			& \armenian{մեռել, մեռեր}
			\\
			Connegative converb & {mer-n-i} &$\sqrt{~}$-{\vx}-{\cncvb} 
			& \armenian{մեռնի}	\\
			Subject participle & {mer-n-oʁ} &$\sqrt{~}$-{\vx}-{\sptcp} 
			& \armenian{մեռնող}
			\\
			Resultative participle * & {mer-ɒt͡sʰ} &$\sqrt{~}$-{\rptcp} 
			& \armenian{մեռած}
			\\
			Sbjv. Present 1PL & {mer-n-e-ŋkʰ} &$\sqrt{~}$-{\vx}-{\thgloss}-1{\pl} & \armenian{մեռնենք} \\
			
			Sbjv. Past   1PL & {mer-n-i-ŋkʰ} &$\sqrt{~}$-{\vx}-{\pst}-1{\pl}& \armenian{մեռնինք} \\
			
			
			Past Pfv. 1PL * & {mer-ɒ-ŋkʰ} &$\sqrt{~}$-{\pst}-1{\pl} & \armenian{մեռանք} \\
			
			Imperative 2SG * & {mer-i} &$\sqrt{~}$-{\imp}.2{\sg}& \armenian{մեռի} \\
			
			
			Imperative 2PL * & {mer-ekʰ} &$\sqrt{~}$-{\imp}.2{\pl} & \armenian{մեռէք}\\
			
			Causative * & {mer-t͡sʰn-e-l} &$\sqrt{~}$-{\caus}-{\thgloss}-{\infgloss} & \armenian{մեռցնել} \\
			
			Passive & N/A &%$\sqrt{~}$-{\vx}-{\pass}-{\thgloss}-{\infgloss}
   & \\
			\lspbottomrule 
		\end{tabular}
	}
\end{table}


For brevity, the above paradigm omits zero morphs (theme vowels). For the finite forms, we only show the 1PL; the other agreement cells behave the same with respect to the nasal. We omit the following:
\begin{itemize}
	\item The negatives that derive from simple prefixation of \textit{{t͡ʃʰ-}} onto a subjunctive or past perfective base.
	\item The positive synthetic future and conditional past that are derived by prefixing \textit{{k(ə)-}} to the subjunctive.
	\item The prohibitives that are derived by adding the proclitic \textit{{mi}} to the imperative base.
\end{itemize}



It is difficult to find a single infixed verb that can be both causativized and passivized (Table \ref{tab:Verb:Irr:Infix:pass}). Causativization generally deletes the nasal morph, as seen in Table \ref{tab:nasal infix drop}. Passivization generally keeps the nasal morph.


\begin{table}
	\caption{Passivization of infixed verbs}\label{tab:Verb:Irr:Infix:pass}
	\begin{tabular}{llll}
		\lsptoprule
		 Active && Passive & \\\midrule
		{{tes-n-e-l}} & $\sqrt{~}$-{\vx}-{\thgloss}-{\infgloss} &{{tes-nə-v-e-l}} & 
		$\sqrt{~}$-{\vx}-{\pass}-{\thgloss}-{\infgloss} \\
		`to see' && `to be seen' &\\
		\armenian{տեսնել} && \armenian{տեսնուել}&
		\\\lspbottomrule 
	\end{tabular}
\end{table}


For a typical infixed verb like \textit{{mer-n-e-l}} `to die', the imperative 2SG is formed by dropping the nasal and using the imperative 2SG suffix \textit{{-i}}. A subset of these infixed verbs have an irregular imperative 2SG. This set is listed in Table \ref{tab:Verb:Irr:Infix:irrImp}. The prohibitive 2SG is derived from this imperative by adding the proclitic \textit{mi}.

\begin{table}
	\caption{Irregular imperative 2SG within irregular infixed verbs}\label{tab:Verb:Irr:Infix:irrImp}
	\begin{tabular}{llll l}
		\lsptoprule 
		&`to see'&`to buy'&`to let/leave'&\\\midrule
		Infinitive& {tes-n-e-l} & {ɒr-n-e-l} & {tʰoʁ-n-e-l}& $\sqrt{~}$-{\vx}-{\thgloss}-{\infgloss} \\
		& \armenian{տեսնել} & \armenian{առնել} & \armenian{թողնել}& \\
		\addlinespace
		Imperative 2SG& {tes}& {ɒr} & {tʰoʁ}& $\sqrt{~}$\\
		& \armenian{տես} & \armenian{առ} & \armenian{թող} & \\ 
		\lspbottomrule
	\end{tabular}
\end{table}





There is no semantic or morphosyntactic correlation that unites the various cells which show the deletion of the nasal. The distribution is morphomic, and is traditionally described as utilizing an aorist stem. The distribution of nasal dropping is the same in {\seaSEA}, and essentially in {\swaSWA} as well. \citet{DolatianGuekguezian-prep-Morphome} analyze the cognate infixed verbs of {\swaSWA} as morphomic and provide an analysis of aorist stems. 

For the infixed verb `to let' [tʰoʁ-n-e-l] \armenian{թողնել}, AS reports that the fricative /ʁ/ can be optionally deleted in some of the inflected forms, such as the imperfective converb [tʰoʁ-n-um] or [tʰo-n-um]. We have not systematically studied this deletion, but it is likely just grammaticalized lenition in a highly-frequent verb. Similar deletion is attested in function words like [əste(ʁ)] `here' (\S\ref{section:funct:demonstrative}).\pagebreak

\subsection{Suppletive verbs}\label{section:verb:irregular:suppletive}
A small class of irregular verbs are suppletive. These inflect as E-Class verbs in many parts of the paradigm. But in other parts, they use a different root allomorph and irregular imperative suffixes. Suppletive verbs can be categorized into three groups or subclasses, which we catalog below.

The first group of verbs is listed in Table \ref{tab:suppletive 1}. For a suppletive verb like `to eat' \textit{{ut-e-l}}, the root maintains a constant form \textit{{ut-}} in many paradigm cells. In some other cells, the root uses a morphologically-conditioned allomorph \textit{{keɻ-}}. We call \textit{{keɻ-}} the restricted allomorph, while \textit{{ut-}} is the elsewhere allomorph.\footnote{For some of our speakers like NK, the suppletive verb \textit{{dən-e-l}} `to put' is pronounced with an initial voiceless stop [t] in all its allomorphs. In contrast, AS and KM report [d], just as in {\seaSEA}. } In the traditional literature, the restricted morph is also called the aorist stem. 

\begin{table}
	\caption{Suppletive verbs in {\iaIA} - Group 1}
	\label{tab:suppletive 1}
	\resizebox{\textwidth}{!}{%
		\begin{tabular}{llllll}
			\lsptoprule
			&`to eat'&`to do'&`to take to' &`to put' &\\\midrule
			Elsewhere allomorph: &{ut-}&{ɒn-}&{tɒn-}&{dən-}
			&
			\\
			Infinitive& {ut-e-l}&{ɒn-e-l}&{tɒn-e-l}&{dən-e-l}&$\sqrt{~}$-{\thgloss}-{\infgloss}
			\\ 
			& \armenian{ուտել} &\armenian{անել} &\armenian{տանել} & \armenian{դնել} & 
			\\
			\addlinespace	Sbjv. present 1PL & {ut-e-ŋkʰ}&{ɒn-e-ŋkʰ}&{tɒn-e-ŋkʰ}&{dən-e-ŋkʰ}&$\sqrt{~}$-{\thgloss}-1{\pl}
			
			\\
			&\armenian{ուտենք} &\armenian{անենք} &\armenian{տանենք} &\armenian{դնենք} & 
			\\ 
			\addlinespace 
			Restricted allomorph:&{keɻ-}&{ɒɻ-} &{tɒɻ-} &{dəɻ-}
			&
			\\
			Past Pfv. 1PL& {keɻ-ɒ-ŋkʰ}&{ɒɻ-ɒ-ŋkʰ}&{tɒɻ-ɒ-ŋkʰ}&{dəɻ-ɒ-ŋkʰ}&$\sqrt{~}$-{\pst}-1{\pl}
			\\ & \armenian{կերանք} & \armenian{արանք} &\armenian{տարանք} &\armenian{դրանք} &
			\\
			\addlinespace
			Imperative 2SG& {keɻ}&{ɒɻ-ɒ}&{tɒɻ}&{diɻ} &$\sqrt{~}$-({\imp}.2{\sg})
			\\
			& \armenian{կեր} & \armenian{արա}& \armenian{տար} & \armenian{դիր}& 
			\\
			\lspbottomrule 
		\end{tabular}
	}
\end{table}


For the verb `to eat', the imperative 2SG is formed by just using the restricted allomorph without further suffixation. In contrast, some suppletive verbs like `to do' use an additional suffix. Some verbs like `to put' use a special additional root allomorph that is only found in the imperative 2SG. We list the imperative 2SG of the suppletive verbs in Table \ref{tab:suppletive 1}. The prohibitive 2SG is derived from this imperative by adding the proclitic \textit{{mi}}.



The above suppletive verbs all use the \textit{{-e-}} theme vowel in their infinitive form. Outside of the imperative 2SG, they pattern the same in the distribution of their root allomorphs.

The partial paradigm in Table \ref{tab:suppletive paradigm eat} lists the distribution of the root allomorphs for Group 1 verbs. An asterisk is placed next to each cell that shows the restricted allomorph.\footnote{As with the infixed verbs, in {\seaSEA}, many of the suppletive verbs are irregular in the past perfective not only because they use a different root allomorph, but also because they use the past T marker /{ɑ}/: [{{keɾ-ɑ-v}}] `he ate' [$\sqrt{~}$-{\pst}-3{\sg}]. But in {\iaIA}, the use of the past T marker /{ɒ}/ is a regular feature for verbs.} The subjunctive forms pattern like E-Class verbs.


\begin{table}
	\caption{Distribution of root allomorphs in {\iaIA} for [{ut-e-l}] `to eat'}
	\label{tab:suppletive paradigm eat}
	\begin{tabular}{lll l}
		\lsptoprule
		Cell&Form&Gloss& \\\midrule
		Infinitive & {ut-e-l} &$\sqrt{~}$-{\thgloss}-{\infgloss}
		& \armenian{ուտել}
		\\
		Imperfective converb & {ut-um} &$\sqrt{~}$-{\impfcvb} &\armenian{ոտում}
		\\
		Future converb & {ut-e-l-u} &$\sqrt{~}$-{\thgloss}-{\infgloss}-{\futcvb} &\armenian{ուտելու}
		
		\\
		Perfective converb * & {keɻ-el, keɻ-eɻ} &$\sqrt{~}$-{\perfcvb} &\armenian{կերել, կերեր}
		
		\\
		Connegative converb & {ut-i} &$\sqrt{~}$-{\cncvb} &\armenian{ուտի}
		\\
		
		Subject participle & {ut-oʁ} &$\sqrt{~}$-{\sptcp} 
		&\armenian{ուտող}\\
		
		Resultative participle * & {keɻ-ɒt͡sʰ} &$\sqrt{~}$-{\rptcp} &\armenian{կերած}
		\\
		
		Sbjv. Present 1PL & {ut-e-ŋkʰ} &$\sqrt{~}$-{\thgloss}-1{\pl} &\armenian{ուտենք}\\
		
		Sbjv. Past   1PL & {ut-i-ŋkʰ} &$\sqrt{~}$-{\pst}-1{\pl} &\armenian{ուտինք} \\
		
		Past Pfv. 1PL * & {keɻ-ɒ-ŋkʰ} &$\sqrt{~}$-{\pst}-1{\pl} &\armenian{կերանք} \\
		
		Imperative 2SG * & {keɻ} &$\sqrt{~}$ &\armenian{կեր} \\
		
		Imperative 2PL * & {keɻ-ekʰ} &$\sqrt{~}$-{\imp}.2{\pl} 
		&\armenian{կերէք}	\\
		\lspbottomrule 
	\end{tabular}
\end{table}

The paradigm in Table \ref{tab:suppletive paradigm eat} omits zero morphs (theme vowels). For the finite forms, we only show the 1PL; the other agreement cells behave the same with respect to the root allomorphy. We omit the following:

\begin{itemize}
	\item The negatives that derive from simple prefixation of \textit{{t͡ʃʰ-}} onto a subjunctive or past perfective base. 
	\item The positive synthetic future and conditional past that are derived by prefixing \textit{{k-}} to the subjunctive. \item The prohibitives that are derived by adding the proclitic \textit{{mi}} to the imperative base.
\end{itemize}

The second group of suppletive verbs consists of only the verb [{{etʰ-ɒ-l}}] `to go'. It acts as an A-Class verb in terms of the distribution of theme vowels, the aorist suffix, and the past marker /{i}/. Its irregularity is that some of its paradigm cells utilize a restricted root allomorph \textit{{ɡən-}}. We show a partial paradigm in Table \ref{tab:suppletive go}. The asterisk is used to mark the cells that utilize the restricted allomorph.\footnote{Some speakers pronounce the elsewhere root allomorph as \textit{{eɻtʰ-}} instead of \textit{{etʰ-}}. Some speakers can make the sbjv. past    utilize the restricted root \textit{{ɡən-}}, e.g. the 1PL form [{{ɡən-ɒj-i-ŋkʰ}}]. Some speakers use the restricted allomorph in the connegative converb: [{{ɡən-ɒ}}] instead of [{{etʰ-ɒ}}]. But others have told us that using \textit{{ɡən-}} root in these contexts sounds more “Eastern” instead of {\iaIA}. In {\seaSEA}, the root \textit{{ɡən-}} is used to form a regular non-suppletive A-Class verb \textit{{ɡən-ɑ-l}} `to go'. Some of our speakers use this separate verb as well.}


\begin{table}
	\caption{Distribution of root allomorphs in {\iaIA} for [{etʰ-ɒ-l}] `to go'\label{tab:suppletive go}}
		\begin{tabular}{llll}
			\lsptoprule
			Cell&Form&Gloss& \\\midrule
			Infinitive & {etʰ-ɒ-l} &$\sqrt{~}$-{\thgloss}-{\infgloss} 
			& \armenian{էթալ}
			\\
			Imperfective converb & {etʰ-um} &$\sqrt{~}$-{\impfcvb}
			& \armenian{էթում}\\
			
			Future converb & {etʰ-ɒ-l-u} &$\sqrt{~}$-{\thgloss}-{\infgloss}-{\futcvb} & \armenian{էթալու}
			\\
			
			Perfective converb * & {ɡən-ɒ-t͡sʰ-el} &$\sqrt{~}$-{\thgloss}-{\aorother}-{\perfcvb} & \armenian{գնացել}
			\\
			
			& {ɡən-ɒ-t͡sʰ-eɻ} & & 
\armenian{գնացեր}			\\
			Connegative converb & {etʰ-ɒ} &$\sqrt{~}$-{\thgloss}
			&\armenian{էթայ}\\
			Subject participle * & {ɡən-ɒ-t͡sʰ-oʁ} &$\sqrt{~}$-{\thgloss}-{\aorother}-{\sptcp} & \armenian{գնացող}
			\\
			
			Resultative participle * & {ɡən-ɒ-t͡sʰ-ɒt͡sʰ} &$\sqrt{~}$-{\thgloss}-{\aorother}-{\rptcp} & \armenian{գնացած}\\
			
			Sbjv. Present 1PL & {etʰ-ɒ-ŋkʰ} &$\sqrt{~}$-{\thgloss}-1{\pl} & \armenian{էթանք}\\
			
			Sbjv. Past    1PL & {etʰ-ɒj-i-ŋkʰ} &$\sqrt{~}$-{\thgloss}-{\pst}-1{\pl} & \armenian{էթայինք} \\
			
			Past Pfv. 1PL * & {ɡən-ɒ-t͡sʰ-i-ŋkʰ} &$\sqrt{~}$-{\thgloss}-{\aorperf}-{\pst}-1{\pl} & \armenian{գնացինք}\\
			
			Imperative 2SG * & {ɡən-ɒ} &$\sqrt{~}$-{\thgloss} & \armenian{գնա}\\
			
			
			Imperative 2PL * & {ɡən-ɒ-t͡sʰ-ekʰ} &$\sqrt{~}$-{\thgloss}-{\aorother}-{\imp}.2{\pl} & \armenian{գնացէք}\\
			
			\lspbottomrule 
		\end{tabular}
\end{table}



Finally, there is a third group of suppletive verbs (Table \ref{tab:suppletive mono}), made up of two members: [{{t-ɒ-l}}] `to give' and [{{ɡ-ɒ-l}}] `to come'. These verbs use the \textit{{-ɒ-}} theme vowel, and the elsewhere root allomorph is a single consonant. These two verbs have restricted allomorphs in the past perfective. Each has a separate allomorph used in the imperative 2SG. 

\begin{table}[H]
	\centering
	\caption{Suppletive verbs with mono-consonantal root}
	\label{tab:suppletive mono}
	\begin{tabular}{llll}
		\lsptoprule
		&`to give'&`to come' & \\
		Elsewhere allomorph & {t-}& {ɡ-} & \\\midrule
		Infinitive& {t-ɒ-l} & {ɡ-ɒ-l}&$\sqrt{~}$-{\thgloss}-{\infgloss}\\
		&\armenian{տալ} &\armenian{գալ} & \\
		Sbjv. Present 1PL & {t-ɒ-ŋkʰ} & {ɡ-ɒ-ŋkʰ}&$\sqrt{~}$-{\thgloss}-1{\pl}\\
		&\armenian{տանք} &\armenian{գանք} &	\\ \addlinespace
		Restricted allomorph& {təv-}&{ek-} &\\
		Past Pfv. 1PL & {təv-ɒ-ŋkʰ}& {ek-ɒ-ŋkʰ}&$\sqrt{~}$-{\pst}-1{\pl}\\
		&\armenian{տուանք}&\armenian{էկանք}&\\
		Imperative 2SG & {tuɻ}& {ɒɻi}& $\sqrt{~}$\\
		&\armenian{տուր}&\armenian{արի} & \\
		\lspbottomrule
	\end{tabular}
\end{table}

These two verbs also use a special construction for forming the imperfective converb (Table \ref{tab:suppletive mono converb}). Whereas A-Class verbs use the template {$\sqrt{~}$-um}, these two verbs use the template {$\sqrt{~}$-ɒ-l-is}. The suffix \textit{{-is}} is an irregular imperfective converb suffix. The final fricative is a latent segment, meaning this segment is deleted when the auxiliary has moved such as in negation. This segment's distribution parallels that of the perfective converb's latent segment; see \S\ref{section:morphophono:auxiliary:imperfective}. 

\begin{table}
	\caption{Imperfective converb for suppletive mono-consonantal root\label{tab:suppletive mono converb}}
	\resizebox{\textwidth}{!}{%
		\begin{tabular}{llll}
			\lsptoprule 
			&`to give'&`to come' & \\\midrule
			Infinitive& {t-ɒ-l} & {ɡ-ɒ-l} &$\sqrt{~}$-{\thgloss}-{\infgloss} \\
			&\armenian{տալ} &\armenian{գալ} & \\%\addlinespace
			Impf. converb & {t-ɒ-l-is} & {ɡ-ɒ-l-is} 	& {$\sqrt{~}$-{\thgloss}-{\infgloss}-{\impfcvb}}\\			
			&\armenian{տալիս} &\armenian{գալիս}& \\%\addlinespace
			Indc. Pres. 1PL & {t-ɒ-l-is e-ŋkʰ} & {ɡ-ɒ-l-is e-ŋkʰ} & {$\sqrt{~}$-{\thgloss}-{\infgloss}-{\impfcvb} {\auxgloss}-1{\pl}}\\
			&\armenian{տալիս ենք} &\armenian{գալիս ենք}&		\\%\addlinespace
			Neg. indc. Pres. 1PL & {t͡ʃʰ-e-ŋkʰ t-ɒ-l-i } & {t͡ʃʰ-e-ŋkʰ ɡ-ɒ-l-i } & {{\neggloss}-{\auxgloss}-1{\pl} $\sqrt{~}$-{\thgloss}-{\infgloss}-{\impfcvb} }	\\
			&\armenian{չենք տալիս} & \armenian{չենք գալիս}& 	\\
			\lspbottomrule
		\end{tabular}} 
\end{table}

The partial paradigm of the verb\largerpage[1]{} `to give' is shown in Table \ref{tab:paradigm give suppletive}. The verb `to come' is inflected similarly.{\interfootnotelinepenalty=10000\footnote{The subject participle of `to come' [ɡ-ɒ-l] is typically [ek-oʁ] `$\sqrt{}$-{\sptcp}' \armenian{էկող}, but NK says the word [ek-ɒ-t͡sʰ-oʁ] `$\sqrt{}$-{\thgloss}-{\aorother}-{\sptcp}' \armenian{էկացող} is also attested, especially in the phrase [ek-ɒ-t͡sʰ-oʁ t͡ʃʰ-i] meaning `he's not coming' with the negative 3SG auxiliary. The participle here is used to mean something like `he's not the type of person to come', such as to a party.}} These verbs further differ from the previous set of suppletive verbs in that their subject participles utilize the restricted allomorph. Their subjunctive forms pattern like A-Class verbs.



\begin{table}
	\caption{Distribution of root allomorphs in {\iaIA} for [{t-ɒ-l}] `to give'}
	\label{tab:paradigm give suppletive}
	\begin{tabular}{llll}
		\lsptoprule 
		Cell&Form&Gloss& \\\midrule
		Infinitive & {t-ɒ-l} &$\sqrt{~}$-{\thgloss}-{\infgloss}& \armenian{տալ}\\
		Imperfective converb & {t-ɒ-l-is} &$\sqrt{~}$-{\thgloss}-{\infgloss}-{\impfcvb}& \armenian{տալիս}\\
		Future converb & {t-ɒ-l-u} &$\sqrt{~}$-{\thgloss}-{\infgloss}-{\futcvb} &\armenian{տալու}\\
		Perfective converb * & {təv-el, təv-eɻ} &$\sqrt{~}$-{\perfcvb} &\armenian{տուել, տուեր}\\
		Connegative converb & {t-ɒ} &$\sqrt{~}$-{\thgloss} &\armenian{տայ}\\
		Subject participle * & {təv-oʁ} &$\sqrt{~}$-{\sptcp} &\armenian{տուող}\\
		Resultative participle * & {təv-ɒt͡sʰ} &$\sqrt{~}$-{\rptcp} &\armenian{տուած}\\
		Sbjv. Present 1PL & {t-ɒ-ŋkʰ} &$\sqrt{~}$-{\thgloss}-1{\pl} &\armenian{տանք}\\
		Sbjv. Past    1PL & {t-ɒj-i-ŋkʰ} &$\sqrt{~}$-{\thgloss}-{\pst}-1{\pl} & \armenian{տայինք} \\
		Past Pfv. 1PL * & {təv-ɒ-ŋkʰ} &$\sqrt{~}$-{\pst}-1{\pl}&\armenian{տուանք} \\
		Imperative 2SG * & {tuɻ} &$\sqrt{~}$ &\armenian{տուր} \\
		Imperative 2PL * & {təv-ekʰ} &$\sqrt{~}$-{\imp}.2{\pl} &\armenian{տուէք} \\
		\lspbottomrule 
	\end{tabular}	
\end{table}



It is difficult to make generalizations when it comes to causativizing or passivizing suppletive verbs. We have come across causatives of [{{ut-e-l}}] `to eat' that use the elsewhere root allomorph: [{{ut-e-t͡sʰn-e-l}}] `to feed' \armenian{ուտեցնել}. But we have also come across speakers who prefer not causativizing this verb at all. For passivization, {\seaSEA} uses the restricted root allomorph to passivize `to take to', `to put', and `to give'. Some (more literate) {\iaIA} speakers do this as well: [{{tɒɻ-v-e-l}}] \armenian{տարուել} `to be taken to', [{{dəɻ-v-e-l}}]  \armenian{դրուել} `to be put', [{{təɻ-v-e-l}}] \armenian{տրուել} `to be given'. Some {\iaIA} speakers prefer not passivizing these at all. 


\subsection{Defective verbs}\label{section:verb:irregular:def}
There is a small set of defective verbs in {\iaIA}. These verbs are defective in not having all possible types of finite and non-finite forms. 

One defective verb is the copula, which only appears in the present tense and the past tense. We discussed the copula in \S\ref{section:verb:aux} under the guise of the auxiliary. 

Two other defective verbs are the verbs `to exist' [{{k-ɒ-m}}] and `to have' [{{un-e-m}}].\footnote{For the verb `to exist', the initial stop is usually voiceless \textit{{k-ɒ-m}}, but some speakers voice it: \textit{{ɡ-ɒ-m}}. } The verb `to exist' is used to mark existential sentences like `there is X'. The verb `to have' is more accurately translated as `to own'. This verb only marks possession and is not an auxiliary. 

We show a partial paradigm in Table \ref{tab:defective exist own} with just the 1SG. Both of these verbs are used only in the indicative present and past, along with the corresponding negated forms. Unlike regular verbs, the indicative of these verbs is formed synthetically. The two verbs use the same T-Agr morphs as the subjunctive of the regular A-Class and E-Class respectively. 

\begin{table}
	\caption{Defective verbs `to exist' and `to own'\label{tab:defective exist own}}
	\begin{tabular}{lllr}
		\lsptoprule
		&`to exist' & `to have'&\\
		Infinitive& N/A& N/A& \\
		\midrule
		Indc. pres. 1SG & {k-ɒ-m} & {un-e-m} &$\sqrt{~}$-{\thgloss}-1{\sg}\\
		& \armenian{կամ} & \armenian{ունեմ} & \\
		Neg. indc. pres. 1SG & {t͡ʃʰə-k-ɒ-m} & {t͡ʃʰ-un-e-m} & {\neggloss}-$\sqrt{~}$-{\thgloss}-1{\sg}\\
		& \armenian{չկամ} & \armenian{չունեմ} & \\\addlinespace
		Indc. past 1SG & {k-ɒj-i-m} 	&{un-$\emptyset$-i-m} & $\sqrt{~}$-{\thgloss}-{\pst}-1{\sg}\\
		& \armenian{կայիմ} & \armenian{ունիմ} & \\
		Neg. indc. past 1SG & {t͡ʃʰə-k-ɒj-i-m}& {t͡ʃʰ-un-$\emptyset$-i-m}
		& {\neggloss}-$\sqrt{~}$-{\thgloss}-{\pst}-1{\sg}\\
		& \armenian{չկայիմ} & \armenian{չունիմ} & \\
		\lspbottomrule
	\end{tabular}	
\end{table}


Note that the past markers are the ones used for the subjunctive past. But for these defective verbs, the meaning can be perfective as in the following examples (\ref{sent:defective past}).

\begin{exe}
	\ex \label{sent:defective past}
	\begin{xlist}
		\ex \gll {kɒtu} {un-$\emptyset$-i-m}
		\\
		cat have-{\thgloss}-{\pst}-1{\sg}
		\\
		\trans 		`I had a cat.' \hfill (NK)
		\\
		\armenian{Կատու ունիմ։}
		\ex \gll {t͡ʃɒʃ} {k-ɒ-$\emptyset$-ɻ}
		\\
		food exist-{\thgloss}-{\pst}-3{\sg}
		\\
		\trans `There was food.'\hfill (NK)
		\\
		\armenian{Ճաշ կար։}
	\end{xlist}
	
\end{exe}


For the verb `to have', all other tenses are expressed by using the regular inchoative verb [{{un-e-n-ɒ-l}}] `to have; own'. %{\added}
For the verb `to exist', other tenses are expressed by using the verb `to be' (\ref{sent:Verb:Def:Exist:Fut}). 

\begin{exe}
\ex \label{sent:Verb:Def:Exist:Fut} %{\added}
 \begin{xlist}
	\ex \gll 
	t͡ʃɒʃ k-el-n-i-$\emptyset$ \\
	food {\fut}-be-{\vx}-{\thgloss}-3{\sg} \\
	\trans `There will be food.' \\
	\armenian{Ճաշ կէլնի։}
	\ex \gll 
t͡ʃɒʃ piti el-n-i-$\emptyset$ \\
food must be-{\vx}-{\thgloss}-3{\sg} \\
\trans `There will be food.' \\
\armenian{Ճաշ պիտի էլնի։}
\end{xlist}	
\end{exe}

%{\added}
Another defective verb is the word for `to be worth', but it is quite restricted in use (\ref{sent:defective:worth}). It has two main functions: to say how much some item is worth or costs, and as part of a social phrase. For {\seaSEA}, it is restricted to the indicative present and past imperfective, but synthetically. It is usable for any person-number combination. However for {\iaIA}, it seems to be mainly used for the third person, and we have not been able to successfully elicit it for other persons. Our online paradigms show  all the possible persons (as they would hypothetically be constructed), but it is possible  there there are paradigm gaps. 



\begin{exe}
	\ex \label{sent:defective:worth}
	\begin{xlist}
		\ex \gll hiŋɡ dolɒɻ ɒɻʒ-i-$\emptyset$ \\
		five dollar worth-{\thgloss}-3{\sg} \\ 
		\trans `It's worth five dollars.' \\
		\armenian{Հինգ դոլար արժի։}
		\ex \gll t͡ʃʰ-ɒɻʒ-i-$\emptyset$ \\
		{\neggloss}-worth-{\thgloss}-3{\sg} \\
		\trans Literal translation: `It's not worth it.' \\
		Functional translation: `You're welcome.' \\
		\armenian{չարժի}
	\end{xlist}
	
\end{exe}


{\seaSEA} has a few additional defective verbs (Table \ref{tab:defective:loss}). But in {\iaIA}, these have either been replaced or are not used in general.\footnote{It is difficult to be sure if these verbs truly do not exist in {\iaAbbre} because of diglossia between {\iaAbbre} and {\seaAbbre}.}

\begin{table}
	\caption{Loss of defective verbs from {\seaSE} to {\iaIA}}\label{tab:defective:loss}
	\begin{tabular}{llll}
		\lsptoprule
		\multicolumn{3}{l}{Defective in {\seaSE}} & Status in {\iaIA} \\
		$\sqrt{~}$-{\thgloss}-1{\sg} & & & \\\midrule
		{hus-ɑ-m} & `I hope' &\armenian{հուսամ}& does not exist\\
		{ɡit-e-m}&`I know' & \armenian{գիտեմ} & replaced by\\
		& & & inchoative [{im-ɒ-n-ɒ-l}] \armenian{իմանալ} \\
		\lspbottomrule
	\end{tabular}
\end{table}


\subsection{Other irregular verbs}\label{section:verb:irregular:other}
This section discusses verbs that have some irregularity in their conjugation, but do not neatly fit into the previous categories. 

Two irregular verbs in Table \ref{tab:irreg imp} are conjugated as regular E-Class verbs in most of the paradigm except for the imperative 2SG (and prohibitive 2SG). 

\begin{table}
	\caption{E-Class verbs that are irregular in only the imperative 2SG}\label{tab:irreg imp}
	\begin{tabular}{llll}
		\lsptoprule
		&`to say' &`to bring' & \\\midrule
		Infinitive & {ɒs-e-l}& {beɻ-e-l} &$\sqrt{~}$-{\thgloss}-{\infgloss}\\
		& \armenian{ասել} & \armenian{բերել} & \\
		Imperative & {ɒs-ɒ} &{beɻ} &$\sqrt{~}$(-{\imp}.2{\sg})\\
		& \armenian{ասա} & \armenian{բեր} & \\ 
		\lspbottomrule
	\end{tabular}
\end{table}

The verb `to bring' has an irregular imperative 2SG also in {\seaAbbre} \textit{beɾ} and  in {\swaAbbre} \textit{pʰeɾ}. The verb `to say' has an irregular imperative 2SG in   {\seaAbbre} \textit{ɑs-ɑ} but not in {\swaAbbre} \textit{əs-e} $\sqrt{}$-{\thgloss}. 

Among inchoative verbs (Table \ref{irregular incoatives}), the verb [{{dɒr-n-ɒ-l}}] has some irregularities. Before the nasal inchoative suffix, the rhotic surfaces as a trill /r/. But before the aorist suffix, the rhotic is a retroflex approximant /ɻ/.\footnote{We have gotten some contradictory information from some informants. It is possible that some more innovative speakers use a retroflex /ɻ/ throughout this verb's paradigm, while other more conservative speakers have the /r/-/ɻ/ change as we describe above. Note that this verb has an irregular imperative 2SG in {\seaSEA}: [{{dɑɾt͡sʰ}}]. In {\iaIA}, the imperative 2SG is regular.} The inchoative `to wash' is irregular because it uses the past T marker /{i}/ in the past perfective. Its imperative 2SG is likewise irregular.\footnote{The origin of the imperative 2SG of `to wash' is likely from the synonymous A-Class verb [{{ləv-ɑ-l}}] which exists in {\seaSEA} but not {\iaIA}.} One can argue this verb is actually heteroclitic (= mixed) with the A-Class because its past perfective and imperative pattern with the A-Class instead of with inchoatives.

\begin{table}
\caption{Two irregular inchoatives against the regular inchoative `to become happy'}\label{irregular incoatives}
\resizebox{\textwidth}{!}{%
\begin{tabular}{l lll }
	\lsptoprule
	&Infinitive & Past Pfv. 1PL & Imperative 2SG\\\midrule
	`to become happy' & {uɻɒχ-ɒ-n-ɒ-l}& {uɻɒχ-ɒ-t͡sʰ-ɒ-ŋkʰ}&{uɻɒχ-ɒ-t͡sʰ-i}
	\\
	&$\sqrt{~}$-{\lvgloss}-{\inch}-{\thgloss}-{\infgloss}&$\sqrt{~}$-{\lvgloss}-{\aorperf}-{\pst}-1{\pl}&$\sqrt{~}$-{\lvgloss}-{\aorother}-{\imp}.2{\sg}
	\\
	& \armenian{ուրախանալ} & \armenian{ուրախացանք}& \armenian{ուրախացի}
	\\ \addlinespace 
	`to turn into'& {dɒr-n-ɒ-l}&{dɒɻ-t͡sʰ-ɒ-ŋkʰ}& {dɒɻ-t͡sʰ-i}
	\\
	&$\sqrt{~}$-{\inch}-{\thgloss}-{\infgloss}&$\sqrt{~}$-{\aorperf}-{\pst}-1{\pl}&$\sqrt{~}$-{\aorother}-{\imp}.2{\sg}
	\\
	& \armenian{դառնալ} & \armenian{դարձանք}& \armenian{դարձի}
	\\ \addlinespace 
	`to wash'&{ləv-ɒ-n-ɒ-l} & {ləv-ɒ-t͡sʰ-i-ŋkʰ}
	& {ləv-ɒ}
	\\
	&$\sqrt{~}$-{\lvgloss}-{\inch}-{\thgloss}-{\infgloss}&$\sqrt{~}$-{\lvgloss}-{\aorperf}-{\pst}-1{\pl}&$\sqrt{~}$-{\lvgloss}
	\\	& \armenian{լուանալ} & \armenian{լուացինք}& \armenian{լուա}
	\\ \lspbottomrule
\end{tabular}}
\end{table}


There is evidence that {\iaIA} has leveled out some irregularities in verbal inflection (Table \ref{irregular loss}). The following verbs are irregular in {\seaSEA} but they either a) are regular verbs in {\iaIA}, or b) have been replaced by regular verbs in {\iaIA}.

\begin{table}[H]
	\caption{Loss of irregulars in {\iaIA}, relative to {\seaSE}}\label{irregular loss}
	
	{%\resizebox{\textwidth}{!}{%
			\begin{tabular}{lllll}
				\lsptoprule
				\multicolumn{2}{l}{Irregular in {\seaAbbre}} &  \multicolumn{3}{l}{Status in {\iaAbbre}} \\\midrule
				{zɑɾk-e-l} &`to hit' & replaced by E-Class &{χəpʰ-e-l} & `to hit'\\
				\armenian{զարկել} && & \armenian{խփել} & \\
				{l-ɑ-l} &`to cry'& replaced by E-Class &{lɒt͡sʰ-e-l} &`to cry'\\
				\armenian{լալ} & && \armenian{լացել} & \\
				{bɑt͡sʰ-e-l} &`to open' & regularized E-Class & & \\
				\armenian{բացել} & & & & \\
				{ken-ɑ-l}&`to stand'& replaced by E-Class &{kɒŋɡ(ə)n-e-l}&`to stand'\\
				\armenian{կենալ} & && \armenian{կանգնել} & \\\lspbottomrule
			\end{tabular}}
	\end{table}
	
	One convoluted case involves the {\seaSE} words [{lin-e-l}] `to be' and [{jel-n-e-l}] `to get up' or `to go up' (Table \ref{irregular shift linel}). The first is suppletive; the second is an infixed verb. In {\iaIA}, the form of the second verb is used as the verb `to be', without an initial glide: [{el-n-e-l}]. The meaning of `to get up' or `to go up' is periphrastic with another verb.
	
	\begin{table}
		\caption{Lexical shift from {\seaSE} to {\iaIA}}\label{irregular shift linel}
		\resizebox{\textwidth}{!}{%
			\begin{tabular}{lllll}
				\lsptoprule
				&\multicolumn{2}{c}{\seaAbbre} & \multicolumn{2}{c}{\iaAbbre}\\\cmidrule(lr){2-3}\cmidrule(lr){4-5}
				&`to be' &`to get up' &`to be' &`to get up'\\\midrule
				Infinitive&{lin-e-l}&{jel-n-e-l}&{el-n-e-l} &{etʰ-ɒ-l veɻev}\\
				&$\sqrt{~}$-{\thgloss}-{\infgloss}&$\sqrt{~}$-{\vx}-{\thgloss}-{\infgloss}&$\sqrt{~}$-{\vx}-{\thgloss}-{\infgloss}&$\sqrt{~}$-{\thgloss}-{\infgloss} up\\
				& \armenian{լինել} & \armenian{ելնել}& \armenian{էլնել} &\armenian{էթալ վերեւ}\\
				\addlinespace
				Past Pfv. 1PL&{jeʁ-ɑ-ŋkʰ}&{jel-ɑ-ŋkʰ}&{el-ɑ-ŋkʰ}&{ɡən-ɒ-t͡sʰ-i-ŋkʰ veɻev}\\
				&$\sqrt{~}$-{\pst}-1{\pl}&$\sqrt{~}$-{\pst}-1{\pl}&$\sqrt{~}$-{\pst}-1{\pl}&$\sqrt{~}$-{\thgloss}-{\aorperf}-{\pst}-1{\pl} up\\
				& \armenian{եղանք} & \armenian{ելանք}& \armenian{էլանք} &\armenian{գնացինք վերեւ}\\ 
				\lspbottomrule
			\end{tabular}%
		}
	\end{table}
	
	We have noted some degree of optional heteroclisis \citep{stump-2006-heteroclisisParadigmLinkage}, meaning that a verb changes its conjugation class in some paradigm cells (Table \ref{irregular heteroclisis}). Consider the common verb \textit{{siɻ-e-l}} `to like'. This verb is primarily a regular E-Class verb and is inflected as such. But in the past perfective, some speakers conjugate the verb as E-Class and some as A-Class. NK sometimes produced perfective converbs with the aorist stem, following the A-Class pattern. 
	
\begin{table}
\caption{Variable aorist stem as a form of heteroclisis}\label{irregular heteroclisis}
\resizebox{\textwidth}{!}{%
	\begin{tabular}{llll}
		\lsptoprule
		&`to like' & & \\\midrule
		Infinitive&{sir-e-l}&$\sqrt{~}$-{\thgloss}-{\infgloss} & \armenian{սիրել}\\
		Past Pfv. 1SG& {siɻ-ɒ-m}&$\sqrt{~}$-{\pst}-1{\sg} & \armenian{սիրամ}\\
		& {siɻ-ɒ-t͡sʰ-i-m}&$\sqrt{~}$-{\thgloss}-{\aorperf}-{\pst}-1{\sg} & \armenian{սիրացիմ}\\ \addlinespace
		Pfv. converb& {siɻ-el, siɻ-eɻ}&$\sqrt{~}$-{\perfcvb} & \armenian{սիրել, սիրեր}\\
		& {siɻ-ɒ-t͡sʰ-el, siɻ-ɒ-t͡sʰ-eɻ}&$\sqrt{~}$-{\thgloss}-{\aorother}-{\perfcvb} & \armenian{սիրացել, սիրացեր}\\ 
		\lspbottomrule
	\end{tabular}}	
\end{table}
	
	Some speakers consider the A-Class forms to be normal, but others perceive them as ``done in jest.'' It is difficult to tell if this is genuine inter-speaker variation, or if it is due to hyper-correction from {\seaSEA}.
	
	Another possible case of heteroclisis that we found was for the A-Class verb [χos-ɒ-l] `to speak'. NK inflects this as an A-Class almost always, but sometimes she produced an imperative 2PL that followed the E-Class pattern [χos-ekʰ] $\sqrt{}$-{\imp}.2{\pl} instead of the A-Class pattern [χos-ɒ-t͡sʰ-ekʰ] $\sqrt{}$-{\thgloss}-{\aorother}-{\imp}.2{\pl}. She likewise once produced an E-Class infinitive [χos-e-l] instead of [χos-ɒ-l]. Obviously, more data is needed to see the extent of lexical or speaker variation in such mixing of conjugation classes. It is possible that such class changes are a form of dialect-mixing between {\iaIA} and {\seaSEA}. 
