\chapter{Nominal morphology}\label{chapter:noun}


This chapter covers   the basics of nominal inflection in {\iaIA}. In general, we have not found any significant differences between {\seaSE} and {\iaIA} in this domain. We thus keep this chapter brief, with an overview of the basic paradigms. For larger paradigms and for work on the noun phrase of Armenian, we refer readers to other sources for {\seaSEA} (\cites{Kozintseva-1995-EasternBook}{Yeghiazaryan-2010-ArmenianCase}[\S4]{Tamrazian-1994-ArmenianSyntax}[\S5]{Megerdoomian-2009-ThesisBook}[\S2.1]{DumTragut-2009-ArmenianReferenceGrammar}[\S2.1.1]{Hodgson-2019-DissRelativeClauseArmenianSyntax}) and {\swaSWA} (\cites{Sigler-1997-SpecificitySWADissertation}[\S2.3]{Khanjian-2013-DissNegativeConcord}{BaleKhanjian-2014-SyntacticComplexityCompetitionSingularPluralWesternArmenian}).


\section{Basic template for nominal inflection}\label{section:noun:template}

Nominal inflection is agglutinative for number, case, possession, and definite marking. The basic template for   nominal inflection is given in Table \ref{tab:template noun}. The rightmost column is dedicated to possessive and definiteness marking, which we refer to collectively as a Determiner slot. We list productive suffixes within each cell, including suffixal allomorphs.

\begin{table}
	\caption{Template for nominal inflection and the set of productive suffixes}
	\label{tab:template noun}
	\begin{tabular}{ l lllll lll l  }
		\lsptoprule 
		N & \multicolumn{3}{c}{Number} & \multicolumn{3}{c}{Case ({\case})} & \multicolumn{3}{c}{Determiner ({\detgloss})}\\ \cmidrule(lr){2-4}\cmidrule(lr){5-7}\cmidrule(lr){8-10}
		& {\sg} & -$\emptyset$& & {\nom}/{\acc} &-$\emptyset$& & unmarked & -$\emptyset$&\\
		& {\pl} & {-eɻ} & \armenian{-եր} & {\dat}/{\gen} & {-i}& \armenian{-ի} & {\possFsg}& {-(ə)s}& \armenian{-ս}\\
		&& {-neɻ} & \armenian{-ներ} & {\abl}& {-it͡sʰ}& \armenian{-ից} & {\possSsg}& {-(ə)t} & \armenian{-դ}\\ 
		& & && {\ins}& {-ov} &\armenian{-ով}& {\defgloss} & {-ə}& \armenian{-ը}\\
		&& & & {\locgloss}& {-um}& \armenian{-ում} & & {-n}& \armenian{-ն}\\
		& & & & & & & & {-ən}&\armenian{-ն}\\
		\lspbottomrule 
	\end{tabular}
\end{table}


Some of the above morphemes have multiple realizations due to pho\-nol\-o\-gi\-cal\-ly-con\-di\-tioned allomorphy. Such allomorphy is discussed in \S\ref{section:morphophono:allomorphy}.


To illustrate nominal inflection, we show the paradigms of a singular case-marked noun, a plural case-marked noun, and a plural case-marked possessed noun (Table \ref{tab:noun basic paradigm}). Note that possessive marking follows   case marking. 

\begin{table}
	\caption{Paradigm for singular noun, plural noun, and plural possessed noun}
	\label{tab:noun basic paradigm}
	\begin{tabular}{ ll l l }
		\lsptoprule 
		& N-{\case} & N-{\pl}-{\case} & N-{\pl}-{\case}-{\possFsg}\\\midrule
		{\nom}/{\acc} & {senjɒk}& {senjɒk-neɻ} & {senjɒk-neɻ-əs}\\
		&\armenian{սենեակ} &\armenian{սենեակներ} &\armenian{սենեակներս}\\
		\addlinespace 
		{\dat}/{\gen} & {senjɒk-i} & {senjɒk-neɻ-i} & {senjɒk-neɻ-i-s} \\
		&\armenian{սենեակի}&\armenian{սենեակների}&\armenian{սենեակներիս}\\
		\addlinespace 
		{\abl} & {senjɒk-it͡sʰ} & {senjɒk-neɻ-it͡sʰ} & {senjɒk-neɻ-it͡sʰ-əs}\\
		&\armenian{սենեակից}
		&\armenian{սենեակներից}
		&\armenian{սենեակներիցս}\\
		\addlinespace 
		{\ins} & {senjɒk-ov} & {senjɒk-neɻ-ov} & {senjɒk-neɻ-ov-əs}\\
		&\armenian{սենեակով}
		&\armenian{սենեակներով}
		&\armenian{սենեակներովս}\\
		\addlinespace 
		{\locgloss} & {senjɒk-um} & {senjɒk-neɻ-um} & {senjɒk-neɻ-um-əs} \\
		&\armenian{սենեակում}
		&\armenian{սենեակներում}
		&\armenian{սենեակներումս}\\
		\addlinespace 
		& `room' &`rooms'&`my rooms'\\
		\lspbottomrule
	\end{tabular}
	
\end{table}

In {\seaSEA}, the word for `case' is /holov/ \armenian{հոլով}. The names of the different cases are in Table \ref{tab:case name}.\largerpage

\begin{table}
	\caption{Names of cases in {\seaSEA}}
	\label{tab:case name}
	\begin{tabular}{lll}
		\lsptoprule
		Nominative & uʁʁɑkɑn  &  \armenian{ուղղական}\\
		Accusative &hɑjt͡sʰɑkɑn  &  \armenian{հայցական}\\
		Genitive &  serɑkɑn & \armenian{սեռական}\\
		Dative &  təɾɑkɑn  & \armenian{տրական}\\
		Ablative &  bɑt͡sʰɑrɑkɑn & \armenian{բացառական}\\
		Instrumental &  ɡoɾt͡sijɑkɑn  & \armenian{գործիական}\\
		Locative & neɾɡojɑkɑn  &  \armenian{ներգոյական}\\ 
		\lspbottomrule  
		\end{tabular}
\end{table}

In terms of syncretism and exponence, nominative and accusative are zero-marked, singular number is unmarked, and dative and genitive are syncretic for common nouns. However, this syncretism     does not apply to personal pronouns, which we discuss in \S\ref{section:funct:personal pronoun}.

{\seaSEA} can use  the instrumental case marker \textit{-ov} to denote either the meaning of `to use X as an instrument' or `to go along with X'. The latter meaning is considered a comitative meaning \citep[93]{DumTragut-2009-ArmenianReferenceGrammar}. {\swaSWA} can likewise use the instrumental as a comitative. However in {\iaIA}, the comitative meaning of the instrumental suffix is considered atypical and odd. Speakers prefer to express the comitative meaning through an alternative postpositional construction.\footnote{%{\added}
	However, a reviewer  states that a possibly more accurate description of {\seaAbbre} is that the instrumental can be used for activities that are carried out as a group (for example as a family), and not alongside a person. If we take this description of {\seaAbbre} as accurate, then both {\seaAbbre} and {\iaAbbre} lack comitative instrumentals, while {\swaAbbre} has them. However, KM did report that she encountered such comitative readings in {\seaAbbre} before, so it is possible that there is variation among {\seaAbbre} speakers. Our {\seaAbbre} consultant AT said that such a comitative  reading is ``not okay'' but that it is possible that someone might use it in a disparaging way, e.g., a misogynist might use the comitative instrumental of the word `sister'. }

For example, sentence (\ref{sentence inst comitative no}) places an instrumental suffix on the noun. The intended interpretation is comitative: to go along with the sister. Such a meaning is possible for some speakers in {\seaSEA}, but not in {\iaIA}. The typical {\iaIA} reading would be purely instrumental: to go to the cinema by using the sister. To express the comitative meaning, speakers strongly prefer using the postposition \textit{het} (\ref{sentence inst comitative yes}).\footnote{For the word `sister', the nominative form is [{kʰuɻ}] \armenian{քուր}. In the dative/genitive, the word uses an irregular allomorph for both the root and the suffix: [{kʰəɻ-ot͡ʃʰ}]. The dative/genitive stem is then further inflected to form the instrumental. Note that the prescriptive form of the irregular dative/genitive suffix is [{-od͡ʒ}], but in {\iaIA} it is more often pronounced as [{-ot͡ʃʰ}].}\largerpage

\begin{exe}
	\ex 
	\begin{xlist}
		\ex \gll {kʰəɻ-ot͡ʃʰ-ov-əs} {ɡən-ɒ-t͡sʰ-i-ŋkʰ} {sinemɒ} 
		\\
		sister-{\dat}-{\ins}-{\possFsg} go-{\thgloss}-{\aorperf}-{\pst}-1{\pl} cinema
		\\
		\trans 	Intended meaning: `We went to the cinema along with my sister.' \label{sentence inst comitative no}
		\\
		Actual meaning: `We went to the cinema by using my sister.'
		\\
		\armenian{Քրոջովս գնացինք սինեմա։} \hfill (KM)
		\ex \gll {kʰəɻ-ot͡ʃʰ-əs} {het} {ɡən-ɒ-t͡sʰ-i-ŋkʰ} {sinemɒ}
		\\
		sister-{\gen}-{\possFsg} with go-{\thgloss}-{\aorperf}-{\pst}-1{\pl} cinema
		\\
		\trans 	`We went to the cinema along with my sister.'\label{sentence inst comitative yes}
		\\
		\armenian{Քրոջս հետ գնացինք սինեմա։} \hfill (KM)
	\end{xlist}
\end{exe}



The   suffixes in Table \ref{tab:template noun} are the  regular  or default suffixes for the corresponding morphosyntactic features. {\iaIA} has limited     morphologically-conditioned allomorphy with irregular suffixes. We have not found any significant differences for irregular inflection in {\iaIA} vs. {\seaSEA}. At most, it seems that {\iaIA} is slowly leveling out irregular inflection.



To illustrate, the regular dative/genitive suffix is \textit{-i}. In both {\seaSE} and {\iaIA}, the dative/genitive suffix has a wide set of irregular allomorphs or realizations. For example, the suffix \armenian{-ութիւն} /{-utʰjun}/ is a productive nominalizer (\ref{sent:Noun:BasicTemplate:utʰjun}). This suffix forms an irregular dative/genitive by using a different allomorph for the entire nominalizer suffix: \armenian{-ութեան} /{-utʰjɒn}/. The use of this    allomorph is the prescriptive rule in {\seaSE} and {\iaIA}, but KM reports that {\iaIA} speakers much more frequently apply a regularized form /{-utʰjun-i}/. 



\begin{exe}
	\ex \textit{Leveling out of irregular dative/genitive of /{-utʰjun}/}
	\label{sent:Noun:BasicTemplate:utʰjun}
	
	\resizebox{\linewidth}{!}{%
		\begin{tabular}{@{}l@{~}llll@{}}
			a. & {uɻɒχ} & & `happy' & \armenian{ուրախ} \\
			& {uɻɒχ-utʰjun} & happy-{\nmlz} & `happiness' & \armenian{ուրախութիւն} \\
			b. & {uɻɒχ-utjɒn} & happy-{\nmlz}.{\dat}/{\gen} & `to/of happiness' & \armenian{ուրախութեան} \\
			c. & {uɻɒχ-utʰjun-i} & happy-{\nmlz}-{\dat}/{\gen} & `to/of happiness' &\armenian{ուրախութիւնի} \\
		\end{tabular}
	}
\end{exe}

For complete paradigms of these irregular declensions in {\seaSEA}, see \citet[\S2.1.2]{DumTragut-2009-ArmenianReferenceGrammar}. These paradigms apply to the formal prescriptive speech of {\iaIA}s.  But in casual speech, KM and AS report the loss of   various irregular case suffixes. 
\section{Constraints on definite marking and case marking}\label{section:noun:caseDefConstraint}


The determiner slot can be realized by either nothing, the 1SG possessive, 2SG possessive, or the definite suffix. The 1SG possessive  and 2SG possessive  can follow any type of case marker. This was illustrated in section \S\ref{section:noun:template} in Table \ref{tab:noun basic paradigm} for the 1SG possessive. However, the definite suffix cannot follow the genitive, ablative, or instrumental (\citealt[104]{DumTragut-2009-ArmenianReferenceGrammar}, \citealt[7]{Yeghiazaryan-2010-ArmenianCase}, \citealt[48]{Hodgson-2019-DissRelativeClauseArmenianSyntax}, \citeyear{Hodgson-202x-GrammaticaliztionDefiniteARticleArmenian}).\footnote{%{\added}
	We treat the definite and possessive morphemes as suffixes and not clitics. Morphosyntactically, there is no obvious evidence for treating them as separate words (clitics) instead of suffixes. Phonologically, these morphemes are unstressed (like clitics). But because these morphemes lack a non-schwa vowel, a suffix account already correctly predicts that they are unstressable (\S\ref{section:phono:suprasegmental:stress:reg}).   }


To illustrate, Table \ref{tab:def case restriction} shows definite marking on singular   and plural nouns. For the genitive, ablative, and instrumental, the noun is semantically ambiguous in terms of being definite or not. %{\added}
The gloss {\case} is a placeholder for case marking.\footnote{%{\added}
	The morpheme sequence of instrumental-definite is judged as ungrammatical by NK. In {\seaSEA} it is also judged as odd. However, BV found around 29 instances of this morpheme sequence as [senjɑk-um-ə] \armenian{սենյակումը} `in the room'   on the EANC. Victoria Khurshudyan reported that such a sequene can be uttered, ``but it will be clearly perceived as a non-standard form.''  }


\begin{table}
	\caption{Paradigm of definite singular noun and definite plural noun}
	\label{tab:def case restriction}
	\begin{tabular}{ll lll }
		\lsptoprule 
		& \makebox[.2cm][l]{}N-{\case}-{\defgloss} && \makebox[.2cm][l]{}N-{\pl}-{\case}-{\defgloss} & \\
		\midrule 
		{\nom}/{\acc} & \makebox[.2cm][l]{}{senjɒk-ə}&\armenian{սենեակը} & \makebox[.2cm][l]{}{senjɒk-neɻ-ə}&\armenian{սենեակնեըր}\\
		{\dat} & \makebox[.2cm][l]{}{senjɒk-i-n} &\armenian{սենեակին}& \makebox[.2cm][l]{}{senjɒk-neɻ-i-n} &\armenian{սենեակներին}\\
		{\gen} & \makebox[.2cm][l]{}{senjɒk-i} &\armenian{սենեակի}& \makebox[.2cm][l]{}{senjɒk-neɻ-i} &\armenian{սենեակների}\\
		& \makebox[.2cm][l]{*}{senjɒk-i-n} & & \makebox[.2cm][l]{*}{senjɒk-neɻ-i-n} & \\
		{\abl} & \makebox[.2cm][l]{}{senjɒk-it͡sʰ} &\armenian{սենեակից}& \makebox[.2cm][l]{}{senjɒk-neɻ-it͡sʰ} &\armenian{սենեակներից}\\
		& \makebox[.2cm][l]{*}{senjɒk-it͡sʰ-ə} & & \makebox[.2cm][l]{*}{senjɒk-neɻ-it͡sʰ-ə} & \\
		{\ins} & \makebox[.2cm][l]{}{senjɒk-ov} &\armenian{սենեակով}& \makebox[.2cm][l]{}{senjɒk-neɻ-ov} &\armenian{սենեակներով}\\
		& \makebox[.2cm][l]{*}{senjɒk-ov-ə} & & \makebox[.2cm][l]{*}{senjɒk-neɻ-ov-ə}& \\
		{\locgloss} & \makebox[.2cm][l]{}{senjɒk-um} &\armenian{սենեակում}& \makebox[.2cm][l]{}{senjɒk-neɻ-um}&\armenian{սենեակներում}\\
		& \makebox[.2cm][l]{*}{senjɒk-um-ə} & & \makebox[.2cm][l]{*}{senjɒk-neɻ-um-ə} & \\
		\midrule 
		& `the room' &&`the rooms'&\\
		\lspbottomrule
	\end{tabular}
\end{table}

It is interesting that the dative and genitive are syncretic with the suffix \textit{{-i}}. However, the definite suffix can be used after the dative form, but not the genitive form. This is illustrated in the following sentences.

In sentence (\ref{sentence:dative def}), the suffix \textit{{-i}} marks dative case. It can take the definite suffix \textit{{-n}}. But in (\ref{sentence:gen def}), the suffix \textit{{-i}} marks genitive case. It cannot be followed by the definite suffix.

\begin{exe}
	\ex 
	\begin{xlist}
		\ex \gll {senjɒk-i-n} {ɡiɻkʰ} {təv-ɒ-m}
		\\
		room-{\dat}-{\defgloss} book give-{\pst}-1{\sg}
		\\
		\trans	`I gave books to the room.' \hfill (NK)\label{sentence:dative def}
		\\
		\armenian{Սենեակին գիրք տուամ։}
		
		
		\ex \gll {senjɒk-i(*-n)} {ɡujn-ə}
		\\
		room-{\gen}-*{\defgloss} color-{\defgloss}
		\\
		\trans			`the color of the room' \hfill (*NK) \label{sentence:gen def}
		\\
		\armenian{սենեակի գոյնը}	\end{xlist}
	
\end{exe}

This co-occurrence restriction applies equally to both non-human nouns and to human nouns, such as the given name Aram (\ref{sent:Noun:Def:cooccurence}). 

\begin{exe}
	\ex \label{sent:Noun:Def:cooccurence}
	\begin{xlist}
		\ex \gll {ɒɻɒm-i-n} {ɡiɻkʰ} {təv-ɒ-m}
		\\
		Aram-{\dat}-{\defgloss} book give-{\pst}-1{\sg}
		\\
		\trans 	`I gave books to Aram.'\hfill (NK)
		\\
		\armenian{Արամին գիրք տուամ։}
		
		
		\ex \gll {ɒɻɒm-i(*-n)} {ɡiɻkʰ-ə}
		\\
		Aram-{\gen}-*{\defgloss} book-{\defgloss}
		\\
		\trans 			`the book of Aram'\hfill (NK)
		\\
		\armenian{Արամի գիրքը}
	\end{xlist}	
\end{exe}





The co-occurrence restriction between the genitive and the definite suffix is limited to just the definite suffix (\ref{sent:Noun:Def:cooccurencePoss}). Other determiner suffixes like the 1SG possessive can freely co-occur with either the dative \textit{{-i}} or the genitive \textit{{-i}}.


\begin{exe}
	\ex \label{sent:Noun:Def:cooccurencePoss}
	\begin{xlist}
		\ex \gll {senjɒk-i-s} {ɡiɻkʰ} {təv-ɒ-m}
		\\
		room-{\dat}-{\possFsg} book give-{\pst}-1{\sg}
		\\
		\trans	`I gave books to my room.'\hfill (NK)
		\\
		\armenian{Սենեակիս գիրք տուամ։}
		
		\ex \gll {senjɒk-i-s} {ɡujn-ə}
		\\
		room-{\gen}-{\possFsg} color-{\defgloss}
		\\
		\trans `the color of my room'\hfill (NK)
		\\
		\armenian{սենեակիս գոյնը}
	\end{xlist}
\end{exe}


%{\added}
The definite suffix has an additional function of helping to mark third person possessives. This is discussed in the following section. 


\section{Constraints on possessive marking}\label{section:noun:PossMarking}
The determiner slot can be occupied by either the possessive suffixes or the definite suffix. There are likewise co-dependencies between this slot and the possessive pronouns.

{\iaIA} has a set of 8 genitive/possessive pronouns which mark possession. The 3SG and 3PL each have two members. One member is intensive or emphatic, while the other member is non-intensive or non-emphatic. This is discussed in \S\ref{section:funct:personal pronoun}. 


If a noun is possessed by the first person singular, then the noun can surface in one of three forms (\ref{sent:Noun:Poss:1sg}). It can surface without a possessive pronoun and with the 1SG possessive suffix. Or, it can surface with the possessive pronoun and the 1SG possessive suffix. Or, it can   surface with the possessive pronoun but with the definite suffix. Similar options are found for 2SG   possessives (\ref{sent:Noun:Poss:2sg}).\pagebreak


\begin{exe}
	\ex 
	
	\begin{xlist}
		\ex \textit{Variation in 1SG possessive marking}\label{sent:Noun:Poss:1sg}
		
		\begin{tabular}{llll}
			a. & & {senjɒk-\textbf{əs}}& \armenian{սենեակս}
			\\
			& &room-{\possFsg}&\\
			b. &{\textbf{im}}& {senjɒk-\textbf{əs}} & \armenian{իմ սենեակս}
			\\
			& my& room-{\possFsg}&
			\\
			c. &{\textbf{im}}& {senjɒk-\textbf{ə}} &\armenian{իմ սենեակը}
			\\
			& my& room-{\defgloss}&
			\\
			&\multicolumn{3}{l}{`my room'}
		\end{tabular}
		\ex \textit{Variation in 2SG possessive marking}\label{sent:Noun:Poss:2sg}
		
		\begin{tabular}{llll}
			a. & & {senjɒk-\textbf{ət}}& \armenian{սենեակդ}
			\\
			& &room-{\possSsg}&\\
			b. &{\textbf{kʰo}}& {senjɒk-\textbf{ət}} &\armenian{քո սենեակս}
			\\
			& my& room-{\possSsg}&
			\\
			c. &{\textbf{kʰo}}& {senjɒk-\textbf{ə}} &\armenian{քո սենեակը}
			\\
			& my& room-{\defgloss}&
			\\
			&\multicolumn{3}{l}{`your room'}
		\end{tabular}
	\end{xlist}
	
\end{exe}

Sociolinguistically, the simultaneous use of the possessive pronoun and the possessive suffix is deemed   prescriptively incorrect for {\seaSEA} \citep[113]{DumTragut-2009-ArmenianReferenceGrammar}. The use of both the pronoun and the possessive suffix is instead restricted to colloquial speech and often stigmatized. But it is the preferred strategy for casual speech in {\iaIA}. 

For the other combinations of person and number, there is no dedicated possessive suffix (Table \ref{tab:Noun:Poss:other12sg}). Instead, the possessed noun takes the genitive/possessive pronoun and the definite suffix.


\begin{table}
	\caption{Possessive marking for person-number combinations beyond 1SG-2SG}\label{tab:Noun:Poss:other12sg}
	\begin{tabular}{lllll}
		\lsptoprule
		3SG & {iɻɒ} & {senjɒk-ə} & `his room'& \armenian{իրա սենեակը}\\
		& {nəɻɒ} & {senjɒk-ə} & `his room'&\armenian{նրա սենեակը}\\\addlinespace
		1PL & {meɻ} & {senjɒk-ə} &`our room'& \armenian{մեր սենեակը}\\\addlinespace
		2PL & {d͡zeɻ} & {senjɒk-ə} &`your.{\pl} room'& \armenian{ձեր սենեակը}\\\addlinespace
		3PL & {iɻɒnt͡sʰ} & {senjɒk-ə} &`their room'& \armenian{իրանց սենեակը}\\
		& {nəɻɒnt͡sʰ} & {senjɒk-ə} & `their room'&\armenian{նրանց սենեակը}\\\addlinespace 
		& {\pro}.{\gen} & room-{\defgloss} & & \\
		\lspbottomrule
	\end{tabular}
\end{table}




\section{Synthetic constructions for plural possessors}
When the noun has a plural possessor, the most typical construction is to use a genitive pronoun and the definite suffix (\ref {sent:PlPoss:Khur:analyic}). Both {\seaSE} and {\iaIA} allow a synthetic alternative that is very restricted in usage \citep[113--114]{DumTragut-2009-ArmenianReferenceGrammar}.  In {\seaSEA}, one can use the plural suffix   \textit{-neɾ} to encode a plural possessor (\ref {sent:PlPoss:Khur:synthetic}). 

\begin{exe}
	\ex {\seaSEA} (adapted from \citealt[339,340]{Khurshudian-2020-someAspectsPossessiveMarkersModernArmenian}) 
	\begin{xlist}
		\ex \gll  meɾ ɑt͡ʃʰkʰ-eɾ-ə, meɾ het-ə  \\
		us.{\gen} eye-{\pl}-{\defgloss}, us.{\gen} with-{\defgloss} \\ 
		\trans `our eyes, with us'  \label{sent:PlPoss:Khur:analyic}\\
		\armenian{մեր աչքերը, մեր հետը}
		\ex \gll   ɑt͡ʃʰkʰ-neɾ-əs, het-neɾ-əs \\
		eye-{\pl}-{\possFsg}, with-{\pl}-{\possFsg} \\ 
		\trans `our eyes, with us'   \label{sent:PlPoss:Khur:synthetic} \\
		\armenian{աչքներս, հետներս}
	\end{xlist}
\end{exe}

For SEA, note how the plural \textit{-neɾ} suffix is supposed to attach only to  polysyllabic stems, while the allomorph \textit{-eɾ} attaches to   monosyllables. But the suffix \textit{-neɾ}  is exceptionally used to mark plural possession on monosyllables in the above examples (\S\ref{section:morphophono:allomorphy: syll}). 

In {\swaSWA},  such constructions are   productive,  using     different morphological templates \citep{Vaux-2013-NumberPolysyllabicPoss,Bezrukov-2016-MA}.  In contrast in {\seaSEA}, the use of this synthetic construction for plural possessors is quite unproductive, and limited to      a small set of concepts, such as talking about one's body parts  `our eyes' or using an adposition `with us'. The {\seaAbbre}-style of plural possessives is   also attested in {\iaIA} (\ref{sent:PlPoss:IA}).\footnote{%{\added}
	In {\seaAbbre}, the prescriptive norm is that the postposition /het/ `with' assigns dative case to its argument. In contrast, {\seaCEAAbbre}  uses   genitive marking \citep[297--299]{DumTragut-2009-ArmenianReferenceGrammar}. {\iaAbbre} also uses   genitive marking.}


\begin{exe}
	\ex {\iaIA}  \label{sent:PlPoss:IA} 
	\begin{xlist}
		\ex \gll  meɻ ɒt͡ʃʰkʰ-eɻ-ə, meɻ het-ə  \\
		us.{\gen} eye-{\pl}-{\defgloss}, us.{\gen} with-{\defgloss} \\ 
		\trans `our eyes, with us'  \hfill (NK)\\
		\armenian{մեր աչքերը, մեր հետը}
		\ex \gll   ɒt͡ʃʰkʰ-neɻ-əs, het-neɻ-əs \\
		eye-{\pl}-{\possFsg}, with-{\pl}-{\possFsg} \\ 
		\trans `our eyes, with us' \hfill (NK)\\
		\armenian{աչքներս, հետներս}
	\end{xlist}
\end{exe}

This construction seems particularly common for body parts which come in pairs, like feet or eyes (\ref{sent:PlPoss:IAMoreBody}).\footnote{%{\added}
	AS reports that for the word `foot', the default form is /votkʰ/, as in [votkʰ-eɻ-əs]. However, the form /vot/ can be used as well: [vot-eɻ-əs]. However, he suspects that such a form is more permissible if the preceding genitive pronoun is singular and not plural. That is, this smaller form is used when there is no plural possessor: /im vot-eɻ-əs/ `my feet'. } 





\begin{exe}
	\ex {\iaIA}  \label{sent:PlPoss:IAMoreBody} 
	\begin{xlist}
		\ex \gll  meɻ votkʰ-eɻ-ə, meɻ d͡zer-eɻ-ə  \\
		us.{\gen} foot-{\pl}-{\defgloss}, us.{\gen} hand-{\pl}-{\defgloss} \\ 
		\trans `our feet, our hands'  \hfill (NK)\\
		\armenian{մեր ոտքերը, մեր ձեռերը}
		\ex \gll   votkʰ-neɻ-əs, d͡zer-neɻ-əs \\
		foot-{\pl}-{\possFsg}, hand-{\pl}-{\possFsg} \\ 
		\trans `our feet, our hands'  \hfill (NK)\\
		\armenian{ոտքներս, ձեռներս}
	\end{xlist}
\end{exe}

As in {\seaSEA}, this construction is restricted and unproductive in {\iaIA} (\ref{sent:PlPoss:IAweird}). NK found it odd  to add it to nouns that were for animals. 

\begin{exe}
	\ex {\iaIA}  \label{sent:PlPoss:IAweird}
	\begin{xlist}
		\ex \gll  meɻ muk-ə, meɻ kov-ə, meɻ kɒtu-n  \\
		us.{\gen} mouse-{\defgloss}, us.{\gen} cow-{\defgloss}, us.{\gen} cat-{\defgloss} \\ 
		\trans `our mouse, our cow, our cat'  \hfill (NK)\\
		\armenian{մեր մուկը, մեր կովը, մեր կատուն}
		\ex \gll  *muk-neɻ-əs, kov-neɻ-əs, kɒtu-neɻ-əs \\
		mouse-{\pl}-{\possFsg}, cow-{\pl}-{\possFsg}, cat-{\pl}-{\possFsg} \\ 
		\trans Intended: `our mouse, our cow, our cat'  \hfill (*NK)\\
		
	\end{xlist}
\end{exe}



\section{Differential object marking}\label{section:noun:DOM}\largerpage
For nouns in the subject position, nominative case is covert or zero. But in the object position, we see a distinction between nouns with human referents and nouns with non-human referents. Non-human nouns are not overtly marked for morphological case, i.e., they take covert accusative case. In contrast, human nouns in object position take dative \textit{{-i}} as a form of differential object marking. The same pattern occurs in {\seaSEA} (\cites[61]{DumTragut-2009-ArmenianReferenceGrammar}{scala-2011-differentialObjectMarkingEasternArmenian}) and the Iranian dialect of Maragha \citep[160]{Adjarian-1926-MaraghaDialect}.

To illustrate, consider the sentences in (\ref{sent:Noun:DOM:basic}). If the object is   non-human (\ref{sentence:obj nonhuman}), then the noun is unmarked for case.  If the object is   human, such as the given name Aram (\ref{sentence:obj human}), then the object must take dative case.  Our consultants felt that if the dative marker was absent (\ref{sentence:obj human def}), then the sentence reads as if Aram was a non-human entity.  

\begin{exe}
	\ex \label{sent:Noun:DOM:basic}
	\begin{xlist}
		\ex \gll {senjɒk-ə} {mɒkʰɻ-ɒ-m}
		\\
		room-{\defgloss} clean-{\pst}-1{\sg}
		\\
		\trans			`I cleaned the room.'\hfill (NK)\label{sentence:obj nonhuman}
		\\
		\armenian{Սենեակը մաքրամ։}
		\ex \gll {ɒɻɒm-i-n} {mɒkʰɻ-ɒ-m}
		\\
		Aram-{\dat}-{\defgloss} clean-{\pst}-1{\sg}
		\\
		\trans	`I cleaned Aram.'\hfill (NK)\label{sentence:obj human}
		\\
		\armenian{Արամին մաքրամ։}
		\ex \gll *{ɒɻɒm-ə} {mɒkʰɻ-ɒ-m}
		\\
		Aram-{\defgloss} clean-{\pst}-1{\sg}
		\\
		\trans	Intended: `I cleaned Aram'.
		\\ Actual: `I cleaned some entity called an ``Aram''.'\label{sentence:obj human def}
	\end{xlist}
	
\end{exe}

The above discussion focused on humans vs. inanimates. Differential object marking on animals is more complicated \citep[\S2.1.1.1]{DumTragut-2009-ArmenianReferenceGrammar}. 

\section{Indefinites and classifiers}\label{section:noun:indf}\largerpage
Like {\seaSEA}, {\iaIA} has grammaticalized the numeral `one' into an indefinite proclitic. {\iaIA} likewise utilizes a classifier [hɒt] for counting. The combination of the indefinite and classifier has some semantic and phonological idiosyncrasies \citep{Hodgson-2020-DiscourseConfigurationalityNounPhraseEasternArmenian,Sargsyan-2022-FormsIndefiniteArticleEasternArmenianPreModernEarlyColloquialEasternArmenianSources}.


The numeral `one' in {\iaIA} is [{mek}]. The \textit{{k}} segment is retained in the citation form (\ref{ex:mek}). But when the numeral is used as a modifier for a noun, the \textit{{k}} can be  dropped: \textit{{me rope}} `one minute' (\ref{ex:merope}).\footnote{For the word `minute', the rhotic is a flap [ɾope] in {\seaSEA}, but it is a trill in NK and KM's speech [rope] (\S\ref{section:phono:segmental:rhotic}).} The \textit{{me}} morph is also grammaticalized as an indefinite proclitic (\ref{meban}). It is spelled as \armenian{մի} <{mi}> because the {\seaSE} equivalent is [{mi}]. 


\begin{exe}
	\ex 
	\begin{xlist}
		
		\ex \gll	{mek}
		\\
		one
		\\
		\trans	`one' \hfill (KM)\label{ex:mek}
		\\
		\armenian{մէկ}
		\ex \gll	{mek/me} rope 
		\\
		one minute
		\\
		\trans	`one minute' \hfill (NK, KM)\label{ex:merope}
		\\
		\armenian{մէկ րոպէ}
		\ex \gll 	{me} bɒn
		\\
		{\indf} thing
		\\
		\trans	`A thing; something' \hfill (NK)\label{meban}
		\\
		\armenian{մի բան}
	\end{xlist}
\end{exe}

The indefinite morph /me/  is also the indefinite article in some of the traditional dialects of Iran (Khoy/Urmia: \citealt[84]{Asatryan-1962-KhoyUrmiaDialect}; Maragha: \citealt[1.78]{Adjarian-1926-MaraghaDialect}; and Salmast).\footnote{For Salmast, BV found an example of an indefinite /me/ in a newspaper article called  \armenian{Խայու Լաճ} from the periodical \armenian{Պսակ} (date  October 11, 1880, volume 30): \url{https://tert.nla.am/archive/NLA\%20TERT/Psak/1880/1880(30).pdf}} The \textit{mek/me} alternation could be connected to how in colloquial Persian, the word [yek] is used to mean the cardinal `one' while [ye] is used as an indefinite article   (\citealt[328]{Mahootian-2002-PersianGrammar}; Geoffrey Haig, p.c.).


The indefinite can be used alongside the classifier \textit{{hɒt}} (\ref{sent:Noun:Indf:more}) \citep{sigler-2003-noteClassifierWesternArmenianHad,BaleKhanjian-2009-ClassifiersNumbeerMarking,Sag-2019-SemanticsNumberMarkingReferenceKindsCountingOptionalClassifiers}. The classifier \textit{{hɒt}} can also be used as a noun meaning `piece' (\ref{ex:a piece}). As in {\seaSE} and {\swaWA}, the classifier is used in number + noun constructions. Here, the \textit{{me}} is on the surface ambiguous between an indefinite proclitic and  a numeral (\ref{ex:a man}). But when it precedes the classifier \textit{{hɒt}}, the morpheme \textit{{me}} is unambiguously a numeral (\ref{ex:a cl man}).\largerpage[2]

\begin{exe}
	\ex \label{sent:Noun:Indf:more}
	\begin{xlist}
		\ex \gll me hɒt
		\\
		{\indf}/one piece\\
		\trans	`a piece; one' \hfill (KM)\label{ex:a piece}
		\\
		\armenian{մի հատ}
		\ex \gll 	me mɒɻtʰ
		\\
		{\indf}/one man
		\\
		\trans			`a/one man' \hfill (KM)\label{ex:a man}
		\\
		\armenian{մի մարդ}
		\ex \gll 	me hɒt mɒɻtʰ
		\\
 one {\cl} man
		\\
		\trans	`one man' \hfill (KM)\label{ex:a cl man}
		\\
		\armenian{մի հատ մարդ}	\end{xlist}
	
\end{exe}

The construction \textit{{me hɒt}} can undergo vowel lowering and fronting as \textit{{mæ hæt}} (\ref{sent:Noun:Indf:mehatMerge}). This phrase can be further reduced into a single morph \textit{{mæt}}. Note the use of [{æ}], which is otherwise a marginal phoneme in {\iaIA}. 

\begin{exe}
	\ex \label{sent:Noun:Indf:mehatMerge}
	\begin{xlist}
		
		\ex \gll {\{mæt} / {mæ} {hæt}\} {mɒɻtʰ}
		\\
		{\indf}.{\cl} / {\indf} {\cl} man
		\\
		\trans	`a man' \hfill (NK)
		\\
		\armenian{մի հատ մարդ}
		\ex \gll \{{mæt} / {mæ} {hæt}\} {χɒʁɒlikʰ}
		\\
		{\indf}.{\cl} / {\indf} {\cl} toy
		\\
		\trans	`a toy' \hfill (AS)
		\\
		\armenian{մի հատ խաղալիք}
		\ex \gll vɒʁ-ə k-eɻtʰ-ɒ-m χɒnutʰ-it͡sʰ  {\textbf{mæt}}  {χɒʁɒlikʰ} veɻ-t͡sʰn-e-m iɻɒ zɒvɒk-neɻ-i hɒmɒɻ 
		\\
		tomorrow-{\defgloss} {\fut}-go-{\thgloss}-1{\sg} store-{\abl} 	\textbf{{\indf}.{\cl}}  toy buy-{\caus}-{\thgloss}-1{\sg}  he.{\gen} child-{\pl}-{\dat} for
		\\
		\trans	`Tomorrow I’m going to go pick up a toy for his children from the store' \hfill (AS)
		\\
		\armenian{Վաղը կէրթամ խանութից մի հատ խաղալիք վերցնեմ իրա զաւակների համար։} % կ՚էրթամ
	\end{xlist}
\end{exe}


The combination of indefinite\,+\,classifier is also used as an adverb to denote a sense of transience, roughly translatable to `for a moment' or `a little bit' (\ref{sent:Noun:Indf:mehat:nuance}).

\begin{exe}
	\ex \label{sent:Noun:Indf:mehat:nuance}
	\begin{xlist}
		
		\ex \gll {mæt} {ɒɻi} {ste}
		\\
		{\indf}.{\cl} come.{\imp}.2{\sg} here
		\\
		\trans			`Come here for a moment.' \hfill (AS)
		\\
		\armenian{Մի հատ արի ստէ։}
		\ex \gll mæt  {mətɒt͡sʰ-i} {mjus-i} {zɡɒt͡sʰmuŋkʰ-neɻ-i} {mɒs-i-n}
		\\
		{\indf}.{\cl} think-{\imp}.2{\sg} other-{\gen} feeling-{\pl}-{\gen} about-{\gen}-{\defgloss}
		\\
		\trans			`Think a little bit about the other person’s feelings.' \hfill (AS)
		\\
		\armenian{Մի հատ մտածէ միւսի զգացմունքների մասին։}
		\ex \gll {mæt} {hɒŋɡəst-ɒ-t͡sʰɻ-u} {senjɒk-um-ət}
		\\
		{\indf}.{\cl} relax-{\lvgloss}-{\caus}-{\imp}.2{\sg} room-{\locgloss}-{\possSsg} 
		\\
		\trans	`Rest for a while in your room.' \hfill (AS)
		\\
		\armenian{Մի հատ հանգստացրու սենեակումդ։}
		
	\end{xlist}
	
\end{exe}
