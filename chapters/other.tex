\chapter{Particles and function words}

\section{Bucket}
\begin{tabular}{l ll lll}
	Infinitival & Aorist & Subject& Resultative & Passive
	\\
{st-ɒ-n-ɒ-l} &{st-ɒ-\t{ts}ʰ-ɒ-m} & {st-ɒ-\t{ts}ʰ-oʁ} &  {st-ɒ-\t{ts}ʰ-ɒ\t{ts}ʰ} &  Ineffable  & `to receive'
	\\
	{goʁ-ɒ-n-ɒ-l} & {goʁ-ɒ-\t{ts}ʰ-ɒ-v} &  {goʁ-ɒ-\t{ts}ʰ-ɒ\t{ts}ʰ} & Ineffable  & `to steal'
	\\
	{goʁ-ɒ-\t{ts}ʰ-ɒ\t{ts}ʰ giɻkʰə}
	\\
	{ləv-ɒ-n-ɒ-l} & `to wash'
	\\
	{mah-ɒ-n-ɒ-l} &  `to die'
	\\
	{\t{ts}ʰɒnk-ɒ-n-ɒ-l}& Ineffable & `to wish'
	
\end{tabular}



Megerdoomian  2004 provides many pairs of verbs that involve causativization. Most of the causativizes have a corresponding intransitive inchoative. Some have a corresponding intransitive simple verb. A rare few involve a corresponding transitive verb. Megerdoomian states that causativization is unproductive in PA. I think this is in contrast to EA and WA where causativization is productive (Khurshudyan paper, Dolatian paper).

The following inchoative-causative pairs are from Megerdoomian 2004.

\begin{tabular}{llll ll}
	{\t{tʃ}ʰoɻ}
	&
	{\t{tʃ}ʰoɻ-ɒnɒl}   
	&
	{\t{tʃ}ʰoɻ-ɒ\t{ts}ʰnel}
	&
	`dry'
	&
	`to dry'
	&
	`to make dry'
	\\
	{me\t{ts}}  
	&
	{me\t{ts}-ɒnɒl}
	&
	{me\t{ts}-ɒ\t{ts}ʰnel}
	&
	`big'
	`to grow'
	`to grow, bring up; exaggerate'
	\\
	{ɒɻɒg}
	&
	{ɒɻɒg-ɒnɒl}
	&
	{ɒɻɒg-ɒ\t{ts}ʰnel}
	&
	`fast'
	&`quicken'
	&`accelerate'
	\\
	{\t{tʃ}ʰɒʁ }
	&
	{\t{tʃ}ʰɒʁ-ɒnɒl   }
	&
	{\t{tʃ}ʰɒʁ-ɒ\t{ts}ʰnel}
	&
	`fat'   
	&`become fat'
	&`fatten'
	\\
	{sev   }
	&
	{sev-ɒnɒl}
	&
	{sev-ɒ\t{ts}ʰnel}
	&
	`black'   
	&`blacken' 
	&`blacken, darken'
	\\
	{lɒjn}
	&
	{lɒjn-ɒnɒl}
	&
	{lɒjn-ɒ\t{ts}ʰnel}
	&
	`wide’   
	&`to widen’   
	&`to widen X'
	\\
	{vɒχ}   
	&{vɒχ-enɒl} 
	&{vɒχ-e\t{ts}ʰnel}
	&`fear'
	&`fear'   
	&`frighten'
	\\
	{uɻɒχ   }
	&
	{uɻɒχ-ɒnɒl}
	&
	{uɻɒχ-ɒ\t{ts}ʰnel}
	&`happy'   
	&`become happy'
	&`make happy'
	\\
	{\t{dʒ}əʁɒjn}   
	& 
	{\t{dʒ}əʁɒjn-ɒnɒl   }
	&
	{\t{dʒ}əʁɒjn-ɒ\t{ts}ʰnel}
	&`angry'
	&`become angry'   
	&`make angry'
	\\
	{zgujʃ}
	& {zgujʃ-ɒnɒl}
	& {zgujʃ-ɒ\t{ts}ʰnel   }
	&`caution'   
	&`be careful’   
	&`warn'
	\\
	{neʁ} &
	&{neʁ-ɒnɒl}
	&{neʁ-ɒ\t{ts}ʰnel   }
	&`thin, shrink'   
	&`disturb, bug'
\end{tabular}


She has one case of the causative of a transitive bleached inchoative.

{hɒsk-ɒnɒl} 'understand' {hɒsk-ɒ\t{ts}ʰnel} 'make understand'

One case of base-derived causative
{kɒɻmiɻ} 'red' {kɒɻməɻ-el} 'redden' {kɒɻmɻ-ɒ\t{ts}ʰnel}  `redden, saute'
Causatives of intransitives, from KM’s paper.










\begin{tabular}{ll ll}
	{lɒ\t{ts}ʰ-el   }
	&
	'cry'   
	&
	{lɒ\t{ts}ʰ-ɒ\t{ts}ʰnel   }
	&'make cry'
	\\
	{vɒz-el   }
	&'run'   
	& {vɒz-ɒ\t{ts}ʰnel}
	&'make run'
	\\
	{\t{ts}i\t{ts}ɒʁ-el}
	&'laugh'
	&{\t{ts}i\t{ts}ɒʁ-ɒ\t{ts}ʰnel}
	& 'make laugh'
	\\
	{χosel}
	&speak, talk'
	& {χos-e\t{ts}ʰnel   }
	& 'make speak, make talk'
	\\
	{kʰən-el}
	& 'sleep'   
	& {kʰən-ɒ\t{ts}ʰnel   }
	&'put to sleep, marinate'
	\\
	{χɒʁ-ɒl}   
	&
	'play'   
	& {χɒʁ-ɒ\t{ts}ʰnel   }
	&'make play;'mess with, mock''
	\\
	{nəst-el   }
	& `sit’   
	&{nəst-ɒ\t{ts}ʰnel   }
	& `to make sit’ 
	\\
	{pətət-el   }
	& `turn’   
	&{pətət-ɒ\t{ts}ʰnel   }
	& `to take for a ride’
	\\
	{\t{ts}aʁk-el}
	&'bloom'   
	& {\t{ts}aʁk-ɒ\t{ts}ʰnel}
	& 'embellish'
\end{tabular}


Some ingestive transitive-based causatives:

\begin{tabular}{ll ll}
	{χəmel}   
	&'drink'
	&{χm-e\t{ts}ʰnel   }
	&make drink'
	
	
	{xəm-ɒ\t{ts}ʰnel}
	\\  
	{utel   }
	&'eat'   
	{ut-e\t{ts}ʰnel   }
	&make eat'
	\\
\end{tabular}





\begin{tabular}{lll}
	{\t{ts}i\t{ts}ɒʁ-e\t{ts}ʰinkʰ}  & {\t{ts}i\t{ts}ɒʁ-ɒ\t{ts}ʰinkʰ}&  (KM)
	\\
	{χm-e\t{ts}ʰnum en} & {χm-ɒ\t{ts}num en } (KM)
	\\
	{vɒz-e\t{ts}ɻe\t{ts}inkʰ  } {vɒz-e\t{ts}ɻɒnkʰ } (KM)
	
\end{tabular}

\section{infixed caus}

idk if just EA from karine paper


\begin{tabular}{llll}
	{mətn-el   }
	& `enter’   
	{mət-\t{ts}ʰnel   }
	& `push/make enter'
	\\
	{tʰərn-el   }
	&'fly'   
	{tʰər-\t{ts}ʰnel   }
	&'steal'
	\\
	{pʰɒχ-nel}
	&
	'escape'
	&
	{pʰɒχ-\t{ts}ʰnel}
	&'kidnap'\end{tabular}


\section{syntax junk}
subj senjɒk-ə mɒkʰuɻ ɒ

senjɒk-neɻ-ə mɒkʰuɻ e-n

senjɒk-əs mɒkʰuɻ ɒ

ɒɻɒm-ə uɻɒχ ɑ

obj 

senjɒk-neɻ-ə mɒkʰɻ-ɒ-m

senjɒk-əs mɒkʰɻ-ɒ-m



dat giɻkʰ em təve senjɒk-i-n

senjɒk-neɻ-i-n giɻkʰ təv-ɒ-m

senjɒk-i-s giɻkʰ təv-ɒ-m


gen 

senjɒk-neɻ-i gujn-ə



jes mɒkʰuɻ e-m
du mɒkʰuɻ e-s
inkʰə mɒkʰuɻ ɒ
nɒ  mɒkʰuɻ ɒ -- prefers e, feels na is formal
menkʰ mɒkʰuɻ e-nkʰ
dukʰ mɒkʰuɻ e-kʰ
iɻɒnkʰ mɒkʰuɻ e-n
nəɻɒnkʰ mɒkʰuɻ e-n


d͡ʒɒn-ə in\t{dz} mɒkʰɻ-ɒ-v
d͡ʒɒn-ə kʰez mɒkʰɻ-ɒ-v
d͡ʒɒn-ə iɻɒn mɒkʰɻ-ɒ-v
d͡ʒɒn-ə mez mɒkʰɻ-ɒ-v
d͡ʒɒn-ə \t{dz}ez mɒkʰɻ-ɒ-v
d͡ʒɒn-ə iɻɒn\t{ts}ʰ mɒkʰɻ-ɒ-v

d͡ʒɒn-ə in\t{dz} giɻkʰ təv-ɒ-v
d͡ʒɒn-ə kʰez giɻkʰ təv-ɒ-v
d͡ʒɒn-ə iɻɒn giɻkʰ təv-ɒ-v
nəɻɒn
d͡ʒɒn-ə mez giɻkʰ təv-ɒ-v
d͡ʒɒn-ə \t{dz}ez giɻkʰ təv-ɒ-v
d͡ʒɒn-ə iɻɒn\t{ts}ʰ giɻkʰ təv-ɒ-v
d͡ʒɒn-ə nəɻɒn\t{ts}ʰ giɻkʰ təv-ɒ-v

in\t{dz}əni\t{ts}ʰ
kʰezni\t{ts}ʰ
iɻɒni\t{ts}ʰ
nəɻɒni\t{ts}ʰ
mezni\t{ts}ʰ
\t{dz}ezni\t{ts}ʰ
iɻɒn\t{ts}ʰi\t{ts}ʰ

im het
kʰo het
iɻɒ het
nəɻɒ het
meɻ het
\t{dz}eɻ het
iɻɒn\t{ts}ʰ het
nəɻɒn\t{ts}ʰ het

often reduced to nɒn\t{ts}ʰ

\section{Allomorphy of the past perfective}\label{section:verb:past}

Iranian Armenian has developed a quite complicated manner of marking the past  perfective tense on verbs. The previous sections described past marking for the different conjugation classes. This section synthesizes that information in order to formalize the generalizations that unify these classes.  

Diachronically, past marking in Iranian developed from past marking in Eastern Armenian.   The main change was the extension of the marked past morph /{-ɒ}/ from the past perfective of irregular verbs to the past perfective of the regular E-Class. Simultaneous with the extension of promotion of the /{-ɒ}/ allomorph, the elsewhere allomorph /{-i}/ became more restricted in its usage.  We formalize these changes in terms of changes in the conditions for the relevant  rules for the past perfective morphemes.



\subsection{Consistent past marker /-i/ in the imperfective}\label{section:verb:past: impf}


In terms of morpho-semantics, the past tense is marked in two types of paradigm cells: the subj. past imperfective and the past perfective. The subj. past imperfective is realized by placing the past suffix /{-i}/ after the verb stem in both Eastern and Iranian. We illustrate below for the regular E-Class and A-Class. We underline the past marker.\footnote{Through this section, what we call the verb stem is the sequence of morphemes that start from the root up until the theme vowel (if overt). }

\begin{table}[H]
	\centering
	\caption{Past marker /{-i}/ for the past   imperfective 3PL in Eastern and Iranian simple classes}	\label{tab:past impf past marker}
	\begin{tabular}{| l| lll| }
		\hline       &  Eastern &  Iranian &  
		\\
		\hline 
		E-Class `to sing'  & {jeɾkʰ-e-l} & {jeɻkʰ-e-l}& $\sqrt{~}$\textsc{-th-inf}  
		\\& \armenian{երգել}& \armenian{երգել}&
		\\
		`(If) they were singing'& {jeɾkʰ-ej-\uline{i}-n} & {jeɻkʰ-$\emptyset$-\uline{i}-n} & $\sqrt{~}$\textsc{-th-pst-3pl} 
		\\& \armenian{երգեին}& \armenian{երգին}& 
		\\
		\hline      
		A-Class `to read'& {kɑɾtʰ-ɑ-l} & {kɒɻtʰ-ɒ-l}& $\sqrt{~}$\textsc{-th-inf}
		\\ & \armenian{կարդալ} & \armenian{կարդալ} &   
		\\
		`(If) they were reading'& {kɑɾtʰ-ɑj-\uline{i}-n} & {kɒɻtʰ-ɒj-\uline{i}-n} & $\sqrt{~}$\textsc{-th-pst-3pl}\\
		& \armenian{կարդային}& \armenian{կարդային}& 
		\\ \hline
	\end{tabular}
	
\end{table}


Vowel hiatus between the verb's theme vowel and the past /{-i}/ is resolved by glide epenthesis in Eastern Armenian. In Iranian Armenian, glide epenthesis applies after the theme vowel /{-ɒ-}/, and vowel deletion for the the theme vowel /{-e-}/.

Throughout Eastern Armenian and Iranian Armenian, the past marker /{-i}/ is consistently used for the past imperfective. All verbs, including complex verbs  and irregular verbs, use the past marker /{-i}/ in the subj. past imperfective.\footnote{The past marker /{-i}/ is likewise used for defective verbs like the copula, verb `to exist', and verb `to have' (\S\ref{section:verb:irregular:def}). Here, the past marker contributes a past perfective reading but it's surface morphotactics resemble the subj. past imperf. of non-defective verbs. We don't formalize the complications of defective verbs.)} 


\begin{table}[H]
	\centering
	\caption{Consistent past marker /{-i}/ for the past   imperfective 3PL throughout   Eastern and Iranian Armenian}	\label{tab:past impf past marker acros paradigm}
	\begin{tabular}{| l| lll| }
		\hline       &  Eastern &  Iranian &  
		\\
		\hline 
		Causative `to teach' & {sovoɾ-e-\t{ts}ʰn-e-l} &  {sovoɻ-e-\t{ts}ʰn-e-l} & $\sqrt{~}$\textsc{-th-caus-th-inf}  
		\\ & \armenian{սովորեցնել}& \armenian{սովորեցնել}&
		\\
		& {sovoɾ-e-\t{ts}ʰn-ej-\uline{i}-n} & {sovoɻ-e-\t{ts}ʰn-$\emptyset$-\uline{i}-n} & $\sqrt{~}$\textsc{-th-caus-th-pst-3pl} 
		\\& \armenian{սովորեցնեին}& \armenian{սովորեցնին}& 
		\\
		\hline      
		Passive `to be sung' 	  & {jeɾkʰ-v-e-l} & {jeɻkʰ-v-e-l} & $\sqrt{~}$\textsc{-pass-th-inf}  
		\\ & \armenian{երգվել}& \armenian{երգուել}&
		\\
		& {jeɾkʰ-v-ej-\uline{i}-n} & {jeɻkʰ-v-$\emptyset$-\uline{i}-n} & $\sqrt{~}$\textsc{-pass-th-pst-3pl} 
		\\& \armenian{երգվեին}& \armenian{երգուին}& 
		\\ \hline
		Inchoative	`to become happy'   &{uɾɑχ-ɑ-n-ɑ-l} & {uɻɒχ-ɒ-n-ɒ-l} & $\sqrt{~}$\textsc{-lv-inch-th-inf}  
		\\ & \armenian{ուրախանալ}& \armenian{ուրախանալ} &
		\\
		& {uɾɑχ-ɑ-n-ɑj-\uline{i}-n} & {uɻɒχ-ɒ-n-ɒj-\uline{i}-n} & $\sqrt{~}$\textsc{-th-pst-3pl}\\
		& \armenian{ուրախանային}& \armenian{ուրախանային}& 
		\\ \hline
		Infixed `to die' & 	   {mer-n-e-l} & {mer-n-e-l} & $\sqrt{~}$\textsc{-x-th-inf}  
		\\ & \armenian{մեռնել}& \armenian{մեռնել}&
		\\
		& {mer-n-ej-\uline{i}-n} & {mer-n-$\emptyset$-\uline{i}-n} & $\sqrt{~}$\textsc{-x-th-pst-3pl} 
		\\& \armenian{մեռնեին}& \armenian{մեռնին}& 
		\\
		\hline      
		
		Suppletiv`to eat' e & 	  {ut-e-l} & {ut-e-l} & $\sqrt{~}$\textsc{-th-inf}  
		\\& \armenian{ուտել}& \armenian{ուտել}&
		\\
		& {ut-ej-\uline{i}-n} & {ut-$\emptyset$-\uline{i}-n} & $\sqrt{~}$\textsc{-th-pst-3pl} 
		\\& \armenian{ուտեին}& \armenian{ուտին}& 
		\\
		\hline      
		
		
		
	\end{tabular}
	
\end{table}


In sum, the past marker is /{-i}/ uniformly for all conjugation classes in the subj. past imperfective. This is summarized by the  rule below. \footnote{In the 3SG past imperfective, the past marker is zero (\ref{rule past impf i}). This morph is tangential to our eventual discussion of the past perfective, so we set it aside. The use of zero in this context is arbitrary allomorphy.}

\begin{exe}
	\ex \textit{ Rules for past tense in the imperfective} (partially repeated from  \ref{rule past impf i})
	
	\begin{tabular}{llll}
		\textsc{T[+pst]} & $\leftrightarrow$ &  \textit{{-i}} &/ elsewhere
		
		
	\end{tabular}
\end{exe}


The next section presents complications from the past perfective.  



\subsection{Variation for past marking in the perfective}\label{section:verb:past: pfv basic}

In the past perfective, we find that Iranian and Eastern Armenian utilize morphologically-conditioned allomorphy for the past marker and the perfective marker. 

Depending on dialect and conjugation class, the past marker in  the past perfective is either /{-i-}/ or a low vowel: /{-ɑ-}/ in Eastern, /{-ɒ-}/ in Iranian. The perfective marker is either zero -$\emptyset$ or an affricate /{-\t{ts}ʰ-}/. We illustrate below with an A-Class verb and an infixed verb. 

\begin{table}[H]
	\centering
	\caption{Templates for past perfective marking in A-Class and irregular infixed verbs}	\label{tab:past perf template aclass infix}
	\begin{tabular}{|l|l l l| }
		\hline   & Eastern & Iranian & 
		\\
		A-Class  `to read' & {kɑɾtʰ-ɑ-l} & {kɒɻtʰ-ɒ-l} & $\sqrt{~}$\textsc{-th-inf}
		\\& \armenian{կարդալ} & \armenian{կարդալ} &   
		\\
		`they read (\textsc{pst}) & {kɑɾtʰ-ɑ-\uline{\t{ts}ʰ}-\uline{i}-n} & {kɒɻtʰ-ɒ-\uline{\t{ts}ʰ}-\uline{i}-n} & $\sqrt{~}$\textsc{-th-pfv-pst-3pl}\\
		& \armenian{կարդացին}& \armenian{կարդացին}& 
		\\ \hline
		Irregular infixed `to die' & 	   {mer-n-e-l} & {mer-n-e-l} & $\sqrt{~}$\textsc{-x-th-inf}  
		\\& \armenian{մեռնել}& \armenian{մեռնել}&
		\\
		`they died'	& {mer-$\emptyset$-$\emptyset$-\uline{$\emptyset$-ɑ}-n} & {mer-$\emptyset$-$\emptyset$-\uline{$\emptyset$-ɒ}-n} & $\sqrt{~}$\textsc{-x-th-pfv-pst-3pl} 
		\\& \armenian{մեռան}& \armenian{մեռան}& 
		\\
		\hline      
	\end{tabular}
	
\end{table}


The past and perfective allomorphs can appear in different combinations. For the A-Class, the pattern is /{-\t{ts}ʰ-i}-/. For the infixed verbs, the pattern is /{-$\emptyset$-ɑ}/ or /{-$\emptyset$-ɒ}/. Other logically possible patterns are also attested: /{-\t{ts}ʰ-ɑ}/ and /{-$\emptyset$-i}/. But these are restricted to special contexts and registers that we discuss later in \S\ref{section:verb:past:other perm}.

As a rule, the  template /{-$\emptyset$-ɒ}/   triggers the deletion of any preceding theme vowels. We repeat the relevant rule below.

\begin{exe}
	\ex \textit{Readjustment rule}: \textit{Delete theme vowels before the past node }/{ɒ}/ (from \ref{rule: delete theme before pst a})
	
	\begin{tabular}{llll}
		/{e}/ &$\rightarrow$&$\emptyset$ &  / \_ {ɒ} 
		\\
		\multicolumn{4}{l}{(where /{e}/ is a theme vowel, and /{ɒ}/ is a past marker)}
	\end{tabular}
\end{exe}


The two templates  /{-\t{ts}ʰ-i}-/ and /{-$\emptyset$-ɒ}/ are both the two main patterns of past perfective  marking in the two dialects. But the two templates have different distributions across the lects. The predominant template in Eastern is  /{-\t{ts}ʰ-i}-/, while in Iranian it is  /{-$\emptyset$-ɒ}/. The different predominances are visible in the E-Class. This class is the elsewhere or default class for verbs, and it is the most populated class.  

\begin{table}[H]
	\centering
	\caption{Different predominant patterns for past perfective marking in the E-Class}	\label{tab:past perf template eclass }
	\begin{tabular}{|l|l l l| }
		\hline   & Eastern & Iranian & 
		\\
		\hline 
		E-Class to sing' & {jeɾkʰ-e-l} & {jeɻkʰ-e-l} & $\sqrt{~}$\textsc{-th-inf}  
		\\& \armenian{երգել}& \armenian{երգել}&
		\\
		`they sang'	& {jeɾkʰ-e-\uline{\t{ts}ʰ}-\uline{i}-n} & {jeɻkʰ-$\emptyset$-\uline{$\emptyset$-ɒ}-n} & $\sqrt{~}$\textsc{-th-pfv-pst-3pl} 		\\
		& \armenian{երգեին}& \armenian{երգան}& 
		\\
		\hline      
	\end{tabular}
	
\end{table}

The generalization is that in Eastern, the default template for the past perfective is /{-\t{ts}ʰ-i}/, while it is /{-$\emptyset$-ɒ}/ in Iranian. In the imperfective, the past is uniformly just /{-i}/. These generalizations are formalized below, based on the A-Class, infixed verbs, and E-Class. 

For the perfective slot, the zero morph -$\emptyset$ is used for irregular infixed verbs in both Eastern and Iranian. The morph /{-\t{ts}ʰ-}/ is elsewhere. In Iranian, the zero morph is elsewhere, while the /{-\t{ts}ʰ-}/ morph is for the A-Class.  Note the switch in which morph is elsewhere.

\begin{exe}
	\ex { Rules for perfective  aspect for A/E-Class and infixed}
	
	\begin{tabular}{llll}
		\multicolumn{3}{l}{\textbf{Eastern}} & \\
		
		\textsc{Asp[pfv]} & $\leftrightarrow$ & -$\emptyset$ & / $\sqrt{~}$   $\frown$ X $\frown$\textsc{th}  $\frown$  \_ (where root is irregular infixed)  
		\\
		& & {{-\t{ts}ʰ}-}& / elsewhere
		\\
		\multicolumn{3}{l}{\textbf{Iranian}}\\
		
		\textsc{Asp[pfv]} & $\leftrightarrow$ & {{-\t{ts}ʰ-}} & / [\textsc{A-Class}]   $\frown$\textsc{th}  $\frown$  \_ 
		\\
		& & -$\emptyset$ & / elsewhere
		\\
		
		
	\end{tabular}
\end{exe}

For the past slot, the marker is /{-i}/ in the imperfective. We treat imperfectivity as the lack of an aspect node. In the perfective in Eastern, the marker is /{-ɑ}/ for  the irregular infixed, and /{-i}/ elsewhere. In Iranian, the /{-i}/ is for the A-Class, while /{-ɒ}/ is elsewhere in the perfective.

\begin{exe}
	\ex { Rules for past marker in A/E-Class and irregular infixed verbs }
	
	\begin{tabular}{llll}
		\multicolumn{3}{l}{\textbf{Eastern}}
		\\
		\textsc{T[+pst]} & $\leftrightarrow$ & {{-ɑ}} & / $\sqrt{~}$ $\frown$ X $\frown$\textsc{th} $\frown$\textsc{Asp[pfv]} $\frown$ \_ (where root is infixed)
		\\
		&& {{-i}} & / elsewhere
		\\
		
		\multicolumn{3}{l}{\textbf{Iranian}}
		\\
		\textsc{T[+pst]} & $\leftrightarrow$ & {{-ɒ}} & / \textsc{Asp[pfv]} $\frown$ \_ 
		\\
		&& {{-i}} & / [\textsc{A-Class}] $\frown$\textsc{th} $\frown $ \textsc{Asp[pfv]} $\frown$ \_ 
		\\
		&&   & / elsewhere  (when verb is imperfective)
		
	\end{tabular}
\end{exe}

The  rules are quiet convoluted. But the core generalization is that in the past perfective, the default template is  /{-\t{ts}ʰ-i}/, while /{-$\emptyset$-ɑ}/ is the marked template. Iranian instead does the reverse, with     /{-$\emptyset$-ɒ}/ as default while  /{-\t{ts}ʰ-i}/ is marked.  The next section goes through the conjugation classes of the two lects in order to reinforce this generalization. 

\subsection{Promotion of past /{ɒ}/ in the perfective}\label{section:verb:past: pfv everywhere}
The E-Class is the default verb class in both lects. Its inflectional system is generally percolated throughout the rest of the conjugation classes in both lects.  There are exceptions however.  These exceptions are more common in Eastern than in Iranian. For Iranian, the promotion of the past perfective /{-$\emptyset$-ɒ}/  for the E-Class has systematically spread throughout the language. 

The previous section showed past marking for the simple verb classes: E-Class and A-Class. For the complex cases of causatives and passives, we see virtually the same behavior. Passives are inflected identically to simple E-Class words in both lects. The past perfective template is /{-\t{ts}ʰ-}i/ in Eastern, while /{-$\emptyset$-ɒ}/ for Iranian. 

\begin{table}[H]
	\centering
	\caption{Different predominant patterns for past perfective marking in  passives} \label{tab:past perf template passive }
	\begin{tabular}{|l|l l l| }
		\hline Passive  & Eastern & Iranian & 
		\\
		\hline 
		`to be sung'	  & {jeɾkʰ-v-e-l} & {jeɻkʰ-v-e-l} & $\sqrt{~}$\textsc{-pass-th-inf}  
		\\     & \armenian{երգվել}& \armenian{երգուել}&
		\\
		`they were sung'	& {jeɾkʰ-v-e-\uline{\t{ts}ʰ}-\uline{i}-n} & {jeɻkʰ-v-$\emptyset$-\uline{$\emptyset$-ɒ}-n} & $\sqrt{~}$\textsc{-pass-th-pfv-pst-3pl} 
		\\& \armenian{երգվեցին}& \armenian{երգուան}& 
		\\ \hline
	\end{tabular}
	\label{tab:past perf template passive}
\end{table}

For causatives,  matters are slightly more complicated. The causative suffix is /{-\t{ts}ʰn-}/ in its elsewhere form, but /{-\t{ts}ʰɾ}/ (Eastern) and /{-\t{ts}ʰɻ}/ (Iranian) in the perfective. In Eastern, the past perfective template is /-{$\emptyset$-i}/ in standard Eastern without a theme vowel. Colloquial Eastern Armenian instead has /{-\t{ts}ʰ-i}/ with the theme vowel.\footnote{in In both standard and colloquial Eastern Armenian, the 3SG uses an overt theme and perfective marker, but no past or Agr suffix: `he taught' [{sovoɾ-e-\t{ts}ʰɾ-e-\t{ts}ʰ}] `$\sqrt{~}$- \textsc{th-caus-th-pfv}'. We set this minor complication aside.} In both registers, the main past allomorph is /{-i}/. In contrast in Iranian, the only template is /{-$\emptyset$-ɒ}/ with past marker /{-ɒ}/.


\begin{table}[H]
	\centering
	\caption{Different predominant patterns for past perfective marking in causatives}\label{tab:past perf template caus}
	\begin{tabular}{|l|ll l l| }
		\hline Causative & & Eastern & Iranian & 
		\\
		\hline 
		`to teach'	&& {sovoɾ-e-\t{ts}ʰn-e-l} &  {sovoɻ-e-\t{ts}ʰn-e-l} & $\sqrt{~}$\textsc{-th-caus-th-inf}  
		\\  &&  \armenian{սովորեցնել}& \armenian{սովորեցնել}&
		\\
		`they taught'	& Std. &  {sovoɾ-e-\t{ts}ʰɾ-$\emptyset$-\uline{$\emptyset$-i}-n} & {sovoɻ-e-\t{ts}ʰɻ-$\emptyset$-\uline{$\emptyset$-ɒ}-n }& $\sqrt{~}$\textsc{-th-caus-th-pfv-pst-3pl} 
		\\& 
		& 	\armenian{սովորեցրին}
		& \armenian{սովորեցրան}
		&
		\\
		& Coll. & {sovoɾ-e-\t{ts}ʰɾ-e-\uline{\t{ts}ʰ-i}-n} & & 
		\\
		& & \armenian{սովորեցրեցին} & & 
		\\
		\hline      
		
		
	\end{tabular}
	
\end{table}




Among infixed verbs, most take the /{-$\emptyset$-ɑ}/ template in Eastern. The exception is one infixed verb which takes the /{-\t{ts}ʰ-i}/ template. In contrast in Iranian, all infixed verbs take the /{-$\emptyset$-ɑ}/ template without exception.

\begin{table}[H]
	\centering
	\caption{Templates for past perfective marking in A-Class and irregular infixed verbs}	\label{tab:past perf template all infix}
	\begin{tabular}{|l|l l l| }
		\hline Infixed  & Eastern & Iranian & 
		\\
		`to die'	 & 	   {mer-n-e-l} & {mer-n-e-l} & $\sqrt{~}$\textsc{-x-th-inf}  
		\\ & \armenian{մեռնել}& \armenian{մեռնել}&
		\\
		`they died'	& {mer-$\emptyset$-$\emptyset$-\uline{$\emptyset$-ɑ}-n} & {mer-$\emptyset$-\uline{$\emptyset$-ɒ}-n} & $\sqrt{~}$\textsc{-x-th-pfv-pst-3pl} 
		\\& \armenian{մեռան}& \armenian{մեռան}& 
		\\
		\hline      
		`to flee' 	 & 	   {pʰɑχ-\t{tʃ}ʰ-e-l} & {pʰɑχ-n-e-l} & $\sqrt{~}$\textsc{-x-th-inf}  
		\\& \armenian{փախչել}& \armenian{փախնել}&
		\\
		`they fled' 	& {pʰɑχ-$\emptyset$-$\emptyset$-\uline{$\emptyset$-ɑ}-n} & {pʰɑχ-$\emptyset$-\uline{$\emptyset$-ɒ}-n} & $\sqrt{~}$\textsc{-x-th-pfv-pst-3pl} 
		\\& \armenian{փախան}& \armenian{փախան}& 
		\\
		\hline      
		Exceptional `to let' & 	   {tʰoʁ-n-e-l} & {tʰoʁ-n-e-l} & $\sqrt{~}$\textsc{-x-th-inf}  
		\\
		& \armenian{թողնել}& \armenian{թողնել}&
		\\
		`they let (\textsc{pst})'		& {tʰoʁ-$\emptyset$-e-\uline{\t{ts}ʰ-i}-n} & {tʰoʁ-$\emptyset$-\uline{$\emptyset$-ɒ}-n} & $\sqrt{~}$\textsc{-x-th-pfv-pst-3pl} 
		\\& \armenian{թողեցին}& \armenian{թողան}& 
		\\
		\hline      
	\end{tabular}
	
\end{table}


For suppletive verbs, both dialects use the marked or aorist stem in the past perfective. In terms of the perfective-past template, Eastern Armenian inflects some of them with the irregular /{-$\emptyset$-ɑ}/ template (Table \ref{tab:past perf template suppletive zero-a east}), and some with the elsewhere /{-\t{ts}ʰ-i/} template (Table \ref{tab:past perf template suppletive zero-a iran but not east}). In Iranian, most of these suppletive verbs have switched to using the /{-$\emptyset$-ɒ}/ template. 



\begin{table}[H]
	\centering
	\caption{Suppletive verbs that take  /{-$\emptyset$-ɑ}/ for the past perfective in Eastern and Iranian}\label{tab:past perf template suppletive zero-a east}
	\begin{tabular}{| l| lll| }
		\hline       &  Eastern &  Iranian &  
		\\
		\hline 
		`to come' & 	  {ɡ-ɑ-l} & {ɡ-ɒ-l} & $\sqrt{~}$\textsc{-th-inf}  
		\\ & \armenian{գալ}& \armenian{գալ}&
		\\
		`they came'		& {jek-$\emptyset$-\uline{$\emptyset$-ɑ}-n} & {ek-$\emptyset$-\uline{$\emptyset$-ɒ}-n} & $\sqrt{~}$\textsc{-th-pfv-pst-3pl} 		\\
		& \armenian{եկան}& \armenian{էկան}& 
		\\
		\hline      
		
		`to eat' & 	  {ut-e-l} & {ut-e-l} & $\sqrt{~}$\textsc{-th-inf}  
		\\ & \armenian{ուտել}& \armenian{ուտել}&
		\\
		`they ate'		& {keɾ-$\emptyset$-\uline{$\emptyset$-ɑ}-n} & {keɻ-$\emptyset$-\uline{$\emptyset$-ɒ}-n} & $\sqrt{~}$\textsc{-th-pfv-pst-3pl} 		\\
		& \armenian{կերան}& \armenian{կերան}& 
		\\
		\hline      
		`to take to' & 	  {tɑn-e-l} & {tɒn-e-l} & $\sqrt{~}$\textsc{-th-inf}  
		\\ & \armenian{տանել}& \armenian{տանել}&
		\\
		`they took'		& {tɑɾ-$\emptyset$-\uline{$\emptyset$-ɑ}-n} & {tɒɻ-$\emptyset$-\uline{$\emptyset$-ɒ}-n} & $\sqrt{~}$\textsc{-th-pfv-pst-3pl} 		\\
		& \armenian{տարան}& \armenian{տարան}& 
		\\
		\hline      
		
	\end{tabular}
	
\end{table}

\begin{table}[H]
	\centering
	\caption{Suppletive verbs that take  /{-\t{ts}ʰ-i}/ for the past perfective in Eastern but not Iranian}\label{tab:past perf template suppletive zero-a iran but not east}
	\begin{tabular}{| l| lll| }
		\hline       &  Eastern &  Iranian &  
		\\
		\hline 
		`to give' & 	  {t-ɑ-l} & {t-ɒ-l} & $\sqrt{~}$\textsc{-th-inf}  
		\\ & \armenian{տալ}& \armenian{տալ}&
		\\
		`they gave'		& {təⱱ-e-\uline{\t{ts}ʰ-i}-n} & {təv-$\emptyset$-\uline{$\emptyset$-ɒ}-n} & $\sqrt{~}$\textsc{-th-pfv-pst-3pl} 		\\
		& \armenian{տվեցին}& \armenian{տուան}& 
		\\
		\hline      
		`to put' & 	  {dən-e-l} & {dən-e-l} & $\sqrt{~}$\textsc{-th-inf}  
		\\ & \armenian{դնել}& \armenian{դնել}&
		\\
		`they put (\textsc{pst})'		& {dən-e-\uline{\t{ts}ʰ-i}-n} & {dən-$\emptyset$-\uline{$\emptyset$-ɒ}-n} & $\sqrt{~}$\textsc{-th-pfv-pst-3pl} 		\\
		& \armenian{դրեցին}& \armenian{դրան}& 
		\\
		\hline      
		`to do' & 	  {ɑn-e-l} & {ɒn-e-l} & $\sqrt{~}$\textsc{-th-inf}  
		\\ & \armenian{անել}& \armenian{անել}&
		\\
		`they did'		& {ɑɾ-e-\uline{\t{ts}ʰ-i}-n} & {ɒɻ-$\emptyset$-\uline{$\emptyset$-ɒ}-n} & $\sqrt{~}$\textsc{-th-pfv-pst-3pl} 		\\
		& \armenian{արեցին}& \armenian{արան}& 
		\\
		\hline      
		
		
	\end{tabular}
	
\end{table}


An exception is the verb `to go'. The expression of this verb is complicated across the lects.   This verb is expressed by a regular A-Class verb in standard Eastern [ɡən-ɑ-l], a suppletive verb [{jeɾtʰ-ɑ-l}] in colloquial Eastern, and a suppletive verb [{e(ɻ)tʰ-ɒ-l}] in Iranian.  the /{-\t{ts}ʰ-i/} template is used for  these verbs.

\begin{table}[H]
	\centering
	\caption{Verb  `to go' and its past perfective in both Eastern and   Iranian}\label{tab:past perf template suppletive zero-a in neither east iran}
	\begin{tabular}{| l| llll| }
		\hline       & Std. Eastern &Coll. Eastern&  Iranian &  
		\\
		\hline 
		`to go' & {ɡən-ɑ-l} &	  {jeɾtʰ-ɑ-l} & {e(ɻ)tʰ-ɒ-l} & $\sqrt{~}$\textsc{-th-inf}  
		\\ & \armenian{գնալ}& \armenian{երթալ}& \armenian{էրթալ, էթալ}&
		\\
		`they went'		& {ɡən-ɑ-\uline{\t{ts}ʰ-i}-n} 	& {ɡən-ɑ-\uline{\t{ts}ʰ-i}-n} & {ɡən-ɒ-\uline{\t{ts}ʰ-i}-n} & $\sqrt{~}$\textsc{-th-pfv-pst-3pl} 		\\
		& \armenian{գնացին}	& \armenian{գնացին}& \armenian{գնացին}& 
		\\
		\hline      
		
		
	\end{tabular}
	
\end{table}

There is an irregular  verb `to be' which is expressed by a suppletive verb [lin-e-l] in Eastern, while it is expressed by an irregular infixed verb in Iranian [el-n-e-l]. Both use the template [{$\emptyset$-ɑ/ɒ}].


\begin{table}[H]
	\centering
	\caption{Irregular verb `to be' in the past perfective for Eastern and Iranian}\label{tab:past perf template verb to be}
	\begin{tabular}{| l| lll| }
		\hline       &  Eastern &  Iranian &  
		\\
		\hline 
		`to be' & 	  {lin-e-l} & {el-n-el} & $\sqrt{~}$\textsc{-(x)-th-inf} \\ & \armenian{լինել}& \armenian{էլնել}&
		\\
		`they were'		& {jeʁ-$\emptyset$-\uline{$\emptyset$-ɑ}-n} & {el-$\emptyset$-$\emptyset$-\uline{$\emptyset$-ɒ}-n} & $\sqrt{~}$\textsc{-(x)-th-pfv-pst-3pl} 		\\
		& \armenian{եղան}& \armenian{էլան}& 
		\\
		\hline      
		
		
		
	\end{tabular}
	
\end{table}




\subsection{Isolated change in past perfective marking}\label{section:verb:past:irreg isolation}
As is clear so far, Iranian Armenian developed from Eastern Armenian by promoting the template /{-$\emptyset$-ɒ}/ into the default pattern for past perfective marking. This change affects verbs across conjugation classes, including both regular and irregular verbs. Such a change did not significantly alter other types of irregular morphology in Iranian, or with the loss of some verbs. 

In Eastern Armenian, there are irregular verbs which are inflected as regular E-Class verbs throughout of the inflectional paradigm. This means that such verbs take the /{-\t{ts}ʰ-i}/ template for the past perfective. Their irregularity is how they form the imperative 2SG. The default strategy is to add the suffix \textit{-iɾ}, but these verbs instead either output the bare root or add some irregular suffix. In Iranian, this irregularity is maintained in the imperative, while the past perfective takes the /{-$\emptyset$-ɒ}/ template.\footnote{In Eastern Armenian, the verb `to say' irregularly takes a theme vowel /{ɑ}/ in the past perfective.}


\begin{table}[H]
	\centering
	\caption{Irregular verbs that are irregular in imperative 2SG but not in their past perfective template}\label{tab:past perf template irregular imp}
	\begin{tabular}{| l| lll| }
		\hline       &  Eastern &  Iranian &  
		\\
		\hline 
		`to say' & 	  {ɑs-e-l} & {ɒs-e-l} & $\sqrt{~}$\textsc{-th-inf} \\ & \armenian{ասել}& \armenian{ասել}&
		\\
		`they said'& {ɑs-ɑ-\uline{\t{ts}ʰ-i}-n} & {ɒs-$\emptyset$-\uline{$\emptyset$-ɒ}-n} & $\sqrt{~}$\textsc{-th-pfv-pst-3pl} 		\\
		& \armenian{ասացին}& \armenian{ասան}& 
		\\
		`say! (\textsc{2sg})''& {ɑs-ɑ} & {ɒs-ɒ} & $\sqrt{~}$\textsc{-th} 		\\
		& \armenian{ասա}& \armenian{ասա}& 
		\\
		\hline      
		`to bring' & 	  {beɽ-e-l} & {beɻ-e-l} & $\sqrt{~}$\textsc{-th-inf} \\ & \armenian{բերել}& \armenian{բերել}&
		\\
		`they brought'	& {beɾ-e-\uline{\t{ts}ʰ-i}-n} & {beɻ-$\emptyset$-\uline{$\emptyset$-ɒ}-n} & $\sqrt{~}$\textsc{-th-pfv-pst-3pl} 		\\
		& \armenian{բերեցին}& \armenian{բերան}& 
		\\
		`bring! (\textsc{2sg})'		& {beɾ} & {beɻ} & $\sqrt{~}$ 		\\
		& \armenian{բեր}& \armenian{բեր}& 
		\\
		\hline      
		
		
	\end{tabular}
	
\end{table}





Thus, the morphological change targeted the past perfective, while leaving other irregularities intact.

There are some verbs which have irregular imperatives in Eastern, but are regularized in Iranian. In the past perfective, these verbs still follow the Iranian /-$\emptyset$-ɒ/ template. 

\begin{table}[H]
	\centering
	\caption{Verbs that were regularized in Iranian Armenian, and their  past perfective}\label{tab:past perf template irregular open}
	\begin{tabular}{| l| lll| }
		\hline       &  Eastern &  Iranian &  
		\\
		\hline 
		`to open' & 	  {bɑ\t{ts}ʰ-e-l} & {bɒ\t{ts}ʰ-e-l} & $\sqrt{~}$\textsc{-th-inf}  
		\\ & \armenian{բացել}& \armenian{բացել}&
		\\
		`they opened'		& {bɑ\t{ts}ʰ-e-\uline{\t{ts}ʰ-i}-n} & {bɒ\t{ts}ʰ-$\emptyset$-\uline{$\emptyset$-ɒ}-n} & $\sqrt{~}$\textsc{-th-pfv-pst-3pl} 		\\
		& \armenian{բացեցին}& \armenian{բացան}& 
		\\
		`open! (\textsc{2sg})'		& {bɑ\t{ts}ʰ} & {bɒ\t{ts}ʰ-i} & $\sqrt{~}$\textsc{-(imp.2sg)} 		\\
		& \armenian{բաց}& \armenian{բացի}& 
		\\
		
		\hline      
		
	\end{tabular}
	
	
	
	
\end{table}



There are also some irregular words which exist in Eastern Armenian but do not in Iranian Armenian. These verbs got replaced by other verbs. They are still inflected with the /-$\emptyset$-ɒ/ template in Iranian.\footnote{Note that `to cry' in Eastern is suppletive with an elsewhere root /l-/ and a marked or aorist root /lɑ\t{ts}ʰ-/. Iranian regularized this form by only using the aorist root.}

\begin{table}[H]
	\centering
	\caption{Verbs that were lost or replaced in Iranian Armenian, and their  past perfective}\label{tab:past perf template irregular open}
	\begin{tabular}{| l| lll| }
		\hline       &  Eastern &  Iranian &  
		\\
		\hline 
		`to hit' & 	  {zɑɾk-e-l} & {χəpʰ-e-l} & $\sqrt{~}$\textsc{-th-inf}  
		\\ & \armenian{զարկել}& \armenian{խփել}&
		\\
		`they hit (\textsc{pst})'		& {zɑɾk-e-\uline{\t{ts}ʰ-i}-n} & {χəpʰ-$\emptyset$-\uline{$\emptyset$-ɒ}-n} & $\sqrt{~}$\textsc{-th-pfv-pst-3pl} 		\\
		& \armenian{զարկեցին}& \armenian{խփան}& 
		\\
		\hline 
		`to cry' & 	  {l-ɑ-l} & {lɒ\t{ts}ʰ-e-l} & $\sqrt{~}$\textsc{-th-inf}  
		\\ & \armenian{լալ}& \armenian{լացել}&
		\\
		`they cried' 		& {lɑ\t{ts}ʰ-e-\uline{\t{ts}ʰ-i}-n} & {lɒ\t{ts}ʰ-$\emptyset$-\uline{$\emptyset$-ɒ}-n} & $\sqrt{~}$\textsc{-th-pfv-pst-3pl} 		\\
		& \armenian{լացեցին}& \armenian{լացան}& 
		\\
		\hline 
		
		
	\end{tabular}
	
	
	
	
\end{table}





\subsection{Other permutations of past and perfective marking}\label{section:verb:past:other perm}
Across the two lects, the predominant templates for the past perfective are /{-\t{ts}ʰ-i}/ and /{-$\emptyset$-ɒ}/ (Eastern: /{-$\emptyset$-ɑ}/). But with these four morphs, there are two other logically possible permutations. Both are attested in either one or both the lects.

The overt perfective morph /{-\t{ts}ʰ-}/ typically surfaces with the past morph /-i-/ in both lects. However, there is one special conjugation class  where the /{-\t{ts}ʰ-}/ surfaces with the past /ɑ/ or /ɒ/. That class is inchoatives. In Eastern, the infinitive form of an inchoative verb is formed by adding the suffix sequence /{-ɑ-n-ɑ-l}/ onto the stem. In the past perfective, the nasal-vowel sequence is deleted and replaced by /{-\t{ts}ʰ-ɑ}/. Iranian behaves the same. 


\begin{table}[H]
	\centering
	\caption{Past perfective template for inchoatives in  Eastern and Iranian}\label{tab:past perf template inch}
	\begin{tabular}{| l| lll| }
		\hline     Inchoative  &  Eastern &  Iranian `to become happy'&  
		\\
		\hline 
		`to become happy'	 & 	  {uɾɑχ-ɑ-n-ɑ-l} & {uɻɒχ-ɒ-n-ɒ-l} & $\sqrt{~}$\textsc{-lv-inch-th-inf} \\ & \armenian{ուրախանալ}& \armenian{ուրախանալ}&
		\\
		`they became happy'	& {uɾɑχ-ɑ-$\emptyset$-$\emptyset$-\uline{\t{ts}ʰ-ɑ}-n} & {uɻɒχ-ɒ-$\emptyset$-$\emptyset$-\uline{\t{ts}ʰ-ɒ}-n} & $\sqrt{~}$\textsc{-lv-inch-th-pfv-pst-3pl} 		\\
		& \armenian{ուրախացան}& \armenian{ուրախացան}& 
		\\
		\hline      
		
		
		
	\end{tabular}
	
\end{table}

The verb `to turn into' is irregular and somewhat suppletive. It's inflected as an inchoative verb throughout its paradigm. But before the perfective suffix /-\t{ts}ʰ-/, its rhotic changes its quality.  Again, the {/-$\emptyset$-ɒ/} template is used regardless of this irregularity.




\begin{table}[H]
	\centering
	\caption{Irregular inchoative with a changing liquid and their past perfective template}\label{tab:past perf template irregular turn into }
	\begin{tabular}{| l| lll| }
		\hline       &  Eastern &  Iranian &  
		\\
		\hline 
		`to turn into'& {dɑr-n-ɑ-l} & 	{dɒr-n-ɒ-l}  & $\sqrt{~}$\textsc{-inch-th-inf} \\ 
		& \armenian{դառնալ}& \armenian{դառնալ}&
		\\
		`they turned into'& {dɑɾ-$\emptyset$-$\emptyset$-\uline{\t{ts}ʰ-ɑ}-n} & {dɒɻ-$\emptyset$-$\emptyset$-\uline{\t{ts}ʰ-ɒ}-n}
		& $\sqrt{~}$\textsc{-inch-th-pfv-pst-3pl} 		\\
		& \armenian{դարձան}& \armenian{դարձան}& 
		\\
		\hline      
		
		
	\end{tabular}
	
\end{table}



The sole exception is the irregular inchoative `to wash'. This verb takes the perfective-past template /{-\t{ts}ʰ-i}/. This verb is also exceptional in that it is morphologically classified as an inchoative, but it is semantically transitive.

\begin{table}[H]
	\centering
	\caption{Past perfective template for irregular inchoatives in  Eastern and Iranian}\label{tab:past perf template inch irreg}
	\begin{tabular}{| l| lll| }
		\hline 		Irregular inchoative      &  Eastern &  Iranian &  
		\\
		\hline 
		`to wash' & 	  {ləv-ɑ-n-ɑ-l} & {ləv-ɒ-n-ɒ-l} & $\sqrt{~}$\textsc{-lv-inch-th-inf} \\ & \armenian{լվանալ}& \armenian{լուանալ}&
		\\
		`they washed'& {ləv-ɑ-$\emptyset$-$\emptyset$-\uline{\t{ts}ʰ-i}-n} & {ləv-ɒ-$\emptyset$-$\emptyset$-\uline{\t{ts}ʰ-i}-n} & $\sqrt{~}$\textsc{-lv-inch-th-pfv-pst-3pl} 		\\
		& \armenian{լվացին}& \armenian{լուացին}& 
		\\
		\hline      
		
		
		
	\end{tabular}
	
\end{table}

The fourth permutation /{$\emptyset$-i}/ is attested in colloquial Eastern but not Iranian. In standard Eastern, the past perfective of the E-Class is formed by adding /{-\t{ts}ʰ-i}/ after the verb's theme vowel. But for colloquial Armenian, it is reported that some speakers can  add the template /-{$\emptyset$-i}/ and delete the theme vowel (\citealt[31,97]{Gharagyulyan-1981-ColloquialArmenian}, \citealt[209]{Zakaryan-1981-ColloquialArmenian}, \citealt[25]{Avetyan-2020-TendenciesAnalogicalArmenianAorist}).


\begin{table}[H]
	\centering
	\caption{Optional /{\t{ts}ʰ-i}/ template in colloquial Eastern}\label{tab:past perf template optional colloq east}
	\begin{tabular}{| l| lll| }
		\hline  		E-Class      &  Standard Eastern &  Colloquial Eastern &  
		\\
		\hline 
		`to write'& 	  {ɡəɾ-e-l} & {ɡəɾ-e-l} & $\sqrt{~}$\textsc{-th-inf} \\ & \armenian{գրել}& \armenian{գրել}&
		\\
		`they wrote'& {ɡəɾ-e-\uline{\t{ts}ʰ-i}-n} & {ɡəɾ-$\emptyset$-\uline{$\emptyset$-i}-n} & $\sqrt{~}$\textsc{-th-pfv-pst-3pl} 		\\
		& \armenian{գրեցին}& \armenian{գրին}& 
		\\
		\hline      
		
		
		
	\end{tabular}
	
\end{table}

\subsection{Diachronic origins of the promotion}\label{section:verb:past:diachrony}

Before we continue, we     briefly discuss the fact that   the shift from /-\t{ts}ʰ-i/ to /-$\emptyset$-ɑ/ has been attested in other Armenian lects. 

For example, \citet[201]{Adjarian-1961-Liakatar4Book2Conj} reports that various other Armenian lects in Iran have extended the use of the past marker /-ɑ/. He specifically cites the dialects of Xoy, Salmast, Tabriz, and Maragha as examples. Although it is unclear how intimate Tehrani Iranian Armenian is with these lects, it is possible that a combination of dialect-internal changes and cross-dialectal contact caused the promotion of the /-$\emptyset$-ɒ/ template. 

Within Eastern Armenian, there are some high-frequency verbs which in standard speech use the /-\t{ts}ʰ-i/ template. Note that in the 3SG, the past marker and Agr marker are covert while the perfective marker /-\t{ts}ʰ/ is present.  But in colloquial speech,  it is reported that some of these verbs can take the  /-$\emptyset$-ɑ/ template, especially in the 3SG. In the 3SG, the past marker /-ɑ/ is preceded by an overt 3SG marker /-v/. The data below are taken from    \citet[230]{DumTragut-2009-ArmenianReferenceGrammar} (citing \citealt[98]{Gharagyulyan-1981-ColloquialArmenian}). The IPA transcriptions are our own. 

\begin{table}[H]
	\centering
	\caption{High-frequency verbs that colloquially   take /-$\emptyset$-ɑ/ in Eastern Armenian}
	\label{tab:eastern high freq a}
	\begin{tabular}{|ll|ll| l| }
		\hline          Infinitival &&  \multicolumn{2}{l|}{Past perfective 3SG}  & \\
		& & Standard Eastern & Colloquial Eastern & 
		\\
		\hline    & $\sqrt{~}$-\textsc{th-inf}  & $\sqrt{~}$-\textsc{th-\uline{pfv-pst}-3sg} & $\sqrt{~}$-\textsc{th-\uline{pfv-pst}-3sg}   & 
		
		\\
		`to sit'&   nəst-e-l    & nəst-e-\uline{\t{ts}-$\emptyset$}-$\emptyset$ & nəst-$\emptyset$-\uline{$\emptyset$-ɑ}-v & `he sat'
		\\
		&     \armenian{նստել}& \armenian{նստեց} & \armenian{նստավ}& 
		\\
		`to bring'&   beɾ-e-l    & beɾ-e-\uline{\t{ts}-$\emptyset$}-$\emptyset$ & beɾ-$\emptyset$-\uline{$\emptyset$-ɑ}-v & `he brought'
		\\
		&     \armenian{բերել}& \armenian{բերեց} & \armenian{բերավ}&
		\\
		`to give'&   t-ɑ-l    & təv-e-\uline{\t{ts}-$\emptyset$}-$\emptyset$ & təv-$\emptyset$-\uline{$\emptyset$-ɑ}-v & `he gave'
		\\
		&     \armenian{տալ}& \armenian{տվեց} & \armenian{տվավ}&
		\\
		`to say'&   ɑs-e-l    & ɑs-ɑ-\uline{\t{ts}-$\emptyset$}-$\emptyset$ & ɑs-$\emptyset$-\uline{$\emptyset$-ɑ}-v & `he said'
		\\
		&     \armenian{ասել}& \armenian{ասաց} & \armenian{ասավ}&
		\\
		`to start'&   skəs-e-l    & skəs-e-\uline{\t{ts}-$\emptyset$}-$\emptyset$ & skəs-$\emptyset$-\uline{$\emptyset$-ɑ}-v & `he started'
		\\
		&     \armenian{սկսել}& \armenian{սկսեց} & \armenian{սկսավ}&
		\\ \hline \end{tabular}
\end{table}

\citet{Avetyan-2020-TendenciesAnalogicalArmenianAorist} provides more examples of high-frequency verbs that show the above variation. He also provides data that this change is also affecting passive verbs.  Personal communication with \citeauthor{Avetyan-2020-TendenciesAnalogicalArmenianAorist} (Sargis Avetyan) suggests that the template has   spread quite quickly to all passives and intransitives in colloquial Eastern. Although most attested examples are for the 3SG, Avetyan informs us that the other persons can also take the /-$\emptyset$-ɑ/ template. Data is from his paper and personal communication. IPA is our own.   


\begin{table}[H]
	\centering
	\caption{Passives  verbs that colloquially   take /-$\emptyset$-ɑ/ in Eastern Armenian}
	\label{tab:eastern high freq passive }
	\begin{tabular}{|ll|ll| l| }
		\hline          Infinitival &&  \multicolumn{2}{l|}{Past perfective 3SG}  & \\
		& & Standard Eastern & Colloquial Eastern & 
		\\
		\hline    & $\sqrt{~}$-\textsc{pass-th-inf}  & \multicolumn{2}{l|}{$\sqrt{~}$-\textsc{pass-th-\uline{\textsc{pfv}}-\uline{\textsc{pst}}-3sg} }& 
		
		\\
		`to be accepted'&   əntʰun-v-e-l    & əntʰun-v-e-\uline{\t{ts}-$\emptyset$}-$\emptyset$ & əntʰun-v-$\emptyset$-\uline{$\emptyset$-ɑ}-v & `they were accepted'
		\\
		&     \armenian{ընդունվել}& \armenian{ընդունվեց} & \armenian{ընդունվավ}&
		\\
		`to be killed'&   spɑn-v-e-l    & spɑn-v-e-\uline{\t{ts}-$\emptyset$}-$\emptyset$ & spɑn-v-$\emptyset$-\uline{$\emptyset$-ɑ}-v & `they were killed'
		\\
		&     \armenian{սպանվել}& \armenian{սպանվեց} & \armenian{սպանվեց}&
		\\ \hline \end{tabular}
\end{table}



Western Armenian has some similar morphological changes but they are more difficult to contrast with Iranian/Eastern \citep{Boyacioglu-2010-HayPayVerbsArmenianOccidentalWestArmenian,boyaciogluDolatian-2020-ArmenianVerbs}.  Some high-frequency irregular verbs like `to bring' are heteroclitic meaning they switch between conjugation classes based on specific person-number features \citep{stump-2006-heteroclisisParadigmLinkage}. In the case of the past perfective, the Western irregular verb `to bring' takes the /-$\emptyset$-ɑ/ template for the 3SG, and /-$\emptyset$-i/ for other persons/numbers. Some verbs like `to sit' take a theme vowel /-i-/, and they use the past perfective template /-\t{ts}ʰ-ɑ/ in standard Western, but /-$\emptyset$-ɑ/ in colloquial speech for all persons/numbers. Furthermore, for colloquial Western Armenian in Lebanon, some passives can take  /-$\emptyset$-ɑ/, especially for the 3SG. Data is from HD.


\begin{table}[H]
	\centering
	\caption{High-frequency verbs that variably   take /-$\emptyset$-ɑ/ in Western Armenian}
	\label{tab:western high freq a }
	
	\resizebox{.92\textwidth}{!}{%
		\begin{tabular}{|ll|ll|ll|   }
			\hline 
			& Infinitival & Std. 3SG  & Coll. 3SG  & Std. 3PL &  Coll. 3PL  
			\\
			& $\sqrt{~}$-\textsc{(pass)-th-inf}& \multicolumn{2}{l|}{$\sqrt{~}$-\textsc{(pass)-th-pfv-pst-3sg}}& \multicolumn{2}{l|}{$\sqrt{~}$-\textsc{(pass)-th-pfv-pst-3pl}}
			\\
			
			
			`to bring' & pʰeɾ-e-l& pʰeɾ-$\emptyset$-\uline{$\emptyset$-ɑ}-v & pʰeɾ-$\emptyset$-\uline{$\emptyset$-ɑ}-v& pʰeɾ-$\emptyset$-\uline{$\emptyset$-i}-n & pʰeɾ-$\emptyset$-\uline{$\emptyset$-i}-n
			\\
			& \armenian{բերել}& \armenian{բերաւ} & \armenian{բերին}& \armenian{բերաւ} & \armenian{բերին}
			\\
			
			`to sit' & nəst-i-l& nəst-e-\uline{\t{ts}ʰ-ɑ}-v& nəst-$\emptyset$-\uline{$\emptyset$-ɑ}-v& nəst-e-\uline{\t{ts}ʰ-ɑ}-n& nəst-$\emptyset$-\uline{$\emptyset$-ɑ}-n 
			\\
			& \armenian{նստիլ}& \armenian{նստեցաւ}& \armenian{նստաւ}& \armenian{նստեցան} & \armenian{նստան} 
			\\
			`to be  & ɑvɾə-v-i-l& ɑvɾə-v-e-\uline{\t{ts}ʰ-ɑ}-v& ɑvɾə-v-$\emptyset$-\uline{$\emptyset$-ɑ}-v& ɑvɾə-v-e-\uline{\t{ts}ʰ-ɑ}-n& ɑvɾə-v-$\emptyset$-\uline{$\emptyset$-ɑ}-n 
			\\
			destroyed' & \armenian{աւրուիլ}& \armenian{աւրուեցաւ}& \armenian{աւրուաւ}& \armenian{աւրուեցան} &\armenian{աւրուան} 
			
			\\ \hline \end{tabular}
	}
\end{table}

All in all, outside of Tehrani Iranian Armenian, there are Armenian lects and registers which have extended the use of /-$\emptyset$-ɑ/ template to some classes or sets of verbs. In the case of colloquial Eastern and Western Armenian, this spread has limited to a small set of irregular, high-frequency, and sometimes passive/intransitive verbs. But in Iranian, this extension has been widespread, affecting verbs of any valency as long as they're in the right class. 

\subsection{Summary}\label{section:verb:past:summary}

The previous sections examined the formation of the past perfective for each regular verb class and for each type of irregular verb. This section summarizes this information and provides some basic statistics on the inflectional templates for the past perfective. 

To summarize, Eastern Armenian uses the template {/-\t{ts}ʰ-i/} as the default template for  verb inflection in the past perfective. A smaller number of irregular verbs instead use the template   {/-$\emptyset$-ɑ/}. In contrast in Iranian, the default template is {/-$\emptyset$-ɒ/}, while the template {/-\t{ts}ʰ-i/}  is significantly restricted in its distribution. 

To reinforce this generalization, Table \ref{tab:verb class past perf stats major} summarizes the  types of regular and irregular verb that we discussed for Eastern and Iranian. The table shows the template for past perfective formation in both Eastern and Iranian. We use checkmarks to state whether some class or verb used the {/-$\emptyset$-ɑ/} template in Eastern. For those classes which don't do so in Eastern, most have shifted to using  {/-$\emptyset$-ɒ} in Iranian. 





\begin{table}[H]
	\centering
	\caption{Distribution of past perfective templates across regular and irregular verbs}
	\label{tab:verb class past perf stats major}
	\begin{tabular}{|ll|ll| l|l|l|   }
		\hline 
		Category & & \multicolumn{2}{l|}{Past perfective template} & Class size & {/-$\emptyset$-ɑ/} in& Shift? 
		\\
		& & Eastern & Iranian &  in Eastern?& Eastern&
		\\
		\hline  \hline
		\multicolumn{7}{|l|}{Regular}
		\\ \hline
		E-Class & (Table \ref{tab:past perf template eclass })& {/-\t{ts}ʰ-i/} &  {/-$\emptyset$-ɒ}/ & 6672 (46.87\%) && \ding{51}
		\\
		Passive & (Table \ref{tab:past perf template passive })& {/-\t{ts}ʰ-i/}&  {/-$\emptyset$-ɒ}/ & 4089 (28.73\%) && \ding{51}
		\\
		Causative & (Table \ref{tab:past perf template caus})&    {/-$\emptyset$-i/} (Std.) &  {/-$\emptyset$-ɒ}/ & 1457 (10.24\%)&& \ding{51}
		\\
		& &  /{\t{ts}ʰ-i}/ (Coll.)& & &&
		\\
		Inchoative &(Table \ref{tab:past perf template inch}) &    {/-\t{ts}ʰ-ɑ/} &  {/-\t{ts}ʰ-ɒ}/ & 1191 (8.37\%) &&
		\\
		A-Class & (Table \ref{tab:past perf template aclass infix}) &   {/-\t{ts}ʰ-i/} &  {/-\t{ts}ʰ-i}/ & 783 (5.5\%) &&
		\\
		\hline \hline
		\multicolumn{7}{|l|}{Irregular or suppletive }
		\\
		\hline
		Infixed &  (Table \ref{tab:past perf template aclass infix})&    {/-$\emptyset$-ɑ/} &  {/-$\emptyset$-ɒ}/ & 29 (0.20\%) & \ding{51}&
		\\
		`to let' &
		(Table \ref{tab:past perf template all infix}) & {/-\t{ts}ʰ-i/} &  {/-$\emptyset$-ɒ}/ & 1&& \ding{51}
		\\
		`to come' &      (Table \ref{tab:past perf template suppletive zero-a east}) &  {/-$\emptyset$-ɑ/} &  {/-$\emptyset$-ɒ}/ & 1& \ding{51}&
		\\
		`to eat' &      (Table \ref{tab:past perf template suppletive zero-a east}) &  {/-$\emptyset$-ɑ/} &  {/-$\emptyset$-ɒ}/ & 1& \ding{51}&
		\\
		`to take to'&      (Table \ref{tab:past perf template suppletive zero-a east}) &  {/-$\emptyset$-ɑ/} &  {/-$\emptyset$-ɒ}/ & 1& \ding{51}&
		\\
		`to give' & (Table \ref{tab:past perf template suppletive zero-a iran but not east}) & {/-\t{ts}ʰ-i/} &  {/-$\emptyset$-ɒ}/ & 1& &\ding{51}
		\\
		`to put'& (Table \ref{tab:past perf template suppletive zero-a iran but not east}) & {/-\t{ts}ʰ-i/} &  {/-$\emptyset$-ɒ}/ & 1&& \ding{51}
		\\
		`to do'& (Table \ref{tab:past perf template suppletive zero-a iran but not east}) & {/-\t{ts}ʰ-i/} &  {/-$\emptyset$-ɒ}/ & 1& &\ding{51}
		\\
		`to go' & (Table \ref{tab:past perf template suppletive zero-a in neither east iran})  & {/-\t{ts}ʰ-i/}  & {/-\t{ts}ʰ-i/}  & 1&&
		\\
		`to be' & (Table \ref{tab:past perf template verb to be}) &  {/-$\emptyset$-ɑ/} &  {/-$\emptyset$-ɒ}/ & 1& \ding{51}&
		\\
		`to say' & (Table \ref{tab:past perf template irregular imp})  & {/-\t{ts}ʰ-i/} &  {/-$\emptyset$-ɒ}/ & 1&& \ding{51}
		\\
		`to bring'& (Table \ref{tab:past perf template irregular imp}) & {/-\t{ts}ʰ-i/} &  {/-$\emptyset$-ɒ}/ & 1&& \ding{51}
		\\
		`to turn into' & (Table \ref{tab:past perf template irregular turn into }) &  {/-$\emptyset$-ɑ/} &  {/-$\emptyset$-ɒ}/ & 1& \ding{51}&
		\\
		`to wash' & (Table \ref{tab:past perf template inch irreg})  & {/-\t{ts}ʰ-i/}  & {/-\t{ts}ʰ-i/} & 1& & 
		\\
		\hline 
		Total & & & &  14234 & & 
		\\\hline 
	\end{tabular}
\end{table}

Table \ref{tab:verb class past perf stats major} also provides a simple frequency count of the regular verb classes and the irregular infixed verbs. The numbers are taken from the lemma count of these verbs from the Eastern Armenian National Corpus \citep{khurshudian-2009-easternArmenianNationalCorpusEANC}. These numbers approximate the class size of these verbs for Eastern Armenian.\footnote{The lexicon was accessed (Sept 2020) from EANC's source-code\url{https://bitbucket.org/timarkh/uniparser-grammar-eastern-armenian/src/master/}. We used the following EANC paradigm indexes to find the relevant classes:  irregular infixed (V12\_main or V13), causative (V14\_main), A-Class (V21), inchoative (V22\_main). Passives are in the V11\_main paradigm, end in \armenian{վել}, and have the "med" tag. E-Class are all other members of the V11\_main paradigm.} 

The EANC lexicon has 14329 verb lemmas.  When comparing Eastern against Iranian, we did not count the frequency of irregular or suppletive verbs in the EANC because some of these words don't exist in Iranian. For irregulars besides basic infixed verbs, we only provide a count of 1. We likewise set aside Eastern verbs which belong to archaic conjugation classes.  This left 14234 verbs, reported in Table \ref{tab:verb class past perf stats major}


The main take-away is that over 80\% of Eastern Armenian verbs (E-Class, passives, causatives) shifted from taking the /-\t{ts}ʰ-i/ template to the /-$\emptyset$-ɑ/ template (/-$\emptyset$-ɒ/ in Iranian). This shift was ultimatley due to analogy to the handful of irregular and suppletive verbs that take the /-$\emptyset$-ɑ/ template. 
