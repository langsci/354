 

\chapter{Syntax}\label{chapter:syntax}

In terms of its syntax, {{\iaIA}}  is largely identical to {{\seaSEA}}. As such, we do not go over the syntax of {{\iaIA}} in depth. In terms of general typological features, {{\iaIA}}  is SOV (\ref{ex:syntax:general:SOV}), has optional post-verbal objects (\ref{ex:syntax:general:SVO}),  uses pro-drop (\ref{ex:syntax:general:drop}), and contextually-implied objects can drop too (\ref{sent:syntax:objDrop}). 

\begin{exe}
	\ex 
	\begin{xlist}
		\ex \gll   {d͡ʒɒn-ə} {ind͡z} {mɒkʰɻ-ɒ-v}
		\\
		John-{\defgloss} I.{\dat} clean-{\pst}-3{\sg}
		\\
		\trans	`John cleaned me.' \label{ex:syntax:general:SOV} \hfill (NK)
		\\
		\armenian{Ջոնը ինձ մաքրաւ։ }
		\ex \gll   {d͡ʒɒn-ə}  {mɒkʰɻ-ɒ-v} {ind͡z}
		\\
		John-{\defgloss}  clean-{\pst}-3{\sg} I.{\dat}
		\\
		\trans		`John cleaned me.'  \label{ex:syntax:general:SVO}\hfill (NK)
		\\
		\armenian{Ջոնը ինձ մաքրաւ։}
		\ex \gll    {mɒkʰuɻ}  e-m
		\\
		clean {\auxgloss}-1{\sg}
		\\
		\trans	`I am clean.'  \label{ex:syntax:general:drop}\hfill (NK)
		\\
		\armenian{Մաքուր եմ։}
		\ex \label{sent:syntax:objDrop} %{\added}
		\begin{xlist}
			\ex\gll d͡ʒuɻ-ə χəm-ɒ-ɻ \\
			water-{\defgloss} drink-{\pst}-2{\sg} \\
			\trans `Did you drink the water?' \hfill (NK) \\
			\armenian{Ջուրը խմա՞ր։}
						\ex\gll ɒjo,   χəm-ɒ-m \\
			yes drink-{\pst}-1{\sg} \\
			\trans `Yes, I drank it ' \hfill (NK) \\
			\armenian{Այո, խմամ։}
		\end{xlist}
	\end{xlist}
\end{exe}

\begin{sloppypar}
More in-depth   studies of {{\seaSEA}} syntax exist \citep{DumTragut-2009-ArmenianReferenceGrammar,Yeghiazaryan-2010-ArmenianCase,Su-2012-syntaxFunctionalProjectionsvPPeriphery,Hodgson-2019-DissRelativeClauseArmenianSyntax,KhurshudayanDonabedian-2021-FocusStrategiesAndCleftConstructionsModerArmenain} and these descriptions largely apply to {{\iaIA}}.  Furthermore, there are some studies of ``Eastern Armenian'', but these are actually done based on data from {{\iaIA}} speakers who are bi-dialectal \citep{Stevick-1955-SyntaxColloquailEasternArmenian,Tamrazian-1994-ArmenianSyntax,Megerdoomian-2009-ThesisBook}.
\end{sloppypar}

This chapter focuses on describing those aspects of {{\iaIA}} syntax that are innovative when compared to {\seaSE}. Some of these are grammaticalized from attested colloquial and optional properties of  {{\seaSEA}}. Some of these changes were likely encouraged by the use of similar structures in Persian \citep[cf. other language-contact effects in the region:][]{donabedianSitaridou-2021-anatolia}. These changes are listed below. 
% \ref{list:Syntax summary}.  

%\begin{exe}
%	\ex Syntactic innovations and grammaticalizations in {{\iaIA}}: \label{list:Syntax summary}
%	\begin{itemize}
	\begin{itemize}
		\item  Using the second person possessive suffix   as an object clitic  (\S\ref{section:syntax:clitic}) $\rightarrow$ borrowed from Persian
		\item  Preference for using resumptive pronouns over case-marked relativizers  (\S\ref{section:syntax:resumptive}) $\rightarrow$ lan\-guage-internal but encouraged   from Persian
		\item  Preference for subjunctive marking  in complement clauses   (\S\ref{section:syntax:subj}) $\rightarrow$ lan\-guage-internal but encouraged   from Persian
		\item  Variation in expressing subject marking in participle clauses  (\S\ref{section:syntax:participleClause}) $\rightarrow$ lan\-guage-internal 
	\end{itemize}
	
In previous sections of this grammar, we did briefly discuss some major aspects of {{\iaIA}} syntax. These include   auxiliary movement (\S\ref{section:morphophono:auxiliary:syntax}) and interrogative questions (\S\ref{section:funct:wh}). Their syntax does not significantly differ from {\seaSEA}. 

Throughout this chapter, Persian sentences were elicited from Nazila Shafiei (NS), an Iranian syntactician. We use the glossing that she provided.   The IPA transcriptions were double-checked with     Koorosh Ariyaee, an Iranian phonologist.\footnote{   Ariyeae   notes that what we transcribed as a Persian [ɒ] may be closer to [ɑ] for Iranian Persian speakers. See footnote \ref{footnote persian a} in \S\ref{section:phono:segmental:vowel} for discussion. } The {\seaSEA} sentences were judged by the consultants mentioned in \S\ref{section: intro: fieldwork}. 

\section{Object clitic for second person}\label{section:syntax:clitic}
Due to contact with Persian, {{\iaIA}}   has extended the use of the 2SG possessive suffix /{-(ə)t}/ into an object clitic. Within Armenian dialectology, the use of /{-(ə)t}/ as an object clitic has been previously attested for Armenian dialects in Iran (\citealt[1159]{SayeedVaux-2017-EvolutionArmenian}, citing  \cites[284]{Adjarian-1911-DialectologyBook}[item 675]{MuradyanEtAl-1977-DialectologyBook}[340]{Khurshudian-2020-someAspectsPossessiveMarkersModernArmenian}{Hodgson-202x-GrammaticaliztionDefiniteARticleArmenian}[87]{Martirosyan-2019-Armeniandialects}[\S 4.1]{Vaux-Salmast}). 

For the Armenian community of Tehran and the diaspora, AS reports that this   use of the clitic is ``prevalent in generation Y's vernacular,'' where generation Y is anyone born in the 80’s or 90’s.   The use of the clitic is stigmatized because it is part of a  ``very informal register.''  Speakers are aware of the register difference. 

Most of our consultants could use the Armenian possessive as an object clitic. Some Iranian Armenians who were born and raised in the diaspora however said they had never heard of such constructions. 



\subsection{General use of the object clitic}\label{section:syntax:clitic:general}

In its typical uses, the morpheme /{-(ə)t}/ acts as a second person possessive suffix on nouns (\ref{sent:Syntax:Clitic:General:basic}). 

\begin{exe}
	\ex \gll {senjɒk-ət}  
	\\
	room={\possSsg}
	\\
	\trans	`your room' \label{sent:Syntax:Clitic:General:basic}
	\\
	\armenian{Սենեակդ։}
\end{exe}

But in {{\iaIA}}, this morpheme also functions as an object clitic (\ref{sent:Syntax:Clitic:General:clitic}).  As a clitic, this morpheme  has some correlations with tense,  mood, and valency. For example, many instances of the clitic are found for verbs with the  synthetic future.  The clitic is mostly used  to replace the direct object of a transitive verb. 

\begin{exe}
	\ex \label{sent:Syntax:Clitic:General:clitic}
	\begin{xlist}
		\ex \gll {kə-χəpʰ-e-m} {kʰez}   
		\\
		{\fut}-hit-{\thgloss}-1{\sg} you.{\sg}.{\dat}
		\\
		\trans	`I {will} hit you.' \hfill   (NK, AP, KM)
		\\
		\armenian{Կը խփեմ քեզ։}
		\ex \gll {kə-χəpʰ-e-m=ət} 
		\\
		{\fut}-hit-{\thgloss}-1{\sg}={\possSsg}
		\\
		\trans	`I {will} hit you.' \hfill   (NK, AP, KM)
		\\
		\armenian{Կը խփեմդ։}
	\end{xlist}
	
\end{exe}

Throughout this chapter, we gloss the /{(ə)t}/ morpheme consistently as a possessive, even when it is functioning as an object clitic /=(ə)t/. 

Although the second person possessive /{-(ə)t}/ can function as an object clitic, the first person possessive /{-(ə)s}/ cannot (\ref{sent:Syntax:Clitic:General:1sg}). 


\begin{exe}
	\ex \label{sent:Syntax:Clitic:General:1sg}
	\begin{xlist}
		\ex \gll {kə-χəpʰ-e-n} {ind͡z}   
		\\
		{\fut}-hit-{\thgloss}-3{\pl} I.{\dat}
		\\
		\trans	`They {will} hit me.' \hfill   (NK)
		\\
		\armenian{Կը խփեն ինձ։}
		\ex \gll *{kə-χəpʰ-e-n=əs} 
		\\
		{\fut}-hit-{\thgloss}-3{\pl}={\possFsg}
		\\
		\trans	Intended: `They  {will} hit me.' \hfill   (*NK)
		
	\end{xlist}
	
\end{exe}

Similarly, the definite suffix is used for third person possessive marking, but it cannot be used as an object clitic (\ref{sent:Syntax:Clitic:General:3sg}). 

\begin{exe}
	\ex \label{sent:Syntax:Clitic:General:3sg}
	\begin{xlist}
		\ex \gll {kə-χəpʰ-e-m} {iɾɒn}   
		\\
		{\fut}-hit-{\thgloss}-1{\sg} he.{\dat}
		\\
		\trans	`I {will} hit him.' \hfill   (NK)
		\\
		\armenian{Կը խփեմ իրան։}
		\ex \gll *{kə-χəpʰ-e-m=ə} 
		\\
		{\fut}-hit-{\thgloss}-1{\sg}={\defgloss}
		\\
		\trans	Intended: `I  {will} hit him.' \hfill   (*NK)
		
	\end{xlist}
	
\end{exe}



There is no clitic option for   plural objects. 


The use of the possessive /{-t}/ as an object clitic likely developed by contact from Persian, which has an entire set of pronominal clitics that act as object clitics for every person-number combination (\cites[138]{Mahootian-2002-PersianGrammar}{SamvelianTseng-2010-persianObjectCliticSyntaxMorphologyInterface}). The object of a transitive verb can be either present (\ref{ex: persian obj}) or absent (\ref{ex: persian objcl}). When the object is absent,  Persian uses object clitics on the verb (\ref{ex: persian objcl}). 



\begin{exe}
	\ex Object cliticization in Persian
	\begin{xlist}
		\ex \gll  {(mæn)} {to=ro} {mi-zæn-æm}
		\\
		(I) you={\om} {\impf}-hit-1{\sg}\label{ex: persian obj}
		\\ 
		\trans	`I'm going to hit you.' \hfill (NS) 
		\\ 
		\textarab{من تو رو میزنم.}
		\ex \gll  {(mæn)} {mi-zæn-æm=et }
		\\
		(I)    {\impf}-hit-1{\sg}=2{\sg}
		\\
		\trans	`I'm going to hit you.'\label{ex: persian objcl}  \hfill (NS)
		\\
		\textarab{من میزنمت.}
	\end{xlist}
\end{exe}

Although Persian allows object clitics for every person-number combination, {\iaIA} has an object clitic /-t/ for only the 2SG.  It is unclear why this restriction exists. Don Stilo (p.c.) suggests that the restriction might exist because of formality. To quote him:

\begin{quote}
		It seems to me that  this use of the possessive clitic as an object clitic only with the 2nd singular further emphasizes the `informal' nature of this pattern. That is, since it is only used in the 2nd singular, this possibly shows that it is only used with friends. Otherwise, what would be the logic of using it only in the 2nd singular when Persian uses these clitics universally in all persons?\end{quote} 

Furthermore,  as we discuss in the following sections, the object clitic prefers certain tenses and moods; it is unclear to us if these restrictions were also copied from Persian. 




\subsection{Object clitic for direct objects in the synthetic future}\label{section:syntax:clitic:future}
As stated earlier, the most typical use of the object clitic is to replace the direct object of a verb in the synthetic future. The synthetic future is marked by the prefix /{k-}/. 



The object clitic can be used for a range of verbs (\ref{sent:Syntax:Clitic:ObjFut:basic}). These all seem to be verbs of physical action. More data is needed to determine if this is a general restriction or a  tendency. For some cases, the use of the clitic carries an emphatic connotation, e.g., \textit{{kə-spɒn-e-m=ət}} `(I am so mad that) I {will} kill you’.

\begin{exe}
	\ex \label{sent:Syntax:Clitic:ObjFut:basic}
	\begin{xlist}
		\ex 
		\begin{xlist}
			\ex \gll {kə-spɒn-e-m} {kʰez}
			\\
			{\fut}-kill-{\thgloss}-1{\sg} you.{\sg}.{\dat}
			\\
			\trans			`I {will} kill you.'  \hfill (NK)
			\\
			\armenian{Կը սպանեմ քեզ։}
			
			\ex \gll {kə-spɒn-e-m=ət} 
			\\
			{\fut}-kill-{\thgloss}-1{\sg}={\possSsg}
			\\
			\trans	`I {will} kill you.' \hfill (NK)
			\\
			\armenian{Կը սպանեմդ։}
			
		\end{xlist}
		\ex 
		\begin{xlist}
			\ex \gll {kə-χeχt-e-m}   {kʰez}  
			\\
			{\fut}-strangle-{\thgloss}-1{\sg} you.{\sg}.{\dat}
			\\
			\trans	`I {will} strangle you.'  \hfill    (NK)
			\\
			\armenian{Կը խեղդեմ քեզ։}
			\ex\gll {kə-χeχt-e-m=ət} 
			\\
			{\fut}-strangle-{\thgloss}-1{\sg}={\possSsg}
			\\
			\trans	`I {will} strangle you.'  \hfill    (NK)
			\\
			\armenian{Կը խեղդեմդ։}
		\end{xlist}
		
		\ex 
		\begin{xlist}
			\ex \gll {kə-bərn-e-m}  {kʰez}   
			\\
			{\fut}-hold-{\thgloss}-1{\sg} you.{\sg}.{\dat}
			\\
			\trans	`I {will} hold you.'  \hfill    (NK)
			\\
			\armenian{Կը բռնեմ քեզ։}
			\ex \gll {kə-bərn-e-m=ət}. 
			\\
			{\fut}-hold-{\thgloss}-1{\sg}={\possSsg}
			\\
			\trans	`I {will} hold you.'  \hfill    (NK)
			\\
			\armenian{Կը բռնեմ քեզ։}
			\\
			\armenian{Կը բռնեմդ։}
	\end{xlist}	\end{xlist}
\end{exe}


For some transitives, the clitic cannot be used by AP (\ref{sent:Syntax:Clitic:ObjFut:apkm}).  Some of them can be used by KM. 

\begin{exe}
	\ex \label{sent:Syntax:Clitic:ObjFut:apkm}
	\begin{xlist}
		\ex 
		\begin{xlist}
			\ex \gll {kə-tɒn-e-m}   {kʰez}
			\\
			{\fut}-take-{\thgloss}-1{\sg}  you.{\sg}.{\dat}
			\\
			\trans			`I {will} take you.' \hfill  (AP).
			\\
			\armenian{Կը տանեմ քեզ։}
			\ex \gll {*kə-tɒn-e-m=ət} 
			\\
			{\fut}-take-{\thgloss}-1{\sg}={\possSsg}
			\\
			\trans			`I {will} take you.' \hfill (*AP, okay KM)
			\\
			\armenian{Կը տանեմդ։}
			
		\end{xlist}
		\ex 
		\begin{xlist}
			\ex \gll {kə-pʰəntr-e-m}  {kʰez} 
			\\
			{\fut}-take-{\thgloss}-1{\sg}  you.{\sg}.{\dat}
			\\
			\trans			`I {will} look for you.' \hfill  (AP)
			\\
			\armenian{Կը փնտռեմ քեզ։}
			\ex \gll {*kə-pʰəntr-e-m=ət}
			\\
			{\fut}-take-{\thgloss}-1{\sg}={\possSsg}
			\\
			\trans	Intended: `I {will} look for you.' \hfill  (*AP)
			
	\end{xlist}\end{xlist}
\end{exe}



The verb [{{mɒt͡ʃʰel}}]  `to kiss’   cannot take the clitic for NK (\ref{sent:Syntax:Clitic:ObjFut:kiss}).

\begin{exe}
	\ex \label{sent:Syntax:Clitic:ObjFut:kiss}
	\begin{xlist}
		\ex \gll {kə-mɒt͡ʃʰ-e-m } {kʰez}
		\\
		{\fut}-kiss-{\thgloss}-1{\sg} you.{\sg}.{\dat}
		\\
		\trans	`I {will} like you.' \hfill    (NK)
		\\
		\armenian{Կը մաչեմ քեզ։}
		\ex {*kə-mɒt͡ʃʰ-e-m=ət} 
		\\
		{\fut}-kiss-{\thgloss}-1{\sg}={\possSsg}
		\\
		\trans	Intended: `I {will} kiss you.' \hfill (*NK)
	\end{xlist}
\end{exe}

In the domain of verbs of speech, the transitive verbs [{{kɒnt͡ʃʰel}}]  `to call’ and [{{zɒŋɡel}}]  `to phone’ can take the clitic for some speakers (\ref{sent:Syntax:Clitic:ObjFut:othertrans}).

\begin{exe}
	\ex \label{sent:Syntax:Clitic:ObjFut:othertrans}
	\begin{xlist}
		\ex 
		\begin{xlist}
			\ex 
			\gll {kə-kɒnt͡ʃʰ-e-m}  {kʰez} 
			\\
			{\fut}-call-{\thgloss}-1{\sg}  you.{\sg}.{\dat}
			\\
			\trans			`I {will} call you.' \hfill  (AP)
			\\
			\armenian{Կը կանչեմ քեզ։}
			\ex \gll {kə-kɒnt͡ʃʰ-e-m=ət }  
			\\
			{\fut}-call-{\thgloss}-1{\sg}={\possSsg}
			\\ 
			\trans	`I {will} call you.' \hfill (AP)
			\\
			\armenian{Կը կանչեմդ։}
		\end{xlist}	
		\ex 
		\begin{xlist}
			\ex \gll {kə-zɒŋɡ-e-m}  {kʰez}
			\\
			{\fut}-phone-{\thgloss}-1{\sg}  you.{\sg}.{\dat}
			\\
			\trans	`I {will} phone you.' \hfill   (AS)
			\\
			\armenian{Կը զանգեմ քեզ։}
			\ex \gll  {kə-zɒŋɡ-e-m=ət}  
			\\
			{\fut}-call-{\thgloss}-1{\sg}={\possSsg}
			\\
			\trans	`I {will} phone you.' \hfill  (AS)
			\\
			\armenian{Կը զանգեմդ։}
		\end{xlist}	
\end{xlist}	\end{exe}

AS provides a common example with the verb  `to see’. He reports that this is a social expression and a calque from Persian (\ref{sent:Syntax:Clitic:ObjFut:AS}). 

\begin{exe}
	\ex \gll {kə-ɡ-ɒ-s} {kə-tesn-e-m=ət} 
	\\
	{\fut}-come-{\thgloss}-2{\sg} {\fut}-see-{\thgloss}-1{\sg}={\possSsg}
	\\
	\trans	`Come, let me see you.' \hfill   (AS)\label{sent:Syntax:Clitic:ObjFut:AS}
	\\
	\armenian{Կը գաս, կը տեսնեմդ}
\end{exe}

Some verbs like [{siɻel}]  `to like’ cannot take the clitic for some speakers (\ref{sent:Syntax:Clitic:ObjFut:like}). It is unclear if this is idiosyncratic, or if it reflects a restriction against verbs of non-physical action.

\begin{exe}
	\ex \label{sent:Syntax:Clitic:ObjFut:like}
	\begin{xlist}
		\ex \gll {kə-siɻ-e-m}  {kʰez }
		\\
		{\fut}-like-{\thgloss}-1{\sg} you.{\sg}.{\dat}
		\\
		\trans	`I {will} like you.' \hfill   (NK)
		\\
		\armenian{Կը սիրեմ քեզ։}
		\ex \gll {*kə-siɻ-e-m=ət}
		\\
		{\fut}-like-{\thgloss}-1{\sg}={\possSsg}
		\\
		\trans	`Intended:  `I {will} like you.' \hfill  (*NK)
		\\
		\armenian{Կը սիրեմդ։}
	\end{xlist}
\end{exe}

\subsection{Object clitic for other tenses and moods}\label{section:syntax:clitic:other tense}
The previous section focused on   examples of the object clitic when the verb is in the synthetic future. It is rather difficult to find cases where the clitic is added for other tenses and moods for some of our consultants.


In other synthetic tenses, NK expressed uncertainty about using the clitic in the subjunctive (\ref{sent:Syntax:Clitic:OtherTense:Subj}). 

\begin{exe}
	\ex \label{sent:Syntax:Clitic:OtherTense:Subj}
	\begin{xlist}
		\ex \gll {uz-um}  {e-m}   {kʰez} {χəpʰ-e-m }  
		\\
		want-{\impfcvb} {\auxgloss}-1{\sg} you.{\sg}.{\dat} hit-{\thgloss}-1{\sg} 
		\\
		\trans	`I want to hit you.' \hfill (NK)
		\\
		\armenian{Ուզում եմ քեզ խփեմ։}
		\ex \gll ?{uz-um}  {e-m} {χəpʰ-e-m=ət} 
		\\
		want-{\impfcvb} {\auxgloss}-1{\sg} hit-{\thgloss}-1{\sg}={\possSsg}
		\\
		\trans		Intended:  `I want to hit you.’ \hfill (?NK)
	\end{xlist}
	
\end{exe}

AS however provides an example  in the subjunctive. The phrase is a   social expression    (\ref{sent:Syntax:Clitic:OtherTense:SubjAS}).

\begin{exe}
	
	\ex \gll {ɒɻi}  {tes-n-e-m=ət}
	\\
	come.{\imp}.2{\sg} see-{\vx}-{\thgloss}-1{\sg}={\possSsg}
	\\
	\trans	`Come, let me see   you.’ \hfill    (AS)\label{sent:Syntax:Clitic:OtherTense:SubjAS}
	\\
	\armenian{Արի, տեսնեմդ։}
\end{exe}

For the past perfective, NK reports that she cannot use the object clitic (\ref{sent:Syntax:Clitic:OtherTense:perf}). 

\begin{exe}
	\ex \label{sent:Syntax:Clitic:OtherTense:perf}
	\begin{xlist}
		
		\ex \gll {χəpʰ-ɒ-m}  {kʰez} 
		\\
		hit-{\pst}-1{\sg} you.{\sg}.{\dat}
		\\
		\trans	`I hit (past) you.' \hfill   (NK)
		\\
		\armenian{Խփամ քեզ։}
		\ex \gll {*χəpʰ-ɒ-m=ət} 
		\\
		hit-{\pst}-1{\sg}={\possSsg}
		\\
		\trans	Intended:  `I hit (past) you.’ \hfill  (*NK)
		
\end{xlist}\end{exe}


For periphrastic tenses, AS reports that the object clitic can be used (\ref{sent:Syntax:Clitic:OtherTense:periphr}). In such cases, the clitic would cliticize onto the auxiliary. Such cliticization is also reported in the Armenian dialect of Urmia in Iran \citep[282]{Gharibyan-1941-SummaryArmenianDialectology}. 

\begin{exe}
	\ex \label{sent:Syntax:Clitic:OtherTense:periphr}
	\begin{xlist}
		\ex 
		\begin{xlist}
			\ex \gll {nɒj-um}  {e-m}  {kʰez}
			\\
			look-{\impfcvb} {\auxgloss}-1{\sg} you.{\sg}.{\dat}
			\\
			\trans	`I am looking at you.' \hfill   (AS)
			\\
			\armenian{Նայում եմ քեզ։}
			\ex \gll {nɒj-um}  {e-m=ət} 
			\\
			look-{\impfcvb} {\auxgloss}-1{\sg}={\possSsg}
			\\
			\trans	`I am looking at you.' \hfill (AS)
			\\
			\armenian{Նայում եմդ։}
		\end{xlist}
		\ex 
		\begin{xlist}
			\ex \gll {spɒs-um}  {e-m} {kʰez}  
			\\
			wait-{\impfcvb} {\auxgloss}-1{\sg} you.{\sg}.{\dat}
			\\
			\trans			`I am waiting for you.' \hfill (AS)
			\\
			\armenian{Սպասում եմ քեզ։}
			\ex \gll {spɒs-um}  {e-m=ət} 
			\\
			wait-{\impfcvb} {\auxgloss}-1{\sg}={\possSsg}
			\\
			\trans	`I am waiting for you.' \hfill (AS)
			\\
			\armenian{Սպասում եմդ։}
		\end{xlist}
		\ex 
		\begin{xlist}
			\ex \gll {kɒɻot-el} {e-m}  {kʰez}   
			\\
			miss-{\impfcvb} {\auxgloss}-1{\sg} you.{\sg}.{\dat}
			\\
			\trans			`I’ve missed you.' \hfill (AS)
			\\
			\armenian{Կարօտել եմ քեզ։}
			\ex \gll {kɒɻot-el} {e-m=ət}
			\\
			miss-{\impfcvb} {\auxgloss}-1{\sg}={\possSsg}
			\\
			\trans	`I’ve missed you.' \hfill   (AS)
			\\
			\armenian{Կարոտել եմդ։}
	\end{xlist}	\end{xlist}
	
\end{exe}


%{\added} 
Don Stilo (p.c.) informs us that Persian can also add the object clitic to some periphrastic tenses, such as the present perfect (\ref{sent:Syntax:Clitic:OtherTense:stilo}). 


\begin{exe}
	\ex Persian   (formal register)
	
	 \gll di-d-e æm=æt  \\
	 {\impf}-look-{\ptcp} {\auxgloss}.1{\sg}=2{\sg} \\
	 \trans `I have looked at you.' \hfill (NS, Don Stilo) \label{sent:Syntax:Clitic:OtherTense:stilo}\\
	 \textarab{دیده‌امت} 
\\ 
More common colloquial version with reduction: [di-d-æm-et]
%	 \textarab{دیده‌امت‎}
%	 
\end{exe}


 

\subsection{Cliticizing other verbal arguments }\label{section:syntax:clitic:otherArgu}
All previous examples were cases where the object clitic replaced the direct object of a transitive verb. For other types of verbal arguments, we have found mixed judgments. We go through these other possible arguments.


The clitic has varying grammaticality when used to replace an indirect object (\ref{sent:Syntax:Clitic:OtherArg:indirect}). NK felt that use of the clitic was possible but sounded  ``silly.''  KM cannot say these.


\begin{exe}
	\ex\label{sent:Syntax:Clitic:OtherArg:indirect}
	\begin{xlist}
		\ex \gll {k-ɒs-e-m}  {kʰez}   
		\\
		{\fut}-say--{\thgloss}-1{\sg} you.{\sg}.{\dat}
		\\
		\trans	`I will tell you.' \hfill (NK)
		\\
		\armenian{Կասեմ քեզ։} % 		\armenian{Կ՚ասեմ քեզ։}
		\ex \gll {k-ɒs-e-m=ət}
		\\
		{\fut}-say-{\thgloss}-1{\sg}={\possSsg}
		\\
		\trans	`I will tell you.' \hfill   (NK,  *KM)
		\\
		\armenian{Կասեմդ։} % 		\armenian{Կ՚ասեմդ։}
		
	\end{xlist}
\end{exe}

AS reports an example of an indirect object in the subjunctive (\ref{sent:Syntax:Clitic:OtherArg:indirectAS}).

\begin{exe}
	\ex \gll {me} {bɒn} {ɒs-e-m=ət }
	\\
	{\indf} thing tell-{\thgloss}-1{\sg}={\possSsg}
	\\
	\trans	`Let me tell you something'. \hfill  (AS)\label{sent:Syntax:Clitic:OtherArg:indirectAS}
	\\
	\armenian{Մի բան ասեմդ}
\end{exe}

As before, the indirect object clitic is not used in the past perfective (\ref{sent:Syntax:Clitic:OtherArg:indirectperf}).

\begin{exe}
	\ex \label{sent:Syntax:Clitic:OtherArg:indirectperf}
	\begin{xlist}
		
		\ex \gll {ɒs-ɒ-m}  {kʰez} 
		\\
		say-{\pst}-1{\sg} you.{\sg}.{\dat}
		\\
		\trans	`I told you.' \hfill   (NK)
		\\
		\armenian{Ասամ քեզ։}
		\ex \gll {*ɒs-ɒ-m=ət}  .
		\\
		say-{\pst}-1{\sg}={\possSsg}
		\\
		\trans		Intended:  `I told you.’ \hfill (*NK)
		
\end{xlist}\end{exe}

So far, it seems there is significant speaker variation for using the object clitic in place of an indirect object. Much stronger negative judgments are found for other possible arguments. For example, benefactive phrases cannot be replaced by the object clitic (\ref{sent:Syntax:Clitic:OtherArg:otherargs}).

\begin{exe}
	\ex \label{sent:Syntax:Clitic:OtherArg:otherargs}
	\begin{xlist}
		\ex \gll {jes}  {kə-jeɻkʰ-e-m}  {kʰo}  {hɒmɒɻ}
		\\
		I {\fut}-sing-{\thgloss}-1{\sg} you.{\sg}.{\gen} for
		\\
		\trans	`I {will} sing for you.' \hfill   (NK)
		\\
		\armenian{Ես կը երգեմ քո համար։}
		\ex \gll *{jes}  {kə-jeɻkʰ-e-m=ət}
		\\
		I {\fut}-sing-{\thgloss}-1{\sg}={\possSsg}
		\\
		\trans	Intended:  `I {will} sing for you.’ \hfill   (*NK)
	\end{xlist}
\end{exe}

However, AP reports that they   can add the clitic onto the benefactive postposition (\ref{sent:Syntax:Clitic:OtherArg:otherargsAP}). 

\begin{exe}
	\ex \label{sent:Syntax:Clitic:OtherArg:otherargsAP}
	\begin{xlist}
		\ex \gll {kʰo}  {hɒmɒɻ}  {kə-jeɻkʰ-e-m}  
		\\
		you.{\sg}.{\gen} for  {\fut}-sing-{\thgloss}-1{\sg}
		\\
		\trans	`I {will} sing for you.' \hfill  (AP)
		\\
		\armenian{Քո համար կը երգեմ։}
		\ex \gll {hɒmɒɻ=ət} {kə-jeɻkʰ-e-m}
		\\
		for={\possSsg}  {\fut}-sing-{\thgloss}-1{\sg} 
		\\
		\trans	`I {will} sing for you.' \hfill  (AP)
		\\
		\armenian{Համարդ կը երգեմ։	}
		
		
	\end{xlist}
\end{exe}

Second-person substantives cannot be replaced by the object clitic (\ref{sent:Syntax:Clitic:OtherArg:subs}). 

\begin{exe}
	\ex \label{sent:Syntax:Clitic:OtherArg:subs}
	\begin{xlist}
		\ex \gll {jes}  {kʰo}  {jeɻkʰ-ə} {kə-jeɻkʰ-e-m}
		\\
		I you.{\sg}.{\gen} song-{\defgloss} {\fut}-sing-{\thgloss}-1{\sg}
		\\
		\trans	`I {will} sing your song.' \hfill    (NK)
		\\
		\armenian{Ես քո երգը կը երգեմ։}
		
		\ex \gll {jes}  {kʰon-ə} {kə-jeɻkʰ-e-m}  
		\\
		I  yours-{\defgloss} {\fut}-sing-{\thgloss}-1{\sg}
		\\
		\trans	`I {will} sing yours.' \hfill    (NK)
		\\
		\armenian{Ես քոնը կը երգեմ։}
		\ex \gll *{jes}  {kə-jeɻkʰ-e-m-ət} 
		\\
		I  {\fut}-sing-{\thgloss}-1{\sg}={\possSsg}
		\\
		\trans	Intended:  `I {will} sing yours’. \hfill   (*NK)
		
	\end{xlist}
\end{exe}

Nor can we turn the indirect object of the verb  `to speak’ into an object clitic (\ref{ex:syntax:xosamet}). More accurately, the restriction could be against comitatives.

\begin{exe}
	\ex 
	\begin{xlist}
		\ex \gll {kə-χos-ɒ-m}  
		\\
		{\fut}-speak-{\thgloss}-1{\sg}
		\\
		\trans	`I {will} speak.' \hfill (NK)
		\\
		\armenian{Կը խօսամ։}
		\ex \gll {jes} {es} {lezu-n}  {kə-χos-ɒ-m} 
		\\
		I this language-{\defgloss} {\fut}-speak-{\thgloss}-1{\sg}
		\\
		\trans	`I {will} speak this language.' \hfill (NK)
		\\
		\armenian{Ես էս լեզուն կը խօսամ։}
		\ex \gll {jes} {es} {lezu-n}  {d͡ʒon-i} {het} {kə-χos-ɒ-m} 
		\\
		I this language-{\defgloss} John-{\gen} with {\fut}-speak-{\thgloss}-1{\sg}
		\\
		\trans	`I {will} speak this language with John.' \hfill (NK)
		\\
		\armenian{Ես էս լեզուն Ջոնի հետ կը խօսամ։}
		
		\ex \gll {jes} {es} {lezu-n}  {kʰo}  {het} {kə-χos-ɒ-m}  
		\\
		I this language-{\defgloss} you.{\sg}.{\gen} with {\fut}-speak-{\thgloss}-1{\sg}
		\\
		\trans	`I {will} speak this language with you.' \hfill (NK)
		\\
		\armenian{Ես էս լեզուն քո հետ կը խօսամ։}
		\ex \gll *{jes} {es} {lezu-n}  {kə-χos-ɒ-m=ət} 
		\\
		I this language-{\defgloss} {\fut}-speak-{\thgloss}-1{\sg}={\possSsg}
		\\
		\trans	Intended:  `I {will} speak this language with you’. \label{ex:syntax:xosamet}\hfill   (*NK)
		
	\end{xlist}
\end{exe}
\section{Resumptive pronouns in relative clauses}\label{section:syntax:resumptive}

In {\seaSEA}, relative clauses   utilize   case marking on the relativizer (relative pronoun \textit{voɾ}:  \ref{ex:resump:ea:comp}). The use of a resumptive pronoun is judged as ungrammatical, unnatural, or excessive for speakers (\ref{ex:resump:ea:pron}),   and it is not even mentioned in the Dum-Tragut grammar (\citeyear[478]{DumTragut-2009-ArmenianReferenceGrammar}).  

\begin{exe}
	\ex {{\seaSEA}} \label{sent:Syntax:Resum:basicSEA}
		\begin{xlist}
		\ex  \gll   ɑjn kin-ǝ \textbf{voɾ-it͡sʰ} ɑjs ɡiɾkʰ-ǝ veɾ-t͡sʰɾ-e-t͡sʰ-i-$\emptyset$  \\
		that woman-{\defgloss} \textbf{that-{\abl}}  this book-{\defgloss} buy-{\caus}-{\thgloss}-{\aorperf}-{\pst}-1{\sg}\\
		\trans `that woman from whom I bought this book' \label{ex:resump:ea:comp} \hfill (MA, VK, VP)\\
		\armenian{այն կինը որից այս գիրքը վերցրեցի} 
		\ex   \gll   ɑjn kin-ǝ \textbf{voɾ}  \textbf{iɾen-it͡sʰ} ɑjs ɡiɾkʰ-ǝ veɾ-t͡sʰɾ-e-t͡sʰ-i-$\emptyset$  \\
		that woman-{\defgloss} \textbf{that} \textbf{she-{\abl}}  this book-{\defgloss} buy-{\caus}-{\thgloss}-{\aorperf}-{\pst}-1{\sg}\\
		\trans `that woman from whom I bought this book'   \label{ex:resump:ea:pron} \hfill (MA, ?VK, *VP)\\
		\armenian{այն կինը որ իրենից այս գիրքը վերցրեցի} 
	\end{xlist}
\end{exe}

MA felt the use of a resumptive pronoun was grammatical but ``includes complexity that we can avoid.''

Similarly for {\swaSWA}, HD's judgments are that using a case-marked relativizer is the norm (\ref{ex:resump:wa:comp}). Using a separate resumptive pronoun (\ref{ex:resump:wa:pron}) doesn't sound   ungrammatical, but   does sound ``excessively clunky.'' It creates a sense that the relative clause is an after-thought. 




\begin{exe}
	\ex {{\swaSWA}}
	\begin{xlist}
		\ex  \gll   ɑjn ɡin-ǝ \textbf{voɾ-m-e} ɑjs kʰiɾkʰ-ǝ kʰənn-e-t͡sʰ-i-$\emptyset$  \\
		that woman-{\defgloss} \textbf{that-{\nx}-{\abl}}  this book-{\defgloss} buy-{\thgloss}-{\aorperf}-{\pst}-1{\sg}\\
		\trans `that woman from whom I bought this book'  \label{ex:resump:wa:comp} \hfill (HD)\\
		\armenian{այն կինը որմէ այս գիրքը գնեցի} 
		\ex   \gll   ?ɑjn ɡin-ǝ \textbf{voɾ}  \textbf{iɾ-m-e} ɑjs kʰiɾkʰ-ǝ kʰənn-e-t͡sʰ-i-$\emptyset$  \\
		that woman-{\defgloss} \textbf{that} \textbf{she-{\nx}-{\abl}}  this book-{\defgloss} buy-{\thgloss}-{\aorperf}-{\pst}-1{\sg}\\
		\trans `that woman from whom I bought this book'  \hfill  \label{ex:resump:wa:pron} (?HD)\\
		\armenian{այն կինը որ իրմէ այս գիրքը գնեցի} 
	\end{xlist}
\end{exe}


However, resumptive pronouns are attested in some   {\seaCEA} registers (\citealt[100]{Polinsky-1995-CrossLinguisticParellelsLanguageLoss}, \citealt[ex:5]{Hodgson-2020-FiniteRelativeClauseColloquialArmenianInverseAttraction}). Such resumptive pronouns are also attested and seem to be more common in Classical and Middle Armenian (\citealt{Hewitt-1978-ArmenianRelativeClause}, \citealt[\S3.3]{Hodgson-2020-FiniteRelativeClauseColloquialArmenianInverseAttraction}) and some other Armenian dialects (Aslanbeg: \citealt[53]{Vaux-2001-ArmenianDialectAslanberg}). 


In contrast, in {\iaIA}, both strategies are attested (\ref{ex:resump:ia:comp}), at least for clauses where the head noun acts as an ablative argument in the relative clause. For a bi-dialectal speaker like KM, both options were possible, while   the resumptive pronoun feels more common (\ref{ex:resump:ia:pron}). For a mono-lectal speaker like NK, the resumptive pronoun strategy was the default, while using a case-marked complementizer felt odd.  

\begin{exe}
	\ex {{\iaIA}}
	\begin{xlist}
		\ex  \gll  en kin-ǝ \textbf{voɻ-ut͡sʰ} es ɡiɻkʰ-ǝ veɻ-t͡sʰɻ-ɒ-m \\
		that woman-{\defgloss} \textbf{that-{\abl}}  this book-{\defgloss} take-{\caus}-{\pst}-1{\sg} \\
		\trans `that woman from whom I took this book'  \label{ex:resump:ia:comp} \hfill (KM, ?NK) \\
		\armenian{էն կինը որուց էս գիրքը վերցրամ} 
		\ex  \gll en kin-ǝ \textbf{voɻ} jes \textbf{iɻɒn-it͡sʰ} es ɡiɻkʰ-ǝ veɻ-t͡sʰɻ-ɒ-m \\
		that woman-{\defgloss} \textbf{that} I \textbf{she-{\abl}} this book--{\defgloss} take-{\caus}-{\pst}-1{\sg} \\
		\trans `that woman from whom I took this book' \label{ex:resump:ia:pron} \hfill (KM, NK) \\
		\armenian{էն կինը որ ես իրանից էս գիրքը վերցրամ} \\
		
\end{xlist}\end{exe}

It's unknown if the preference for resumptive pronouns is constant across all possible types of case-marking (nominative, accusative, genitive/dative, ablative, instrumental, and locative). However, as Katherine Hodgson reminds us,   the Relativization Accessibility Hierarchy \citep{Keenan-1977-NounPhraseAccessibilityUniversalGrammar} says that resumptives should be more common with lower roles like ablative than with higher ones like subject. 

The preference for resumptive pronouns is likely due to contact with Persian (\ref{sent:Syntax:Resum:persian}). In Persian, if the head noun has oblique case in the relative clause, then  the only strategy   is to use a resumptive pronominal clitic   (\citealt[34]{Mahootian-2002-PersianGrammar}, \citealt[2]{Abdollahnejad-2018-CompetingGrammarLanguageAcquisitionPersianRelativeCLause}). The relativizer /ke/ cannot be case-marked. 

\begin{exe}
	\ex Persian \\\gll  un zæn-i ke æz-æʃ in ketɒb-ɾo xæɾid-æm  \\
	that woman-{\defgloss}  that from-her this book-{\om} bought-1{\sg} \\
	\trans `the woman from whom I bought this book'  \hfill (NS) \label{sent:Syntax:Resum:persian} \\
	\textarab{اون زنی که ازش این کتاب رو خریدم}
\end{exe}







\section{Subjunctive marking in complement clauses}\label{section:syntax:subj}

In {\seaSEA}, a modal verb like `want' can select complement clauses where the verb is an infinitive (\ref{ex:subjclause:ea:inf}). The implicit subject of the complement clause is the subject of the main clause. An alternative strategy is to include a complementizer \textit{voɾ}, and then change the verb into   a finite subjunctive verb (\ref{ex:subjclause:ea:vor}). Both of these two options are judged as prescriptive norms. A third alternative however is to omit the complementizer but still use a subjunctive verb (\ref{ex:subjclause:ea:subj}). This third alternative is judged as quite colloquial      \citep[425--427]{DumTragut-2009-ArmenianReferenceGrammar}.  

\begin{exe}
	\ex  {{\seaSEA}}
	\begin{xlist}
		\ex  \gll   uz-um e-n ind͡z ɡoɾt͡s-i \textbf{dən-e-l} \\
		want-{\impfcvb} {\auxgloss}-3{\pl}  I.{\dat}   work-{\dat}  put-{\thgloss}-{\infgloss}  \\
		\trans `They want to make me work.'  \label{ex:subjclause:ea:inf} \hfill (MA, VK, VP) \\
		\armenian{Ուզում են ինձ գործի դնել։} 
		\ex  \gll   uz-um e-n \textbf{voɾ} ind͡z ɡoɾt͡s-i \textbf{dən-e-n} \\
		want-{\impfcvb} {\auxgloss}-3{\pl} that I.{\dat}   work-{\dat}  put-{\thgloss}-3{\pl}  \\
		\trans `They want to make me work.'   \label{ex:subjclause:ea:vor}  \hfill (MA, VK, VP) \\
		\armenian{Ուզում են որ ինձ գործի դնեն։} 
		\ex  \gll   uz-um e-n ind͡z ɡoɾt͡s-i \textbf{dən-e-n} \\
		want-{\impfcvb} {\auxgloss}-3{\pl}  I.{\dat}   work-{\dat}  put-{\thgloss}-3{\pl}  \\
		\trans `They want to make me work.'  \label{ex:subjclause:ea:subj}  \hfill (MA, VK, VP) \\
		\armenian{Ուզում են ինձ գործի դնեն։} 
	\end{xlist}
\end{exe}


Similar   judgments   apply for {\swaSWA}. The norm is to use an infinitive (\ref{ex:subjclause:wa:inf}) or a complementizer (\ref{ex:subjclause:wa:vor}). Using a subjunctive (\ref{ex:subjclause:wa:subj}) is possible in colloquial speech. When the complement clause includes multiple items besides the verb,   HD feels that using a subjunctive sounds more natural than using an infinitive. 

\begin{exe}
	\ex  {{\swaSWA}}
	\begin{xlist}
		\ex  \gll   ɡ-uz-e-n ind͡z-i  \textbf{ɑʃχɑt-t͡sən-e-l} \\
		{\ind}-want-{\thgloss}-3{\pl}  I-{\dat}  work-{\caus}-{\thgloss}-{\infgloss}  \\
		\trans `They want to make me work.'  \label{ex:subjclause:wa:inf} \hfill (HD) \\
		\armenian{Կուզեն ինծի աշխատցնել։} %		\armenian{Կ՚ուզեն ինծի աշխատցնել։} 
		\ex  \gll   ɡ-uz-e-n  \textbf{voɾ} ind͡z-i  \textbf{ɑʃχɑt-t͡sən-e-n} \\
		{\ind}-want-{\thgloss}-3{\pl} that I-{\dat}   work-{\caus}-{\thgloss}-3{\pl}  \\
		\trans `They want to make me work.'   \label{ex:subjclause:wa:vor}  \hfill (HD) \\
		\armenian{Կուզեն որ ինծի աշխատցնեն։}  % 		\armenian{Կ՚ուզեն որ ինծի աշխատցնեն։} 
		\ex  \gll   ɡ-uz-e-n ind͡z-i  \textbf{ɑʃχɑt-t͡sən-e-n} \\
		{\ind}-want-{\thgloss}-3{\pl}  I-{\dat}  work-{\caus}-{\thgloss}-3{\pl}  \\
		\trans `They want to make me work.'  \label{ex:subjclause:wa:subj}  \hfill (HD) \\
		\armenian{Կուզեն ինծի աշխատցնեն։}  % 		\armenian{Կ՚ուզեն ինծի աշխատցնեն։} 
	\end{xlist}
\end{exe}


In contrast, in {\iaIA}, the use of a finite subjunctive verb is more common (\ref{ex:subjclause:ia:subj}). NK personally felt that using an infinitive was odd or ungrammatical (\ref{ex:subjclause:ia:inf}). 



\begin{exe}
	\ex  {{\iaIA}}
	\begin{xlist}
		\ex  \gll   ?uz-um e-n ind͡z-i ɡoɻt͡s-i \textbf{kʰɒʃ-e-l} \\
		want-{\impfcvb} {\auxgloss}-3{\pl}  I-{\dat}   work-{\dat}  drive-{\thgloss}-{\infgloss}  \\
		\trans `They want to make me work.'  \label{ex:subjclause:ia:inf} \hfill (?NK) \\
		\armenian{Ուզում են ինձի գործի քաշել։} 
		\ex  \gll   uz-um e-n \textbf{voɾ} ind͡z-i ɡoɻt͡s-i \textbf{kʰɒʃ-e-n} \\
		want-{\impfcvb} {\auxgloss}-3{\pl} that I-{\dat}   work-{\dat}  drive-{\thgloss}-3{\pl}  \\
		\trans `They want to make me work.'   \label{ex:subjclause:ia:vor}  \hfill (NK) \\
		\armenian{Ուզում են որ ինձի գործի քաշեն։} 
		\ex  \gll   uz-um e-n ind͡z-i ɡoɻt͡s-i \textbf{kʰɒʃ-e-n} \\
		want-{\impfcvb} {\auxgloss}-3{\pl}  I-{\dat}   work-{\dat}  drive-{\thgloss}-3{\pl}  \\
		\trans `They want to make me work.'  \label{ex:subjclause:ia:subj}  \hfill (KM, NK) \\
		\armenian{Ուզում են ինձի գործի քաշեն։} 
	\end{xlist}
\end{exe}



AS reports more examples of embedded verbs where {\seaSEA} would prefer an infinitive form, while {\iaIA} prefers a subjunctive form (\ref{sent:Syntax:Subj:AS}). 

\begin{exe}
	\ex   \label{sent:Syntax:Subj:AS}
	\begin{xlist}
		\ex {\iaIA}
		\begin{xlist}
			\ex \gll  t͡ʃʰ-e-m kɒɻ-oʁ ɒs-e-m   \\
			{\neggloss}-{\auxgloss}-1{\sg} can-{\sptcp} say-{\thgloss}-1{\sg} \\
			\trans `I cannot say.' \hfill (AS) \\
			\armenian{Չեմ կարող ասեմ։} 
			\ex \gll   uz-um $\emptyset$-i-m ɒn-$\emptyset$-i-m    \\
			want-{\impfcvb}  {\auxgloss}-{\pst}-1{\sg} do-{\thgloss}-{\pst}-1{\sg} \\
			\trans `I wanted to do (it).' \hfill (AS) \\
			\armenian{Ուզում իմ անիմ։}  
			\ex \gll int͡ʃʰ   $\emptyset$-i-ɻ uz-um ɒs-$\emptyset$-i-ɻ   \\
			what  {\auxgloss}-{\pst}-2{\sg} want-{\impfcvb}  say-{\thgloss}-{\pst}-2{\sg} \\
			\trans `What did you want to say?'  \hfill (AS) \\
			\armenian{Ի՞նչ իր ուզում ասիր։} 
		\end{xlist}
		\ex {\seaSEA}
		\begin{xlist}
			\ex \gll  t͡ʃʰ-e-m kɑɾ-oʁ ɑs-e-l  \\
			{\neggloss}-{\auxgloss}-1{\sg} can-{\sptcp} say-{\thgloss}-{\infgloss} \\
			\trans `I cannot say.' \hfill (AS) \\
			\armenian{Չեմ կարող ասել։} 
			\ex \gll uz-um ej-i-$\emptyset$ ɑn-e-l    \\
			want-{\impfcvb}  {\auxgloss}-{\pst}-1{\sg} do-{\thgloss}-{\infgloss} \\
			\trans `I wanted to do (it).'  \\
			\armenian{Ուզում էի անել։}  
			\ex \gll int͡ʃʰ ej-i-ɾ  uz-um ɑs-e-l    \\
			what  {\auxgloss}-{\pst}-2{\sg} want-{\impfcvb}  say-{\thgloss}-{\infgloss} \\
			\trans `What did you want to say?'   \\
			\armenian{Ի՞նչ էիր ուզում ասել։} 
		\end{xlist}
		
		
		
	\end{xlist}
	
\end{exe}

In Iran, the Salmast dialect of Armenian  likewise prefers using subjunctive forms \citep[\S 4.5]{Vaux-Salmast}.  

The preference for subjunctive forms is likely due to language-internal development that got encouraged by language contact with Persian (\ref{sent:Syntax:Subj:persian}). In Persian, verbs like `want' select subjunctive verbs    \citep[29]{Mahootian-2002-PersianGrammar}.\pagebreak

\begin{exe}
	\ex Persian\label{sent:Syntax:Subj:persian}
	\begin{xlist}
		\ex \gll mi-tun-æm be-ɾ-æm  \\
		{\prog}-can-1{\sg}  {\subj}-go-1{\sg} \\ 
		\trans `I can go.' \hfill (NS) \\
		\textarab{میتونم برم.}
		\ex \gll ne-mi-tun-æm be-ɡ-æm \\
		{\neggloss}-{\prog}-can-1{\sg}  {\subj}-say-1{\sg} \\ 
		\trans `I cannot say.' \hfill (NS) \\
		\textarab{نمیتونم بگم.}
		\ex \gll mi-x-ænd mæn be-ɾ-æm\\
		{\prog}-want-3{\pl}  I {\subj}-go-1{\sg} \\ 
		\trans `They want me to go.' \hfill (NS) \\
		\textarab{میخواند من برم.}
		
	\end{xlist}
\end{exe}





\section{Agreement-marking in nominalized relative clauses or participial clauses}\label{section:syntax:participleClause}

A small area of microvariation concerns agreement marking on nominalized relative clauses. For a sentence like (\ref{ex:partagr:ea:3full}), the relative clause is   expressed   as a post-nominal relative  clause with a finite verb. In contrast, this sentence can be paraphrased as in (\ref{ex:partagr:ea:3part}), but where the relative clause is now pre-nominal, and the finite verb is replaced by a participle. 

\begin{exe}
	\ex {\seaSEA} 
	\begin{xlist}
		\ex \gll     ɡiɾkʰ-\textbf{ǝ}  voɾ  iŋkʰ-ə kɑɾtʰ-ɑ-t͡sʰ-$\emptyset$-$\emptyset$\\
		book-{\defgloss}  that he-{\defgloss} read-{\thgloss}-{\aorperf}-{\pst}-3{\sg} \\
		\trans `the book that he read.'  \label{ex:partagr:ea:3full} \hfill (VP)  \\
		\armenian{գիրքը որ ինքը կարդաց}  
		\ex \gll   (iɾ)  kɑɾtʰ-ɑ-t͡sʰ-ɑt͡s  ɡiɾkʰ-\textbf{ǝ}\\
		(he.{\gen}) read-{\thgloss}-{\aorother}-{\rptcp} book-{\defgloss} \\
		\trans `the book that he read.'   \label{ex:partagr:ea:3part} \hfill (VP) \\
		\armenian{(իր) կարդացած գիրքը}  
		
		
	\end{xlist}
\end{exe}

A special subcategory of such relative clause constructions is when the subject or `doer' of the verb is either the first or second person singular (\ref{ex:partagr:ea:1full}). We focus on the first person for illustration. When  such a relative clause is converted to a participle clause (\ref{ex:partagr:ea:1partSpart}), the subject is expressed by the first person possessive suffix \textit{-(ə)s}.

\begin{exe}
	\ex {\seaSEA}
	\begin{xlist}
		\ex \gll     ɡiɾkʰ-\textbf{ǝ}  voɾ kɑɾtʰ-ɑ-t͡sʰ-i-$\emptyset$\\
		book-{\defgloss}  that  read-{\thgloss}-{\aorperf}-{\pst}-1{\sg} \\
		\trans `the book that I read.' \label{ex:partagr:ea:1full} \hfill (MA, VK, VP)  \\
		\armenian{գիրքը որ կարդացի}  
		\ex \gll     kɑɾtʰ-ɑ-t͡sʰ-ɑt͡s-\textbf{ǝs} ɡiɾkʰ-\textbf{ǝ}\\
		read-{\thgloss}-{\aorother}-{\rptcp}-{\possFsg} book-{\defgloss} \\
		\trans `the book that I read.' \label{ex:partagr:ea:1partSpart}  \hfill (MA, VK, VP) \\
		\armenian{կարդացածս գիրքը}  
		\ex \gll   \textbf{im}  kɑɾtʰ-ɑ-t͡sʰ-ɑt͡s-\textbf{ǝs} ɡiɾkʰ-\textbf{ǝ}\\
		I.{\gen} read-{\thgloss}-{\aorother}-{\rptcp}-{\possFsg} book-{\defgloss} \\
		\trans `the book that I read.' \label{ex:partagr:ea:1partSpart:overty}  \hfill (*MA, ?VK, *VP) \\
		\armenian{իմ կարդացածս գիրքը}  
		
		
	\end{xlist}
\end{exe}

Our {\seaSEA} consultants       all felt that using an overt genitive pronoun alongside the possessive suffix on the participle   (\ref{ex:partagr:ea:1partSpart:overty}) was odd or ungrammatical.   



For these participle clauses, there is dialectal variation in how the subject or doer of the action is marked for the first/second person singular. In {\seaSEA}, the norm is (i) to not use an overt genitive pronoun, (ii) to place a subject-marking  possessive suffix \textit{-əs} on the participle, and (iii) to mark the head noun as definite  \citep[508--509]{DumTragut-2009-ArmenianReferenceGrammar}. 

In contrast  in {\swaSWA} (\ref{ex:partagr:wa:norm}), the norm is to (ii') make the participle unmarked, while (iii') the noun gets the possessive suffix. The pronoun is optional (i'). For more data, see \citet{AckermanNikolaeva-1997-IdentityFormDifferenceFunctionPersonNumberArmenianOstyak,Ackerman-1998-ConstructionsMixedCategorySemanticPersonNumber,AckemaNeeleman-2004-BeyondMorpho}, and \citet[284ff]{AckermanNikolaeve-2014-DescriptiveTypologyLingTheoryRelativeClause}.  For {\seaSE}, such  constructions are deemed ``okay but not default'' for VK and ``not preferable'' for   VP (\ref{ex:partagr:ea:snoun}). Neither consultant approved  of adding the pronoun.





\begin{exe}
	\ex 
	\begin{xlist}
		\ex {\swaSWA} \\ \gll  (\textbf{im}) ɡɑɾtʰ-ɑ-t͡sʰ-ɑd͡z kʰiɾkʰ-\textbf{əs} \\
		I.{\gen} read-{\thgloss}-{\aorother}-{\rptcp}   book-{\possFsg} \\
		\trans `the book that I read' \label{ex:partagr:wa:norm} \hfill (HD) \\
		\armenian{(իմ) կարդացած գիրքս}  
		\ex {\seaSEA} \\ \gll (*\textbf{im})   kɑɾtʰ-ɑ-t͡sʰ-ɑt͡s ɡiɾkʰ-\textbf{əs} \\
		I.{\gen} read-{\thgloss}-{\aorother}-{\rptcp}   book-{\possFsg} \\
		\trans `the book that I read' \label{ex:partagr:ea:snoun} \hfill (VK, ?VP) \\
		\armenian{(իմ) կարդացած գիրքս}  
		
\end{xlist}\end{exe}


Note that some speakers like MA feel that having the possessive on the noun (\ref{ex:partagr:ea:snoun}) was grammatical but had a distinct meaning of `I own the book and I read it.'  In contrast, when the possessive suffix is on the participle (\ref{ex:partagr:ea:1partSpart}), there is no information concerning who the owner of the book is. 



In contrast, in {\iaIA}, it seems that there is optionality across these parameters. We can either follow {\seaAbbre} and   place the possessive on the participle (\ref{ex:partagr:ia:sea}), or we can follow {\swaAbbre} and place the possessive on the noun (\ref{ex:partagr:ia:wa}).  An intermediate option is to not use a possessive suffix at all (\ref{ex:partagr:ia:inter}). 

\begin{exe}
	\ex {{\iaIA}}
	\begin{xlist}
		\ex \gll     ɡiɻkʰ-\textbf{ǝ}  voɾ kɒɻtʰ-ɒ-t͡sʰ-i-m\\
		book-{\defgloss}  that  read-{\thgloss}-{\aorperf}-{\pst}-1{\sg} \\
		\trans `the book that I read.' \label{ex:partagr:ia:1full} \hfill (NK)  \\
		\armenian{գիրքը որ կարդացիմ}  
		
		\ex \gll   (\textbf{im}) kɒɻtʰ-ɒ-t͡sʰ-ɒt͡s-\textbf{ǝs}    ɡiɻkʰ-\textbf{ǝ}   \\
		(I.{\gen}) read-{\thgloss}-{\aorother}-{\rptcp}-{\possFsg} book-{\defgloss} \\
		\trans \armenian{իմ կարդացածս գիրքը} \label{ex:partagr:ia:sea}  \hfill (KM, ?NK)
		\ex \gll  \textbf{im} kɒɻtʰ-ɒ-t͡sʰ-ɒt͡s  ~ ~  ~  ~   ~ ~    ~ ~   ɡiɻkʰ-\textbf{ǝ} \\
		I.{\gen} read-{\thgloss}-{\aorother}-{\rptcp}  ~ ~ ~   ~  ~ ~  ~   ~ book-{\defgloss} \\
		\trans \armenian{իմ կարդացած գիրքը} \label{ex:partagr:ia:inter} \hfill (KM, NK)
		\ex \gll (\textbf{im})  kɒɻtʰ-ɒ-t͡sʰ-ɒt͡s ~ ~  ~    ~  ~ ~   ~    ɡiɻkʰ-\textbf{ǝs} \\
		(I.{\gen})  read-{\thgloss}-{\aorother}-{\rptcp}  ~ ~    ~ ~  ~ ~  ~  book-{\possFsg} \\
		\trans  `the book that I read'
		\\
		\armenian{(իմ) կարդացած գիրքս} \label{ex:partagr:ia:wa} \hfill (KM, ?NK) \\
		
	\end{xlist}
\end{exe}

For a bi-dialectal  speaker like KM, all of the options were acceptable. For a mono-lectal speaker like NK, the intermediate option (\ref{ex:partagr:ia:inter}) was judged as the best option, the {\seaAbbre}-style sentences were judged as odd  (\ref{ex:partagr:ia:sea}), while the {\swaAbbre}-sentences (\ref{ex:partagr:ia:wa}) were judged as better than the {\seaAbbre}-style  ones, but not as good as the intermediate. 

This intermediate option (\ref{ex:partagr:ia:inter}) was likewise accepted for {\seaAbbre} (\ref{sent:Syntax:Part:EAinter}) by our consultants; VK and MA went as far to say this intermediate option is as good as the norm (\ref{ex:partagr:ea:1partSpart}). Katherine Hodgson informs us that all this variation is likewise attested in \seaCEA. 

\begin{exe}
	\ex  {{{\seaSEA}}}\\
	\gll  \textbf{im} kɑɾtʰ-ɑ-t͡sʰ-ɑt͡s ɡiɾkʰ-\textbf{ǝ}   \\
	I.{\gen} read-{\thgloss}-{\aorother}-{\rptcp}    book-{\defgloss} ~  \\
	\trans `the book that I read.'\hfill (MA, VK, VP) \label{sent:Syntax:Part:EAinter}
	\\ \armenian{իմ կարդացած գիրքը} 
	
\end{exe}




Among these various options for {\iaIA}, KM reports that the intermediate option is relatively more preferred (\ref{ex:partagr:ia:inter}). The {\swaAbbre}-style option is attested but rather stigmatized (\ref{ex:partagr:ia:wa}).  The {\seaAbbre}-style option is prescriptively the rule  but rather uncommon (\ref{ex:partagr:ia:sea}). It seems that at some point, {\seaSEA} developed this intermediate option as an acceptable colloquial alternative. {\iaIA} then grammaticalized this intermediate option as the norm.  

