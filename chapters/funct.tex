\chapter{Function words}\label{chapter:funct}

We go over basic function words in this chapter, including personal pronouns (\S\ref{section:funct:personal pronoun}), demonstratives (\S\ref{section:funct:demonstrative}), interrogative pronouns or wh-words (\S\ref{section:funct:wh}),  numerals (\S\ref{section:funct:num}), and other function words (\S\ref{section:funct:other}). We have not found many significant differences between {\iaIA} and {\seaSEA} when it comes to pronouns. 

\section{Personal pronouns}\label{section:funct:personal pronoun}

{\iaIA} uses the    personal pronouns in Table \ref{tab:personal pronoun}. Whereas common nouns are syncretic for dative and genitive, pronouns distinguish the two cases. The {\iaIA} pronouns do not significantly differ from {\seaSE} \citep[123]{DumTragut-2009-ArmenianReferenceGrammar} except that the intensive 3SG dative is \textit{iɾen} in {\seaSE} but \textit{iɻɒn} in {\iaIA}.  The form [iɾɑn] is attested in {\seaCEA} \citep[128]{DumTragut-2009-ArmenianReferenceGrammar}. 

\begin{table}
	\caption{Paradigm of personal pronouns in {\iaIA}\label{tab:personal pronoun}}
	\resizebox{\textwidth}{!}{%
		\begin{tabular}{l l   l   l   l   l   l}
			\lsptoprule
			&Nominative & Acc/Dative & Genitive & Ablative & Instrumental & Locative\\
			& {\pro} & {\pro}-({\dat})& {\pro}& {\pro}-({\nx})-{\abl}& {\pro}-({\nx})-{\ins}& {\pro}-({\nx})-{\locgloss}\\\midrule
			1SG & {jes} & {ind͡z, ind͡z-i}  & {im} & {ind͡z-ɒn-it͡sʰ}& {{ind͡z-ɒn-ov}}& {{ind͡z-ɒn-um}}\\
			& \armenian{ես} & \armenian{ինձ, ինձի}   &\armenian{իմ} & \armenian{ինձանից} & \armenian{ինձանով}& \armenian{ինձանում}\\\addlinespace
			2SG & {du}  &  {kʰez, kʰez-i} &{kʰo}&   {kʰez-ɒn-it͡sʰ} & {{kʰez-ɒn-ov}}& {{kʰez-ɒn-um}}\\
			& \armenian{դու} & \armenian{քեզ, քեզի}  &\armenian{քո} & \armenian{քեզանից} & \armenian{քեզանով}& \armenian{քեզանում}\\\addlinespace
			3SG &{iŋkʰ-ə}   & {iɻɒn} & {iɻɒ}&   {iɻɒn-it͡sʰ}		& {{iɻɒn-ov}}& {{iɻɒn-um}}\\
			& \armenian{ինքը} & \armenian{իրան}   &\armenian{իրա} & \armenian{իրանից}& \armenian{իրանով}& \armenian{իրանում}\\
			& {nɒ} &   nəɻɒn& {nəɻɒ} & {nəɻɒn-it͡sʰ}	& {{nəɻɒn-ov}}& {{nəɻɒn-um}}\\
			& \armenian{նա} & \armenian{նրան}   &\armenian{նրա} & \armenian{նրանից} & \armenian{նրանով}& \armenian{նրանում}\\\addlinespace
			1PL & {meŋkʰ}   & {mez, mez-i}& {meɻ}&  {mez-ɒn-it͡sʰ} & {{mez-ɒn-ov}}& {{mez-ɒn-um}}\\
			& \armenian{մենք} & \armenian{մեզ, մեզի}  &\armenian{մեր} & \armenian{մեզանից} &\armenian{մեզանով}& \armenian{մեզանում}\\\addlinespace 
			2PL & {dukʰ} &     {d͡zez, d͡zez-i} & {d͡zeɻ}& {d͡zez-ɒn-it͡sʰ} & {{d͡zez-ɒn-ov}}& {{d͡zez-ɒn-um}}\\
			& \armenian{դուք}& \armenian{ձեզ, ձեզի}   &\armenian{ձեր} & \armenian{ձեզանից} & \armenian{ձեզանով}& \armenian{ձեզանում}\\\addlinespace
			3PL & {iɻɒŋkʰ}    & {iɻɒnt͡sʰ} &{iɻɒnt͡sʰ} &  {iɻɒnt͡sʰ-it͡sʰ}	& {{iɻɒnt͡sʰ-ov}}& {{iɻɒnt͡sʰ-um}}\\
			& \armenian{իրանք} & \armenian{իրանց}    &\armenian{իրանց} & \armenian{իրանցից} & \armenian{իրանցով}& \armenian{իրանցում}\\
			& {nəɻɒŋkʰ}  & {nəɻɒnt͡sʰ} &   {nəɻɒnt͡sʰ}  & {{nəɻɒnt͡sʰ-it͡sʰ}}	& {{nəɻɒnt͡sʰ-ov}}& {{nəɻɒnt͡sʰ-um}}\\ 
			& \armenian{նրանք} & \armenian{նրանց}    &\armenian{նրանց} & \armenian{նրանցից} & \armenian{նրանցով}& \armenian{նրանցում}\\
			\lspbottomrule 
		\end{tabular}%
	}
\end{table}

For the 3SG and 3PL, there are two series of pronouns. One series is intensive \citep[126]{DumTragut-2009-ArmenianReferenceGrammar} or emphatic \citep{Donabedian-2018-WestArmTypoSocio} and starts with the segment \textit{{i}}, while the other series is a generic third person pronoun and starts with \textit{{n}}. For the syntactic distribution of Armenian pronouns, see \citet{sigler-2001-logophoricPronounWesternArmenian,donabedian-2007-rechercheLogophorWesternArmenian}. For both NK and KM, the intensive pronoun is considered more ``conversational'', while the non-intensive pronoun feels more formal.
For the 3PL non-intensive, the initial /{nəɻɒ-}/ sequence was often lenited in NK's speech, e.g., {\acc}/{\dat}/{\gen} plural [{nəɻɒnt͡sʰ}] or lenited [{nɒnt͡sʰ}] `they'.\footnote{Compare New Julfa  {\acc}/{\dat}/{\gen}   plural [nu̯ont͡sʰ] \armenian{նոնց}, which in its Indian subdialect is [nɑnt͡sʰɑn] \armenian{նանցան} `those over there.{\acc}/{\dat}/{\gen}',  ablative [nɑnt͡sʰɑne] \armenian{նանցանէ} ‘from those' \citep[\S 266]{Adjarian-1940-NewJulfaDialect}. }

For the accusative/dative series, outside of the third person, the pronoun has two forms: one bare and one suffixed with \textit{{-i}}. For example, accusative/dative 1SG is \textit{{ind͡z}} or \textit{{ind͡z-i}}. The bare form is the more common form, but there is significant speaker variation on the preferred form. For example, NK almost always used the bare form in our elicitations, while AS reports that his consultants often used the suffixed form.  

In pronouns, the accusative is syncretic with the dative (and with the genitive in the 3PL). This syncretism is shown in the following sentences (\ref{sent:Pronoun:Personal:syncretic}).

\begin{exe}
	\ex \label{sent:Pronoun:Personal:syncretic}
	\begin{xlist}
		
		\ex \gll{d͡ʒɒn-ə} {ind͡z} {mɒkʰɻ-ɒ-v}
		\\
		John-{\defgloss} me.{\dat} clean-{\pst}-3{\sg}
		\\
		\trans		`John cleaned (or washed) me.'\hfill (NK)
		\\
		\armenian{Ջոնը ինձ մաքրաւ։}
		\ex \gll {d͡ʒɒn-ə} {ind͡z} ɡiɻkʰ  {təv-ɒ-v}
		\\
		John-{\defgloss} me.{\dat} book give-{\pst}-3{\sg}
		\\
		\trans	`John gave a book to me.'\hfill (NK)
		\\
		\armenian{Ջոնը ինձ գիրք տուաւ։}
		
	\end{xlist}
	
\end{exe}

Morphotactically, the ablative, instrumental, and locative are built on top of the dative form. For the non-third person series, the dative form and the added case suffix are separated by the meaningless morph \textit{{-ɒn-}}.  This morph sequence can be weakened to either \textit{{-ən-}} or \textit{{-n-}}: [ind͡z-ɒn-it͡sʰ, ind͡z-n-it͡sʰ] `I-{\nx}-{\abl}'. 

We have received conflicting judgments on the frequency of such weakening. NK always lenited the 1SG obliques to   \textit{{-ən-}}, e.g. 1SG ablative \textit{{ind͡z-ən-it͡sʰ}}. Yet she always lenited the other non-third person series  to just \textit{{-n-}}, e.g., dative 2SG \textit{{kʰez-n-it͡sʰ}}.  In contrast, AS reports that for speakers in Iran, the deletion of /{ɒ}/ is not frequent.

For the instrumental and locative series, they are quite difficult to elicit   in natural speech. Alternative syntactic strategies are preferred. For example, for instrumentals, the comitative meaning of the instrumental (`to go alongside X') is expressed by using a postpositional construction with the genitive pronoun (Table \ref{sent:Pronoun:Personal:com}).  Similarly, the locative meaning is   expressed by using a postposition [met͡ʃʰ] \armenian{մէջ} `in'.

\begin{table}
	\caption{Expressing comitative-instrumental with postpositions}\label{sent:Pronoun:Personal:com}
	\begin{tabular}{lllll}
		\lsptoprule
		1SG & {im} & {het} & `with me' & \armenian{իմ հետ} \\
		2SG & {kʰo} & {het} & `with you.{\sg}' & \armenian{քո հետ} \\
		3SG & {iɻɒ} & {het} & `with him' & \armenian{իրա հետ} \\
		& {nəɻɒ} & {het} & `with him' & \armenian{նրա հետ} \\
		1PL & {meɻ} & {het} & `with us' & \armenian{մեր հետ} \\
		2PL & {d͡zeɻ} & {het} & `with you.{\pl}' & \armenian{ձեր հետ} \\
		3PL & {iɻɒnt͡sʰ} & {het} & `with them' & \armenian{իրանց հետ} \\
		& {nəɻɒnt͡sʰ} & {het} & `with them' & \armenian{նրանց հետ} \\
		& {\pro}.{\gen} & with & & \\ 
		\lspbottomrule
		\end{tabular}
\end{table}



\section{Demonstratives}\label{section:funct:demonstrative}


{\iaIA} uses a small set of demonstrative pronouns. These show a three-way contrast for deixis: proximal, medial, and distal. There are different forms for when the pronoun is a modifier in a noun phrase vs. when the pronoun stands on its own as a substantive. 

For illustration, we focus on the proximal series in (\ref{sent:Pronoun:Dem:basic}). This series is characterized by starting with the segmental sequence /{es-}/ or /s/. When the proximal pronoun is a modifier in a noun phrase, it is realized as [{es}]. It can modify either a singular noun or plural noun.\largerpage[0.5]

\begin{exe}
	\ex \label{sent:Pronoun:Dem:basic}
	\begin{multicols}{2}
	\begin{xlist}
		
		\ex \gll {es} {ɡiɻkʰ-ə}
		\\
		this book-{\defgloss}
		\\
		\trans	`this book'
		\\
		\armenian{էս գիրքը}
		\ex \gll {es} {ɡiɻkʰ-eɻ-ə} 
		\\
		this book-{\pl}-{\defgloss}
		\\
		\trans		`these books'
		\\
		\armenian{էս գիրքերը}
		
	\end{xlist}
	\end{multicols}
\end{exe}



Table \ref{tab:demons modifier} shows the set of demonstrative pronouns when the pronoun is a modifier. 

\begin{table}
	\caption{Demonstrative pronouns when acting as a modifier\label{tab:demons modifier}}
 	\begin{tabular}{ll lll ll }
 	\lsptoprule
 	& \multicolumn{2}{c}{Proximal}&  \multicolumn{2}{c}{Medial} &  \multicolumn{2}{c}{Distal}  \\\midrule
 	&   {es}  & \armenian{էս} & {et} &\armenian{էտ}& {en}  &\armenian{էն}\\
 	& \multicolumn{2}{ l }{`this'}&  \multicolumn{2}{l }{`that (close)'} &  \multicolumn{2}{l}{`that (yonder)'} 		\\ \addlinespace
 	Usage & \multicolumn{2}{l}{The item is by}& \multicolumn{2}{l}{The item is by}& \multicolumn{2}{l}{The item is not by}\\
 	& \multicolumn{2}{l}{the speaker}& \multicolumn{2}{l}{the listener}& \multicolumn{2}{l}{ the speaker  or listener}\\ 
 	\lspbottomrule
 \end{tabular}
\end{table}


%{\added}

In {\seaSEA}, these demonstratives have cognate forms that are phonologically larger. For example, the proximal-medial-distal series in {\seaSEA} is \{/ɑjs/, /ɑjd/ or /ɑjt/, /ɑjn/\} (\armenian{այս, այդ, այն}). The {\iaIA} forms /es, et, en/    are likely diachronically reduced versions of these larger {\seaSEA} forms. A reviewer informs us that these reduced forms are also attested in {\seaCEA} in Armenia.  BV reports that this is just the regular change of Classical /ɑi̯/ <ay, \armenian{այ} > to /e/  in Eastern dialects. 


When the   pronoun is substantivized and stands for an entire noun phrase, it   can be realized in one of three forms (\ref{sent:Pronoun:Dem:subs}). For the proximal pronoun, the singular forms are   \textit{{es}}, \textit{{esi}}, and \textit{{esikə}}. The plural form of the substantivized pronoun is \textit{{səɻɒŋkʰ}}.

\begin{exe}
	\ex \label{sent:Pronoun:Dem:subs}
	\begin{xlist}
		
		\ex \gll {es/esi/esikə} {ɡiɻkʰ} ={ɒ}  
		\\
		this book-{\defgloss}  ={\auxgloss}
		\\
		\trans		`This is a book.'\hfill (NK)
		\\
		\armenian{Էս/էսի/Էսիկը գիրք ա։}
		\ex \gll {səɻɒŋkʰ} {ɡiɻkʰ-eɻ} {=e-n}  
		\\
		these book-{\pl} ={\auxgloss}-3{\pl}
		\\
		\trans		`These are books.'\hfill (NK)
		\\
		\armenian{Սրանք   գիրքեր են։}
		
	\end{xlist}
\end{exe}



The final schwa of the long pronoun \textit{{esikə}} is likely part of the definite suffix (\ref{sent:Pronoun:Dem:esikn}). Evidence for this is that the schwa becomes a schwa-nasal sequence when cliticized. See similar patterns for the definite suffix in \S\ref{section:morphophono:allomorphy: det}. 

\begin{exe}
	\ex \gll {esik-ən} {e-m} {uz-um}
	\\
	this-{\defgloss} {\auxgloss}-1{\sg} want-{\impfcvb}
	\\
	\trans		`I want this one.'\hfill (NK) \label{sent:Pronoun:Dem:esikn}
	\\
	\armenian{Էսիկն եմ ուզում։}
\end{exe} 

Etymologically, it is possible that forms like /esik-ə/ `this' derive from adding the definite suffix onto a hypothetical earlier form like *\textit{esik}  \citep[cf.][195ff]{Adjarian-1954-Liakatar2}. Alternatively, BV suggests that the modern complex form /esik-ə/ may have a more complicated origin. First, the form was  *\textit{esikɒ}. Second, the form  underwent final vowel reduction to *\textit{esikə}.  Third, the form underwent  morphological reanalysis as /esik-ə/ with a definite suffix. But Hrach Martirosyan (p.c.) suggests the first is more probable.  

When these demonstratives are substantivized, they inflect for case (\ref{sent:Pronoun:Dem:case}).

\begin{exe}
	\ex \label{sent:Pronoun:Dem:case}
	\begin{xlist}
		
		\ex \gll    {səɻɒn} {t͡ʃɒʃ} {təv-ɒ-m}  
		\\
		this.{\dat} food give-{\pst}-1{\sg}
		\\
		\trans		`I gave food to this one.'\hfill (NK)
		\\
		\armenian{Սրան ճաշ տուամ։}
		\ex \gll    {səɻɒ} {ɡujn-ə}
		\\
		this.{\gen} color-{\defgloss}
		\\
		\trans	`the color of this one'\hfill (NK)
		\\
		\armenian{սրա գոյնը}
	\end{xlist}
\end{exe}



\begin{sloppypar}
Table \ref{tab:demon subst} shows the paradigm of substantivized demonstratives.   Note that the inflected forms of the substantivized distal are identical to the non-intensive third-person personal pronouns from Table \ref{tab:personal pronoun}. The {\iaIA} paradigm does not significantly differ from that of {\seaSEA} \citep[129]{DumTragut-2009-ArmenianReferenceGrammar}. For the medial series, the plurals and the case-marked forms use [d] in {\seaSEA}: [dəɾɑŋkʰ, dəɾɑ]. Some {\iaIA} speakers like KM use [d] too,  while some {\iaIA} speakers like NK use [t]. 
\end{sloppypar}

\begin{table}
	\caption{Paradigm for substantivized demonstratives\label{tab:demon subst}}
	\resizebox{\textwidth}{!}{%
		\begin{tabular}{l l l l l l l }
			\lsptoprule 
			&Nom/Acc  & Dative & Genitive & Ablative & Instrumental & Locative\\
			& {\pro} & {\pro}& {\pro}& {\pro}-{\abl}& {\pro}-{\ins}& {\pro}-{\locgloss}\\\midrule
			\multicolumn{2}{l}{Singular}\\
			Prox.  & {es, esi, esikə}   & {səɻɒn} & {səɻɒ}& {səɻɒn-it͡sʰ}&  {səɻɒn-ov} & {səɻɒn-um} \\
			& \armenian{էս, էսի, էսիկը} & \armenian{սրան} &\armenian{սրա} &\armenian{սրանից} &\armenian{սրանով} &\armenian{սրանում}\\
			% 		& esi & & & & &  \\
			% 		& \armenian{էսի} & & & & &  \\
			
			% 				& {esikə} & & & & &  \\ 
			% 		& \armenian{էսիկը} & & & & &  \\
			\addlinespace 			Med.  & {et, eti, etikə}  & {dəɻɒn} & {dəɻɒ}& {dəɻɒn-it͡sʰ}&  {dəɻɒn-ov} & {dəɻɒn-um} \\
			&   & {təɻɒn} & {təɻɒ}& {təɻɒn-it͡sʰ}&  {təɻɒn-ov} & {təɻɒn-um} \\
			
			& \armenian{էտ, էտի, էտիկը} & \armenian{դրան} &\armenian{դրա} &\armenian{դրանից} &\armenian{դրանով} &\armenian{դրանում}\\
			% 		& eti & & & & &  \\
			% 		& \armenian{էդի} & & & & &  \\
			
			% 				& {etikə} & & & & &  \\ 
			% 		& \armenian{էդիկը} & & & & &  \\
			\addlinespace 			Dist.  & {en, eni, enikə}   & {nəɻɒn} & {nəɻɒ}& {nəɻɒn-it͡sʰ}&  {nəɻɒn-ov} & {nəɻɒn-um} \\
			& \armenian{էն, էնի, էնիկը} & \armenian{նրան} &\armenian{նրա} &\armenian{նրանից} &\armenian{նրանով} &\armenian{նրանում}\\
			% 		& eni & & & & &  \\
			% 		& \armenian{էնի} & & & & &  \\
			
			% 				& {enikə} & & & & &  \\ 
			% 		& \armenian{էնիկը} & & & & &  \\
			\midrule 
			\multicolumn{2}{l}{Plural}\\
			Prox.     &{səɻɒŋkʰ}& {səɻɒnt͡sʰ}&{səɻɒnt͡sʰ}&{səɻɒnt͡sʰ-it͡sʰ} &{səɻɒnt͡sʰ-ov} & {səɻɒnt͡sʰ-um}\\
			& \armenian{սրանք}& \armenian{սրանց} & \armenian{սրանց} & \armenian{սրանցից}  & \armenian{սրանցով} & \armenian{սրանցում}\\ 
			\addlinespace 	Med. 		& {{dəɻɒŋkʰ}}& {{dəɻɒnt͡sʰ}}&  {{dəɻɒnt͡sʰ}}&  {{dəɻɒnt͡sʰ-it͡sʰ}}&{{dəɻɒnt͡sʰ-ov}}&{{dəɻɒnt͡sʰ-um}}\\
			& {{təɻɒŋkʰ}}& {{təɻɒnt͡sʰ}}&  {{təɻɒnt͡sʰ}}&  {{təɻɒnt͡sʰ-it͡sʰ}}&{{təɻɒnt͡sʰ-ov}}&{{təɻɒnt͡sʰ-um}}\\ 
			& \armenian{դրանք}& \armenian{դրանց} & \armenian{դրանց} & \armenian{դրանցից}  & \armenian{դրանցով} & \armenian{դրանցում}\\ 
			\addlinespace 	Dist. & {{nəɻɒŋkʰ}}& {{nəɻɒnt͡sʰ}}&  {{nəɻɒnt͡sʰ}}&  {{nəɻɒnt͡sʰ-it͡sʰ}}&{{nəɻɒnt͡sʰ-ov}}&{{nəɻɒnt͡sʰ-um}}\\
			& \armenian{նրանք}& \armenian{նրանց} & \armenian{նրանց} & \armenian{նրանցից}  & \armenian{նրանցով} & \armenian{նրանցում}\\ 
			\lspbottomrule
		\end{tabular}}
\end{table}



\section{Interrogative pronouns}\label{section:funct:wh}
\begin{sloppypar}
{\iaIA} seems to use the same set of interrogative pronouns (wh-words) as {\seaSEA} \citep[247]{DumTragut-2009-ArmenianReferenceGrammar}. Full declension paradigms are found in the Dum-Tragut grammar for {\seaSEA}. We have not found   significant differences between {\seaSE} and {\iaIA}  when it comes to the use or form of these interrogative pronouns, and therefore keep this section rather brief. 
In the following sentences, we provide   examples of the different types of interrogative pronouns in bold.  
\end{sloppypar}


The pronoun `who' (\ref{sent:Pronoun:Intr:who}) is [ov] in the nominative (\ref{ex:wh word: who: nom}). But it uses a different root allomorph \textit{um} when case suffixes are added.\footnote{This allomorph /um/ is actually the genitive-dative form of this morpheme in {\seaSEA} \citep[148]{DumTragut-2009-ArmenianReferenceGrammar}.  } Instrumentals and locative suffixes are generally avoided, and replaced with postpositional constructions.

\begin{exe}
	\ex \label{sent:Pronoun:Intr:who}
	\begin{xlist}
		\ex \gll \textbf{ov} ɒ uɻɒχ \\
		who  {\auxgloss} happy \\
		\trans `Who is happy?'\label{ex:wh word: who: nom} \hfill (NK)\\ 
		\armenian{Ո՞վ ա ուրախ։}
		\ex \gll \textbf{um}-i-n e-s mɒkʰɻ-um \\
		who-{\dat}-{\defgloss} {\auxgloss}-2{\sg} clean-{\impfcvb} \\
		\trans `Who are you washing?'\hfill (NK)
		\\ \armenian{Ումի՞ն ես մաքրում։}
		\ex \gll ɡiɻkʰ-ə \textbf{um}-i-n e-s t-ɒ-l-i \\
		book-{\defgloss} who-{\dat}-{\defgloss} {\auxgloss}-2{\sg} give-{\thgloss}-{\infgloss}-{\impfcvb} \\
		\trans `Who do you give the book to?'\hfill (NK)
		\\ \armenian{Գիրքը ումի՞ն ես տալի։}
		\ex \gll \textbf{um}-i ɡiɻkʰ-ə \\
		who-{\gen} book{\defgloss} \\
		\trans `Whose book?'\hfill (NK)
		\\ \armenian{Ումի՞ գիրքը։}
		\ex \gll \textbf{um}-it͡sʰ \\
		who-{\abl} \\
		\trans `From who?'\hfill (NK)
		\\ \armenian{Ումի՞ց։}
		\ex \gll \textbf{um}-i het, \textbf{um}-i met͡ʃʰ \\
		who-{\gen} with, who-{\gen} in \\
		\trans `With who? In who?'\hfill (NK)
		\\ \armenian{Ումի՞ հետ։ Ումի՞ մէջ։}
	\end{xlist}
\end{exe}

The pronoun `what' is [int͡ʃʰ], and there is no case-conditioned suppletion or stem allomorphy involved (\ref{sent:Pronoun:Intr:what}). 

\begin{exe}
	\ex \label{sent:Pronoun:Intr:what} \begin{xlist}
		\ex \gll \textbf{int͡ʃʰ} ɒ kɒput. \textbf{int͡ʃʰ} e-s uz-um \\
		what  {\auxgloss} blue. what {\auxgloss}-2{\sg} want-{\impfcvb} \\
		\trans `What is blue? What do you want?'\hfill (NK)
		\\ \armenian{Ի՞նչ ա կապուտ։ Ի՞նչ ես ուզում։}
		\ex \gll \textbf{int͡ʃʰ}-i(-n) e-s t-ɒ-l-i ɡiɻkʰ-ə \\
		what-{\dat}{(-{\defgloss})} {\auxgloss}-2{\sg} give-{\thgloss}-{\infgloss}-{\impfcvb} book-{\defgloss} \\
		\trans `To what do you give the book?'\hfill (NK)
		\\ \armenian{Ինչի՞ն/Ինչի՞ ես տալի գիրքը։}
		\ex \gll \textbf{int͡ʃʰ}-i ɡujn-ə \\
		what-{\gen} color-{\defgloss} \\
		\trans `The color of what?'\hfill (NK)
		\\ \armenian{Ինչի՞ գոյնը։}
		\ex \gll \textbf{int͡ʃʰ}-it͡sʰ, \textbf{int͡ʃʰ}-ov, \textbf{int͡ʃʰ}-um \\
		what-{\abl}, what-{\ins}, what-{\locgloss}\\
		\trans `From what? With what? In what?' \hfill (NK)
		\\ \armenian{Ինչի՞ց։ Ինչո՞վ։ Ինչո՞ւմ}
		
	\end{xlist}
\end{exe}


The word for `where' can vary between [voɻteʁ] and [uɻ]. NK reports that [uɻ] feels more informal (\ref{sent:Pronoun:Intr:where}). 

\begin{exe}
	\ex \label{sent:Pronoun:Intr:where} \begin{xlist}
		\ex \gll keɻɒkuɻ-ə \textbf{voɻteʁ} ɒ \\
		food-{\defgloss} where  {\auxgloss} \\ 
		\trans `Where is the food?' \hfill (NK) \\
		\armenian{Կերակուրը որտե՞ղ ա։}
		\ex \gll keɻɒkuɻ-ə \textbf{uɻ} ɒ \\
		food-{\defgloss} where  {\auxgloss} \\ 
		\trans `Where is the food?' \hfill (NK) \\
		\armenian{Կերակուրը ո՞ւր ա։}
		\ex \gll \textbf{voɻteʁ}-it͡sʰ. \textbf{voɻteʁ}-um e-s t͡sən-v-e \\
		where-{\abl}. where-{\locgloss} {\auxgloss}-2{\sg} born-{\pass}-{\impfcvb} \\
		\trans `From where? Where were you born?'\hfill (NK)
		\\ \armenian{Որտեղի՞ց։ Որտեղո՞ւմ ես ծնուէ։}
		
	\end{xlist}
\end{exe}

The pronoun `when' is prescriptively [jeɻpʰ], but the rhotic can be deleted in colloquial speech [jepʰ] (\ref{ex:wh word: when: base}). The pronoun takes a special dative/genitive suffix \textit{-vɒn} or \textit{-vɒ} (\ref{ex:wh word: when: va}). This suffix is also used before oblique case suffixes like the ablative (\ref{ex:wh word: when: va stuff}), as a type of oblique stem. 

\begin{exe}
	
	
	\ex 
	\begin{xlist}
		\ex \gll tɒɻedɒɻt͡sʰ-ət \textbf{jeɻpʰ/jepʰ} ɒ \\
		birthday-{\possSsg} when  {\auxgloss} \\
		\trans `When is your birthday?'\label{ex:wh word: when: base} \hfill (NK)
		\\ \armenian{Տարեդարձդ ե՞րբ ա։}
		\ex \gll \textbf{jeɻpʰ}-vɒ   \\
		when-{\gen} \\
		\trans `Of when?'\label{ex:wh word: when: va}\hfill (NK) \\
		\armenian{Երբուա՞յ։}
		\ex \gll \textbf{jeɻpʰ}-vɒn-it͡sʰ \\
		when-{\dat}-{\abl} \\
		\trans `From when?'\label{ex:wh word: when: va stuff} \hfill (NK)\\
		\armenian{Երբուանի՞ց։}
	\end{xlist}
\end{exe}

For the pronoun `why' (\ref{sent:Pronoun:Intr:why}), the Eastern Armenian version is [int͡ʃʰu]. This word is used by the {\iaIA} community as well, but it has a formal connotation. A common colloquial version is [heɻ] \armenian{հեր}, which Sargsyan et al.  \citep[vol. 4: p.  227]{DialectDictionary-2001} report for  New Nakhichevan and several dialects around Lake Van (Moks, Shatakh, Mush, Van). Adjarian  \citep{Adjarian-1979-Etymology}  cites a form /heɾ/  \armenian{հէր} for Tabriz (p. 658) and Maragha (p.  119) and derives it from Classical Armenian /ēɾ/ \armenian{էր}, also meaning ‘why'. Given the presence of [heɻ] \armenian{հեր} in so many of the neighboring southeastern dialects, particularly in Iran, we should not be surprised to come across it in Tehran. 


\begin{exe}
	\ex \label{sent:Pronoun:Intr:why}\begin{xlist} 
		
		\ex \gll \textbf{int͡ʃʰu} \\
		why \\
		\trans `Why?'\hfill (NK) \\
		\armenian{Ինչո՞ւ։}
		
		\ex \gll \textbf{heɻ} uʃ-ɒ-t͡sʰ-ɒ-n \\
		why late-{\lvgloss}-{\aorperf}-{\pst}-3{\pl} \\
		\trans `Why are they late?" \hfill (AS) \\
		\armenian{Հե՞ր ուշացան։}
\end{xlist}	\end{exe}

NK reports that her family uses [heɻ] more often than [int͡ʃʰu] (\ref{sent:Pronoun:Intr:hermore}). She further reports that [int͡ʃʰu] is restricted to more formal speech. 


\begin{exe}
	\ex \label{sent:Pronoun:Intr:hermore}
	\begin{xlist}
		\ex \gll \textbf{heɻ} e-s et hɒkʰ-e \\ 
		why {\auxgloss}-2{\sg} that wear-{\perfcvb} \\
		\trans `Why are you wearing that?' \hfill (NK) \\ 
		\armenian{Հե՞ր ես էտ հագէ։}
		\ex \gll \textbf{heɻ} e-s et ut-um \\ 
		why {\auxgloss}-2{\sg} that eat-{\impfcvb} \\
		\trans% Literal translation: `Why have you worn that'. \\
	%	Functional translation: 
		`Why are you eating that?' 
		\hfill (NK) \\ 
		\armenian{Հե՞ր ես  էտ ուտում։}
		\ex \gll \textbf{heɻ} t͡ʃʰ-e-s zɒŋɡ-um \\ 
		why {\neggloss}-{\auxgloss}-2{\sg}  call-{\impfcvb} \\
		\trans `Why don't you call?' \hfill (NK) \\ 
		\armenian{Հե՞ր չես զանգում։}
	\end{xlist}
\end{exe}

For the pronoun `how', {\seaSEA} uses [int͡ʃʰpes] while {\seaCEA} uses [vont͡sʰ] \citep[154]{DumTragut-2009-ArmenianReferenceGrammar}.   {\iaIA} uses [int͡ʃʰpes] (\ref{sent:Pronoun:Intr:how}).  The modifier version is [int͡ʃʰpesi]. 
 
\begin{exe}
	\ex \label{sent:Pronoun:Intr:how}
	\begin{xlist}
		\ex \gll keɻɒkuɻ-ət \textbf{int͡ʃʰpes} ɒ \\
		food-{\possSsg} how  {\auxgloss} \\
		\trans `How is your food?'\hfill (NK)
		\\ \armenian{Կերակուրդ ինչպէ՞ս ա։}
		\ex \gll \textbf{int͡ʃʰpesi} mɒɻtʰ ɒ \\
		what.kind man   {\auxgloss} \\
		\trans `What kind of man is he?'\hfill (NK)
		\\ \armenian{Ինչպէսի մա՞րդ ա։}
	\end{xlist}
\end{exe}
\section{Numerals}\label{section:funct:num}

{\iaIA} uses essentially the same set of numerals and morphological operations to create complex numerals, as   {\seaSEA}. We focus on cardinals  (\S\ref{section:funct:num:card}) and ordinals (\S\ref{section:funct:num:ord}). For cardinals, there are only minor lexical differences between {\seaSEA} and {\iaIA}. For ordinals, {\iaIA} displays a   difference from {\seaSEA} in the use of irregular morphology in complex numerals. All numeral data in this section was gathered from NK. She gave useful meta-linguistic judgements on variation within the Iranian Armenian community in Los Angeles. {\seaSEA} forms were taken from Wiktionary and double-checked against grammars, the EANC's lexicon,\footnote{\url{https://bitbucket.org/timarkh/uniparser-grammar-eastern-armenian/src/master/}}  and speakers. 

\subsection{Cardinal numerals}\label{section:funct:num:card}

Table \ref{tab:numeral:cardinal:1to10} lists the basic numerals from 0 to 10. Numeral 9 includes the definite suffix /-ə/. We include stress markers because ordinals will later present exceptional stress patterns. 

\begin{table}
	\caption{Cardinal numerals 0--10}\label{tab:numeral:cardinal:1to10}
	\begin{tabular}{lllllll}
		\lsptoprule
		Value & \multicolumn{3}{l}{{\iaIA}}  & \multicolumn{3}{l}{cf. {\seaAbbre}} \\\midrule
		0     & zeˈɻo     & zero      & \armenian{զէրօ}   & zəˈɾo     & zero      & \armenian{զրո}   \\
		1     & ˈmek     & one      & \armenian{մէկ}   & ˈmek     & one      & \armenian{մեկ}   \\
		2     & eɻˈku    & two      & \armenian{էրկու} & jeɾˈku   & two      & \armenian{երկու} \\
		3     & jeˈɻekʰ  & three    & \armenian{երեք}  & jeɾekʰ  & three    & \armenian{երեք}  \\
		4     & ˈt͡ʃʰoɻs & four     & \armenian{չորս}  & ˈt͡ʃʰoɾs & four     & \armenian{չորս}  \\
		5     & ˈhiŋɡ    & five     & \armenian{հինգ}  & ˈhiŋɡ    & five     & \armenian{հինգ}  \\
		6     & ˈvet͡sʰ  & six      & \armenian{վեց}   & ˈvet͡sʰ  & six      & \armenian{վեց}   \\
		7     & ˈjotʰ    & seven    & \armenian{եօթ}   & ˈjotʰ    & seven    & \armenian{յօթ}   \\
		8     & ˈutʰ     & eight    & \armenian{ութ}   & ˈutʰ     & eight    & \armenian{ութ}   \\
		9     & ˈinn-ə   & nine-{\defgloss} & \armenian{իննը}  & ˈin-ə    & nine-{\defgloss} & \armenian{ինը}   \\
		10    & ˈtɒs     & ten      & \armenian{տաս}   & ˈtɑs-ə   & ten-{\defgloss}  & \armenian{տասը}\\
		\lspbottomrule   
	 \end{tabular}
\end{table}

Some minor points of difference between {\seaSE} and {\iaIA}:
a) the numeral 0 has different vowels in {\seaAbbre}    and  {\iaAbbre}, b) the numeral 2 has an initial glide in {\seaAbbre} [jeɾku]  but not  in {\iaAbbre}  [eɻku],\footnote{As discussed in \S\ref{section:morphophono:morphophono:root initial glide}, many polysyllabic words start with /je/ in {\seaAbbre} but an initial /e/ in {\iaAbbre}. It is odd how the numerals 2 and 3 are both bisyllabic but behave differently. }    c) the numeral 9 has an extra nasal  [inn-ə]  in {\iaAbbre}, and d) numeral 10 includes a definite suffix in {\seaAbbre} but not {\iaAbbre}. Note however that an unsuffixed form [tɑs]  is attested in {\seaCEA}.

 The final schwa in these cardinals is morphologically the definite suffix, but it is being used here meaninglessly without contributing definiteness. One cannot add another definite suffix onto  these suffixed roots. And also, this schwa /-ə/ shows the same allomorphy patterns as the definite suffix (\S\ref{section:morphophono:allomorphy: det}), such as a prevocalic /-n-/ (\ref{ex:function:numeral:card:def}). 

\begin{exe}
	\ex \label{ex:function:numeral:card:def} \glll tɑs-n =e ({\seaAbbre}) \\
	 tɒs-n =ɒ ({\iaAbbre}) \\
	ten-{\defgloss} {\auxgloss} \\
	\trans `(The time) is ten.'\\
	\armenian{Տասն է/ա։}
\end{exe}\largerpage[-1]

For numerals 11–19, {\iaIA} admits more variability than {\seaSEA} (Table \ref{tab:numeral:cardinal:11to10}). In {\seaSEA}, a number like 11 is expressed by concatenating the numerals for 10  [tɑs]  and 1  [mek]; the two numerals are separated by the definite suffix /-n-/ and a meaningless connective suffix /-ə-/: [tɑs-n-ə-mek]. {\seaCEA} allows   a simpler construction whereby the intervening `{\defgloss}-{\con}' morphs are omitted: [tɑs-mek]. NK reports that in her {\iaIA} community, both strategies are attested, and she feels that neither is dominant over the other. She   reports that she herself uses the `{\defgloss}-{\con}' template  more often for 15 than for   16.   She also had vowel hiatus in words like 12.


\begin{table}
\caption{Cardinal numerals 11–19 in {\iaIA}}\label{tab:numeral:cardinal:11to10}
\resizebox{\textwidth}{!}{%
	\begin{tabular}{lllllll}
		\lsptoprule 
		Value & \multicolumn{3}{l}{Using {\seaAbbre}-style template}  & \multicolumn{3}{l}{Using {\seaCEAAbbre}-style template}  \\\midrule
		11 & tɒs-n-ə-ˈmek     & 10-{\defgloss}-{\con}-1              & \armenian{տասնմէկ}  & tɒs-ˈmek     & 10-1              & \armenian{տասմէկ}   \\
		12 & tɒs-n-ə-eɻˈku    & 10-{\defgloss}-{\con}-2              & \armenian{տասէրկու} & tɒs-eɻˈku    & 10-2              & \armenian{տասէրկու} \\
		13 & tɒs-n-ə-jeˈɻekʰ  & 10-{\defgloss}-{\con}-3            & \armenian{տասներեք} & tɒs-jeˈɻekʰ  & 10-3            & \armenian{տասերեք}  \\
		14 & tɒs-n-ə-ˈt͡ʃʰoɻs & 10-{\defgloss}-{\con}-4             & \armenian{տասնչորս} & tɒs-ˈt͡ʃʰoɻs & 10-4             & \armenian{տասչորս}  \\
		15 & tɒs-n-ə-ˈhiŋɡ    & 10-{\defgloss}-{\con}-5             & \armenian{տասնհինգ} & tɒs-ˈhiŋɡ    & 10-5             & \armenian{տասհինգ}  \\
		16 & tɒs-n-ə-ˈvet͡sʰ  & 10-{\defgloss}-{\con}-6              & \armenian{տասնվեց}  & tɒs-ˈvet͡sʰ  & 10-6              & \armenian{տասվեց}   \\
		17 & tɒs-n-ə-ˈjotʰ    & 10-{\defgloss}-{\con}-7            & \armenian{տասնեօթ}  & tɒs-ˈjotʰ    & 10-7            & \armenian{տասեօթ}   \\
		18 & tɒs-n-ə-ˈutʰ     & 10-{\defgloss}-{\con}-8            & \armenian{տասնութ}  & tɒs-ˈutʰ     & 10-8            & \armenian{տասութ}   \\
		19 & tɒs-n-ə-ˈinn-ə   & 10-{\defgloss}-{\con}-9-{\defgloss} & \armenian{տասնիննը} & tɒs-ˈinn-ə   & 10-9-{\defgloss} & \armenian{տասիննը} \\ 
		\lspbottomrule
	 \end{tabular}%
 }
\end{table}

A point of difference between {\iaIA} and {\seaSEA} concerns numerals 12, 13, and 18 where the ones digit starts with a glide or vowel: SEA 2 [jeɾku], 3  [jeɾekʰ], 8  [utʰ]. For {\seaAbbre}, the connective schwa and glide are absent: 12  [tɑs-n-eɾku],  13 [tɑs-n-eɾekʰ], 18 [tɑs-n-utʰ]. {\seaCEA} allows the retention of the schwa and of the  numeral's glide: 12  [tɑs-n-ə-jeɾku],  13  [tɑs-n-ə-jeɾekʰ], 18  [tɑs-n-ə-utʰ]. {\iaIA} patterns like {\seaCEAAbbre} in keeping the connective and the glide, except for 12.\footnote{No such differences arise for numeral 17: {\seaAbbre} [tɑs-n-ə-jotʰ] and {\iaAbbre} [tɒs-n-ə-jotʰ].} 

Moving onto the higher numbers (Table \ref{tab:numeral:cardinal:higherNumber}), most     multiples of ten like 30 consist of a root and suffix /-sun/. For illustration, we don't separately segment the root and suffix because their allomorphy is quite opaque. 

\begin{table}
	\caption{Higher cardinal numerals (decades, 100, 1000) in {\iaIA}}\label{tab:numeral:cardinal:higherNumber}
	\begin{tabular}{llll}
		\lsptoprule
		20   & ˈkʰsɒn      & twenty   & \armenian{քսան}       \\
		30   & jeɻeˈsun    & thirty   & \armenian{երեսուն}    \\
		40   & kʰɒrɒˈsun   & forty    & \armenian{քառասուն}   \\
		50   & hiˈt͡sʰun   & fifty    & \armenian{յիսուն}     \\
		60   & vɒˈt͡sʰun   & sixty    & \armenian{վաթսուն}    \\
		70   & jotʰɒnɒˈsun & seventy  & \armenian{եօթանասուն} \\
		80   & utˈt͡sʰun     & eighty   & \armenian{ութսուն}    \\
		90   & innəˈsun    & ninety   & \armenian{իննսուն}    \\
		100  & hɒˈɻuɻ      & hundred  & \armenian{հարուր}     \\
		1000 & hɒˈzɒɻ      & thousand & \armenian{հազար}  \\ 
		\lspbottomrule
	\end{tabular}
\end{table}
	
	
	Numbers 20, 100, and 1000 have their own special forms. For the decade 20, the initial consonant cluster can contain a schwa in careful speech [kʰəsɒn], but it is usually omitted in natural speech \citep[cf. SEA data from][]{Hovakimyan-2016-EasternArmenianClusters}.  NK never produced a schwa for this form. 
	
	The lects differ   for numerals 50, 60, 80, and 100. For {\seaSEA}, these numerals end in /sun/: 50 [hi-sun], 60  [vɑtʰ-sun], 80 [utʰ-sun]. In {\seaCEA}, it's possible to affricate the /s/ in these numerals, as in [hit͡sʰun, vɑt͡sʰun, ut͡sʰun].   {\iaIA} speaker   NK  always affricates these numerals, sometimes also including a /t/ before the affricate: [hi-t͡sʰun, vɒt͡sʰun, utt͡sʰun]. 
	
	For the number 100, {\seaSEA} uses [hɑɾjuɾ] with a glide, while {\iaIA} uses [hɒɻuɻ] without a glide. 
	
	To create complex cardinals, {\iaIA} and {\seaSEA} use the same strategy as English. Numerals are concatenated from the highest number to the lowest. For example, the number 35 is just a concatenation of the numerals 30 and 5: [jeɻesun jeɻekʰ] \armenian{երեսուն երեք}. Our archive includes more examples of complex cardinals that we elicited. 
	
	
	As a final note, these cardinals can act as nouns and take nominal inflection (\ref{sent:cardinal:3:infl}).   When the numeral 2 takes inflection,  it uses a special allomorph [eɻkus] (\ref{sent:cardinal:2:infl}). 
	
	\begin{exe}
		\ex \begin{xlist}
			\ex \gll{jeɻekʰ-ən} e-m uz-um
			\\
			three-{\defgloss} {\auxgloss}-1{\sg} want-{\impfcvb}		\\
			\trans		`I want the three of them.' \label{sent:cardinal:3:infl}\hfill (NK)
			\\
			\armenian{Երեքն եմ ուզում։}
			\ex \gll{eɻkus-ən} e-m uz-um
			\\
			two-{\defgloss} {\auxgloss}-1{\sg} want-{\impfcvb}		\\
			\trans		`I want the two of them.'\label{sent:cardinal:2:infl}\hfill (NK)
			\\
			\armenian{Էրկուսն եմ ուզում։}
		\end{xlist}
	\end{exe}
	\subsection{Ordinal numerals}\label{section:funct:num:ord}
	
	{\iaIA} uses essentially the same set of ordinal numerals and ordinal morphology as {\seaSEA}. However, the two varieties differ in the use of irregular allomorphy in complex ordinals \citep{stump-2010-derivationCompoundOrdinalNumeralImplicationMorphologicalTheory,Dolatian-prep-ordinal}. Briefly, the numeral one displays allomorphy for `first' but not for higher numerals. Numerals 2–4 show allomorphy for their simple ordinals, but their allomorphy is variably percolated to higher numbers.\largerpage 
	
	First, consider numerals 1–10 (Table \ref{tab:numeral:ordinal:1to10}). The ordinal of 1 [mek] is a special suppletive lexeme [ɒrɒt͡ʃʰin].{\interfootnotelinepenalty=10000\footnote{The ordinal [{ɒrɒt͡ʃʰin}] `first' is morphologically related to the word [{ɒrɒt͡ʃʰ}] which means `forward, before' in the modern language. In Classical Armenian, the word also had other meanings like `previous', while the root had other meanings like `front'. The etymological connection between these words is cross-linguistically common \citep[441]{veselinova-1997-suppletionDerivationOrdinalNumeral}.}}  	Numerals 2--4 utilize allomorphy with a special root allomorph and short suffix allomorph /-ɻoɻtʰ/. For example, 2 is [eɻku] but 2\textsuperscript{nd} is [jek-ɻoɻtʰ].{\interfootnotelinepenalty=10000\footnote{For 2, NK  uses a glide in the ordinal but not the cardinal. AM reports more ordinal variation as [je(ɻ)ɡ-ɻoɻtʰ, je(ɻ)k-ɻoɻtʰ].}}
 The ordinals of 5--10 are formed by combining the cardinal root with the default ordinal suffix /-eɻoɻtʰ/: 5 [hiŋɡ] and 5\textsuperscript{th} [hiŋɡ-eɻoɻtʰ]. The ordinal suffixes /-ɻoɻtʰ, -eɻoɻtʰ/ are morphologically exceptional because they are prosodically prestressing (\S\ref{section:phono:suprasegmental:stress:irreg}).\largerpage
	
\begin{table}
	\caption{Ordinal numerals 1–10}\label{tab:numeral:ordinal:1to10}
	\begin{tabular}{llllll}
		\lsptoprule 
		Value & \multicolumn{3}{l}{{\iaIA}}  & \multicolumn{2}{l}{cf. {\seaAbbre}} \\\midrule
		1\textsuperscript{st}  & ɒrɒˈt͡ʃʰin     & first        & \armenian{առաջին}    & ɑrɑˈt͡ʃʰin     & \armenian{առաջին}    \\
		2\textsuperscript{nd}  & ˈjek-ɻoɻtʰ     & two-{\ord}   & \armenian{երկրորդ}   & ˈjeɾk-ɾoɾtʰ    & \armenian{երկրորդ}   \\
		3\textsuperscript{rd}  & ˈje-ɻoɻtʰ      & three-{\ord} & \armenian{երրորդ}    & ˈjeɾ-ɾoɾtʰ     & \armenian{երրորդ}    \\
		4\textsuperscript{th}  & ˈt͡ʃʰo-ɻoɻtʰ   & four-{\ord}  & \armenian{չորրորդ}   & ˈt͡ʃʰoɾ-ɾoɾtʰ  & \armenian{չորրորդ}   \\
		5\textsuperscript{th}  & ˈhiŋɡ-eɻoɻtʰ   & five-{\ord}  & \armenian{հինգերորդ} & ˈhiŋɡ-eɾoɾtʰ   & \armenian{հինգերորդ} \\
		6\textsuperscript{th}  & ˈvet͡sʰ-eɻoɻtʰ & six-{\ord}   & \armenian{վեցերորդ}  & ˈvet͡sʰ-eɾoɾtʰ & \armenian{վեցերորդ}  \\
		7\textsuperscript{th}  & ˈjotʰ-eɻoɻtʰ   & seven-{\ord} & \armenian{եօթերորդ}  & ˈjotʰ-eɾoɾtʰ   & \armenian{յոթերորդ}  \\
		8\textsuperscript{th}  & ˈutʰ-eɻoɻtʰ    & six-{\ord}   & \armenian{ութերորդ}  & ˈutʰ-eɾoɾtʰ    & \armenian{ութերորդ}  \\
		9\textsuperscript{th}  & ˈinn-eɻoɻtʰ    & nine-{\ord}  & \armenian{իններորդ}  & ˈin-n-eɾoɾtʰ   & \armenian{իններորդ}  \\
		10\textsuperscript{th} & ˈtɒs-eɻoɻtʰ    & ten-{\ord}   & \armenian{տասերորդ}  & ˈtɑs-n-eɾoɾtʰ  & \armenian{տասներորդ}\\ 
		\lspbottomrule
		\end{tabular}
\end{table}
	
	
	
	{\seaSEA} uses essentially the same morphemes, with some additional segments for ordinals 2--4, cf. {\seaAbbre} [jeɾk-ɾoɾtʰ]  against {\iaAbbre} [jek-ɾoɾtʰ] `2nd'.  Ordinals 9 and 10 include the definite suffix /-n-/ in {\seaAbbre}.  
	
	The ordinal suffix /-eɻoɻtʰ/ is the default suffix for ordinal formation. Higher numbers like decades use this suffix as well (Table \ref{tab:numeral:ordinal:higherNumber}). 
		
		For complex numbers like 35, the default strategy is to add the ordinal suffix /-eɻoɻtʰ/ to the entire complex cardinal. For example, 35 is [jeɻesun hiŋɡ] \armenian{երեսուն հինգ}, thus the ordinal `thirty-fifth' is [jeɻesun hiŋɡ-eɻoɻtʰ] \armenian{երեսուն հինգերորդ}.\largerpage
		
		Complications arise for complex numerals where the ones digit is 1--4. Recall that for the numeral 1, the cardinal is [mek] and the ordinal is [ɒrɒt͡ʃʰin]. For numerals 2--4, the cardinal is one root allomorph like 2 [eɻku], while the ordinal uses   special root and suffix allomorphs [jek-ɻoɻtʰ]. These two groups of numerals differ in whether their allomorphy is inherited by higher complex cardinals.
		
		
		First consider the numeral 1 and its higher forms (Table \ref{tab:numeral:ordinal:higherOne}). For complex ordinals like 31\textsuperscript{st}, we simply add the ordinal suffix without using the lexeme [ɒrɒt͡ʃʰin], such as [jeɻesun-mek-eɻoɻtʰ]. The lexeme [ɒrɒt͡ʃʰin] is not used for higher forms *\textit{jeɻesun-ɒrɒt͡ʃʰin}.
		
\begin{table}
\caption{Higher ordinal numerals (decades, 100, 1000) in {\iaIA}}\label{tab:numeral:ordinal:higherNumber}
\begin{tabular}{llll}
	\lsptoprule 
	20\textsuperscript{th}   & kʰsɒn-eɻoɻtʰ       & twenty-{\ord}   & \armenian{քսաներորդ}       \\
	30\textsuperscript{th}   & jeɻeˈsun-eɻoɻtʰ    & thirty-{\ord}   & \armenian{երեսուներորդ}    \\
	40\textsuperscript{th}   & kʰɒrɒˈsun-eɻoɻtʰ   & forty-{\ord}    & \armenian{քառասուներորդ}   \\
	50\textsuperscript{th}   & hiˈt͡sʰun-eɻoɻtʰ   & fifty-{\ord}    & \armenian{յիսուներորդ}     \\
	60\textsuperscript{th}   & vɒˈt͡sʰun-eɻoɻtʰ   & sixty-{\ord}    & \armenian{վաթսուներորդ}    \\
	70\textsuperscript{th}   & jotʰɒnɒˈsun-eɻoɻtʰ & seventy-{\ord}  & \armenian{եօթանասուներորդ} \\
	80\textsuperscript{th}   & utˈt͡sʰun-eɻoɻtʰ   & eighty-{\ord}   & \armenian{ութսուներորդ}    \\
	90\textsuperscript{th}   & innəˈsun-eɻoɻtʰ    & ninety-{\ord}   & \armenian{իննսուներորդ}    \\
	100\textsuperscript{th}  & hɒˈɻuɻ-eɻoɻtʰ      & hundred-{\ord}  & \armenian{հարուրերորդ}     \\
	1000\textsuperscript{th} & hɒˈzɒɻ-eɻoɻtʰ      & thousand-{\ord} & \armenian{հազարերորդ}     \\
	 \lspbottomrule    
\end{tabular}
\end{table}
		
\begin{table}
\caption{Allomorphy of numeral 1 in complex ordinals  in {\iaIA}}\label{tab:numeral:ordinal:higherOne}
	\begin{tabular}{llll}
		\lsptoprule
		1 & mek & 1 &    \armenian{մէկ}  \\
		1\textsuperscript{st}& ɑrɑˈt͡ʃʰin  &  first & \armenian{առաջին}    \\
		21 & kʰsɒn-ˈmek    & 20-1 & \armenian{քսան մէկ}    \\
		21\textsuperscript{st}& kʰsɒn-ˈmek-eɻoɻtʰ    & 20-1-{\ord} & \armenian{քսան մէկերորդ} \\     
		31 & jeɻesun-ˈmek  & 30-1 & \armenian{երեսուն մէկ}  \\
		31\textsuperscript{st}   & jeɻesun-ˈmek-eɻoɻtʰ  & 30-1-{\ord} & \armenian{երեսուն մէկերորդ} \\
		41 & kʰɒrɒsun-ˈmek & 40-1 & \armenian{քառասուն մէկ}  \\
		41\textsuperscript{st} & kʰɒrɒsun-ˈmek-eɻoɻtʰ & 40-1-{\ord} & \armenian{քառասուն մէկերորդ} \\
		\lspbottomrule    
	\end{tabular}
\end{table}
			
			Such patterns of limited allomorphy in higher numbers have been called external marking \citep{stump-2010-derivationCompoundOrdinalNumeralImplicationMorphologicalTheory}. The idea is that the ordinal of a complex cardinal like 31 is treated as an exocentric construction, and that the component 1 numeral cannot use its special allomorph in   complex cardinals. 
				
			\begin{sloppypar}
			   {\seaSEA} shows the same patterns for the non-use of [ɑrɑt͡ʃʰin] in higher numbers (\citealt[120]{DumTragut-2009-ArmenianReferenceGrammar}). For example, 21\textsuperscript{st} in {\seaSEA} is simply [jeɾesun-mek-eɾoɾtʰ] and not *\textit{jeɾesun-ɑrɑt͡ʃʰin}. 
		   \end{sloppypar}
			
			Different behavior is found  for complex ordinals where the ones digit is 2--4. Consider the numeral 2 [eɻku]. Its ordinal is [jek-ɻoɻtʰ] with special root-suffix allomorphs.  NK reports that she uses the same allomorphs for both simplex ordinals like 2  and complex ordinals like 32: [jeɻesun-jek-ɻoɻtʰ]  (Table \ref{tab:numeral:ordinal:higherTWoFour}). Such patterns are typologically called internal-marking \citep{stump-2010-derivationCompoundOrdinalNumeralImplicationMorphologicalTheory}, metaphorically meaning that the complex ordinal is treated like an endocentric compound. 
			
			
			
\begin{table}
	\caption{Allomorphy of numerals 2--4 in complex ordinals  in {\iaIA} from NK}\label{tab:numeral:ordinal:higherTWoFour}
	\begin{tabular}{llll}
		\lsptoprule
		2                      & eɻˈku                 & 2           & \armenian{էրկու}            \\
		2\textsuperscript{nd}  & ˈjek-ɻoɻtʰ            &  2-{\ord}  & \armenian{երկրորդ}          \\
		22                     & kʰsɒn-eɻˈku           & 20-2        & \armenian{քսան էրկու}       \\
		22\textsuperscript{nd} & kʰsɒn-ˈjek-ɻoɻtʰ      & 20-2-{\ord} & \armenian{քսան երկրորդ}     \\
		32                     & jeɻesun-eɻˈku         & 30-2        & \armenian{երեսուն էրկու}    \\
		32\textsuperscript{nd} & jeɻesun-ˈjek-ɻoɻtʰ    & 30-2-{\ord} & \armenian{երեսուն երկրորդ}  \\
		42                     & kʰɒrɒsun-eɻˈku        & 40-2        & \armenian{քառասուն էրկու}   \\
		42\textsuperscript{nd} & kʰɒrɒsun-ˈjek-ɻotʰ    & 40-2-{\ord} & \armenian{քառասուն երկրորդ} \\
		\midrule
		3                      & jeˈɻekʰ               & 3           & \armenian{երեք}             \\
		3\textsuperscript{rd}  & ˈje-ɻoɻtʰ             & 3-{\ord}    & \armenian{երրորդ}           \\
		23                     & kʰsɒn-jeˈɻekʰ         & 20-3        & \armenian{քսան երեք}        \\
		23\textsuperscript{rd} & kʰsɒn-ˈje-ɻoɻtʰ       & 20-3-{\ord} & \armenian{քսան երրորդ}      \\
		33                     & jeɻesun-jeˈɻekʰ       & 30-3        & \armenian{երեսուն երեք}     \\
		33\textsuperscript{rd} & jeɻesun-ˈje-ɻoɻtʰ     & 30-3-{\ord} & \armenian{երեսուն երրորդ}   \\
		43                     & kʰɒrɒsun-jeˈɻekʰ      & 40-3        & \armenian{քառասուն երեք}    \\
		43\textsuperscript{rd} & kʰɒrɒsun-ˈje-ɻoɻtʰ    & 40-3-{\ord} & \armenian{քառասուն երրորդ}  \\
		\midrule
		4                      & ˈt͡ʃʰoɻs              & 4           & \armenian{չորս}             \\
		4\textsuperscript{th}  & ˈt͡ʃʰo-ɻoɻtʰ          & 4-{\ord}    & \armenian{չորրորդ}          \\
		24                     & kʰsɒn-ˈt͡ʃʰoɻs        & 20-4        & \armenian{քսան չորս}        \\
		24\textsuperscript{th} & kʰsɒn-ˈt͡ʃʰo-ɻoɻtʰ    & 20-4-{\ord} & \armenian{քսան չորրորդ}     \\
		34                     & jeɻesun-ˈt͡ʃʰoɻs      & 30-4        & \armenian{երեսուն չորս}     \\
		34\textsuperscript{th} & jeɻesun-ˈt͡ʃʰo-ɻoɻtʰ  & 30-4-{\ord} & \armenian{երեսուն չորրորդ}  \\
		44                     & kʰɒrɒsun-ˈt͡ʃʰoɻs     & 40-4        & \armenian{քառասուն չորս}    \\
		44\textsuperscript{th} & kʰɒrɒsun-ˈt͡ʃʰo-ɻoɻtʰ & 40-4-{\ord} & \armenian{քառասուն չորրորդ}\\ 
		\lspbottomrule
	\end{tabular}
\end{table}
			
			{\seaSEA} crucially differs from NK's {\iaAbbre} ideolect in this regard. In {\seaAbbre}, a numeral like 2  [jeɾku] cannot percolate its irregular form [jeɾk-ɾoɾtʰ] to higher numerals. Thus, the ordinal of 32 in {\seaAbbre} is [jeresun-eɾku-eɾoɾtʰ] with the default ordinal suffix, and not *\textit{jeɾesun-eɾk-ɾoɾtʰ} with the special allomorphs (\citealt[209]{Sargsyan-1985-WesternEasternArmenian}, \citealt[308]{Hagopian-2007-ArmenianTextbookEveryone}). For discussion on such ordinal variation in Armenian, see \citet{Dolatian-prep-ordinal}. 
			
			NK informs us that, because of this difference between {\seaSEA} and {\iaIA},  her colleagues and family gave her contradictory judgments on the correct formation of complex ordinals like 32. Some recommended the use of the {\seaAbbre}-style ordinal with the default ordinal suffix /-eɻoɻtʰ/ (like [jeɻesun-eɻku-eɻoɻtʰ]), while she and her friends preferred the use of the irregular ordinal suffix /-ɻoɻtʰ/ (like [jeɻesun-jek-ɻoɻtʰ]). Anooshik Melikian (AM, an {\iaIA} speaker from Tehran) likewise reports that NK's colloquial constructions are   attested across educated and non-educated speakers in Tehran. The use of the {\seaAbbre}-style construction is obviously due to the prestige of {\seaAbbre}, as a form of prescriptivism. 

\section{Other function words}\label{section:funct:other}
The following  are lists of   function  words that we have elicited which do not  fit neatly into the previous sections. As of writing this grammar, we have not been able to study these function words extensively.  




{\iaIA} uses the   adverbial function words in Table \ref{tab:loc adverb loc} to indicate location, e.g., the equivalent of English `here' and `there'. As with demonstratives, these locational words distinguish between proximal, medial, and distal locations. We specify the source of the items.  

\begin{table}
	\caption{Location adverbs in {\iaIA}}
	\label{tab:loc adverb loc}
	\begin{tabular}{lll}
	\lsptoprule
	Proximal  &    {esteʁ} (KM, NK), {ste} (AS), {steʁ} (AS) & `this place'\\
	& \armenian{էստեղ, ստեղ, ստէ} & 		\\\addlinespace
	Medial & {etteʁ} (AS, NK)  & `that place'\\
	& \armenian{էդտեղ} & 		\\\addlinespace
	Distal &{əndeʁ}  (AS), {ənde} (AS), &  `that place yonder'\\
	&  {ənne} (NK, KM), ənneʁ (KM)  & \\ 
	& \armenian{ընտեղ, ընտէ, ըննէ, ըննեղ} & 		\\
	\lspbottomrule	
\end{tabular}
\end{table}


All these words like [esteʁ] `this place' are morphologically derived from a demonstrative like [es] `this' and the word `place' [teʁ]. Note how the [t] becomes [d] after the nasal in [əndeʁ] `that place yonder'. Post-nasal voicing seems limited to such function words. 

To illustrate, the following sentence shows a location adverb (\ref{sent:loc adverb loc:ex}).

\begin{exe}
	\ex \gll {ɡən-ɒ} {ənne} 
	\\
	go-{\thgloss} there'
	\\
	\trans		`Go over there.' \hfill (NK)\label{sent:loc adverb loc:ex}
	\\
	\armenian{Գնա ըննէ։}
\end{exe}

We likewise elicited the following adverbs of manner (Table \ref{tab:loc adverb manner}).


\begin{table}
	\caption{Manner adverbs in {\iaIA} from AS}
	\label{tab:loc adverb manner}
	\begin{tabular}{llll}
		\lsptoprule 
		Proximal  &    {esent͡sʰ}, {sent͡sʰi}  & \armenian{էսենց, սենցի} & `like this'\\  
		Medial & {etent͡sʰ}, {tent͡sʰi}    & \armenian{էտենց, տենցի} & `like that'\\
		Distal & {nent͡sʰi}   & \armenian{նենցի} & `like that yonder'\\
		\lspbottomrule
	\end{tabular}
\end{table}

An additional adverb of manner is [hent͡sʰ], which has a broad range of uses, often translatable to the English word `just' (\ref{sent:Pronoun:Other:hents}). 

\begin{exe}
	\ex \label{sent:Pronoun:Other:hents}\begin{xlist}
		\ex \gll  hent͡sʰ et \\
just that \\
		\trans `That's it' \hfill (NK) \\
		\armenian{Հէնց էտ։}
		\ex \gll  hent͡sʰ himɒ \\
just now \\
		\trans `Right now' \hfill (NK) \\
		\armenian{Հէնց հիմա։}
	\end{xlist}
\end{exe}


{\iaIA} has   a modal word [piti]  that roughly translates to `must'  (\ref{sent:Pronoun:Other:piti}).  It is used to create a debitive or obligative mood \citep[263]{DumTragut-2009-ArmenianReferenceGrammar}. 


\begin{exe}
	\ex \label{sent:Pronoun:Other:piti} \begin{xlist}
		\ex \gll   piti etʰ-ɒ-m  \\ 
		must go-{\thgloss}-1{\sg} \\ 
		\trans `I have to go.' \hfill (NK) \\ 
		\armenian{Պիտի էթամ։}
		\ex \gll   piti ut-e-m  \\ 
		must eat-{\thgloss}-1{\sg} \\ 
		\trans `I have to eat.' \hfill (NK) \\ 
		\armenian{Պիտի ուտեմ։}
		
	\end{xlist}
\end{exe}

This word is related to the syntactic construction [petʰk ɒ] which is used to mean `it is needed' or `it is necessary' (\ref{sent:Pronoun:Other:petka}). 

\begin{exe}
	\ex \label{sent:Pronoun:Other:petka}
	\begin{xlist}
		\ex \gll   petkʰ ɒ    \\ 
		need {\auxgloss} \\ 
		\trans `It is needed.' \hfill (NK) \\ 
		\armenian{Պէտք ա։}
		\ex \gll   petkʰ ɒ ut-e-m  \\ 
		need {\auxgloss} eat-{\thgloss}-1{\sg} \\ 
		\trans `I have to eat.' \hfill (NK) \\
		Literally: `It is needed that I eat.'\\
		\armenian{Պէտք ա ուտեմ։}
	\end{xlist}
\end{exe}
